\documentclass{article}
\usepackage{doc,url,verbatim,fancyvrb}
\usepackage{pifont}
\usepackage[authoryear]{natbib}
\usepackage[pdftex]{graphicx}
\usepackage{gretl}
\usepackage[letterpaper,body={6.3in,9.15in},top=.8in,left=1.1in]{geometry}
\usepackage[pdftex,hyperfootnotes=false]{hyperref}

%\usepackage[a4paper,body={6.1in,9.7in},top=.8in,left=1.1in]{geometry}

\begin{document}

\setlength{\parindent}{0pt}
\setlength{\parskip}{1ex}

\newcommand{\argname}[1]{\textsl{#1}}

\title{dbnomics for gretl, version 0.1}
\author{Jack Lucchetti \and Allin Cottrell}
\maketitle

\section{Introduction}

This package offers a preliminary version of an interace to
\textsf{dbnomics} for gretl. For anyone who hasn't yet caught on,
\textsf{dbnomics} makes available a large number of macroeconomic data
series from many ``providers'' in Europe, the USA and elsewhere, an
admirable service.

At this stage the package does not provide a means of browsing the
offerings of \textsf{dbnomics} within gretl (other than, as proof of
concept, getting a list of data providers). For the core business of
accessing individual time-series, users will first of all have to
browse the (new version of the) \textsf{dbnomics} site
(\url{https://next.nomics.world/}) themselves to discover the series
IDs for data of interest.

\section{Command-line usage}

The two functions that are likely to be of greatest interest in the
first instance are
\begin{code}
dbnomics_get_series(const string datacode, bool verbose[0])
\end{code}
which returns a gretl bundle, and
\begin{code}
dbnomics_bundle_get_data(const bundle b, series *x)
\end{code}
which read a bundle (sucessfully) returned by
\verb|dbnomics_get_series| and writes the data values into the series
given as the second argument in ``pointer'' form.

The \texttt{datacode} argument to the first of these functions should
be a slash-separated triplet, \texttt{provider/dataset/series}, for
example
\begin{code}
ECB/IRS/M.IT.L.L40.CI.0000.EUR.N.Z
\end{code}
where the provider is \texttt{ECB} (the European Central Bank), the
dataset is \texttt{IRS} (interest rates) and the series is
\texttt{M.IT.L.L40.CI.0000.EUR.N.Z} (yes, it's a bit of a mouthful,
but it identifies the Italian 10-year interest rate). After browsing
the \textsf{dbnomics} site for a while these fields ahould become more
familiar!

If you wish to add the above-mentioned series to a suitable gretl
dataset (or use it to start a dataset when no data are currently
loaded in gretl), the basic routine would be something like
\begin{code}
bundle b = dbnomics_get_series("ECB/IRS/M.IT.L.L40.CI.0000.EUR.N.Z")
series IT_10yr = NA
dbnomics_bundle_get_data(b, &IT_10yr)
\end{code}


\section{GUI usage}




\end{document}
