\documentclass{article}
\usepackage{smallhds,lucidabr}
\usepackage{url}

\begin{document}

\setlength{\parskip}{1ex}
\setlength{\parindent}{0pt}

\begin{center}
  {\large \textbf{Building a gretl disk image for OS X}}\\
Allin Cottrell, December 2007
\end{center}

\section{Objective}

To build a stand-alone disk image (dmg) of gretl for Mac OS X.  The
final user should be able to download the dmg file, double-click to
mount it, and drag the Gretl.app folder (found ``inside'' the image)
to an Applications folder.  You'll use Fink in the build process but
the final dmg should not be dependent on Fink in any way; it will,
however, be dependent on Apple's X11.

\section{Overview}

First, here are the prerequisites:

\begin{itemize}
\item A fully functional installation of OS X.
\item Apple's X11 and the Xcode development package.  If these are
  not already installed, they should be found on the OS X installation
  DVDs.
\item A basic installation of Fink.
\item Source code for gretl and gnuplot.
\item A skeleton for Gretl.app plus some auxiliary scripts.
\end{itemize}

The method is as follows:

\begin{itemize}
\item Install the Gretl.app skeleton.  This provides the ``space''
  into which you'll install gretl and gnuplot.
\item Under Fink, install the various required third-party packages
  (including the ``dev'' or developer components).  This includes
  GTK+ and friends (glib, atk, gdk, pango).
\item Configure and build gretl; install gretl into the right place
  inside the Gretl.app folder; delete some extraneous files.
\item Patch, configure and build gnuplot; install gnuplot into
  Gretl.app; delete extraneous files.
\item Tar up various run-time files from your Fink installation
  and dump them into the appropriate place in Gretl.app (hence
  removing the dependency on Fink at run time).
\item Grab the latest gretl documentation and dump it into Gretl.app.
\item Create a compressed disk image containing Gretl.app.
\end{itemize}

The following sections expand on each of the steps.

\section{The Gretl.app skeleton}

I'll make a gzipped tar file available.  This will contain a mostly
empty directory tree, but I'll include some ``generic'' files that
shouldn't depend on the particular OS X build platform.  This should
be unzipped in some suitable location; on the OS X system to which I
have access I've put it under \texttt{/Users/allin/dist}.

\section{Required Fink packages}

The exact line-up of these packages depends somewhat on the specific
OS X variant.  If a given package is available via OS X itself, then
you don't need to, and probably don't want to, install the
corresponding Fink package.  A case in point is libxml2, which is
supplied on recent OS X (but was not supplied in earlier variants).

The required packages will presumably include gtk+2, gtk+2-dev and
fftw3;  recode may also be required; gnuplot is not required since
we'll be building that ourselves.  Libxml2 will hopefully be supplied
by OS X, and dlcompat doesn't seem to be needed any longer.

\section{Configuring and building gretl}

Needed: sample configure params; postinst shell script.


\section{Patching and building gnuplot}

Needed: patch plus instructions, sample configure params;
post-installation script.

\section{Copying Fink run-time files}

Explanation; sample script.

\section{Documentation files}

The canonical PDF documentation for gretl is available from
\url{http://ricardo.ecn.wfu.edu/pub/gretl/manual/en/}.  You should do

\begin{verbatim}
wget http://ricardo.ecn.wfu.edu/pub/gretl/manual/en/gretl-guide.pdf
wget http://ricardo.ecn.wfu.edu/pub/gretl/manual/en/gretl-ref.pdf
cp gretl-guide.pdf XXX
cp gretl-ref.pdf XXX

\end{verbatim}


\section{Creation of dmg}

Below is a shell script to create the final .dmg file.  There's a copy
of this in the gretl source package, in the osx subdirectory (called
dmg.sh).  Obviously, you'll need to edit the line that defines TOPDIR;
hopefully the rest should be portable.  Note that you need the
top-level README.pdf for users of gretl on OS X.  That's also
available in the osx subdirectory of the gretl source.

\begin{verbatim}
#!/bin/bash

# the directory above Gretl.app
TOPDIR=/Users/allin/dist

HERE=`pwd`
KB=`du -ks ~/dist | awk '{ print $1 }'`
KB=$((KB+640))
hdiutil create -size ${KB}k tmp.dmg -layout NONE
MYDEV=`hdid -nomount tmp.dmg`
sudo newfs_hfs -v gretl $MYDEV
hdiutil eject $MYDEV
hdid tmp.dmg
cd $TOPDIR && \
cp -a Gretl.app /Volumes/gretl && \
cp -a README.pdf /Volumes/gretl
cd $HERE
hdiutil eject $MYDEV
hdiutil convert -format UDZO tmp.dmg -o gretl.dmg && rm tmp.dmg

\end{verbatim}

\end{document}

%%% Local Variables: 
%%% mode: latex
%%% TeX-master: t
%%% End: 
