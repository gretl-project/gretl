\documentclass{article}
\usepackage[T1]{fontenc}
\usepackage[body={6.5in,9in},top=1in,left=1in,nohead]{geometry}
\usepackage{smallhds,lucidabr}
\usepackage{verbatim,fancyvrb}
\usepackage{url}

\DefineVerbatimEnvironment%
{code}{Verbatim}
{fontsize=\small, xleftmargin=1em}


\begin{document}

\setlength{\parskip}{1ex}
\setlength{\parindent}{0pt}

\begin{center}
  {\large \textbf{Building a gretl disk image for OS X}}\\[6pt]
Allin Cottrell, December 2007
\end{center}

\section{Objective}

To build a stand-alone disk image (dmg) of gretl, including a suitably
configured version of gnuplot, for Mac OS X.  The final user should be
able to download the dmg file, double-click to mount it, and drag the
\texttt{Gretl.app} folder (found ``inside'' the image) to an
Applications folder.  You'll use Fink in the build process but the
final dmg should not be dependent on Fink in any way; it will,
however, be dependent on Apple's X11.

\section{Overview}

Here are the basic prerequisites:

\begin{itemize}
\item A fully functional installation of OS X.
\item Apple's X11 and the Xcode development package.  If these are
  not already installed, they should be found on the OS X installation
  DVDs.
\item A basic installation of Fink.
\item Source code for gretl and gnuplot.
\item A skeleton for \texttt{Gretl.app} plus some auxiliary scripts.
\end{itemize}

The method is as follows:

\begin{enumerate}
\item Install the \texttt{Gretl.app} skeleton.  This provides the
  ``space'' into which you'll install gretl and gnuplot.
\item Under Fink, install various required third-party packages
  (including the ``dev'' or developer components).  This includes
  GTK+ and friends (glib, atk, gdk, pango).
\item Configure and build gnuplot; install gnuplot into
  the \texttt{Gretl.app} folder.
\item Configure and build gretl; install gretl into the right place
  inside \texttt{Gretl.app}; delete some extraneous files and add some
  extras.
\item Tar up various run-time files from your Fink installation and
  dump them into the appropriate place in \texttt{Gretl.app} (hence
  removing the dependency on Fink at run time).  This is the trickiest
  part.
\item Grab the latest gretl documentation and dump it into place.
\item Create a compressed disk image containing \texttt{Gretl.app}.
\end{enumerate}

Steps 1, 2, 3 and 5 only need to be done once; thereafter you can
update the disk image with just steps 4, 6 and 7.

The following sections expand on each of the steps.

\section{The Gretl.app skeleton}

I'm making a gzipped tar file available.  This is mostly an empty
directory tree, but it includes some ``generic'' files that shouldn't
depend on the particular OS X build platform (though see the final
section below).  This should be unzipped in some suitable location; on
the OS X system to which I have access I've put it under
\texttt{/Users/allin/dist}.

\url{http://ricardo.ecn.wfu.edu/~cottrell/gretl-osx/Gretl.app.tar.gz}

\section{Required Fink packages}

The exact line-up of these packages depends somewhat on the specific
OS X variant.  If a given package is available via OS X itself, then
you don't need to, and probably don't want to, install the
corresponding Fink package.  A case in point is libxml2, which is
supplied on recent OS X (but was not supplied in earlier variants).

The required packages will presumably include gtk+2, gtk+2-dev and
fftw3; recode may also be required; gnuplot is not required since
we'll be building that ourselves.  Libxml2 will hopefully be supplied
by OS X, and dlcompat doesn't seem to be needed any longer. It may be
helpful to install wget via Fink for build purposes.

Since we're building gnuplot, the libraries to be installed via Fink
also include those needed by gnuplot (more on this below).

\section{Building gnuplot}

Grab the patched source for gnuplot 4.2.2,
\url{http://ricardo.ecn.wfu.edu/~cottrell/gretl-osx/gnuplot-4.2.2-ac.tar.gz}.
Untar and configure.  FIXME: complete this section.


\section{Configuring and building gretl}

There's a file \texttt{myconf} in the osx subdirectory of the gretl
source.  You should use this, or a variant of it, to configure gretl.
Here's what it looks like:

\begin{code}
export CFLAGS="-O2 -I/sw/include"
export LDFLAGS=-L/sw/lib
export CPPFLAGS=$CFLAGS
export PKG_CONFIG_PATH="/usr/lib/pkgconfig:/usr/X11R6/lib/pkgconfig:/sw/lib/pkgconfig"
export PATH=/Users/allin/dist/Gretl.app/Contents/Resources/bin:$PATH
./configure --prefix=/Users/allin/dist/Gretl.app/Contents/Resources \
  --disable-rpath --enable-build-doc
\end{code}

The ``\texttt{export PATH}'' line is designed to ensure that the
version of gnuplot installed at the previous step is found during the
gretl configuration process.  This may not be necessary if a version
of gnuplot that supports PNG output is already in your path.

After doing \texttt{make} and \texttt{make install} we run a script
named postinst to clear out unnecessary files and add a few extra
things needed for OS X.  This is also in the \texttt{osx} subdir of
the source.

\begin{code}
#!/bin/sh
# postinst: run this in the gretl build directory

# The directory above Gretl.app
TOPDIR=/Users/allin/dist
PREFIX=$TOPDIR/Gretl.app/Contents/Resources

rm -f $PREFIX/bin/gretl
rm -rf $PREFIX/include
rm -rf $PREFIX/share/aclocal
rm -rf $PREFIX/share/info
rm -rf $PREFIX/info
rm -rf $PREFIX/lib/pkgconfig
rm -f $PREFIX/lib/gretl-gtk2/*.la
rm -rf $PREFIX/share/emacs

install -m 644 osx/README.pdf $TOPDIR
install -m 755 osx/gretl.sh $PREFIX/bin/gretl
\end{code}

\section{Copying Fink run-time files}

As mentioned above, this is a bit tricky.  The general idea is that we
want to identify all the files provided by Fink that are necessary to
support gretl and/or gnuplot, and copy these into the right places
under \texttt{Gretl.app}.  To keep the disk image as compact as
possible, we want to try to ensure that we copy \textit{only} those
files that are really necessary.  This includes all shared libraries
that are not provided by OS X itself; it also includes some additional
run-time files required by GTK+.

FIXME: complete this section.

\textit{Testing}: To check that the \texttt{Gretl.app} is
self-contained, you need to run gretl with Fink disabled.  To help
with this I have two little scripts, as follows:

\begin{code}
# disable Fink
sudo mv /sw /hidden.sw
hash -r

# enable Fink
sudo mv /hidden.sw /sw
hash -r

\end{code}

The \texttt{hash -r} command is required to ensure that common
utilities such as \texttt{cp}, which are present in both \texttt{/bin}
and \texttt{/sw/bin}, are found after the switch.

\section{Documentation files}

The canonical PDF documentation for gretl is available from
\url{ricardo.ecn.wfu.edu}.  You should do something like the following
(wget can be installed via Fink; you could use curl instead if you
prefer):

\begin{code}
TOPDIR=/Users/allin/dist
DOCDIR=$TOPDIR/Gretl.app/Contents/Resources/share/gretl/doc
rm -f gretl-guide.pdf
wget http://ricardo.ecn.wfu.edu/pub/gretl/manual/PDF/gretl-guide.pdf
cp gretl-guide.pdf $DOCDIR
rm -f gretl-ref.pdf
wget http://ricardo.ecn.wfu.edu/pub/gretl/manual/PDF/gretl-ref.pdf
cp gretl-ref.pdf $DOCDIR
\end{code}

\section{Creation of dmg}

Below is a shell script to create the final compressed .dmg file.
There's a copy in the gretl source package, in the osx subdirectory
(called \texttt{dmg.sh}).  Obviously, you'll need to edit the line
that defines \texttt{TOPDIR}; hopefully the rest should be portable.

This script should be run from some ``neutral'' location outside of
the distribution tree; you don't want to get a recursive thing going,
whereby the dmg is included within itself.  I run \texttt{dmg.sh} from
\verb+~/bin+.

\begin{code}
#!/bin/bash

# the directory above Gretl.app
TOPDIR=/Users/allin/dist

HERE=`pwd`
KB=`du -ks $TOPDIR | awk '{ print $1 }'`
KB=$((KB+1024))
hdiutil create -size ${KB}k tmp.dmg -layout NONE
MYDEV=`hdid -nomount tmp.dmg`
sudo newfs_hfs -v gretl $MYDEV
hdiutil eject $MYDEV
hdid tmp.dmg
cd $TOPDIR && \
cp -a Gretl.app /Volumes/gretl && \
cp -a README.pdf /Volumes/gretl
cd $HERE
hdiutil eject $MYDEV
hdiutil convert -format UDZO tmp.dmg -o gretl.dmg && rm tmp.dmg
\end{code}

%$
This script uses the \texttt{-a} flag to the \texttt{cp} command: that
is not supported by \texttt{/bin/cp} under OS X, but it is supported by 
Fink's \texttt{/sw/bin/cp}.

\section{Extra: ScriptExec stuff}

Something else should be mentioned.  To make Gretl.app into a proper
OS X Application ``bundle'' we use the ScriptExec apparatus from
gimp.app, at \url{http://gimp-app.sourceforge.net/}.  This apparatus
allows for launching gretl from an icon, and automatic startup of X11.

The \texttt{Gretl.app} skeleton mentioned above contains all the files
generated in association with ScriptExec as built on OS X 10.4.11.
So with any luck you should not have to mess with this.  On the other
hand it's possible that the files generated on Tiger don't work
properly with Leopard (or higher), so it may be worth regenerating
them.

An account of how to do this (for gimp, but \textit{mutatis mutandis}
for gretl) is given at the gimp.app URL above.  I can't add much to
what's said there as I have no expertise in Xcode and I worked by trial
and error.  The basic idea is that you have to build ScriptExec as an
Xcode project, then copy the generated bits and pieces into place
under \texttt{Gretl.app}.

\end{document}

%%% Local Variables: 
%%% mode: latex
%%% TeX-master: t
%%% End: 
