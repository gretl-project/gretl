\chapter{User-written functions}

Main structure:

\begin{flushleft}
\texttt{function \emph{type} \emph{funcname}(\emph{arguments})}\\
   \quad \ldots\\
   \quad \emph{function body}\\
   \quad \ldots \\
\texttt{end function}
\end{flushleft}
\section{Parameter passing and return values}
\label{sec:params-returns}



\section{Recursion}

Obligatory factorial example
\begin{code}
function scalar factorial(scalar n)
    if (n<0) || (n>floor(n))
        # filter out everything that isn't a 
        # non-negative integer
        return NA
    elif n==0
        return 1
    else
        return n*factorial(n-1)
    endif
end function

loop i=0..6 --quiet
    printf "%d! = %d\n", i, factorial(i)
end loop
\end{code}

Note: this is fun, but in practice, you'll be much better off using
the pre-cooked gamma function (or, better still, its logarithm).

%%% Local Variables: 
%%% mode: latex
%%% TeX-master: "hansl-primer"
%%% End: 
