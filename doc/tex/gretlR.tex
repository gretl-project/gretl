\chapter{Gretl and R}
\label{chap:gretlR}

\section{Introduction}
\label{R-intro}

\app{R} is, by far, the largest free statistical
project.\footnote{\app{R}'s homepage is at
  \url{http://www.r-project.org/}.} Like \app{gretl}, it is a GNU
project and the two have a lot in common; however, \app{gretl}'s
approach focuses on ease of use much more than \app{R}, which instead
aims to encompass the widest possible range of statistical procedures.

As is natural in the free software ecosystem, we don't view ourselves
as competitors to \app{R},\footnote{OK, who are we kidding? But it's
  \emph{friendly} competition!} but rather as projects sharing a common
goal who should support each other whenever possible. For this reason,
\app{gretl} provides a way to interact with \app{R} and thus enable
users to pool the capabilities of the two packages.

In this chapter, we will explain how to exploit \app{R}'s power from
within \app{gretl}. We assume that the reader has a working
installation of \app{R} available and a basic grasp of \app{R}'s
syntax.\footnote{The main reference for \app{R} documentation is
  \url{http://cran.r-project.org/manuals.html}.  In addition, \app{R}
  tutorials abound on the Net; as always, Google is your friend.}

Despite several valiant attempts, no graphical shell has gained wide
acceptance in the \app{R} community: by and large, the standard method
of working with \app{R} is by writing scripts, or by typing commands
at the \app{R} prompt, much in the same way as one would write
\app{gretl} scripts or work with the \app{gretl} console. In this
chapter, the focus will be on the methods available to execute \app{R}
commands without leaving \app{gretl}.

\section{Starting an interactive \app{R} session}
\label{sec:R-interactive}

The easiest way to use \app{R} from \app{gretl} is in interactive
mode.  Once you have your data loaded in \app{gretl}, you can select
the menu item ``Tools, Start GNU R'' and an interactive \app{R}
session will be started, with your dataset automatically pre-loaded.

\subsection{A simple example: OLS on cross-section data}
\label{sec:R-ols-ex}

For this example we use Ramanathan's dataset \texttt{data4-1}, one of
the sample files supplied with \app{gretl}.  We first run, in
\app{gretl}, an OLS regression of \texttt{price} on \texttt{sqft},
\texttt{bedrms} and \texttt{baths}.  The basic results are shown in
Table \ref{tab:data4-1-gretlOLS}.

\begin{table}[htbp]
\caption{OLS house price regression via \app{gretl}}
\label{tab:data4-1-gretlOLS}
\begin{center}

\begin{tabular*}{0.75\textwidth}{@{\extracolsep{\fill}}
l% col 1: varname
  D{.}{.}{-1}% col 2: coeff
    D{.}{.}{-1}% col 3: sderr
      D{.}{.}{-1}% col 4: t-stat
        D{.}{.}{4}}% col 5: p-value (or slope)
Variable &
  \multicolumn{1}{c}{Coefficient} &
    \multicolumn{1}{c}{Std.\ Error} &
      \multicolumn{1}{c}{$t$-statistic} &
        \multicolumn{1}{c}{p-value} \\[1ex]
const &
  129.062 &
    88.3033 &
      1.4616 &
        0.1746 \\
sqft &
  0.154800 &
    0.0319404 &
      4.8465 &
        0.0007 \\
bedrms &
  -21.587 &
    27.0293 &
      -0.7987 &
        0.4430 \\
baths &
  -12.192 &
    43.2500 &
      -0.2819 &
        0.7838 \\
\end{tabular*}
\end{center}
\end{table}

We will now replicate the above results using \app{R}. Select 
the menu item ``Tools, Start GNU R''. A window similar to the one
shown in figure \ref{fig:Rwind1} should appear.

\begin{figure}[htbp]
  \centering
  \includegraphics[scale=0.7]{figures/Rwindow-1}
  \caption{\app{R} window}
  \label{fig:Rwind1}
\end{figure}

The actual look of the \app{R} window may be somewhat different from
what you see in Figure~\ref{fig:Rwind1} (especially for Windows
users), but this is immaterial. The important point is that you have a
window where you can type commands to \app{R}. If the above procedure
doesn't work and no \app{R} window opens, it means that \app{gretl}
was unable to launch \app{R}.  You should ensure that \app{R} is
installed and working on your system and that \app{gretl} knows where
it is. The relevant settings can be found by selecting the ``Tools,
Preferences, General'' menu entry, under the ``Programs'' tab.

Assuming \app{R} was launched successfully, you will see notification
that the data from gretl are available.  In the background,
\app{gretl} has arranged for two \app{R} commands to be executed, one
to load the \app{gretl} dataset in the form of a \emph{data frame}
(one of several forms in which \app{R} can store data) and one to
\emph{attach} the data so that the variable names defined in the
\app{gretl} workspace are available as valid identifiers within
\app{R}.

In order to replicate \app{gretl}'s OLS estimation, go into the \app{R}
window and type at the prompt
\begin{code}
  model <- lm(price ~ sqft + bedrms + baths)
  summary(model)
\end{code}

\begin{figure}[htbp]
  \centering
  \includegraphics[scale=0.7]{figures/Rwindow-2}
  \caption{OLS regression on house prices via \app{R}}
  \label{fig:Rwind2}
\end{figure}

You should see something similar to Figure~\ref{fig:Rwind2}. Surprise
--- the estimates coincide! To get out, just close the \app{R} window
or type \verb|q()| at the \app{R} prompt.

\subsection{Time series data}
\label{sec:R-ols-arma}

We now turn to an example which uses time series data: we
will compare \app{gretl}'s and \app{R}'s estimates of Box and Jenkins'
immortal ``airline'' model. The data are contained in the \texttt{bjg}
sample dataset. The following \app{gretl} code
\begin{code}
open bjg
arima 0 1 1 ; 0 1 1 ; lg --nc
\end{code}
produces the estimates shown in Table \ref{tab:airline-gretl}.

\begin{table}[htbp]
\caption{Airline model from Box and Jenkins (1976) --- selected
  portion of \app{gretl}'s estimates}
\label{tab:airline-gretl}
\begin{center}

\begin{tabular*}{0.75\textwidth}{@{\extracolsep{\fill}}
l% col 1: varname
  D{.}{.}{-1}% col 2: coeff
    D{.}{.}{-1}% col 3: sderr
      D{.}{.}{-1}% col 4: t-stat
        D{.}{.}{4}}% col 5: p-value (or slope)
Variable &
  \multicolumn{1}{c}{Coefficient} &
    \multicolumn{1}{c}{Std.\ Error} &
      \multicolumn{1}{c}{$t$-statistic} &
        \multicolumn{1}{c}{p-value} \\[1ex]
$\theta_{1}$ &
  -0.401824 &
    0.0896421 &
      -4.4825 &
        0.0000 \\
$\Theta_{1}$ &
  -0.556936 &
    0.0731044 &
      -7.6184 &
        0.0000 \\
\end{tabular*}

\begin{tabular}{lD{.}{.}{-1}}
Variance of innovations & 0.00134810 \\
Log-likelihood & 244.696 \\
Akaike information criterion & -483.39 
\end{tabular}
\end{center}
\end{table}

If we now open an \app{R} session as described in the previous
subsection, the data-passing mechanism is slightly different.  Since
our data were defined in \app{gretl} as time series, we use an \app{R}
\emph{time-series} object (\emph{ts} for short) for the transfer.  In
this way we can retain in \app{R} useful information such as the
periodicity of the data and the sample limits.  The downside is that
the names of individual series, as defined in \app{gretl}, are not
valid identifiers. In order to extract the variable \texttt{lg}, one
needs to use the syntax \verb|lg <- gretldata[, "lg"]|.

ARIMA estimation can be carried out by issuing the following two
\app{R} commands:
\begin{code}
lg <- gretldata[, "lg"]
arima(lg, c(0,1,1), seasonal=c(0,1,1))
\end{code}

which yield

\begin{code}
Coefficients:
          ma1     sma1
      -0.4018  -0.5569
s.e.   0.0896   0.0731

sigma^2 estimated as 0.001348:  log likelihood = 244.7,  aic = -483.4
\end{code}

Happily, the estimates again coincide.

\section{Running an \app{R} script}
\label{sec:R-scripts}

Opening an \app{R} window and keying in commands is a convenient
method when the job is small. In some cases, however, it would be
preferable to have \app{R} execute a script prepared in advance. 
One way to do this is via the \texttt{source()} command in
\app{R}.  Alternatively, \app{gretl} offers the facility to edit an
\app{R} script and run it, having the current dataset pre-loaded
automatically. This feature can be accessed via the ``File, Script
Files'' menu entry.  By selecting ``User file'', one can load a
pre-existing \app{R} script; if you want to create a new script
instead, select the ``New script, R script'' menu entry.

\begin{figure}[htbp]
  \centering
  \includegraphics[scale=0.7]{figures/R-edit1}
  \caption{Editing window for \app{R} scripts}
  \label{fig:R-edit1}
\end{figure}
In either case, you are presented with a window very similar to
the editor window used for ordinary \app{gretl} scripts, as in
Figure~\ref{fig:R-edit1}.

There are two main differences.  First, you get syntax highlighting for
\app{R}'s syntax instead of \app{gretl}'s. Second, clicking on the
Execute button (the gears icon), launches an instance of \app{R} in
which your commands are executed.  Before \app{R} is actually run, you
are asked if you want to run \app{R} interactively or not (see
Figure~\ref{fig:R-exec-mode}).

\begin{figure}[htbp]
  \centering
  \includegraphics[scale=0.7]{figures/R-exec-mode}
  \caption{Editing window for \app{R} scripts}
  \label{fig:R-exec-mode}
\end{figure}

An interactive run opens an \app{R} instance similar to the one seen
in the previous section: your data will be pre-loaded (if the
``pre-load data'' box is checked) and your commands will
be executed. Once this is done, you will find yourself at the \app{R}
prompt, where you can enter more commands.

A non-interactive run, on the other hand, will execute your script,
collect the output from \app{R} and present it to you in an output
window; \app{R} will be run in the background. If, for example, the
script in Figure~\ref{fig:R-edit1} is run non-interactively, a window
similar to Figure~\ref{fig:R-output1} will appear.

\begin{figure}[htbp]
  \centering
  \includegraphics[scale=0.7]{figures/R-output1}
  \caption{Output from a non-interactive \app{R} run}
  \label{fig:R-output1}
\end{figure}

\section{Taking stuff back and forth}
\label{sec:R-passing-data}

As regards the passing of data between the two programs, so far we
have only considered passing series from \app{gretl} to \app{R}. In
order to achieve a satisfactory degree of interoperability, more is
needed.  In the following sub-sections we see how matrices can be
exchanged, and how data can be passed from \app{R} back to
\app{gretl}.

\subsection{Passing matrices from \app{gretl} to \app{R}}

For passing matrices from \app{gretl} to \app{R}, you can use the
\texttt{mwrite} matrix function described in section
\ref{sec:matrix-csv}. For example, the following \app{gretl} code
fragment generates the matrix 
\[ 
A = \left[
  \begin{array}{ccc}
    3 &  7 &  11 \\ 
    4 &  8 &  12 \\ 
    5 &  9 &  13 \\ 
    6 & 10 &  14 
  \end{array}
\right]
\] 
and stores it into the file \texttt{mymatfile.mat}.
\begin{code}
  matrix A = mshape(seq(3,14),4,3)
  err = mwrite(A, "mymatfile.mat")
\end{code}
In order to retrieve this matrix from \app{R}, all you have to do is
\begin{code}
  A <- as.matrix(read.table("mymatfile.mat", skip=1))
\end{code}

Although in principle you can give your matrix file any valid
filename, a couple of conventions may prove useful. First, you may
want to use an informative file suffix such as ``\texttt{.mat}'', but
this is a matter of taste. More importantly, the exact location of the
file created by \texttt{mwrite} could be an issue. By default, if no
path is specified in the file name, \app{gretl} stores matrix files in
the current work directory. However, it may be wise for the purpose at
hand to use the directory in which \app{gretl} stores all its
temporary files, whose name is stored in the built-in string
\texttt{dotdir} (see section~\ref{sec:named-strings}). The value of
this string is automatically passed to \app{R} as the string variable
\texttt{gretl.dotdir}, so the above example may be rewritten more
cleanly as

\app{Gretl} side:
\begin{code}
  matrix A = mshape(seq(3,14),4,3)
  err = mwrite(A, "@dotdir/mymatfile.mat")
\end{code}
\app{R} side:
\begin{code}
  fname <- paste(gretl.dotdir, "mymatfile.mat", sep="")
  A <- as.matrix(read.table(fname, skip=1))
\end{code}

\subsection{Passing data from \app{R} to \app{gretl}}
\label{sec:Rpassing-data}

For passing data in the opposite direction, \app{gretl} defines a
special function that can be used in the \app{R} environment. An
\app{R} object will be written as a temporary file in \app{gretl}'s
\texttt{dotdir} directory, from where it can be easily retrieved from
within \app{gretl}.

The name of this function is \texttt{gretl.export()}, and it accepts
one argument, the object to be exported. At present, the objects that
can be exported with this method are matrices, data frames and
time-series objects. The function creates a text file, with the same
name as the exported object, in \app{gretl}'s temporary
directory. Data frames and time-series objects are stored as CSV
files, and can be retrieved by using \app{gretl}'s \texttt{append}
command.  Matrices are stored in a special text format that is
understood by \app{gretl} (see section~\ref{sec:matrix-csv}); the file
suffix is in this case \texttt{.mat}, and to read the matrix in
\app{gretl} you must use the \texttt{mread()} function.

As an example, we take the airline data and use them to estimate a
structural time series model \`a la \cite{harvey89}. The model we will 
use is the \emph{Basic Structural Model} (BSM), in which a time series
is decomposed into three terms:
\[
  y_t = \mu_t + \gamma_t + \varepsilon_t
\]
where $\mu_t$ is a trend component, $\gamma_t$ is a seasonal component
and $\varepsilon_t$ is a noise term. In turn, the following is assumed
to hold:
\begin{eqnarray*}
  \Delta \mu_t & = & \beta_{t-1} + \eta_t \\
  \Delta \beta_t & = & \zeta_t \\
  \Delta_s \gamma_t & = & \Delta \omega_t
\end{eqnarray*}
where $\Delta_s$ is the seasonal differencing operator, $(1-L^s)$, and
$\eta_t$, $\zeta_t$ and $\omega_t$ are mutually uncorrelated white
noise processes. The object of the analysis is to estimate the
variances of the noise components (which may be zero) and to recover
estimates of the latent processes $\mu_t$ (the ``level''), $\beta_t$
(the ``slope'') and $\gamma_t$.

\app{Gretl} does not provide (yet) a command for estimating this class
of models, so we will use \app{R}'s \texttt{StructTS} command and
import the results back into \app{gretl}. Once the \texttt{bjg}
dataset is loaded in \app{gretl}, we pass the data to \app{R} and execute
the following script:
\begin{code}
# extract the log series 
y <- gretldata[, "lg"]
# estimate the model
strmod <- StructTS(y)
# save the fitted components (smoothed)
compon <- as.ts(tsSmooth(strmod))
# save the estimated variances
vars <- as.matrix(strmod$coef)

# export into gretl's temp dir
gretl.export(compon)
gretl.export(vars)
\end{code}
%$

Running this script via \app{gretl} produces minimal output:
\begin{code}
current data loaded as ts object "gretldata"
wrote /home/cottrell/.gretl/compon.csv 
wrote /home/cottrell/.gretl/vars.mat 
\end{code}
However, we are now able to pull the results back into \app{gretl} by
executing the following commands, either from the console or by
creating a small script:\footnote{This example will work on Linux and
  presumably on OSX without modifications. On the Windows platform,
  you may have to substitute the ``\texttt{/}'' character with
  ``$\backslash$''.}
\begin{code}
append @dotdir/compon.csv
vars = mread("@dotdir/vars.mat")
\end{code}
The first command reads the estimated time-series components from a
CSV file, which is the format that the passing mechanism employs for
series. The matrix \texttt{vars} is read from the file
\texttt{vars.mat}.

\begin{figure}[htbp]
  \centering
  \includegraphics{figures/BSM-output}
  \caption{Estimated components from BSM}
  \label{fig:BSM-output}
\end{figure}

After the above commands have been executed, three new series will
have appeared in the \app{gretl} workspace, namely the estimates of
the three components; by plotting them together with the original
data, you should get a graph similar to
Figure~\ref{fig:BSM-output}. The estimates of the variances can be
seen by printing the \texttt{vars} matrix, as in

\begin{code}
? print vars
vars (4 x 1)

  0.00077185 
      0.0000 
   0.0013969 
      0.0000 
\end{code}

That is,
\begin{equation*}
  \hat{\sigma}^2_{\eta} = 0.00077185, \quad
  \hat{\sigma}^2_{\zeta} = 0, \quad
  \hat{\sigma}^2_{\omega} = 0.0013969, \quad
  \hat{\sigma}^2_{\varepsilon} = 0
\end{equation*}
Notice that, since $\hat{\sigma}^2_{\zeta} = 0$, the estimate for
$\beta_t$ is constant and the level component is simply a random walk
with a drift.

\section{Interacting with \app{R} from the command line}
\label{sec:foreign-command}

Up to this point we have spoken only of interaction with \app{R} via
the GUI program. In order to do the same from the command line
interface, \app{gretl} provides the \texttt{foreign} command. This
enables you to embed non-native commands within a \app{gretl}
script.

A ``foreign'' block takes the form
\begin{code}
foreign language=R [--send-data] [--quiet]
    ... R commands ...
end foreign
\end{code}
and achieves the same effect as submitting the enclosed \app{R}
commands via the GUI in the non-interactive mode (see
section~\ref{sec:R-scripts} above). The \option{send-data} option
arranges for auto-loading of the data present in the \app{gretl}
session.  The \option{quiet} option prevents the output from \app{R}
from being echoed in the \app{gretl} output.

\begin{script}[htbp]
  \caption{Estimation of the Basic Structural Model --- simple}
\begin{scode}
open bjg.gdt

foreign language=R --send-data
    y <- gretldata[, "lg"]
    strmod <- StructTS(y)
    compon <- as.ts(tsSmooth(strmod))
    vars <- as.matrix(strmod$coef)
            
    gretl.export(compon)
    gretl.export(vars)
end foreign

append @dotdir/compon.csv
rename level lg_level
rename slope lg_slope
rename sea lg_seas

vars = mread("@dotdir/vars.mat")
\end{scode}
\label{RStructTS-simple}
\end{script}
%$

Using this method, replicating the example in the previous subsection
is rather easy: basically, all it takes is encapsulating the content
of the \app{R} script in a \texttt{foreign}\ldots\texttt{end foreign}
block; see example \ref{RStructTS-simple}.

\begin{script}[htbp]
  \caption{Estimation of the Basic Structural Model --- via a function}
\begin{scode}
function list RStructTS(series myseries)

    smpl ok(myseries) --restrict
    sx = argname(myseries)

    foreign language=R --send-data --quiet
        @sx <- gretldata[, "myseries"]
        strmod <- StructTS(@sx)
        compon <- as.ts(tsSmooth(strmod))
        gretl.export(compon)
    end foreign

    append @dotdir/compon.csv
    rename level @sx_level
    rename slope @sx_slope
    rename sea @sx_seas

    list ret = @sx_level @sx_slope @sx_seas
    return ret
end function

# ------------ main -------------------------

open bjg.gdt
list X = RStructTS(lg)
\end{scode}

\label{RStructTS-func}
\end{script}

The above syntax, despite being already quite useful by itself, shows
its full power when it is used in conjunction with user-written
functions.  Example~\ref{RStructTS-func} shows how to define a
\app{gretl} function that calls \app{R} internally.

\section{Performance issues with \app{R}}
\label{sec:R-performance}

\app{R} is a large and complex program, which takes an appreciable
time to initialize itself.\footnote{About one third of a second on an
  Intel Core Duo machine of 2009 vintage.}  In interactive use this
not a significant problem, but if you have a \app{gretl} script that
calls \app{R} repeatedly the cumulated start-up costs can become
bothersome.  To get around this, \app{gretl} calls the \app{R} shared
library by preference; in this case the start-up cost is borne only
once, on the first invocation of \app{R} code from within \app{gretl}.

Support for the \app{R} shared library is built into the \app{gretl}
packages for MS Windows and OS X --- but the advantage is realized
only if the library is in fact available at run time.  If you are
building \app{gretl} yourself on Linux and wish to make use of the
\app{R} library, you should ensure (a) that \app{R} has been built
with the shared library enabled (specify \verb|--enable-R-shlib| when
configuring your build of \app{R}), and (b) that the \verb|pkg-config|
program is able to detect your \app{R} installation.  We do not link
to the \app{R} library at build time, rather we open it dynamically on
demand. The \app{gretl} GUI has an item under the
\textsf{Tools/Preferences} menu which enables you to select the
path to the library, if it is not detected automatically.  

If you have the \app{R} shared library installed but want to force
\app{gretl} to call the \app{R} executable instead, you can do
\begin{code}
set R_lib off
\end{code}

\section{Further use of the \app{R} library}
\label{sec:R-functions}

Besides improving performance, as noted above, use of the \app{R}
shared library makes possible a further refinement.  That is, you can
define functions in \app{R}, within a \texttt{foreign} block, then
call those functions later in your script much as if they were
\app{gretl} functions.  This is illustrated below.  
%
\begin{code}
set R_functions on
foreign language=R
  plus_one <- function(q) {
     z = q+1
     invisible(z)
  }
end foreign

scalar b=R.plus_one(2)
\end{code}
%
The \app{R} function \verb|plus_one| is obviously trivial in itself,
but the example shows a couple of points.  First, for this mechanism
to work you need to enable \verb|R_functions| via the \texttt{set}
command.  Second, to avoid collision with the \app{gretl} function
namespace, calls to functions defined in this way must be prefixed
with ``\texttt{R.}'', as in \verb|R.plus_one|.

Built-in \app{R} functions may also be called in this way, once
\verb|R_functions| is set \texttt{on}.  For example one can invoke
\app{R}'s \texttt{choose} function, which computes binomial
coefficients:
%
\begin{code}
set R_functions on
scalar b=R.choose(10,4)
\end{code}
%
Note, however, that the possibilities for use of built-in \app{R}
functions are limited; only functions whose arguments and return
values are sufficiently generic (basically scalars or matrices) will
work.

%%% Local Variables: 
%%% mode: latex
%%% TeX-master: "gretl-guide"
%%% End: 

