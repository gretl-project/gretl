\chapter{Gretl and Ox}
\label{chap:gretlOx}

\section{Introduction}
\label{Ox-intro}

\app{Ox}, written by Jurgen A. Doornik \citep[see][]{doornik07}, is
described by its author as ``an object-oriented statistical system. At
its core is a powerful matrix language, which is complemented by a
comprehensive statistical library. Among the special features of Ox
are its speed [and] well-designed syntax\dots{}.  Ox comes in two
versions: Ox Professional and Ox Console. Ox is available for Windows,
Linux, Mac (OS X), and several Unix platforms.''
(\url{www.doornik.com})

\app{Ox} is proprietary, closed-source software.  The command-line
version of the program is, however, available free of change for
academic users.  Quoting again from Doornik's website: ``The
Console (command line) versions may be used freely for academic
research and teaching purposes only\dots{}. The Ox syntax is public,
and, of course, you may do with your own Ox code whatever you wish.''
If you wish to use \app{Ox} in conjunction with gretl please
refer to \url{doornik.com} for further details on licensing.

As the reader will no doubt have noticed, most other software that
we discuss in this Guide is open-source and freely available for all
users.  We make an exception for \app{Ox} on the grounds that it is
indeed fast and well designed, and that its statistical
library---along with various add-on packages that are also
available---has exceptional coverage of cutting-edge techniques in
econometrics.  The gretl authors have used \app{Ox} for benchmarking
some of gretl's more advanced features such as dynamic panel models
and state space models.\footnote{For a review of \app{Ox}, see
  \cite{cribari-neto03} and for a (somewhat dated) comparison of
  \app{Ox} with other matrix-oriented packages such as \app{GAUSS},
  see \cite{steinhaus99}.}

\section{\app{Ox} support in gretl}
\label{sec:Ox-support}

The support offered for \app{Ox} in gretl is similar to that
offered for \app{R}, as discussed in chapter~\ref{chap:gretlR}.

\tip{To enable support for \app{Ox}, go to the
  Tools/Preferences/General menu item and look under the Programs
  tab. Find the entry for the path to the \texttt{oxl} executable,
  that is, the program that runs \app{Ox} files (on MS Windows it is
  called \texttt{oxl.exe}). Adjust the path if it's not already right
  for your system and you should be ready to go.}
  
With support enabled, you can open and edit \app{Ox} programs in the
gretl GUI.  Clicking the ``execute'' icon in the editor window
will send your code to \app{Ox} for execution.
Figures~\ref{fig:Oxedit} and Figure~\ref{fig:Oxout} show an \app{Ox}
program and part of its output.

\begin{figure}[htbp]
  \centering
  \includegraphics[scale=0.7]{figures/Oxedit}
  \caption{\app{Ox} editing window}
  \label{fig:Oxedit}
\end{figure}

\begin{figure}[htbp]
  \centering
  \includegraphics[scale=0.7]{figures/Oxout}
  \caption{Output from \app{Ox}}
  \label{fig:Oxout}
\end{figure}

In addition you can embed \app{Ox} code within a gretl script
using a \texttt{foreign} block, as described in connection with
\app{R}.  A trivial example, which simply prints the gretl data
matrix within \app{Ox}, is shown in Listing~\ref{ex:ox-simple}.

\begin{script}
  \caption{Simple example of \app{Ox} usage}
  \label{ex:ox-simple}
\begin{scode}
open data4-1
matrix m = {dataset}
mwrite(m, "gretl.mat", 1)

foreign language=Ox 
#include <oxstd.h>
main()
{
   decl gmat = gretl_loadmat("gretl.mat");
   print(gmat);
}
end foreign
\end{scode}
\end{script}

The listing illustrates how a matrix can be passed from gretl to
\app{Ox}.  We use the \texttt{mwrite} function to write a matrix into
the user's ``dotdir'' (see section~\ref{sec:named-strings}), then in
\app{Ox} we use the function \verb|gretl_loadmat| to retrieve the
matrix.

How does \verb|gretl_loadmat| come to be defined?  When gretl
writes out the \app{Ox} program corresponding to your \texttt{foreign}
block it does two things in addition.  First, it writes a small
utility file named \verb|gretl_io.ox| into your dotdir.  This contains
a definition for \verb|gretl_loadmat| and also for the function
\verb|gretl_export| (see below).  Second, gretl interpolates
into your \app{Ox} code a line which includes this utility file (it is
inserted right after the inclusion of \texttt{oxstd.h}, which is
needed in all \app{Ox} programs).  Note that \verb|gretl_loadmat|
expects to find the named file in the user's dotdir.

\section{Illustration: replication of DPD model}
\label{sec:dpd-replication}

Listing~\ref{script:Ox-DPD} shows a more ambitious case.  This script
replicates one of the dynamic panel data models in
\cite{arellano-bond91}, first using gretl and then using \app{Ox}; we
then check the relative differences between the parameter estimates
produced by the two programs (which turn out to be reassuringly
small).

Unlike the previous example, in this case we pass the dataset from
gretl to \app{Ox} as a CSV file in order to preserve the
variable names.  Note the use of the internal variable \verb|csv_na|
to get the right representation of missing values for use with
\app{Ox}---and also note that the \verb|--send-data| option for the
\texttt{foreign} command is not available in connection with Ox.

We get the parameter estimates back from \app{Ox} using
\verb|gretl_export| on the \app{Ox} side and \texttt{mread} on the
gretl side.  The \verb|gretl_export| function takes two arguments, a
matrix and a file name.  The file is written into the user's dotdir,
from where it can be picked up using \texttt{mread}.  The final
portion of the output from Listing~\ref{script:Ox-DPD} is shown below:
%
\begin{code}
? matrix oxparm = mread("oxparm.mat", 1)
Generated matrix oxparm
? eval abs((parm - oxparm) ./ oxparm)
  1.4578e-13 
  3.5642e-13 
  5.0672e-15 
  1.6091e-13 
  8.9808e-15 
  2.0450e-14 
  1.0218e-13 
  2.1048e-13 
  9.5898e-15 
  1.8658e-14 
  2.1852e-14 
  2.9451e-13 
  1.9398e-13 
\end{code}

\begin{script}[htbp]
  \caption{Estimation of dynamic panel data model via gretl and \app{Ox}}
  \label{script:Ox-DPD}
\begin{scode}
open abdata.gdt

# 1-step GMM estimation
dpanel 2 ; n w w(-1) k ys ys(-1) 0 --time-dummies --dpdstyle
matrix parm = $coeff

# Write CSV file for Ox
set csv_na .NaN
store @dotdir/abdata.csv

# Replicate using the Ox DPD package
foreign language=Ox
#include <oxstd.h>
#import <packages/dpd/dpd>

main ()
{
    decl dpd = new DPD();
    dpd.Load("@dotdir/abdata.csv"); 
    dpd.SetYear("YEAR");

    dpd.Select(Y_VAR, {"n", 0, 2});
    dpd.Select(X_VAR, {"w", 0, 1, "k", 0, 0, "ys", 0, 1});
    dpd.Select(I_VAR, {"w", 0, 1, "k", 0, 0, "ys", 0, 1});

    dpd.Gmm("n", 2, 99);  // GMM-type instrument
    dpd.SetDummies(D_CONSTANT + D_TIME);
    dpd.SetTest(2, 2); // Sargan, AR 1-2 tests
    dpd.Estimate();    // 1-step estimation
    decl parm = dpd.GetPar();
    gretl_export(parm, "oxparm.mat");
   
    delete dpd;
}
end foreign

# Compare the results
matrix oxparm = mread("oxparm.mat", 1)
eval abs((parm - oxparm) ./ oxparm)
\end{scode}
\end{script}

%%% Local Variables: 
%%% mode: latex
%%% TeX-master: "gretl-guide"
%%% End: 

