\chapter{Matrix manipulation}
\label{chap:matrices}

\section{Introduction}
\label{matrix-intro}

Since version 1.5.1, gretl has offered the facility of creating and
manipulating user-defined matrices.  There are a few changes in this
respect in version 1.6.1.

\section{Creating matrices}
\label{matrix-create}

Matrices can be created using any of these methods:

\begin{enumerate}
\item By direct specification of the scalar values that compose the
  matrix, in numerical form, by reference to pre-existing
  scalar variables, or using computed values.
\item By providing a list of data series.
\item By providing a \textit{named list} of series.
\item Using a formula of the same general type that is used
  with the \texttt{genr} command, whereby a new matrix is defined
  in terms of existing matrices and/or scalars, or via some
  special functions.
\end{enumerate}

To specify a matrix \textit{directly in terms of scalars}, the syntax
is, for example:

\begin{code}
matrix A = { 1, 2, 3 ; 4, 5, 6 }
\end{code}

The matrix is defined by rows; the elements on each row are separated
by commas and the rows are separated by semi-colons.  The whole
expression must be wrapped in braces.  Spaces within the braces are
not significant.  The above expression defines a $2\times3$ matrix.
Each element should be a numerical value, the name of a scalar
variable, or an expression that evaluates to a scalar.  Directly after
the closing brace you can append a single quote (\texttt{'}) to obtain
the transpose.

To specify a matrix \textit{in terms of data series} the syntax is,
for example,
%
\begin{code}
matrix A = { x1, x2, x3 }
\end{code}
%
where the names of the variables are separated by commas.  Besides
names of existing variables, you can use expressions that evaluate to
a series.  For example, given a series \texttt{x} you could do
%
\begin{code}
matrix A = { x, x^2 }
\end{code}
%
Each variable occupies a column (and there can only be one variable
per column).  You cannot use the semi-colon as a row separator in this
case: if you want the series arranged in rows, append the transpose
symbol.  The range of data values included in the matrix depends on
the current setting of the sample range.

Please note: while gretl's built-in statistical functions are
capable of handling missing values, the matrix arithmetic functions
are not.  \emph{When you build a matrix from series that include missing
values, observations for which at least one series has a missing value
are skipped}.  

Instead of giving an explicit list of variables, you may instead
provide the \textit{name of a saved list} (see
Chapter~\ref{chap-persist}), as in
%
\begin{code}
list xlist = x1 x2 x3
matrix A = { xlist }
\end{code}
%
When you provide a named list, the data series are by default placed
in columns, as is natural in an econometric context: if you want them
in rows, append the transpose symbol.

As a special case of constructing a matrix from a list of variables,
you can say
%
\begin{code}
matrix A = { dataset }
\end{code}
%
This builds a matrix using all the series in the current dataset,
apart from the constant (variable 0).  When this dummy list is used, it
must be the sole element in the matrix definition \texttt{\{...\}}.  You
can, however, create a matrix that includes the constant along with
all other variables using column-wise concatenation (see below), as in
%
\begin{code}
matrix A = {const}~{dataset}
\end{code}
%

You can create new matrices, or replace existing matrices, by means of
various transformations just as with scalars and data series.  The
relevant mechanisms are discussed in the next several sections.

\tip{Names of matrices must satisfy the same requirements as names of
  gretl variables in general: the name can be no longer than 15
  characters, must start with a letter, and must be composed of
  nothing but letters, numbers and the underscore character.}

\section{Matrix operators}
\label{matrix-op}

The following binary operators are available for matrices:

\begin{center}
\begin{tabular}{ll}
\texttt{+} & addition \\
\texttt{-} & subtraction \\
\texttt{*} & ordinary matrix multiplication \\
\texttt{'} & pre-multiplication by transpose \\
\texttt{/} & matrix ``division'' (see below) \\
\texttt{.*} & element-wise multiplication \\
\texttt{./} & element-wise division \\
\verb+.^+ & element-wise exponentiation \\
\verb+~+ & column-wise concatenation \\
\verb+|+ & row-wise concatenation \\
\texttt{**} & Kronecker product \\
\texttt{=} & test for equality 
\end{tabular}
\end{center}

Here are explanations of the less obvious cases. 

For matrix addition and subtraction, in general the two matrices have
to be of the same dimensions but an exception to this rule is granted
if one of the operands is a $1\times 1$ matrix or scalar.  The scalar
is implicitly promoted to the status of a matrix of the correct
dimensions, all of whose elements are equal to the given scalar value.
For example, if $A$ is an $m \times n$ matrix and $k$ a scalar, then
the commands
%
\begin{code}
matrix C = A + k
matrix D = A - k
\end{code}
%
both produce $m \times n$ matrices, with elements $c_{ij} = 
a_{ij} + k$ and $d_{ij} = a_{ij} - k$ respectively.

By ``pre-multiplication by transpose'' we mean, for example, that 
%
\begin{code}
matrix C = X'Y
\end{code}
%
produces the product of $X$-transpose and $Y$.  In effect, 
the expression \texttt{X'Y} is shorthand for \texttt{X'*Y}
(which is also valid).

In matrix ``division'', $A/B$ is algebraically equivalent to
$B^{-1}A$ (pre-multiplication by the inverse of the ``divisor'').
Therefore the following two expressions are equivalent in principle:
%
\begin{code}
matrix C = A / B
matrix C = inv(B) * A
\end{code}
%
where \texttt{inv()} is the matrix inversion function (see below for
more on matrix functions).  The first form, however, may be more
accurate than the second; the solution is obtained via LU
decomposition, without the explicit calculation of the inverse.

In \textit{element-wise multiplication} if we write
%
\begin{code}
matrix C = A .* B
\end{code}
% 
then the result depends on the dimensions of $A$ and $B$.  Let $A$ be
an $m \times n$ matrix and let $B$ be $p \times q$.  
%
\begin{itemize}
\item If $m=p$ and $n=q$ then $C$ is $m\times n$ with $c_{ij} = a_{ij}
  \times b_{ij}$.  This is known as the \emph{Hadamard product}.
\item Otherwise, if $m=1$ and $n=q$, or $n=1$ and $m=p$, then $C$ is
  $p\times q$ with $c_{ij} = a_k \times b_{ij}$, where $k=j$ if $m=1$
  else $k=i$.
\item Otherwise, if $p=1$ and $n=q$, or $q=1$ and $m=p$, then $C$ is
  $m\times n$ with $c_{ij} = a_{ij} \times b_k$, where $k=j$ if $p=1$
  else $k=i$.
\item If none of the above conditions are satisfied the product is
  undefined and an error is flagged.
\end{itemize}

Element-wise division and element-wise exponentiation work in a manner
exactly analogous to element-wise multiplication: simply replace
$\times$ by $\div$, or the exponentation operation, in the account
given for multiplication.

In \textit{column-wise concatenation} of an $m\times n$ matrix $A$ and
an $m\times p$ matrix $B$, the result is an $m\times (n+p)$ matrix.
That is,
%
\begin{code}
matrix C = A ~ B
\end{code}
% 
produces $C = \left[ \begin{array}{cc} A & B \end{array} \right]$.

\textit{Row-wise concatenation} of an $m\times n$ matrix $A$ and
an $p\times n$ matrix $B$ produces an $(m+p) \times n$ matrix.
That is,
%
\begin{code}
matrix C = A | B
\end{code}
% 
produces $C = \left[ \begin{array}{cc} A \\ B \end{array} \right]$.

\section{Matrix--scalar operators}
\label{matrix-scalar-op}

For $A$ a matrix and $k$ a scalar, the operators shown in
Table~\ref{tab:matrix-scalar-ops} are available.  (The cases of
addition and subtraction were discussed in section~\ref{matrix-op} but
we include them in the table for completeness.)  In addition, for
square $A$ and integer $k \geq 1$, \verb|B = A^k| produces $B = A
\times A \times A\dots$, that is, the matrix $A$ multiplied by itself
$k-1$ times.

\begin{table}[htbp]
\centering
\begin{tabular}{ll}
\textit{Expression} & \textit{Effect} \\[4pt]
\texttt{matrix B = A * k} & $b_{ij} = k a_{ij}$ \\
\texttt{matrix B = A / k} & $b_{ij} = a_{ij} / k$ \\
\texttt{matrix B = k / A} & $b_{ij} = k / a_{ij}$ \\
\texttt{matrix B = A + k} & $b_{ij} = a_{ij} + k$ \\
\texttt{matrix B = A - k} & $b_{ij} = a_{ij} - k$ \\
\texttt{matrix B = k - A} & $b_{ij} = k - a_{ij}$ \\
\texttt{matrix B = A \% k} & $b_{ij} = a_{ij} \mbox{ modulo } k$ \\
\end{tabular}
\caption{Matrix--scalar operators}
\label{tab:matrix-scalar-ops}
\end{table}


\section{Matrix functions}
\label{matrix-func}

\begin{table}[htbp]
\centering
\begin{tabular}{lllll}
\texttt{abs}       &
\texttt{atan}      &
\texttt{cdemean}   &
\texttt{cholesky}  &
\texttt{cnorm}     \\
\texttt{cols}      &
\texttt{cos}       &
\texttt{det}       &
\texttt{diag}      &
\texttt{dnorm}     \\
\texttt{dsort}     &
\texttt{eigengen}  &
\texttt{eigensym}  &
\texttt{exp}       &
\texttt{gamma}     \\
\texttt{I}         &
\texttt{int}       &
\texttt{inv}       &
\texttt{ldet}      &
\texttt{lngamma}   \\
\texttt{log}       &
\texttt{mcorr}     &
\texttt{mcov}      &
\texttt{meanc}     &
\texttt{meanr}     \\
\texttt{mexp}      &
\texttt{mlag}      &
\texttt{mnormal}   &
\texttt{mshape}    &
\texttt{muniform}  \\
\texttt{nullspace} &
\texttt{onenorm}   &
\texttt{ones}      &
\texttt{qform}     &
\texttt{qnorm}     \\
\texttt{qrdecomp}  &
\texttt{rcond}     &
\texttt{rows}      &
\texttt{sin}       &
\texttt{sort}      \\
\texttt{sqrt}      &
\texttt{sumc}      &
\texttt{sumr}      &
\texttt{svd}       &
\texttt{tan}       \\
\texttt{tr}        &
\texttt{transp}    &
\texttt{unvech}    &
\texttt{values}    &
\texttt{vec}       \\
\texttt{vech}      &
\texttt{zeros}     
\end{tabular}      
\caption{Table of matrix functions}
\label{tab:matrix_funcs}
\end{table}

Table \ref{tab:matrix_funcs} lists the matrix function that
\app{gretl} provides. The following functions are available for
\textit{element-by-element transformations} of matrices: \texttt{log},
\texttt{exp}, \texttt{sin}, \texttt{cos}, \texttt{tan}, \texttt{atan},
\texttt{int}, \texttt{abs}, \texttt{sqrt}, \texttt{dnorm},
\texttt{cnorm}, \texttt{qnorm}, \texttt{gamma} and \texttt{lngamma}.
These functions have the effects documented in relation to the
\texttt{genr} command.  For example, if a matrix \texttt{A} is already
defined, then
%
\begin{code}
matrix B = sqrt(A)
\end{code}
%
generates a matrix such that $b_{ij} = \sqrt{a_{ij}}$.  All of these
functions require a single matrix as argument, or an expression which
evaluates to a single matrix.

Note that to find the ``matrix square root'' you need the
\texttt{cholesky()} function (see below); moreover, the \texttt{exp()}
function computes the exponential element by element, and therefore
does \emph{not} return the matrix exponential unless the matrix is
diagonal --- to get the matrix exponential, use \texttt{mexp()}.
  
The functions \texttt{sort()}, \texttt{dsort()} and \texttt{values()}
are available for matrices as well as data series.  In the matrix case
the argument to these functions must be a vector ($p \times 1$ or
$1\times p$).  For \texttt{sort} and \texttt{dsort} the return value
is a vector containing the elements of the input vector sorted in
ascending (\texttt{sort}) or descending (\texttt{dsort}) order of
magnitude.  For \texttt{values} the return is a vector containing the
distinct values in the input vector, sorted in ascending order.

Several matrix-specific functions are available.  These functions fall
into four categories:
%
\begin{enumerate}
\item Those taking a single matrix as argument and returning a scalar.
\item Those taking a single matrix as argument (plus in some cases an
  additional parameter) and returning a matrix.
\item Those taking one or two dimensions as arguments and
  returning a matrix.
\item Those taking two matrices as arguments and returning a matrix.
\item Those taking one or two matrices as arguments and returning one
  or two matrices.
\end{enumerate}
%
These sets of functions are discussed in turn below.

\subsection{Matrix to scalar functions}
\label{matrix-to-scalar}

The functions which take a single matrix as argument and return a
scalar are:

\begin{center}
\begin{tabular}{ll}
\texttt{det()} & determinant \\
\texttt{ldet()} & log-determinant \\
\texttt{tr()} & trace \\
\texttt{onenorm()} & 1-norm \\
\texttt{rcond()} & reciprocal condition number \\
\texttt{rows()} & number of rows \\
\texttt{cols()} & number of columns 
\end{tabular}
\end{center}

The single matrix argument to these functions may be given as the name
of an existing matrix or as an expression that evaluates to a single
matrix.  Note that the functions \texttt{det}, \texttt{ldet} and
\texttt{tr} require a square matrix as input.  

The \texttt{onenorm} function returns the 1-norm of a matrix --- that
is, the maximum across the columns of the matrix of the sums of the
absolute values of the column elements.  The function \texttt{rcond}
returns the reciprocal condition number for a symmetric, positive
definite matrix.

\subsection{Matrix to matrix functions}
\label{matrix-to-matrix}

The functions which take a single matrix as argument and return a
matrix are:

\begin{center}
\begin{tabular}{ll}
\texttt{sumc()} & sum by column \\
\texttt{sumr()} & sum by row \\
\texttt{meanc()} & mean by column \\
\texttt{meanr()} & mean by row \\
\texttt{mcov()} & covariance matrix by columns \\
\texttt{mcorr()} & correlation matrix by columns \\
\texttt{mexp()} & matrix exponential \\
\texttt{inv()} & inverse \\
\texttt{cholesky()} & Cholesky decomposition \\
\texttt{diag()} & extract principal diagonal \\
\texttt{transp()} & transpose \\
\texttt{cdemean()} & subtract column means \\ 
\texttt{vec()} & organize elements as column vector \\
\texttt{vech()} & vectorize lower triangle \\
\texttt{unvech()} & undo \texttt{vech} \\
\texttt{mlag()} & matrix lag or lead \\
\texttt{nullspace()} & right nullspace
\end{tabular}
\end{center}

As with the previous set of functions, the argument may be given as
the name of an existing matrix or as an expression that evaluates to a
single matrix.

For an $m \times n$ matrix $A$, \texttt{sumc(A)} returns a row vector
holding the $n$ column sums, and \texttt{sumr(A)} returns a column
vector with the $m$ row sums.  \texttt{meanc(A)} returns a row vector
with the $n$ column means, and \texttt{meanr(A)} a column vector with
the $m$ row means.

For a $T \times k$ matrix $A$, \texttt{mcov(A)} and \texttt{mcorr(A)}
both return $k \times k$ symmetric matrices, in the first case
containing the variances (on the diagonal) and covariances of the
variables in the columns of $A$, and in the second, containing the
correlations of the variables.

For an $n \times n$ matrix $A$, \texttt{mexp(A)} returns an $n \times
n$ matrix holding an approximation to the matrix exponential
\[
e^A = \sum_k^{\infty} \frac{A^k}{k!}
\]

The \texttt{cholesky} function computes the Cholesky decomposition $L$
of a symmetric positive definite matrix $A$: $A = LL'$; $L$ is lower
triangular (has zeros above the diagonal).  

The \texttt{diag} function returns the principal diagonal of an
$n\times n$ matrix $A$ as a column vector --- that is, an
$n$-vector $v$ such that $v_i = a_{ii}$.

The \texttt{cdemean} function applied to an $m \times n$ matrix $A$
returns an $m \times n$ matrix $B$ such that $b_{ij} = a_{ij} -
\bar{A}_j$, where $\bar{A}_j$ denotes the mean of column $j$ of $A$.  

The \texttt{vec} function applied to an $m \times n$ matrix $A$
returns a column vector of length $mn$ formed by stacking the columns
of $A$.  

The \texttt{vech} function applied to an $n \times n$ matrix $A$
returns a column vector of length $n(n+1)/2$ formed by stacking the
elements of the lower triangle of $A$, column by column.  Note that
$A$ must be square; for the operation to make sense $A$ should also
be symmetric.  The \texttt{unvech} function performs the inverse
operation, producing a symmetric matrix.

The \texttt{mlag} function requires two arguments, a matrix and a
scalar lag order, $m$.  Applied to an $T \times k$ matrix $A$, this
function returns a $T \times k$ matrix $B$ such that
%
\[
  b_{ij} = \left\{ 
    \begin{array}{ll} 
      a_{i-m,j} & 1 \leq i - m \leq T \\ 
      0 & \mbox{otherwise}
    \end{array}
    \right.
\]
%
That is, the columns of $B$ are lagged versions of the columns of $A$,
with missing values replaced by zeros.  The order $m$ may be negative
to generate leads instead of lags.

The \texttt{nullspace} function yields $X$, the right null space of a
matrix $A$ (it is assumed that A has full row rank): $X$ satisfies $A
\cdot X = 0$.


\subsection{Matrix filling functions}
\label{matrix-fill}

The functions taking one or two dimensions as arguments and returning
a matrix are:

\begin{center}
\begin{tabular}{ll}
\texttt{I(}\textsl{n}\texttt{)} & $n\times n$ identity matrix \\
\texttt{zeros(}\textsl{m}\texttt{,}\textsl{n}\texttt{)} & 
   $m\times n$ zero matrix \\
\texttt{ones(}\textsl{m}\texttt{,}\textsl{n}\texttt{)} &
   $m\times n$ matrix filled with 1s \\
\texttt{muniform(}\textsl{m}\texttt{,}\textsl{n}\texttt{)} &
   $m\times n$ matrix filled with uniform random values \\
\texttt{mnormal(}\textsl{m}\texttt{,}\textsl{n}\texttt{)} &
   $m\times n$ matrix filled with normal random values \\
\end{tabular}
\end{center}

The dimensions \textsl{m} and \textsl{n} may be given numerically, 
by reference to pre-existing scalar variables, or as expressions that
evaluate to scalars.  
%
\begin{code}
scalar m = 4
scalar n = 5
matrix A = mnormal(m,n)
\end{code}
%
The \texttt{muniform()} and \texttt{mnormal()} matrix functions fill the
matrix with drawings from the uniform (0--1) distribution and the
standard normal distribution respectively.

\subsection{Matrix reshaping}
\label{matrix-mshape}

A matrix can also be created by re-arranging the elements of a
pre-existing matrix. This is accomplished via the \texttt{mshape()}
function. It takes three arguments: the input matrix, $A$, and the
rows and columns of the target matrix, $r$ and $c$ respectively.
Elements are read from $A$ and written to the target in column-major
order.  If $A$ contains fewer elements than $n = r \times c$, they are
repeated cyclically; if $A$ has more elements, only the first $n$ are
used.

For example:
\begin{code}
matrix a = mnormal(2,3)
a
matrix b = mshape(a,3,1)
b
matrix b = mshape(a,5,2)
b
\end{code}
produces
\begin{code}
?   a
a

      1.2323      0.99714     -0.39078
     0.54363      0.43928     -0.48467

?   matrix b = mshape(a,3,1)
Generated matrix b
?   b
b

      1.2323
     0.54363
     0.99714

?   matrix b = mshape(a,5,2)
Replaced matrix b
?   b
b

      1.2323     -0.48467
     0.54363       1.2323
     0.99714      0.54363
     0.43928      0.99714
    -0.39078      0.43928
\end{code}

\subsection{Single-return two-matrix functions}
\label{matrix-two}

The function \texttt{qform()} constructs a quadratic form in a matrix
$A$ and a conformable symmetric matrix $X$.  The command
%
\begin{code}
B = qform(A, X)
\end{code}
%
calculates $B = A X A^{\prime}$.  This is computed more efficiently than
the alternative command \texttt{B = A*X*A'}.  In addition, the result
is symmetric by construction.


\subsection{Multiple-return matrix functions}
\label{matrix-multiples}

The functions that take one or more matrices as arguments and compute
one or more matrices are:

\begin{center}
\begin{tabular}{ll}
\texttt{qrdecomp()} & QR decomposition \\
\texttt{eigensym()} & Eigen-analysis of symmetric matrix \\
\texttt{eigengen()} & Eigen-analysis of general matrix \\
\texttt{svd()}      & Singular value decomposition (SVD) 
\end{tabular}
\end{center}

The syntax for all but the last of these functions is of the form
%
\begin{code}
matrix B = func(A, &C)
\end{code}
%
while for \texttt{svd()} it is
%
\begin{code}
matrix B = func(A, &C, &D)
\end{code}
%
The first argument, \texttt{A}, represents the input data, that is,
the matrix whose decomposition or analysis is required.

The second argument (and in the case of \texttt{svd}, the third) must
be either the name of an existing matrix preceded by \verb+&+ (to
indicate the ``address'' of the matrix in question), in which case an
auxiliary result is written to that matrix, or the keyword
\texttt{null}, in which case the auxiliary result is not produced, or
is discarded.

In case a non-null second argument is given, the specified matrix will
be over-written with the auxiliary result.  (It is not required that
the existing matrix be of the right dimensions to receive the result.)

The \texttt{qrdecomp} function computes the QR decomposition of an $m
\times n$ matrix $A$: $A = QR$, where $Q$ is an $m \times n$
orthogonal matrix and $R$ is an $n \times n$ upper triangular matrix.
The matrix $Q$ is returned directly, while $R$ can be retrieved via
the second argument.  Here are two examples:
%
\begin{code}
matrix R
matrix Q = qrdecomp(M, &R)
matrix Q = qrdecomp(M, null)
\end{code}
%
In the first example, the triangular $R$ is saved as \texttt{R}; in
the second, $R$ is discarded.  The first line above shows an example
of a ``simple declaration'' of a matrix: \texttt{R} is
declared to be a matrix variable but is not given any explicit value.
In this case the variable is initialized as a $1\times 1$ matrix whose
single element equals zero.

The function \texttt{eigensym} computes the eigenvalues, and
optionally the right eigenvectors, of a symmetric $n \times n$ matrix.
The eigenvalues are returned directly in a column vector of length
$n$; if the eigenvectors are required, they are returned in an $n
\times n$ matrix.  For example:
%
\begin{code}
matrix V
matrix E = eigensym(M, &V)
matrix E = eigensym(M, null)
\end{code}
%
In the first case \texttt{E} holds the eigenvalues of \texttt{M} and
\texttt{V} holds the eigenvectors.  In the second, \texttt{E} holds
the eigenvalues but the eigenvectors are not computed.

The function \texttt{eigengen} computes the eigenvalues, and
optionally the eigenvectors, of a general $n \times n$ matrix.  The
eigenvalues are returned directly in a column vector of length $2n$:
the first $n$ elements are the real components and the remaining $n$
are the imaginary components.  If the eigenvectors are required (that
is, if the second argument to \texttt{eigengen} is not \texttt{null}),
they are returned in an $n \times n$ matrix.

The function \texttt{svd} computes all or part of the simgular value
decomposition of the real $m \times n$ matrix $A$.  The decomposition
is
\[
A = U \Sigma V'
\]
where $\Sigma$ is an $m \times n$ matrix which is zero except for its
$k = \mbox{min}(m, n)$ diagonal elements, $U$ is an $m \times m$
orthogonal matrix, and $V$ is an $n \times n$ orthogonal matrix.  The
diagonal elements of $\Sigma$ are the singular values of $A$; they are
real and non-negative, and are returned in descending order.  The
first $\mbox{min}(m, n)$ columns of $U$ and $V$ are the left and right
singular vectors of $A$.

The \texttt{svd} function returns the singular values, in a vector of
length $k$.  The left and/or right singular vectors may be obtained by
supplying non-null values for the second and/or third arguments
respectively.  For example:
%
\begin{code}
matrix s = svd(A, &U, &V)
matrix s = svd(A, null, null)
matrix s = svd(A, null, &V)
\end{code}
%
In the first case both sets of singular vectors are obtained, in the
second case only the singular values are obtained; and in the third,
the right singular vectors are obtained but $U$ is not computed.
\emph{Please note}: when the third argument is non-null, it is
actually $V'$ that is provided.


\section{Matrix accessors}
\label{matrix-accessors}

In addition to the matrix functions discussed above,
various ``accessor'' strings allow you to create copies of internal
matrices associated with models previously estimated:

\begin{center}
\begin{tabular}{ll}
\texttt{\$coeff} & vector of estimated coefficients \\
\texttt{\$stderr} & vector of estimated standard errors \\
\texttt{\$uhat} & vector of residuals \\
\texttt{\$yhat} & vector of fitted values \\
\texttt{\$vcv} & covariance matrix (see below) \\
\texttt{\$rho} & autoregressive coefficients for error process \\
\texttt{\$jalpha} & matrix $\alpha$ (loadings) from Johansen's procedure \\
\texttt{\$jbeta} & matrix $\beta$ (cointegration vectors) from
Johansen's procedure \\
\texttt{\$jvbeta} & covariance matrix for the unrestricted elements of
$\beta$ from Johansen's procedure
\end{tabular}
\end{center}

If these accessors are given without any prefix, they retrieve results
from the last model estimated, if any.  Alternatively, they may be
prefixed with the name of a saved model plus a period (\texttt{.}), in
which case they retrieve results from the specified model.  Here are
some examples:
%
\begin{textcode}
matrix u = \$uhat\\
matrix b = m1.\$coeff\\
matrix v2 = m1.\$vcv[1:2,1:2]
\end{textcode}
%
The first command grabs the residuals from the last model; the second
grabs the coefficient vector from model \texttt{m1}; and the third
(which uses the mechanism of sub-matrix selection described in the
following section) grabs a portion of the covariance matrix from model
\texttt{m1}.

If the ``model'' in question is actually a system (a VAR or VECM, or
system of simultaneous equations), \texttt{\$uhat} retrieves the
matrix of residuals (one column per equation) and \texttt{\$vcv} gets
the cross-equation covariance matrix; in the special case of a VAR or
a VECM, \texttt{\$coeff} returns the companion matrix. At present the
other accessors are not available for equation systems.

After a vector error correction model is estimated via Johansen's
procedure, the matrices \texttt{\$jalpha} and \texttt{\$jbeta} are
also available. These have a number of columns equal to the chosen
cointegration rank; therefore, the product
\begin{code}
matrix Pi = $jalpha * $jbeta'
\end{code}
returns the reduced-rank estimate of $A(1)$. Since $\beta$ is
automatically identified via the Phillips normalization (see section
\ref{sec:johansen-ident}), its unrestricted elements do have a proper
covariance matrix, which can be retrieved through the
\texttt{\$jvbeta} accessor.

\section{Selecting sub-matrices}
\label{matrix-sub}

You can select sub-matrices of a given matrix using the syntax

\texttt{A[}\textsl{rows},\textsl{cols}\texttt{]}

where \textsl{rows} can take any of these forms:

\begin{center}
\begin{tabular}{ll}
empty & selects all rows \\
a single integer & selects the single specified row \\
two integers separated by a colon & selects a range of rows \\
the name of a matrix & selects the specified rows \\
\end{tabular}
\end{center}

With regard to the second option, the integer value can be given
numerically, as the name of an existing scalar variable, or as an
expression that evaluates to a scalar.  With the last option, the
index matrix given in the \textsl{rows} field must be either $p\times
1$ or $1\times p$, and should contain integer values in the range 1 to
$n$, where $n$ is the number of rows in the matrix from which the
selection is to be made.

The \textsl{cols} specification works in the same way, \textit{mutatis
  mutandis}.  Here are some examples.
%
\begin{code}
matrix B = A[1,]
matrix B = A[2:3,3:5]
matrix B = A[2,2]
matrix idx = { 1, 2, 6 }
matrix B = A[idx,]
\end{code}
%
The first example selects row 1 from matrix \texttt{A}; the second
selects a $2\times 3$ submatrix; the third selects a scalar; and
the fourth selects rows 1, 2, and 6 from matrix \texttt{A}.

In addition there is a pre-defined index specification, \texttt{diag},
which selects the principal diagonal of a square matrix, as in
\texttt{B[diag]}, where \texttt{B} is square.

You can use selections of this sort on either the right-hand side of
a matrix-generating formula or the left.  Here is an example of use of
a selection on the right, to extract a $2\times 2$ submatrix $B$ from a
$3\times 3$ matrix $A$:
%
\begin{code}
matrix A = { 1, 2, 3; 4, 5, 6; 7, 8, 9 }
matrix B = A[1:2,2:3]
\end{code}
%
And here are examples of selection on the left.  The second line below
writes a $2\times 2$ identity matrix into the bottom right corner of the
$3\times 3$ matrix $A$.  The fourth line replaces the diagonal of $A$ 
with 1s.
%
\begin{code}
matrix A = { 1, 2, 3; 4, 5, 6; 7, 8, 9 }
matrix A[2:3,2:3] = I(2)
matrix d = { 1, 1, 1 }
matrix A[diag] = d
\end{code}

\section{Namespace issues}
\label{matrix-namespace}

Matrices share a common namespace with data series and scalar
variables.  In other words, no two objects of any of these types can
have the same name.  It is an error to attempt to change the type of
an existing variable, for example:
%
\begin{code}
scalar x = 3
matrix x = ones(2,2) # wrong!
\end{code}
%
It is possible, however, to delete or rename an existing variable then
reuse the name for a variable of a different type:
\begin{code}
scalar x = 3
delete x
matrix x = ones(2,2) # OK
\end{code}


\section{Creating a data series from a matrix}
\label{matrix-create-series}

Section~\ref{matrix-create} above describes how to create a matrix
from a data series or set of series.  You may sometimes wish to go in
the opposite direction, that is, to copy values from a matrix 
into a regular data series.  The syntax for this operation is
%
\begin{textcode}
series \textsl{sname} = \textsl{mspec}
\end{textcode}
%
where \ttsl{sname} is the name of the series to create and
\ttsl{mspec} is the name of the matrix to copy from, possibly followed
by a matrix selection expression.  Here are two examples.
%
\begin{code}
series s = x
series u1 = U[,1]
\end{code}
%
It is assumed that \texttt{x} and \texttt{U} are pre-existing
matrices.  In the second example the series \texttt{u1} is formed from
the first column of the matrix \texttt{U}.

For this operation to work, the matrix (or matrix selection) must be a
vector with length equal to either the full length of the current
dataset, $n$, or the length of the current sample range, $n^{\prime}$.
If $n^{\prime} < n$ then only $n^{\prime}$ elements are drawn from the
matrix; if the matrix or selection comprises $n$ elements, the
$n^{\prime}$ values starting at element $t_1$ are used, where $t_1$
represents the starting observation of the sample range.  Any values
in the series that are not assigned from the matrix are set to the
missing code.


\section{Matrices and lists}
\label{matrix-and-list}

To facilitate the manipulation of named lists of variables (see
Chapter~\ref{chap-persist}), it is possible to convert between
matrices and lists.  In section~\ref{matrix-create} above we mentioned
the facility for creating a matrix from a list of variables, as in
%
\begin{code}
matrix M = { listname }
\end{code}
%
That formulation, with the name of the list enclosed in braces, builds
a matrix whose columns hold the variables referenced in the list.
What we are now describing is a different matter: if we say
%
\begin{code}
matrix M = listname
\end{code}
%
(without the braces), we get a row vector whose elements are
the ID numbers of the variables in the list.  This special case
of matrix generation cannot be embedded in a compound
expression.  The syntax must be as shown above, namely simple
assignment of a list to a matrix.

To go in the other direction, you can include a matrix on the
right-hand side of an expression that defines a list, as in
%
\begin{code}
list Xl = M
\end{code}
%
where \texttt{M} is a matrix.  The matrix must be suitable for
conversion; that is, it must be a row or column vector containing
non-negative whole-number values, none of which exceeds the highest ID
number of a variable (series or scalar) in the current dataset.

Example~\ref{normalize-list} illustrates the use of this sort of
conversion to ``normalize'' a list, moving the constant (variable 0)
to first position.

\begin{script}[htbp]
  \caption{Manipulating a list}
  \label{normalize-list}
\begin{scode}
function normalize_list (matrix *x)
  # If the matrix (representing a list) contains var 0,
  # but not in first position, move it to first position

  if (x[1] != 0)
     scalar k = cols(x)
     loop for (i=2; i<=k; i++) --quiet
        if (x[i] = 0)
            x[i] = x[1]
            x[1] = 0
            break
         endif
     end loop
  end if
end function

open data9-7
list Xl = 2 3 0 4
matrix x = Xl
normalize_list(&x)
list Xl = x
\end{scode}
\end{script}


\section{Deleting a matrix}
\label{matrix-delete}

To delete a matrix, just write
%
\begin{code}
delete M
\end{code}
%
where \texttt{M} is the name of the matrix to be deleted.

\section{Printing a matrix}

To print a matrix, you can simply give the name of the matrix in
question on a line by itself, or you can use the \cmd{print} command:
%
\begin{code}
matrix M = mnormal(100,2)
M
print M
\end{code}

\section{Example: OLS using matrices}
\label{matrix-example}

Example \ref{matrixOLS} shows how matrix methods can be used to
replicate gretl's built-in OLS functionality.

\begin{script}[htbp]
  \caption{OLS via matrix methods}
  \label{matrixOLS}
\begin{scode}
open data4-1
matrix X = { const, sqft }
matrix y = { price }
matrix b = inv(X'X) * X'y
printf "estimated coefficient vector\n"
b
matrix u = y - X*b
scalar SSR = u'u
scalar s2 = SSR / (rows(X) - rows(b))
matrix V = s2 * inv(X'X)
V
matrix se = sqrt(diag(V))
printf "estimated standard errors\n"
se
# compare with built-in function
ols price const sqft --vcv
\end{scode}
\end{script}



















