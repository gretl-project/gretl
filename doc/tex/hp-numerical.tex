\chapter{Numerical methods}
\label{chap:numerical}

\section{Numerical optimization}
\label{sec:hp-numopt}

Many, if not most, cases in which an econometrician wants to use a
programming language such as hansl, rather than relying on pre-canned
routines, involve some form of numerical optimization. This could take
the form of maximization of a likelihood or similar methods of
inferential statistics. Alternatively, optimization could be used in a
more general and abstract way, for example to solve portfolio choice
or analogous resource allocation problems.

Since hansl is Turing-complete, in principle any numerical
optimization technique could be programmed in hansl itself. Some such
techniques, however, are included in hansl's set of native
functions, in the interest of both simplicity of use and
efficiency. These are geared towards the most common kind of problem
encountered in economics and econometrics, that is unconstrained
optimization of differentiable functions.

In this chapter, we will briefly review what hansl offers to solve
generic problems of the kind
\[
\hat{\mathbf{x}} \equiv \argmax_{\mathbf{x} \in \Re^k} f(\mathbf{x}; \mathbf{a}),
\]
where $f(\mathbf{x}; \mathbf{a})$ is a function of $\mathbf{x}$, whose
shape depends on a vector of parameters $\mathbf{a}$. The objective
function $f(\cdot)$ is assumed to return a scalar real value. In most
cases, it will be assumed it is also continuous and differentiable,
although this need not necessarily be the case. (Note that while
hansl's built-in functions maximize the given objective function,
minimization can be achieved simply by flipping the sign of
$f(\cdot)$.)

A special case of the above occurs when $\mathbf{x}$ is a vector of
parameters and $\mathbf{a}$ represents ``data''. In these cases, the
objective function is usually a (log-)likelihood and the problem is
one of estimation. For such cases hansl offers several special
constructs, reviewed in section~\ref{sec:est-blocks}. Here we deal
with more generic problems; nevertheless, the differences are only in
the hansl syntax involved: the mathematical algorithms that gretl
employs to solve the optimization problem are the same.

The reader is invited to read the ``Numerical methods'' chapter of
\GUG\ for a comprehensive treatment. Here, we will only give a small
example which should give an idea of how things are done.

\begin{code}
function scalar Himmelblau(matrix x)
    /* extrema:
    f(3.0, 2.0) = 0.0, 
    f(-2.805118, 3.131312) = 0.0,
    f(-3.779310, -3.283186) = 0.0
    f(3.584428, -1.848126) = 0.0
    */
    scalar ret = (x[1]^2 + x[2] - 11)^2
    return -(ret + (x[1] + x[2]^2 - 7)^2)
end function

# ----------------------------------------------------------------------

set max_verbose 1

matrix theta1 = { 0, 0 }
y1 = BFGSmax(theta1, "Himmelblau(theta1)")
matrix theta2 = { 0, -1 }
y2 = NRmax(theta2, "Himmelblau(theta2)")

print y1 y2 theta1 theta2
\end{code}

We use for illustration here a classic ``nasty'' function from the
numerical optimization literature, namely the Himmelblau function,
which has four different minima; $f(x, y) = (x^2+y-11)^2 +
(x+y^2-7)^2$. The example proceeds as follows.
\begin{enumerate}
\item First we define the function to optimize: it must return a
  scalar and have among its arguments the vector to optimize. In this
  particular case that is its only argument, but there could have been
  other ones if necessary.  Since in this case we are solving for a
  minimum our definition returns the negative of the Himmelblau
  function proper.
\item We next set \verb|max_verbose| to 1. This is another example of
  the usage of the \cmd{set} command; its meaning is ``let me see how
  the iterations go'' and it defaults to 0. By using the \cmd{set}
  command with appropriate parameters, you control several features of
  the optimization process, such as numerical tolerances,
  visualization of the iterations, and so forth.
\item Define $\theta_1 = [0, 0]$ as the starting point.
\item Invoke the \cmd{BFGSmax} function; this will seek the maximum
  via the BFGS technique. Its base syntax is \texttt{BFGSmax(arg1,
    arg2)}, where \texttt{arg1} is the vector contining the
  optimization variable and \texttt{arg2} is a string containing the
  invocation of the function to maximize. BFGS will try several values
  of $\theta_1$ until the maximum is reached. On successful
  completion, the vector \texttt{theta1} will contain the final
  point. (Note: there's \emph{much} more to this. For details, be sure
  to read \GUG\ and the \GCR.)
\item Then we tackle the same problem but with a different starting
  point and a different optimization technique. We start from
  $\theta_2 = [0, -1]$ and use Newton--Raphson instead of BFGS,
  calling the \cmd{NRmax()} function instead if \cmd{BFGSmax()}. The
  syntax, however, is the same.
\item Finally we print the results.
\end{enumerate}

Table~\ref{tab:optim-output} on page \pageref{tab:optim-output}
contains a selected portion of the output. Note that the second run
converges to a different local optimum than the first one. This is a
consequence of having initialized the algorithm with a different
starting point. In this example, numerical derivatives were used, but
you can supply analytically computed derivatives to both methods if
you have a hansl function for them; see \GUG\ for more detail.

\begin{table}[ht]
  \begin{footnotesize}
\begin{scode}
? matrix theta1 = { 0, 0 }
Replaced matrix theta1
? y1 = BFGSmax(theta1, "Himmelblau(11, theta1)")
Iteration 1: Criterion = -170.000000000
Parameters:       0.0000      0.0000
Gradients:        14.000      22.000 (norm 0.00e+00)

Iteration 2: Criterion = -128.264504038 (steplength = 0.04)
Parameters:      0.56000     0.88000
Gradients:        33.298      39.556 (norm 5.17e+00)

...

--- FINAL VALUES: 
Criterion = -1.83015730011e-28 (steplength = 0.0016)
Parameters:       3.0000      2.0000
Gradients:    1.7231e-13 -3.7481e-13 (norm 7.96e-07)

Function evaluations: 39
Evaluations of gradient: 16
Replaced scalar y1 = -1.83016e-28
? matrix theta2 = { 0, -1 }
Replaced matrix theta2
? y2 = NRmax(theta2, "Himmelblau(11, theta2)")
Iteration 1: Criterion = -179.999876556 (steplength = 1)
Parameters:   1.0287e-05     -1.0000
Gradients:        12.000  2.8422e-06 (norm 7.95e-03)

Iteration 2: Criterion = -175.440691085 (steplength = 1)
Parameters:      0.25534     -1.0000
Gradients:        12.000  4.5475e-05 (norm 1.24e+00)

...

--- FINAL VALUES: 
Criterion = -3.77420797114e-22 (steplength = 1)
Parameters:       3.5844     -1.8481
Gradients:   -2.6649e-10  2.9536e-11 (norm 2.25e-05)

Gradient within tolerance (1e-07)
Replaced scalar y2 = -1.05814e-07
? print y1 y2 theta1 theta2

             y1 = -1.8301573e-28

             y2 = -1.0581385e-07

theta1 (1 x 2)

  3   2 

theta2 (1 x 2)

      3.5844      -1.8481 
\end{scode}
    
  \end{footnotesize}
  \caption{Output from maximization}
  \label{tab:optim-output}
\end{table}

The optimization methods hansl puts at your disposal are:
\begin{itemize}
\item BFGS, via the \cmd{BFGSmax()} function. This is in most cases
  the best compromise between performance and robustness. It assumes
  that the function to maximize is differentiable and will try to
  approximate its curvature by clever use of the change in the
  gradient between iterations. You can supply it with an
  analytically-computed gradient for speed and accuracy, but if you
  don't, the first derivatives will be computed numerically.
\item Newton--Raphson, via the \cmd{NRmax()} function. Actually, the
  function is less specific than the name implies. This is a
  ``curvature-based'' method, relying on the iterations
  \[
    x_{i+1} = -\lambda_i C(x_i)^{-1} g(x_i)
  \]
  where $g(x)$ is the gradient and $C(x_i)$ is some measure of curvature of
  the function to optimize; if $C(x)$ is the Hessian matrix, you get
  Newton--Raphson proper. Again, you can code your own functions for
  $g(\cdot)$ and $C(\cdot)$, but if you don't then numerical approximations
  to the gradient and the Hessian will be used, respectively. Other popular
  optimization methods (such as BHHH and the scoring algorithm) can be
  implemented by supplying to \cmd{NRmax()} the appropriate curvature matrix
  $C(\cdot)$. This method is very efficient when it works, but is less
  robust than BFGS; for example, if $C(x_i)$ happens to be non-negative
  definite at some iteration convergence may become problematic.
\item Derivative-free methods: hansl offers simulated annealing, via
  the \cmd{simann} function, and the Nelder--Mead simplex algorithm
  (also known as the ``amoeba'' method) via the \cmd{NMmax} function.
  These methods work even when the function to be maximized has some
  form of disconinuity or is not everywhere differentiable; however,
  they may be slow and CPU-intensive.
\end{itemize}

\section{Numerical differentiation}
\label{sec:hp-numdiff}

For numerical differentiation we have \texttt{fdjac}. For example:

\begin{code}
set echo off
set messages off

function scalar beta(scalar x, scalar a, scalar b)
    return x^(a-1) * (1-x)^(b-1)
end function

function scalar ad_beta(scalar x, scalar a, scalar b)
    scalar g = beta(x, a-1, b-1)
    f1 = (a-1) * (1-x)
    f2 = (b-1) * x
    return (f1 - f2) * g
end function

function scalar nd_beta(scalar x, scalar a, scalar b)
    matrix mx = {x}
    return fdjac(mx, beta(mx, a, b))
end function

a = 3.5
b = 2.5

loop for (x=0; x<=1; x+=0.1)
    printf "x = %3.1f; beta(x) = %7.5f, ", x, beta(x, a, b)
    A = ad_beta(x, a, b)
    N = nd_beta(x, a, b)
    printf "analytical der. = %8.5f, numerical der. = %8.5f\n", A, N
endloop
\end{code}

returns 

\begin{code}
x = 0.0; beta(x) = 0.00000, analytical der. =  0.00000, numerical der. =  0.00000
x = 0.1; beta(x) = 0.00270, analytical der. =  0.06300, numerical der. =  0.06300
x = 0.2; beta(x) = 0.01280, analytical der. =  0.13600, numerical der. =  0.13600
x = 0.3; beta(x) = 0.02887, analytical der. =  0.17872, numerical der. =  0.17872
x = 0.4; beta(x) = 0.04703, analytical der. =  0.17636, numerical der. =  0.17636
x = 0.5; beta(x) = 0.06250, analytical der. =  0.12500, numerical der. =  0.12500
x = 0.6; beta(x) = 0.07055, analytical der. =  0.02939, numerical der. =  0.02939
x = 0.7; beta(x) = 0.06736, analytical der. = -0.09623, numerical der. = -0.09623
x = 0.8; beta(x) = 0.05120, analytical der. = -0.22400, numerical der. = -0.22400
x = 0.9; beta(x) = 0.02430, analytical der. = -0.29700, numerical der. = -0.29700
x = 1.0; beta(x) = 0.00000, analytical der. = -0.00000, numerical der. =       NA
\end{code}

Details on the algorithm used can be found in the \GCR. Suffice it to
say here that you have a \dtkttt{fdjac_quality} setting that goes
from 0 to 2. The default value is to 0, which gives you
forward-difference approximation: this is the fastest algorithm, but
sometimes may not be precise enough. The value of 1 gives you
bilateral difference, while 2 uses Richardson extrapolation. Higher
values give greater accuracy but the method becomes considerably more
CPU-intensive.

% \section{Random number generation}

% \begin{itemize}
% \item Mersenne Twister in its various incarnations
% \item Ziggurat vs Box--Muller
% \item Other distributions
% \end{itemize}

%%% Local Variables: 
%%% mode: latex
%%% TeX-master: "hansl-primer"
%%% End: 

