\documentclass{article}
\usepackage{gretl-lite}
\usepackage[letterpaper,body={6.3in,9.15in},top=.8in,left=1.1in]{geometry}

\begin{document}

\begin{center}
\textbf{Keyboard shortcuts}
\end{center}

\textsf{Gretl} offers keyboard shortcuts for various commands. Some of
these are standard, and are indicated in the menus --- for example,
\texttt{Ctrl-s} in the main window saves the current dataset, as
indicated under the \textsf{File} menu. Others are specific to
\textsf{gretl} and do not appear in the menus; here we document the
latter.

\begin{center}
\begin{tabular}{lll}
\textit{Context} & \textit{Key} & \textit{Effect} \\[6pt]
Most windows & \texttt{Alt-w} & 
  pop up a menu listing open \textsf{gretl} windows \\[6pt]
Main window & \texttt{g} & 
  open dialog for defining a new variable \\
 & \texttt{Alt-x} & 
open dialog for executing a single command

% console: up, down: history
% console: ctrl-a: start of line
% console: tab: command completion (or varname)
% datafiles window: enter: open selected file


\end{tabular}
\end{center}

\end{document}
