\chapter{Discrete and censored dependent variables}
\label{chap:probit}

This chapter deals with models for dependent variables that are
discrete or censored or otherwise limited (as in event counts or
durations, which must be positive) and that therefore call for
estimation methods other than the classical linear model. We discuss
several estimators (mostly based on the Maximum Likelihood principle),
adding some details and examples to complement the material on
these methods in the \emph{Gretl Command Reference}.

\section{Logit and probit models}
\label{sec:logit-probit}

It often happens that one wants to specify and estimate a model in
which the dependent variable is not continuous, but discrete. A
typical example is a model in which the dependent variable is the
occupational status of an individual (1 = employed, 0 = unemployed). A
convenient way of formalizing this situation is to consider the
variable $y_i$ as a Bernoulli random variable and analyze its
distribution conditional on the explanatory variables $x_i$.  That is,
%
\begin{equation}
  \label{eq:qr-Bernoulli}
  y_i = \left\{ 
    \begin{array}{ll} 
      1 & P_i \\ 0 & 1 - P_i 
    \end{array}
    \right.
\end{equation}
%
where $P_i = P(y_i = 1 | x_i) $ is a given function of the explanatory
variables $x_i$.

In most cases, the function $P_i$ is a cumulative distribution
function $F$, applied to a linear combination of the $x_i$s. In the
probit model, the normal cdf is used, while the logit model employs
the logistic function $\Lambda()$. Therefore, we have
%
\begin{eqnarray}
  \label{eq:qr-link}
  \textrm{probit} & \qquad & P_i = F(z_i) = \Phi(z_i)  \\
  \textrm{logit}  & \qquad & P_i = F(z_i) = \Lambda(z_i) = \frac{1}{1 + e^{-z_i}} \\
  & &z_i = \sum_{j=1}^k x_{ij} \beta_j
\end{eqnarray}
%
where $z_i$ is commonly known as the \emph{index} function. Note that
in this case the coefficients $\beta_j$ cannot be interpreted as the
partial derivatives of $E(y_i | x_i)$ with respect to
$x_{ij}$.  However, for a given value of $x_i$ it is possible to
compute the vector of ``slopes'', that is
\[
  \mathrm{slope}_j(\bar{x}) = \left. \pder{F(z)}{x_j} \right|_{z =
    \bar{z}}
\]
Gretl automatically computes the slopes, setting each
explanatory variable at its sample mean.

Another, equivalent way of thinking about this model is in terms of
an unobserved variable $y^*_i$ which can be described thus:
%
\begin{equation}
  \label{eq:qr-latent}
  y^*_i = \sum_{j=1}^k x_{ij} \beta_j + \varepsilon_i = z_i  +
  \varepsilon_i 
\end{equation}
%
We observe $y_i = 1$ whenever $y^*_i > 0$ and $y_i = 0$ otherwise. If
$\varepsilon_i$ is assumed to be normal, then we have the probit
model. The logit model arises if we assume that the density function
of $\varepsilon_i$ is
%
\[
  \lambda(\varepsilon_i) =
  \pder{\Lambda(\varepsilon_i)}{\varepsilon_i} =
  \frac{e^{-\varepsilon_i}}{(1 + e^{-\varepsilon_i})^2}
\]

Both the probit and logit model are estimated in gretl via
maximum likelihood, where the log-likelihood can be written as
\begin{equation}
  \label{eq:qr-loglik}
  L(\beta) = \sum_{y_i=0} \ln [ 1 - F(z_i)] + \sum_{y_i=1} \ln F(z_i),
\end{equation}
which is always negative, since $0 < F(\cdot) < 1$.  Since the score
equations do not have a closed form solution, numerical optimization
is used. However, in most cases this is totally transparent to the
user, since usually only a few iterations are needed to ensure
convergence. The \option{verbose} switch can be used to track the
maximization algorithm.

\bigskip

By way of example, the commands below reproduce results given in
chapter 21 of \cite{greene00}, regarding the effectiveness of a
program for teaching economics.
\begin{code}
open greene19_1
logit GRADE const GPA TUCE PSI
probit GRADE const GPA TUCE PSI
\end{code}
The binary dependent variable, \texttt{GRADE}, equals 1 if a student's
grade improved over a certain period, 0 otherwise. The independent
variables are \texttt{PSI} (= 1 if the student participated in the
program in question, otherwise 0) plus two controls, the student's
initial Grade Point Average (\texttt{GPA}) and score on a prior
economics test (\texttt{TUCE}).  Output is shown in
Table~\ref{tab:logit-probit}. Note that for the probit model a
conditional moment test on skewness and kurtosis \citep*{JBL84} is
printed automatically as a test for normality.

\begin{table}
  \centering
\begin{outbit}
Model 1: Logit estimates using the 32 observations 1-32
Dependent variable: GRADE

      VARIABLE       COEFFICIENT        STDERROR      T STAT       SLOPE
                                                                  (at mean)
  const               -13.0213           4.93132      -2.641
  GPA                   2.82611          1.26294       2.238      0.533859   
  TUCE                  0.0951577        0.141554      0.672      0.0179755  
  PSI                   2.37869          1.06456       2.234      0.449339   

  Mean of GRADE = 0.344
  Number of cases 'correctly predicted' = 26 (81.2%)
  f(beta'x) at mean of independent vars = 0.189
  McFadden's pseudo-R-squared = 0.374038
  Log-likelihood = -12.8896
  Likelihood ratio test: Chi-square(3) = 15.4042 (p-value 0.001502)
  Akaike information criterion (AIC) = 33.7793
  Schwarz Bayesian criterion (BIC) = 39.6422
  Hannan-Quinn criterion (HQC) = 35.7227

           Predicted
             0    1
  Actual 0  18    3
         1   3    8

Model 2: Probit estimates using the 32 observations 1-32
Dependent variable: GRADE

      VARIABLE       COEFFICIENT        STDERROR      T STAT       SLOPE
                                                                  (at mean)
  const                -7.45232          2.54247      -2.931
  GPA                   1.62581          0.693883      2.343      0.533347   
  TUCE                  0.0517288        0.0838903     0.617      0.0169697  
  PSI                   1.42633          0.595038      2.397      0.467908   

  Mean of GRADE = 0.344
  Number of cases 'correctly predicted' = 26 (81.2%)
  f(beta'x) at mean of independent vars = 0.328
  McFadden's pseudo-R-squared = 0.377478
  Log-likelihood = -12.8188
  Likelihood ratio test: Chi-square(3) = 15.5459 (p-value 0.001405)
  Akaike information criterion (AIC) = 33.6376
  Schwarz Bayesian criterion (BIC) = 39.5006
  Hannan-Quinn criterion (HQC) = 35.581

           Predicted
             0    1
  Actual 0  18    3
         1   3    8

Test for normality of residual -
  Null hypothesis: error is normally distributed
  Test statistic: Chi-square(2) = 3.61059
  with p-value = 0.164426
\end{outbit}
  \caption{Example logit and probit output}
  \label{tab:logit-probit}
\end{table}

In this context, the \dollar{uhat} accessor function takes a special
meaning: it returns generalized residuals as defined in
\citet*{gourieroux87}, which can be interpreted as unbiased estimators
of the latent disturbances $\varepsilon_i$. These are defined as
%
\begin{equation}
  \label{eq:QR-genres}
  u_i = \left\{
    \begin{array}{ll}
      y_i - \hat{P}_i & \textrm{for the logit model} \\
      y_i\cdot \frac{\phi(\hat{z}_i)}{\Phi(\hat{z}_i)} - 
      ( 1 - y_i ) \cdot \frac{\phi(\hat{z}_i)}{1 - \Phi(\hat{z}_i)}
      & \textrm{for the probit model} \\
    \end{array}
    \right.
\end{equation}

Among other uses, generalized residuals are often used for diagnostic
purposes.  For example, it is very easy to set up an omitted variables
test equivalent to the familiar LM test in the context of a linear
regression; example \ref{ex:addition-test} shows how to perform a variable
addition test.

\begin{script}[htbp]
  \scriptinfo{addition-test}{Variable addition test in a probit model}
\begin{scode}
open greene19_1

probit GRADE const GPA PSI
series u = $uhat 
ols u const GPA PSI TUCE -q
printf "Variable addition test for TUCE:\n"
printf "Rsq * T = %g (p. val. = %g)\n", $trsq, pvalue(X,1,$trsq) 
\end{scode}
\end{script}

\subsection{Odds ratios}

A noteworthy feature of the binary logit model is that the regression
coefficients have an interpretation as log odds ratios, where the odds
ratio is $0 < P(y=1)/P(y=0) < \infty$. In the logit example above the
coefficient on \texttt{TUCE} has a value of 0.095. The corresponding
odds ratio is then $e^{0.095} = 1.10$, meaning that the estimated
effect of a unit increase in \texttt{TUCE} is to move the odds ratio
by 10 percent in favor of \texttt{GRADE} = 1.

When a binary logit model is estimated via the gretl GUI, the
\textsf{Analysis} menu in the model output window incudes an ``Odds
ratios'' item. This opens a window showing the odds ratios along with
standard errors (obtained via the delta method) plus a 95 percent
confidence interval, as illustrated below.
%
\begin{code}
95% confidence intervals
z(0.025) = 1.9600

             odds ratio   std. error     low        high
  ---------------------------------------------------------
  GPA         16.8797     21.3181      1.42019    200.624
  TUCE         1.09983     0.155686    0.833365     1.45150
  PSI         10.7907     11.4874      1.33934     86.9380
\end{code}
%
Note, however, that confidence intervals shown are not calculated
using the delta-method standard errors; rather, the bounds are
obtained by exponentiating the bounds of regular confidence intervals
for the coefficients. This makes sense on the assumption that the
coefficients themselves are more likely to be normally distributed
than their exponentials.

Odds ratio information can also be retrieved following binary logit
estimation via scripting. In this case it takes the form of a matrix,
provided by the \dollar{oddsratios} accessor or as
\dollar{model.oddsratios}.

\subsection{The perfect prediction problem}
\label{sec:perfpred}

One curious characteristic of logit and probit models is that (quite
paradoxically) estimation is not feasible if a model fits the data
perfectly; this is called the \emph{perfect prediction problem}. The
reason why this problem arises is easy to see by considering equation
(\ref{eq:qr-loglik}): if for some vector $\beta$ and scalar $k$ it's
the case that $z_i < k$ whenever $y_i=0$ and $z_i > k$ whenever
$y_i=1$, the same thing is true for any multiple of $\beta$. Hence,
$L(\beta)$ can be made arbitrarily close to 0 simply by choosing
enormous values for $\beta$. As a consequence, the log-likelihood has
no maximum, despite being bounded.

Gretl has a mechanism for preventing the algorithm from
iterating endlessly in search of a non-existent maximum. One sub-case
of interest is when the perfect prediction problem arises because of
a single binary explanatory variable. In this case, the offending
variable is dropped from the model and estimation proceeds with the
reduced specification. Nevertheless, it may happen that no single
``perfect classifier'' exists among the regressors, in which case
estimation is simply impossible and the algorithm stops with an
error. This behavior is triggered during the iteration process if
\[
  \stackunder{i: y_i = 0}{\max z_i} \, < \,
  \stackunder{i: y_i = 1}{\min z_i}  
\]
If this happens, unless your model is trivially mis-specified (like
predicting if a country is an oil exporter on the basis of oil
revenues), it is normally a small-sample problem: you probably just
don't have enough data to estimate your model. You may want to drop
some of your explanatory variables.

This problem is well analyzed in \cite{stokes04}; the results therein
are replicated in the example script \texttt{murder\_rates.inp}. 

\section{Ordered response models}
\label{sec:ordered}

These models constitute a simple variation on ordinary logit/probit
models, and are usually applied when the dependent variable is a
discrete and ordered measurement---not simply binary, but on an
ordinal rather than an interval scale.  For example, this sort of
model may be applied when the dependent variable is a qualitative
assessment such as ``Good'', ``Average'' and ``Bad''.  

In the general case, consider an ordered response variable, $y$, that
can take on any of the $J+1$ values ${0,1,2,\dots,J}$.  We suppose, as
before, that underlying the observed response is a latent variable,
\[
  y^* = X\beta + \varepsilon = z + \varepsilon
\]
Now define ``cut points'', $\alpha_1 < \alpha_2 < \cdots < \alpha_J$,
such that
%
\begin{equation*}
  \begin{array}{ll}
    y = 0 & \textrm{if } \, y^* \leq \alpha_1 \\
    y = 1 & \textrm{if } \, \alpha_1 < y^* \leq \alpha_2 \\
    \vdots \\
    y = J & \textrm{if } \, y^* > \alpha_J \\
  \end{array}
\end{equation*}
For example, if the response takes on three values there will be two
such cut points, $\alpha_1$ and $\alpha_2$. 

The probability that individual $i$ exhibits response $j$, conditional
on the characteristics $x_i$, is then given by
%
\begin{equation}
  \label{eq:QR-ordered}
  P(y_i = j \,|\, x_i) = \left\{
    \begin{array}{ll}
      P(y^* \leq \alpha_1 \,|\, x_i) = F(\alpha_1 - z_i) & \textrm{for }\, j = 0 \\
      P(\alpha_j < y^* \leq \alpha_{j+1} \,|\, x_i) = 
      F(\alpha_{j+1} - z_i) -  F(\alpha_j - z_i) & \textrm{for }\, 0 < j < J \\
      P(y^* > \alpha_J \,|\, x_i) = 1 -  F(\alpha_J - z_i) & \textrm{for }\, j = J 
    \end{array}
    \right.
\end{equation}
%
The unknown parameters $\alpha_j$ are estimated jointly with the
$\beta$s via maximum likelihood.  The $\hat{\alpha}_j$ estimates are
reported by gretl as \texttt{cut1}, \texttt{cut2} and so on. For
the probit variant, a conditional moment test for normality
constructed in the spirit of \cite{chesher-irish87} is also included.

Note that the $\alpha_j$ parameters can be shifted arbitrarily by
adding a constant to $z_i$, so the model is under-identified if there
is some linear combination of the explanatory variables which is
constant. The most obvious case in which this occurs is when the model
contains a constant term; for this reason, gretl drops
automatically the intercept if present. However, it may happen that
the user inadventently specifies a list of regressors that may be
combined in such a way to produce a constant (for example, by using a
full set of dummy variables for a discrete factor). If this happens,
gretl will also drop any offending regressors.

In order to apply these models in gretl, the dependent variable
must either take on only non-negative integer values, or be explicitly
marked as discrete.  (In case the variable has non-integer values, it
will be recoded internally.)  Note that gretl does not provide a
separate command for ordered models: the \texttt{logit} and
\texttt{probit} commands automatically estimate the ordered version if
the dependent variable is acceptable, but not binary.

\begin{script}[htbp]
  \scriptinfo{ordered-probit}{Ordered probit model}
\begin{scode}
/*
  Replicate the results in Wooldridge, Econometric Analysis of Cross
  Section and Panel Data, section 15.10, using pension-plan data from
  Papke (AER, 1998).

  The dependent variable, pctstck (percent stocks), codes the asset
  allocation responses of "mostly bonds", "mixed" and "mostly stocks"
  as {0, 50, 100}.

  The independent variable of interest is "choice", a dummy indicating
  whether individuals are able to choose their own asset allocations.
*/

open pension.gdt

# demographic characteristics of participant
list DEMOG = age educ female black married
# dummies coding for income level
list INCOME = finc25 finc35 finc50 finc75 finc100 finc101

# Papke's OLS approach
ols pctstck const choice DEMOG INCOME wealth89 prftshr
# save the OLS choice coefficient 
choice_ols = $coeff(choice)

# estimate ordered probit
probit pctstck choice DEMOG INCOME wealth89 prftshr

k = $ncoeff
matrix b = $coeff[1:k-2]
a1 = $coeff[k-1]
a2 = $coeff[k]

/* 
   Wooldridge illustrates the 'choice' effect in the ordered probit 
   by reference to a single, non-black male aged 60, with 13.5 years 
   of education, income in the range $50K - $75K and wealth of $200K, 
   participating in a plan with profit sharing.
*/
matrix X = {60, 13.5, 0, 0, 0, 0, 0, 0, 1, 0, 0, 200, 1}

# with 'choice' = 0
scalar Xb = (0 ~ X) * b
P0 = cdf(N, a1 - Xb)
P50 = cdf(N, a2 - Xb) - P0
P100 = 1 - cdf(N, a2 - Xb)
E0 = 50 * P50 + 100 * P100

# with 'choice' = 1
Xb = (1 ~ X) * b
P0 = cdf(N, a1 - Xb)
P50 = cdf(N, a2 - Xb) - P0
P100 = 1 - cdf(N, a2 - Xb)
E1 = 50 * P50 + 100 * P100

printf "\nWith choice, E(y) = %.2f, without E(y) = %.2f\n", E1, E0
printf "Estimated choice effect via ML = %.2f (OLS = %.2f)\n", E1 - E0,
  choice_ols
\end{scode}
\end{script}

Listing \ref{ex:ordered-probit} reproduces the results presented in
section 15.10 of \cite{wooldridge-panel}.  The question of interest in
this analysis is what difference it makes, to the allocation of assets
in pension funds, whether individual plan participants have a choice
in the matter.  The response variable is an ordinal measure of the
weight of stocks in the pension portfolio.  Having reported the
results of estimation of the ordered model, Wooldridge illustrates the
effect of the \texttt{choice} variable by reference to an ``average''
participant.  The example script shows how one can compute this effect
in gretl.

After estimating ordered models, the \dollar{uhat} accessor yields
generalized residuals as in binary models; additionally, the
\dollar{yhat} accessor function returns $\hat{z}_i$, so it is
possible to compute an unbiased estimator of the latent variable
$y^*_i$ simply by adding the two together.

\section{Multinomial logit}
\label{sec:mlogit}

When the dependent variable is not binary and does not have a natural
ordering, \emph{multinomial} models are used.  Multinomial logit is
supported in gretl via the \option{multinomial} option to the
\texttt{logit} command.  Simple models can also be handled via the
\texttt{mle} command (see chapter \ref{chap:mle}). We give here an
example of such a model.  Let the dependent variable, $y_i$, take on
integer values $0,1,\dots p$.  The probability that $y_i = k$ is given
by
\[
  P(y_i = k |  x_i) = \frac{\exp(x_i \beta_k)}{\sum_{j=0}^p \exp(x_i \beta_j)}
\]
For the purpose of identification one of the outcomes must be taken as
the ``baseline''; it is usually assumed that $\beta_0 = 0$, in which case
\[
  P(y_i = k |  x_i) = \frac{\exp(x_i \beta_k)}{1 + \sum_{j=1}^p \exp(x_i \beta_j)} 
\]
and
\[
  P(y_i = 0 |  x_i) = \frac{1}{1 + \sum_{j=1}^p \exp(x_i \beta_j)} .
\]

\begin{script}[htbp]
  \caption{Multinomial logit}
  \label{ex:mlogit}
\begin{scode}
open keane.gdt
smpl year==87 --restrict
logit status 0 educ exper expersq black --multinomial
\end{scode}
Output (selected portions):
\begin{outbit}
Model 1: Multinomial Logit, using observations 1-1738 (n = 1717)
Missing or incomplete observations dropped: 21
Dependent variable: status
Standard errors based on Hessian

             coefficient   std. error      z      p-value 
  --------------------------------------------------------
  status = 2
  const      10.2779       1.13334       9.069    1.20e-19 ***
  educ       -0.673631     0.0698999    -9.637    5.57e-22 ***
  exper      -0.106215     0.173282     -0.6130   0.5399  
  expersq    -0.0125152    0.0252291    -0.4961   0.6199  
  black       0.813017     0.302723      2.686    0.0072   ***
  status = 3
  const       5.54380      1.08641       5.103    3.35e-07 ***
  educ       -0.314657     0.0651096    -4.833    1.35e-06 ***
  exper       0.848737     0.156986      5.406    6.43e-08 ***
  expersq    -0.0773003    0.0229217    -3.372    0.0007   ***
  black       0.311361     0.281534      1.106    0.2687  

Mean dependent var   2.691322   S.D. dependent var   0.573502
Log-likelihood      -907.8572   Akaike criterion     1835.714
Schwarz criterion    1890.198   Hannan-Quinn         1855.874

Number of cases 'correctly predicted' = 1366 (79.6%)
Likelihood ratio test: Chi-square(8) = 583.722 [0.0000]  
\end{outbit}
\end{script}

Listing~\ref{ex:mlogit} reproduces Table 15.2 in
\cite{wooldridge-panel}, based on data on career choice from
\cite{keane97}.  The dependent variable is the occupational status of
an individual (0 = in school; 1 = not in school and not working; 2 =
working), and the explanatory variables are education and work
experience (linear and square) plus a ``black'' binary variable.  The
full data set is a panel; here the analysis is confined to a
cross-section for 1987. 

\section{Bivariate probit}
\label{sec:biprobit}

The bivariate probit model is a two-equation system in which each
equation is a probit model and the two disturbance terms may not be
independent. In formulae,
\begin{eqnarray}
  y^*_{1,i} = \sum_{j=1}^{k_1} x_{ij} \beta_j + \varepsilon_{1,i}  & \qquad &
  y_{1,i}=1  \Longleftrightarrow y^*_{1,i}>0 \\
  y^*_{2,i} = \sum_{j=1}^{k_2} z_{ij} \gamma_j + \varepsilon_{2,i} & \qquad &
  y_{2,i}=1  \Longleftrightarrow y^*_{2,i}>0 \\
  \left[ \begin{array}{c}
      \varepsilon_{1,i} \\ \varepsilon_{2,i}
    \end{array} \right] \sim 
  N \left[ 0, \left( \begin{array}{cc}
      1 & \rho \\ \rho & 1
    \end{array} \right) \right] 
\end{eqnarray}

If $\rho$ were 0, ML estimation of the parameters $\beta_j$ and
$\gamma_j$ could be accomplished by estimating the two equations
separately. In the general case, however, joint estimation is required
for maximal efficiency.

The gretl command for this model is \cmd{biprobit}, which performs ML
estimation via numerical optimization using the Newton--Raphson method
with analytical derivatives. An example of usage is provided in the
\texttt{biprobit.inp} sample script.  The command takes either three
or four arguments, the first three being series names for $y_{1}$ and
$y_{2}$ and a list of explanatory variables. In the common case when
the regressors are the same for the two equations this is sufficient,
but if $z$ differs from $x$ a second list should be appended following
a semicolon, as in:
\begin{code}
  biprobit y1 y2 X ; Z
\end{code}
Output from estimation includes a Likelihood Ratio test for the
hypothesis $\rho = 0$.\footnote{Note that if the \option{robust}
  option is given to \cmd{biprobit}---and therefore the estimator is
  meant to be QMLE---this test may not be valid, even asymptotically.}
This can be retrieved in the form of a bundle named
\texttt{independence\_test} under the \dollar{model} accessor, as in
\begin{code}
? eval $model.independence_test
bundle:
  dfn = 1
  test = 204.066
  pvalue = 2.70739e-46
\end{code}
%$

Since \cmd{biprobit} estimates a two-equation system, the
\dollar{uhat} and \dollar{yhat} accessors provide matrices rather than
series as usual. Specifically, \dollar{uhat} gives a two-column matrix
containing the generalized residuals, while \dollar{yhat} contains
four columns holding the estimated probabilities of the possible
joint outcomes: $(y_{1,i}, y_{1,i}) = (1,1)$ in column 1, $(y_{1,i},
y_{2,i}) = (1,0)$ in column 2, $(y_{1,i}, y_{2,i}) =
(0,1)$ in column 3 and $(y_{1,i}, y_{2,i}) = (0,0)$ in column 4.


\section{Panel estimators}
\label{sec:REprobit}

When your dataset is a panel, the traditional choice for binary
dependent variable models was, for many years, to use logit with fixed
effects and probit with random effects (see \ref{sec:FE-vs-RE} for a
brief discussion of this dichotomy in the context of linear
models). Nowadays the choice is somewhat wider but the two
traditional models are by and large what practitioners use as routine
tools.

Gretl provides FE logit via the function package
\textsf{felogit},\footnote{See
  \url{http://gretl.sourceforge.net/current_fnfiles/felogit.gfn}.}
RE probit natively. Provided your dataset has a panel structure, the
latter option can be obtained by adding the \option{random} option to
the \cmd{probit} command:
\begin{code}
  probit depvar const indvar1 indvar2 --random
\end{code}
as exemplified in the \texttt{reprobit.inp} sample script. The
numerical technique used for this particular estimator is
\emph{Gauss-Hermite quadrature}, which we'll now briefly
describe. Generalizing equation \eqref{eq:qr-latent} to a panel
context, we get
\begin{equation}
  y^*_{i,t} = \sum_{j=1}^k x_{ijt} \beta_j + \alpha_i + \varepsilon_{i,t} = z_{i,t}  +
  \omega_{i,t} 
\end{equation}
in which we assume that the individual effect, $\alpha_i$, and the
disturbance term, $\varepsilon_{i,t}$, are mutually independent
zero-mean Gaussian random variables. The composite error term,
$\omega_{i,t} = \alpha_i + \varepsilon_{i,t}$, is therefore a normal
r.~v.~with mean zero and variance $1 + \sigma^2_{\alpha}$. Because of
the individual effect, $\alpha_i$, observations for the same unit are
not independent; the likelihood therefore has to be evaluated on a
per-unit basis, as
\[
\LogLik_i = \log P\left[ y_{i,1}, y_{i,2}, \ldots , y_{i,T}\right] .
\]
and there's no way to write the above as a product of individual
terms.

However, the above probability \emph{could} be written as a product if
we were to treat $\alpha_i$ as a constant; in that case we would have
\[
\LogLik_i | \alpha_i = \sum_{t=1}^T 
\Phi\left[ 
(2 y_{i,t} - 1) \frac{x_{ijt} \beta_j + \alpha_i}{\sqrt{1 + \sigma^2_{\alpha}}}
\right]
\]
so that we can compute $\LogLik_i$ by integrating $\alpha_i$ out as
\[
\LogLik_i = E\left( \LogLik_i | \alpha_i \right) =
\int_{-\infty}^{\infty} (\LogLik_i | \alpha_i)
\frac{\varphi(\alpha_i)}{\sqrt{1 + \sigma^2_{\alpha}}} \mathrm{d} \alpha_i 
\]

The technique known as Gauss--Hermite quadrature is simply a way of
approximating the above integral via a sum of carefully chosen
terms:\footnote{Some have suggested using a more refined method called
  \emph{adaptive} Gauss-Hermite quadrature; this is not implemented in
  gretl.}
\[
\LogLik_i \simeq \sum_{k=1}^{m} (\LogLik_i | \alpha_i = n_k ) w_k
\]
where the numbers $n_k$ and $w_k$ are known as \emph{quadrature
  points} and \emph{weights}, respectively. Of course, accuracy
improves with higher values of $m$, but so does CPU usage. Note that
this technique can also be used in more general cases by using the
\cmd{quadtable()} function and the \cmd{mle} command via the apparatus
described in chapter \ref{chap:mle}. Here, however, the calculations
were hard-coded in C for maximal speed and efficiency.

Experience shows that a reasonable compromise can be achieved in most
cases by choosing $m$ in the order of 20 or so; gretl uses 32 as
a default value, but this can be changed via the \option{quadpoints}
option, as in
\begin{code}
  probit y const x1 x2 x3 --random --quadpoints=48
\end{code}

 
\section{The Tobit model}
\label{sec:tobit}

The Tobit model is used when the dependent variable of a model is
\emph{censored}. Assume a latent variable $y^*_i$ can be described
as
\[
  y^*_i = \sum_{j=1}^k x_{ij} \beta_j + \varepsilon_i ,
\]
where $\varepsilon_i \sim N(0,\sigma^2)$. If $y^*_i$ were observable,
the model's parameters could be estimated via ordinary least squares.
On the contrary, suppose that we observe $y_i$, defined as
%
\begin{equation}
  \label{eq:tobit}
  y_i = \left\{ 
    \begin{array}{ll}
      a & \mathrm{for} \quad y^*_i \le a \\
      y^*_i & \mathrm{for} \quad a < y^*_i < b \\ 
      b & \mathrm{for} \quad y^*_i \ge b 
    \end{array}
    \right. 
\end{equation}
In most cases found in the applied literature, $a=0$ and $b=\infty$,
so in practice negative values of $y^*_i$ are not observed and are
replaced by zeros.

In this case, regressing $y_i$ on the $x_i$s does not yield consistent
estimates of the parameters $\beta$, because the conditional mean
$E(y_i|x_i)$ is not equal to $\sum_{j=1}^k x_{ij} \beta_j$.  It can be
shown that restricting the sample to non-zero observations would not
yield consistent estimates either. The solution is to estimate the
parameters via maximum likelihood. The syntax is simply
\begin{code}
tobit depvar indvars
\end{code}

As usual, progress of the maximization algorithm can be tracked via
the \option{verbose} switch, while \dollar{uhat} returns the
generalized residuals. Note that in this case the generalized residual
is defined as $\hat{u}_i = E(\varepsilon_i | y_i = 0)$ for censored
observations, so the familiar equality $\hat{u}_i = y_i - \hat{y}_i$
only holds for uncensored observations, that is, when $y_i>0$.

An important difference between the Tobit estimator and OLS is that
the consequences of non-normality of the disturbance term are much
more severe: non-normality implies inconsistency for the Tobit
estimator. For this reason, the output for the Tobit model includes
the \cite{chesher-irish87} normality test by default.

The general case in which $a$ is nonzero and/or $b$ is finite can be
handled by using the options \option{llimit} and \option{rlimit}. So,
for example,
\begin{code}
tobit depvar indvars --llimit=10
\end{code}
would tell gretl that the left bound $a$ is set to 10.

\section{Interval regression}
\label{sec:intreg}

The interval regression model arises when the dependent variable is
unobserved for some (possibly all) observations; what we observe
instead is an interval in which the dependent variable lies.  In other
words, the data generating process is assumed to be
\[
  y^*_i = x_i \beta + \epsilon_i
\] 
but we only know that $m_i \le y^*_i \le M_i$, where the interval may
be left- or right-unbounded (but not both). If $m_i = M_i$, we
effectively observe $y^*_i$ and no information loss occurs. In
practice, each observation belongs to one of four categories:
\begin{enumerate}
\item left-unbounded, when $m_i = -\infty$,
\item right-unbounded, when $M_i = \infty$,
\item bounded, when $-\infty < m_i < M_i <\infty$ and
\item point observations when $m_i = M_i$.
\end{enumerate}

It is interesting to note that this model bears similarities to other
models in several special cases:
\begin{itemize}
\item When all observations are point observations the model trivially
  reduces to the ordinary linear regression model.
\item The interval model could be thought of as an ordered probit
  model (see \ref{sec:ordered}) in which the cut points (the
  $\alpha_j$ coefficients in eq. \ref{eq:QR-ordered}) are observed and
  don't need to be estimated.
\item The Tobit model (see \ref{sec:tobit}) is a special case of the
  interval model in which $m_i$ and $M_i$ do not depend on $i$, that
  is, the censoring limits are the same for all observations. As a
  matter of fact, gretl's \texttt{tobit} command is handled
  internally as a special case of the interval model.
\end{itemize}

The gretl command \texttt{intreg} estimates interval models by
maximum likelihood, assuming normality of the disturbance term
$\epsilon_i$.  Its syntax is
%
\begin{code}
intreg minvar maxvar X
\end{code}
%
where \texttt{minvar} contains the $m_i$ series, with \texttt{NA}s for
left-unbounded observations, and \texttt{maxvar} contains $M_i$, with
\texttt{NA}s for right-unbounded observations.  By default, standard
errors are computed using the negative inverse of the Hessian.  If the
\option{robust} flag is given, then QML or Huber--White standard
errors are calculated instead. In this case the estimated covariance
matrix is a ``sandwich'' of the inverse of the estimated Hessian and
the outer product of the gradient.

If the model specification contains regressors other than just a
constant, the output includes a chi-square statistic for testing the
joint null hypothesis that none of these regressors has any effect on
the outcome.  This is a Wald statistic based on the estimated
covariance matrix.  If you wish to construct a likelihood ratio test,
this is easily done by estimating both the full model and the null
model (containing only the constant), saving the log-likelihood in
both cases via the \dollar{lnl} accessor, and then referring twice the
difference between the two log-likelihoods to the chi-square
distribution with $k$ degrees of freedom, where $k$ is the number of
additional regressors (see the \texttt{pvalue} command in the
\GCR). Also included is a conditional moment normality test, similar
to those provided for the probit, ordered probit and Tobit models (see
above).  An example is contained in the sample script
\texttt{wtp.inp}, provided with the gretl distribution.

\begin{script}[ht]
  \scriptinfo{interval-estimation}{Interval model on artificial data}
\begin{scode}
nulldata 100
# generate artificial data
set seed 201449 
x = normal()
epsilon = 0.2*normal()
ystar = 1 + x + epsilon
lo_bound = floor(ystar)
hi_bound = ceil(ystar)

# run the interval model
intreg lo_bound hi_bound const x

# estimate ystar
gen_resid = $uhat
yhat = $yhat + gen_resid
corr ystar yhat 
\end{scode}

Output (selected portions):
\begin{outbit}
Model 1: Interval estimates using the 100 observations 1-100
Lower limit: lo_bound, Upper limit: hi_bound

             coefficient   std. error   t-ratio    p-value 
  ---------------------------------------------------------
  const       0.993762     0.0338325     29.37    1.22e-189 ***
  x           0.986662     0.0319959     30.84    8.34e-209 ***

Chi-square(1)        950.9270   p-value              8.3e-209
Log-likelihood      -44.21258   Akaike criterion     94.42517
Schwarz criterion    102.2407   Hannan-Quinn         97.58824

sigma = 0.223273
Left-unbounded observations: 0
Right-unbounded observations: 0
Bounded observations: 100
Point observations: 0

...

corr(ystar, yhat) = 0.98960092
Under the null hypothesis of no correlation:
 t(98) = 68.1071, with two-tailed p-value 0.0000
\end{outbit}
\end{script}

As with the probit and Tobit models, after a model has been estimated
the \dollar{uhat} accessor returns the generalized residual, which is
an estimate of $\epsilon_i$: more precisely, it equals
$y_i - x_i \hat{\beta}$ for point observations and
$E(\epsilon_i| m_i, M_i, x_i)$ otherwise. Note that it is possible to
compute an unbiased predictor of $y^*_i$ by summing this estimate to
$x_i \hat{\beta}$. Listing \ref{ex:interval-estimation} shows an
example. As a further similarity with Tobit, the interval regression
model may deliver inconsistent estimates if the disturbances are
non-normal; hence, the \cite{chesher-irish87} test for normality is
included by default here too.


\section{Sample selection model}
\label{sec:heckit}

In the sample selection model (also known as ``Tobit II'' model),
there are two latent variables:
%
\begin{eqnarray}
  \label{eq:heckit1}
  y^*_i & = & \sum_{j=1}^k x_{ij} \beta_j + \varepsilon_i \\
  \label{eq:heckit2}
  s^*_i & = & \sum_{j=1}^p z_{ij} \gamma_j + \eta_i 
\end{eqnarray}
%
and the observation rule is given by
%
\begin{equation}
  \label{eq:tobitII}
  y_i = \left\{ 
    \begin{array}{ll} 
      y^*_i & \mathrm{for} \quad s^*_i > 0 \\ 
      \diamondsuit & \mathrm{for} \quad s^*_i \le 0 
    \end{array}
    \right. 
\end{equation}

In this context, the $\diamondsuit$ symbol indicates that for some
observations we simply do not have data on $y$: $y_i$ may be 0, or
missing, or anything else. A dummy variable $d_i$ is normally used to
set censored observations apart.

One of the most popular applications of this model in econometrics is
a wage equation coupled with a labor force participation equation: we
only observe the wage for the employed. If $y^*_i$ and $s^*_i$ were
(conditionally) independent, there would be no reason not to use OLS
for estimating equation (\ref{eq:heckit1}); otherwise, OLS does not
yield consistent estimates of the parameters $\beta_j$.

Since conditional independence between $y^*_i$ and $s^*_i$ is
equivalent to conditional independence between $\varepsilon_i$ and
$\eta_i$, one may model the co-dependence between $\varepsilon_i$ and
$\eta_i$ as 
\[
  \varepsilon_i = \lambda \eta_i + v_i ;
\]
substituting the above expression in (\ref{eq:heckit1}), you obtain
the model that is actually estimated:
\[
  y_i = \sum_{j=1}^k x_{ij} \beta_j + \lambda \hat{\eta}_i + v_i ,
\]
so the hypothesis that censoring does not matter is equivalent to the
hypothesis $H_0: \lambda = 0$, which can be easily tested.

The parameters can be estimated via maximum likelihood under the
assumption of joint normality of $\varepsilon_i$ and $\eta_i$;
however, a widely used alternative method yields the so-called
\emph{Heckit} estimator, named after \cite{heckman79}. The procedure
can be briefly outlined as follows: first, a probit model is fit on
equation (\ref{eq:heckit2}); next, the generalized residuals are
inserted in equation (\ref{eq:heckit1}) to correct for the effect of
sample selection.

Gretl provides the \texttt{heckit} command to carry out
estimation; its syntax is
%
\begin{code}
heckit y X ; d Z
\end{code}
%
where \texttt{y} is the dependent variable, \texttt{X} is a list of
regressors, \texttt{d} is a dummy variable holding 1 for uncensored
observations and \texttt{Z} is a list of explanatory variables for the
censoring equation.

Since in most cases maximum likelihood is the method of
choice, by default gretl computes ML estimates. The 2-step
Heckit estimates can be obtained by using the \option{two-step}
option. After estimation, the \dollar{uhat} accessor contains the
generalized residuals. As in the ordinary Tobit model, the residuals
equal the difference between actual and fitted $y_i$ only for
uncensored observations (those for which $d_i = 1$).

\begin{script}[htbp]
  \scriptinfo{heckit-estimation}{Heckit model}
\begin{scode}
open mroz87.gdt

series EXP2 = AX^2
series WA2 = WA^2
series KIDS = (KL6+K618)>0

# Greene's specification

list X = const AX EXP2 WE CIT
list Z = const WA WA2 FAMINC KIDS WE

heckit WW X ; LFP Z --two-step 
heckit WW X ; LFP Z 

# Wooldridge's specification

series NWINC = FAMINC - WW*WHRS
series lww = log(WW)
list X = const WE AX EXP2
list Z = X NWINC WA KL6 K618

heckit lww X ; LFP Z --two-step 
\end{scode}
\end{script}

Listing \ref{ex:heckit-estimation} shows two estimates from the
dataset used in \cite{mroz87}: the first one replicates Table 22.7 in
\cite{greene03},\footnote{Note that the estimates given by gretl do
  not coincide with those found in the printed volume.  They do,
  however, match those found on the errata web page for Greene's book:
  \url{http://pages.stern.nyu.edu/~wgreene/Text/Errata/ERRATA5.htm}.}
while the second one replicates table 17.1 in \cite{wooldridge-panel}.

\section{Count data}
\label{sec:count}

Here the dependent variable is assumed to be a non-negative
integer---for example, the number of Nobel Prize winners in a given
country per year, the number of vehicles crossing a certain
intersection per hour, the number of bank failures per year. A
probabilistic description of such a variable must hinge on some
discrete distribution and the one most commonly employed is the
Poisson, according to which, for a random variable $Y$ and a specific
realization $y$,
\[
P(Y = y) = \frac{e^{-\lambda}\lambda^y}{y!}, \quad y=0,1,2\dots
\]
where the single parameter $\lambda$ is both the mean and the variance
of $Y$.  In an econometric context we generally want to treat
$\lambda$ as specific to the observation, $i$, and driven by
covariates $X_i$ via a parameter vector $\beta$. The standard way of
allowing for this is the exponential mean function,
\[
  \lambda_i \equiv \exp(X_i\beta)
\]
hence leading to
\[
P(Y_i = y) = \frac{\exp(-\exp(X_i\beta))(\exp(X_i\beta))^y}{y!}
\]
Given this model the log-likelihood for $n$ observations can be
written as
\[
\ell = \sum_{i=1}^n (-\exp(X_i\beta) + y_iX_i\beta - \log y_i!
\]
Maximization of this quantity is quite straightforward, and is carried
out in gretl using the syntax
\begin{code}
  poisson depvar indep
\end{code}

In some cases, an ``offset'' variable is needed: the count of
occurrences of the outcome of interest in a given time is assumed to
be strictly proportional to the offset variable $t_i$. In the
epidemiology literature, the offset is known as ``population at
risk''. In this case $\lambda$ is modeled as
\[
  \lambda_i = t_i \exp(X_i\beta)
\]
The log-likelihood is not greatly complicated thereby.  Here's another
way of thinking about the offset variable: its natural log is just
another explanatory variable whose coefficient is constrained to equal
1.

If an offset variable is needed, it should be specified at the end of
the command, separated from the list of explanatory variables by a
semicolon, as in
\begin{code}
  poisson depvar indep ; offset
\end{code}

\subsection{Overdispersion}

As mentioned above, in the Poisson model
$E(Y_i | X_i) = V(Y_i | X_i) = \lambda_i$, that is, the conditional
mean equals the conditional variance by construction. In many cases
this feature is at odds with the data; the conditional variance is
often larger than the mean, a phenomenon known as overdispersion. The
output from the \texttt{poisson} command includes a conditional moment
test for overdispersion (as per \citet{davidson-mackinnon04}, section
11.5), which is printed automatically after estimation.

Overdispersion can be attributed to unmodeled heterogeneity between
individuals. Two data points with the same observable characteristics
$X_i = X_j$ may differ because of some unobserved scale factor $s_i
\ne s_j$ so that
\[
   E(Y_i | X_i, s_i) = \lambda_i s_i \ne \lambda_j s_j = E(Y_i | X_j, s_j)
\]
even though $\lambda_i = \lambda_j$. In other words, $Y_i$ is a
Poisson random variable conditional on both $X_i$ and $s_i$, but since
$s_i$ is unobservable, the only thing we can we can use,
$P(Y_i | X_i)$, will \textit{not} conform to the Poisson distribution.

It is often assumed that $s_i$ can be represented as a gamma random
variable with mean 1 and variance $\alpha$. The parameter $\alpha$,
which measures the degree of heterogeneity between individuals, is
then estimated jointly with the vector $\beta$.

In this case, the conditional probability that $Y_i = y$ given $X_i$
can be shown to be
  \begin{equation}
    \label{eq:negbin2}
  P(Y_i = y | X_i) = 
  \frac{\Gamma(y + \alpha^{-1})}{\Gamma(\alpha^{-1})\Gamma(y + 1)}
  \left[ \frac{\lambda_i} {\lambda_i + \alpha^{-1} }\right]^{y}
  \left[ \frac{\alpha^{-1}} {\lambda_i + \alpha^{-1} }\right]^{\alpha^{-1}}
\end{equation}
which is known as the Negative Binomial Model. The conditional
mean is still $E(Y_i | X_i) = \lambda_i$, but the variance is $V(Y_i |
X_i) = \lambda_i \left( 1 + \lambda_i \alpha \right)$.

To estimate the Negative Binomial model in gretl, just substitute the
keyword \texttt{negbin} for \texttt{poisson} in the commands shown
above.

To be precise, the model \ref{eq:negbin2} is that labeled NEGBIN2 by
\cite{cameron-trivedi86}. There's also a lesser-used NEGBIN1 variant,
in which the conditional variance is a scalar multiple of the
conditional mean; that is,
$V(Y_i | X_i) = \lambda_i \left( 1 + \gamma \right)$. This can be
invoked in gretl by appending the option \option{model1} to the
\texttt{negbin} command.\footnote{The ``1'' and ``2'' in these labels
  indicate the power to which $\lambda_i$ is raised in the conditional
  variance expression.}

The two accessors \dollar{yhat} and \dollar{uhat} return the predicted
values and generalized residuals, respectively. Note that
\dollar{uhat} is \emph{not} equal to the difference between the
dependent variable and \dollar{yhat}.

\subsection{Examples}

Among the sample scripts supplied with gretl you can find
\texttt{camtriv.inp}. This exemplifies the count-data estimators
described above, based on a dataset analysed by
\cite{cameron-trivedi98}. The gretl package also contains a relevant
dataset used by \cite{mccullagh-nelder83}, namely
\texttt{mccullagh.gdt}, on which the Poisson and Negative Binomial
estimators may be tried.

\section{Duration models}
\label{sec:duration}

In some contexts we wish to apply econometric methods to measurements
of the duration of certain states. Classic examples include the
following:
\begin{itemize}
\item From engineering, the ``time to failure'' of electronic or
  mechanical components: how long do, say, computer hard drives
  last until they malfunction?
\item From the medical realm: how does a new treatment affect the
  time from diagnosis of a certain condition to exit from that
  condition (where ``exit'' might mean death or full recovery)?
\item From economics: the duration of strikes, or of spells of
  unemployment.
\end{itemize}

In each case we may be interested in how the durations are
distributed, and how they are affected by relevant covariates.  There
are several approaches to this problem; the one we discuss
here---which is currently the only one supported by gretl---is
estimation of a parametric model by means of Maximum Likelihood.  In
this approach we hypothesize that the durations follow some definite
probability law and we seek to estimate the parameters of that law,
factoring in the influence of covariates.

We may express the density of the durations as $f(t, X, \theta)$,
where $t$ is the length of time in the state in question, $X$ is a
matrix of covariates, and $\theta$ is a vector of parameters.  The
likelihood for a sample of $n$ observations indexed by $i$ is then
\[
L = \prod_{i=1}^n f(t_i, x_i, \theta)
\]

Rather than working with the density directly, however, it is standard
practice to factor $f(\cdot)$ into two components, namely a
\emph{hazard function}, $\lambda$, and a \emph{survivor function},
$S$.  The survivor function gives the probability that a state lasts
at least as long as $t$; it is therefore $1 - F(t, X, \theta)$ where
$F$ is the CDF corresponding to the density $f(\cdot)$. The hazard
function addresses this question: given that a state has persisted as
long as $t$, what is the likelihood that it ends within a short
increment of time beyond $t$---that is, it ends between $t$ and $t +
\Delta$?  Taking the limit as $\Delta$ goes to zero, we end up with
the ratio of the density to the survivor function:\footnote{For a
  fuller discussion see, for example, \cite{davidson-mackinnon04}.}
\begin{equation}
\label{eq:surv-decomp}
\lambda(t, X, \theta) = \frac{f(t, X, \theta)}{S(t, X, \theta)}
\end{equation}
so the log-likelihood can be written as
\begin{equation}
\label{eq:surv-loglik}
\ell = \sum_{i=1}^n \log f(t_i, x_i, \theta) = 
\sum_{i=1}^n \log \lambda(t_i, x_i, \theta) + 
\log S(t_i, x_i, \theta)
\end{equation}

One point of interest is the shape of the hazard function, in
particular its dependence (or not) on time since the state began.  If
$\lambda$ does not depend on $t$ we say the process in question exhibits
\emph{duration independence}: the probability of exiting the state at
any given moment neither increases nor decreases based simply on how
long the state has persisted to date. The alternatives are positive
duration dependence (the likelihood of exiting the state rises, the
longer the state has persisted) or negative duration dependence (exit
becomes less likely, the longer it has persisted).  Finally, the
behavior of the hazard with respect to time need not be monotonic;
some parameterizations allow for this possibility and some do not.

Since durations are inherently positive the probability distribution
used in modeling must respect this requirement, giving a density of
zero for $t \leq 0$.  Four common candidates are the exponential,
Weibull, log-logistic and log-normal, the Weibull being the most
common choice. The table below displays the density and the hazard
function for each of these distributions as they are commonly
parameterized, written as functions of $t$ alone. ($\phi$ and $\Phi$
denote, respectively, the Gaussian PDF and CDF.)

\begin{center}
\setlength\tabcolsep{1.5em}
\begin{tabular}{lll}
 & density, $f(t)$ & hazard, $\lambda(t)$ \\ [1ex]
Exponential & $\displaystyle 
\gamma \exp\,(-\gamma t)$ &$\displaystyle
\gamma$ \\ [1ex]
Weibull & $\displaystyle
\alpha\gamma^{\alpha}t^{\alpha-1}\exp\left[-(\gamma t)^\alpha\right]$ 
& $\displaystyle \alpha\gamma^{\alpha}t^{\alpha-1}$ \\ [1ex]
Log-logistic & $\displaystyle \gamma\alpha
\frac{(\gamma t)^{\alpha-1}}
{\left[1 + (\gamma t)^\alpha\right]^2}$ 
& $\displaystyle \gamma\alpha
\frac{(\gamma t)^{\alpha-1}}
{\left[1 + (\gamma t)^\alpha\right]}$ \\ [2.5ex]
Log-normal & $\displaystyle
\frac{1}{\sigma t} \phi\left[(\log t - \mu)/\sigma \right]$ 
& $\displaystyle
\frac{1}{\sigma t} \frac{\phi\left[(\log t - \mu)/\sigma \right]}
{\Phi\left[-(\log t - \mu)/\sigma \right]}$
\end{tabular}
\end{center}

The hazard is constant for the exponential distribution.  For the
Weibull, it is monotone increasing in $t$ if $\alpha > 1$, or
monotone decreasing for $\alpha < 1$. (If $\alpha = 1$ the Weibull
collapses to the exponential.)  The log-logistic and log-normal
distributions allow the hazard to vary with $t$ in a non-monotonic
fashion.

Covariates are brought into the picture by allowing them to govern one
of the parameters of the density, so that durations are not
identically distributed across cases.  For example, when using the
log-normal distribution it is natural to make $\mu$, the expected
value of $\log t$, depend on the covariates, $X$.  This is typically
done via a linear index function: $\mu = X\beta$.

Note that the expressions for the log-normal density and hazard
contain the term $(\log t - \mu)/\sigma$.  Replacing $\mu$ with
$X\beta$ this becomes $(\log t - X\beta)/\sigma$.  As in
\cite{kalbfleisch02}, we define a shorthand label for this
term:
\begin{equation}
\label{eq:duration-w}
w_i \equiv (\log t_i - x_i\beta)/\sigma
\end{equation}
It turns out that this constitutes a useful simplifying change of
variables for all of the distributions discussed here.  The
interpretation of the scale factor, $\sigma$, in the expression above
depends on the distribution. For the log-normal, $\sigma$ represents
the standard deviation of $\log t$; for the Weibull and the
log-logistic it corresponds to $1/\alpha$; and for the exponential it
is fixed at unity. For distributions other than the log-normal,
$X\beta$ corresponds to $-\log \gamma$, or in other words
$\gamma = \exp(-X\beta)$.

With this change of variables, the density and survivor functions may
be written compactly as follows (the exponential is the same as the
Weibull).

\begin{center}
\setlength\tabcolsep{1.5em}
\begin{tabular}{lll}
 & density, $f(w_i)$ & survivor, $S(w_i)$ \\ [4pt]
Weibull & 
$\exp\left(w_i - e^{w_i}\right)$ & $\exp(-e^{w_i})$
\\ [4pt]
Log-logistic & 
$e^{w_i} \left(1 + e^{w_i}\right)^{-2}$ 
& $\left(1 + e^{w_i}\right)^{-1}$ \\ [1ex]
Log-normal & $\phi(w_i)$ & $\Phi(-w_i)$
\end{tabular}
\end{center}

In light of the above we may think of the generic parameter vector
$\theta$, as in $f(t, X, \theta)$, as composed of the coefficients on
the covariates, $\beta$, plus (in all cases but the exponential) the
additional parameter $\sigma$.

A complication in estimation of $\theta$ is posed by ``incomplete
spells''. That is, in some cases the state in question may not have
ended at the time the observation is made (e.g.\ some workers remain
unemployed, some components have not yet failed).  If we use $t_i$ to
denote the time from entering the state to either (a) exiting the
state or (b) the observation window closing, whichever comes first,
then all we know of the ``right-censored'' cases (b) is that the
duration was at least as long as $t_i$. This can be handled by
rewriting the the log-likelihood (compare \ref{eq:surv-loglik}) as
\begin{equation}
\label{eq:surv-loglik2}
\ell_i = \sum_{i=1}^n \delta_i\log S\left(w_i\right)
+ \left(1-\delta_i\right) 
\left[-\log\sigma + \log f\left(w_i\right)\right]
\end{equation}
where $\delta_i$ equals 1 for censored cases (incomplete spells), and
0 for complete observations. The rationale for this is that the
log-density equals the sum of the log hazard and the log survivor
function, but for the incomplete spells only the survivor function
contributes to the likelihood. So in (\ref{eq:surv-loglik2}) we are
adding up the log survivor function alone for the incomplete cases,
plus the full log density for the completed cases.

\subsection{Implementation in gretl and illustration}

The \texttt{duration} command accepts a list of series on the usual
pattern: dependent variable followed by covariates. If right-censoring
is present in the data this should be represented by a dummy variable
corresponding to $\delta_i$ above, separated from the covariates by
a semicolon. For example,
\begin{code}
duration durat 0 X ; cens
\end{code}
where \texttt{durat} measures durations, \texttt{0} represents the
constant (which is required for such models), \texttt{X} is a named
list of regressors, and \texttt{cens} is the censoring dummy.

By default the Weibull distribution is used; you can substitute any of
the other three distributions discussed here by appending one of the
option flags \option{exponential}, \option{loglogistic} or 
\option{lognormal}.

Interpreting the coefficients in a duration model requires some care,
and we will work through an illustrative case. The example comes from
section 20.3 of \cite{wooldridge-panel} and it concerns criminal
recidivism.\footnote{Germ\'an Rodr\'iguez of Princeton University has a
  page discussing this example and displaying estimates from
  \app{Stata} at \url{http://data.princeton.edu/pop509/recid1.html}.}
The data (filename \texttt{recid.gdt}) pertain to a sample of 1,445
convicts released from prison between July 1, 1977 and June 30,
1978. The dependent variable is the time in months until they are
again arrested. The information was gathered retrospectively by
examining records in April 1984; the maximum possible length of
observation is 81 months.  Right-censoring is important: when the date
were compiled about 62 percent had not been rearrested.  The dataset
contains several covariates, which are described in the data file; we
will focus below on interpretation of the \texttt{married} variable, a
dummy which equals 1 if the respondent was married when imprisoned.

Listing~\ref{ex:duration} shows the gretl commands for Weibull and
log-normal models along with most of the output.  Consider first the
Weibull scale factor, $\sigma$. The estimate is 1.241 with a standard
error of 0.048.  (We don't print a $z$ score and $p$-value for this
term since $H_0: \sigma = 0$ is not of interest.)  Recall that
$\sigma$ corresponds to $1/\alpha$; we can be confident that $\alpha$
is less than 1, so recidivism displays negative duration dependence.
This makes sense: it is plausible that if a past offender manages to
stay out of trouble for an extended period his risk of engaging in
crime again diminishes. (The exponential model would therefore not be
appropriate in this case.)

\begin{script}[htbp]
  \scriptinfo{duration}{Models for recidivism data}
\begin{scode}
open recid.gdt
list X = workprg priors tserved felon alcohol drugs \
 black married educ age
duration durat 0 X ; cens
duration durat 0 X ; cens --lognormal
\end{scode}

Partial output:
\begin{outbit}
Model 1: Duration (Weibull), using observations 1-1445
Dependent variable: durat

             coefficient   std. error      z      p-value 
  --------------------------------------------------------
  const       4.22167      0.341311      12.37    3.85e-35 ***
  workprg    -0.112785     0.112535      -1.002   0.3162  
  priors     -0.110176     0.0170675     -6.455   1.08e-10 ***
  tserved    -0.0168297    0.00213029    -7.900   2.78e-15 ***
  felon       0.371623     0.131995       2.815   0.0049   ***
  alcohol    -0.555132     0.132243      -4.198   2.69e-05 ***
  drugs      -0.349265     0.121880      -2.866   0.0042   ***
  black      -0.563016     0.110817      -5.081   3.76e-07 ***
  married     0.188104     0.135752       1.386   0.1659  
  educ        0.0289111    0.0241153      1.199   0.2306  
  age         0.00462188   0.000664820    6.952   3.60e-12 ***
  sigma       1.24090      0.0482896                      

Chi-square(10)       165.4772   p-value              2.39e-30
Log-likelihood      -1633.032   Akaike criterion     3290.065

Model 2: Duration (log-normal), using observations 1-1445
Dependent variable: durat

             coefficient   std. error       z      p-value 
  ---------------------------------------------------------
  const       4.09939      0.347535      11.80     4.11e-32 ***
  workprg    -0.0625693    0.120037      -0.5213   0.6022  
  priors     -0.137253     0.0214587     -6.396    1.59e-10 ***
  tserved    -0.0193306    0.00297792    -6.491    8.51e-11 ***
  felon       0.443995     0.145087       3.060    0.0022   ***
  alcohol    -0.634909     0.144217      -4.402    1.07e-05 ***
  drugs      -0.298159     0.132736      -2.246    0.0247   **
  black      -0.542719     0.117443      -4.621    3.82e-06 ***
  married     0.340682     0.139843       2.436    0.0148   **
  educ        0.0229194    0.0253974      0.9024   0.3668  
  age         0.00391028   0.000606205    6.450    1.12e-10 ***
  sigma       1.81047      0.0623022                       

Chi-square(10)       166.7361   p-value              1.31e-30
Log-likelihood      -1597.059   Akaike criterion     3218.118
\end{outbit}
\end{script}

On a priori grounds, however, we may doubt the monotonic decline in
hazard that is implied by the Weibull specification. Even if a person
is liable to return to crime, it seems relatively unlikely that he
would do so straight out of prison. In the data, we find that only 2.6
percent of those followed were rearrested within 3 months. The
log-normal specification, which allows the hazard to rise and then
fall, may be more appropriate.  Using the \texttt{duration} command
again with the same covariates but the \option{lognormal} flag, we get
a log-likelihood of $-1597$ as against $-1633$ for the Weibull,
confirming that the log-normal gives a better fit.

Let us now focus on the \texttt{married} coefficient, which is
positive in both specifications but larger and more sharply estimated
in the log-normal variant. The first thing is to get the
interpretation of the sign right.  Recall that $X\beta$ enters
negatively into the intermediate variable $w$
(equation~\ref{eq:duration-w}). The Weibull hazard is
$\lambda(w_i) = e^{w_i}$, so being married reduces the hazard of
re-offending, or in other words lengthens the expected duration out of
prison.  The same qualitative interpretation applies for the
log-normal.

To get a better sense of the married effect, it is useful to show its
impact on the hazard across time. We can do this by plotting the
hazard for two values of the index function $X\beta$: in each case the
values of all the covariates other than \texttt{married} are set to
their means (or some chosen values) while \texttt{married} is set
first to 0 then to 1. Listing~\ref{ex:hazard-plots} provides a script
that does this, and the resulting plots are shown in
Figure~\ref{fig:hazard-plots}. Note that when computing the hazards we
need to multiply by the Jacobian of the transformation from $t_i$ to
$w_i = \log (t_i - x_i\beta)/\sigma$, namely $1/t$.  Note also that
the estimate of $\sigma$ is available via the accessor \dollar{sigma},
but it is also present as the last element in the coefficient vector
obtained via \dollar{coeff}.

A further difference between the Weibull and log-normal specifications
is illustrated in the plots. The Weibull is an instance of a
\emph{proportional hazard} model. This means that for any sets of
values of the covariates, $x_i$ and $x_j$, the ratio of the associated
hazards is invariant with respect to duration. In this example the
Weibull hazard for unmarried individuals is always 1.1637 times that
for married. In the log-normal variant, on the other hand, this ratio
gradually declines from 1.6703 at one month to 1.1766 at 100 months.


\begin{script}[htbp]
  \scriptinfo{hazard-plots}{Create plots showing conditional hazards}
\begin{scode}
open recid.gdt -q

# leave 'married' separate for analysis
list X = workprg priors tserved felon alcohol drugs \
 black educ age

# Weibull variant
duration durat 0 X married ; cens
# coefficients on all Xs apart from married
matrix beta_w = $coeff[1:$ncoeff-2]
# married coefficient
scalar mc_w = $coeff[$ncoeff-1]
scalar s_w = $sigma

# Log-normal variant
duration durat 0 X married ; cens --lognormal
matrix beta_n = $coeff[1:$ncoeff-2]
scalar mc_n = $coeff[$ncoeff-1]
scalar s_n = $sigma

list allX = 0 X
# evaluate X\beta at means of all variables except marriage
scalar Xb_w = meanc({allX}) * beta_w
scalar Xb_n = meanc({allX}) * beta_n

# construct two plot matrices
matrix mat_w = zeros(100, 3)
matrix mat_n = zeros(100, 3)

loop t=1..100
  # first column, duration
  mat_w[t, 1] = t
  mat_n[t, 1] = t
  wi_w = (log(t) - Xb_w)/s_w
  wi_n = (log(t) - Xb_n)/s_n
  # second col: hazard with married = 0
  mat_w[t, 2] = (1/t) * exp(wi_w)
  mat_n[t, 2] = (1/t) * pdf(z, wi_n) / cdf(z, -wi_n)
  wi_w = (log(t) - (Xb_w + mc_w))/s_w
  wi_n = (log(t) - (Xb_n + mc_n))/s_n
  # third col: hazard with married = 1
  mat_w[t, 3] = (1/t) * exp(wi_w)
  mat_n[t, 3] = (1/t) * pdf(z, wi_n) / cdf(z, -wi_n)
endloop

cnameset(mat_w, "months unmarried married")
cnameset(mat_n, "months unmarried married")

gnuplot 2 3 1 --with-lines --supp --matrix=mat_w --output=weibull.plt
gnuplot 2 3 1 --with-lines --supp --matrix=mat_n --output=lognorm.plt
\end{scode}
\end{script}

\begin{figure}[htbp]
\centering
\includegraphics[scale=0.85]{figures/weibull}
\includegraphics[scale=0.85]{figures/lognorm}
\caption{Recidivism hazard estimates for married and unmarried
  ex-convicts}
\label{fig:hazard-plots}
\end{figure}

\subsection{Alternative representations of the Weibull model}

One point to watch out for with the Weibull duration model is that the
estimates may be represented in different ways.  The representation
given by gretl is sometimes called the \textit{accelerated
  failure-time} (AFT) metric. An alternative that one sometimes sees
is the \textit{log relative-hazard} metric; in fact this is the metric
used in Wooldridge's presentation of the recidivism example.  To get
from AFT estimates to log relative-hazard form it is necessary to
multiply the coefficients by $-\sigma^{-1}$. For example, the
\texttt{married} coefficient in the Weibull specification as shown
here is 0.188104 and $\hat{\sigma}$ is 1.24090, so the alternative
value is $-0.152$, which is what Wooldridge shows
(\citeyear{wooldridge-panel}, Table 20.1).

\subsection{Fitted values and residuals}

By default, gretl computes fitted values (accessible via
\dollar{yhat}) as the conditional mean of duration.  The formulae
are shown below (where $\Gamma$ denotes the gamma function, and the
exponential variant is just Weibull with $\sigma = 1$).

\begin{center}
\setlength\tabcolsep{1em}
\begin{tabular}{ccc}
Weibull & Log-logistic & Log-normal \\ [4pt]
$\exp(X\beta)\Gamma(1 + \sigma)$ &
$\displaystyle \exp(X\beta)\frac{\pi \sigma}{\sin(\pi \sigma)}$ &
$\exp(X\beta + \sigma^2/2)$
\end{tabular}
\end{center}

The expression given for the log-logistic mean, however, is valid only
for $\sigma < 1$; otherwise the expectation is undefined, a point that
is not noted in all software.\footnote{The \texttt{predict} adjunct to
  the \texttt{streg} command in \app{Stata} 10, for example, gaily
  produces large negative values for the log-logistic mean in duration
  models with $\sigma > 1$.}

Alternatively, if the \option{medians} option is given, gretl's
\texttt{duration} command will produce conditional medians as the
content of \dollar{yhat}.  For the Weibull the median is
$\exp(X\beta)(\log 2)^\sigma$; for the log-logistic and log-normal it
is just $\exp(X\beta)$.

The values we give for the accessor \dollar{uhat} are generalized
(Cox--Snell) residuals, computed as the integrated hazard function,
which equals the negative log of the survivor function:
\[
\epsilon_i = \Lambda(t_i, x_i, \theta) = -\log S(t_i, x_i, \theta)
\]
Under the null of correct specification of the model these generalized
residuals should follow the unit exponential distribution, which has
mean and variance both equal to 1 and density $\exp(-\epsilon)$. See
chapter 18 of \cite{cameron-trivedi05} for further discussion.

%%% Local Variables: 
%%% mode: latex
%%% TeX-master: "gretl-guide"
%%% End: 
