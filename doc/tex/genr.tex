\chapter{Special functions in genr}
\label{chap:genr}

\section{Introduction}
\label{genr-intro}

The \cmd{genr} command provides a flexible means of defining new
variables.  It is documented in the \GCR.  This chapter offers a more
expansive discussion of some of the special functions available via
\verb+genr+ and some of the finer points of the command.


\section{Cumulative densities and p-values}
\label{sec:genr-cdf}

The two functions \cmd{cdf} and \cmd{pvalue} provide complementary
means of examining values from several probability distributions: the
standard normal, Student's $t$, $\chi^2$, $F$, gamma, and binomial.
The syntax of these functions is set out in the \GCR; here we expand
on some subtleties.

The cumulative density function or CDF for a random variable
is the integral of the variable's density from its lower limit
(typically either $-\infty$ or 0) to any specified value $x$.  The
p-value (at least the one-tailed, right-hand p-value as returned by
the \cmd{pvalue} function) is the complementary probability, the
integral from $x$ to the upper limit of the distribution, typically
$+\infty$.

In principle, therefore, there is no need for two distinct functions:
given a CDF value $p_0$ you could easily find the corresponding
p-value as $1-p_0$ (or vice versa).  In practice, with
finite-precision computer arithmetic, the two functions are not
redundant.  This requires a little explanation.  In gretl, as in
most statistical programs, floating point numbers are represented as
``doubles'' --- double-precision values that typically have a storage
size of eight bytes or 64 bits.  Since there are only so many bits
available, only so many floating-point numbers can be represented:
\textit{doubles do not model the real line}.  Typically doubles can
represent numbers over the range (roughly) $\pm 1.7977 \times
10^{308}$, but only to about 15 digits of precision.

Suppose you're interested in the left tail of the $\chi^2$ distribution
with 50 degrees of freedom: you'd like to know the CDF value for $x =
0.9$.  Take a look at the following interactive session:
\begin{code}
? scalar p1 = cdf(X, 50, 0.9)
Generated scalar p1 = 8.94977e-35
? scalar p2 = pvalue(X, 50, 0.9)
Generated scalar p2 = 1
? scalar test = 1 - p2
Generated scalar test = 0
\end{code}

The \cmd{cdf} function has produced an accurate value, but the
\cmd{pvalue} function gives an answer of 1, from which it is not
possible to retrieve the answer to the CDF question.  This may seem
surprising at first, but consider: if the value of \texttt{p1} above
is correct, then the correct value for \texttt{p2} is $1 - 8.94977
\times 10^{-35}$.  But there's no way that value can be represented as
a double: that would require over 30 digits of precision.

Of course this is an extreme example.  If the $x$ in question is not
too far off into one or other tail of the distribution, the \cmd{cdf}
and \cmd{pvalue} functions will in fact produce complementary
answers, as shown below:
\begin{code}
? scalar p1 = cdf(X, 50, 30)
Generated scalar p1 = 0.0111648
? scalar p2 = pvalue(X, 50, 30)
Generated scalar p2 = 0.988835
? scalar test = 1 - p2
Generated scalar test = 0.0111648
\end{code}
But the moral is that if you want to examine extreme values
you should be careful in selecting the function you need, in the
knowledge that values very close to zero can be represented as doubles
while values very close to 1 cannot.


\section{Retrieving internal variables}
\label{sec:genr-internal}

The \cmd{genr} command provides a means of retrieving various values
calculated by the program in the course of estimating models or
testing hypotheses.  The variables that can be retrieved in this way
are listed in the \GCR; here we say a bit more about the special
variables \dollar{test} and \dollar{pvalue}.

These variables hold, respectively, the value of the last test
statistic calculated using an explicit testing command and the p-value
for that test statistic.  If no such test has been performed at the
time when these variables are referenced, they will produce the
missing value code.  The ``explicit testing commands'' that work in
this way are as follows: \cmd{add} (joint test for the significance of
variables added to a model); \cmd{adf} (Augmented Dickey--Fuller test,
see below); \cmd{arch} (test for ARCH); \cmd{chow} (Chow test for a
structural break); \cmd{coeffsum} (test for the sum of specified
coefficients); \cmd{cusum} (the Harvey--Collier $t$-statistic);
\cmd{kpss} (KPSS stationarity test, no p-value available);
\cmd{modtest} (see below); \cmd{meantest} (test for difference of
means); \cmd{omit} (joint test for the significance of variables
omitted from a model); \cmd{reset} (Ramsey's RESET); \cmd{restrict}
(general linear restriction); \cmd{runs} (runs test for randomness);
and \cmd{vartest} (test for difference of variances). In most cases
both a \dollar{test} and a \dollar{pvalue} are stored; the exception
is the KPSS test, for which a p-value is not currently available.

An important point to notice about this mechanism is that the internal
variables \dollar{test} and \dollar{pvalue} are over-written each time
one of the tests listed above is performed.  If you want to reference
these values, you must do so at the correct point in the sequence of
gretl commands.

\begin{itemize}
\item By default, the \cmd{adf} command generates three variants of
  the Dickey--Fuller test: one based on a regression including a
  constant, one using a constant and linear trend, and one using a
  constant and a quadratic trend.  When you wish to reference
  \dollar{test} or \dollar{pvalue} in connection with this command, you
  can control the variant that is recorded by using one of the flags
  \option{nc}, \option{c}, \option{ct} or \option{ctt} with
  \verb+adf+.
\item The \cmd{modtest} command (which must follow an estimation
  command) offers several diagnostic tests; the particular test
  performed depends on the option flag provided. Please see the \GCR\
  and chapter~\ref{chap:var} of this \textit{Guide} for details.
\end{itemize}


%%% Local Variables:
%%% mode: latex
%%% TeX-master: "gretl-guide"
%%% End:
