\documentclass[oneside]{book}
\usepackage{url,verbatim,fancyvrb}
\usepackage{pifont}
\usepackage[latin1]{inputenc}
\usepackage[pdftex]{graphicx}
%\usepackage[authoryear]{natbib}
\usepackage{color,gretl}
\usepackage[letterpaper,body={6.3in,9.15in},top=.8in,left=1.1in]{geometry}
\usepackage[pdftex,hyperfootnotes=false]{hyperref}
\usepackage{dcolumn,amsmath,bm}

%% \pdfimageresolution=120
\hypersetup{pdftitle={A Hansl Primer},
            pdfsubject={The scripting language of gretl},
            pdfauthor={Riccardo (Jack) Lucchetti},
            colorlinks=true,
            linkcolor=blue,
            urlcolor=red,
            citecolor=steel,
            bookmarks=true,
            bookmarksnumbered=true,
            plainpages=false
}

\begin{document}

\VerbatimFootnotes

\setlength{\parindent}{0pt}
\setlength{\parskip}{1ex}
\setcounter{tocdepth}{1}

%% titlepage

\thispagestyle{empty}

\begin{center}
\pdfbookmark[1]{A Hansl Primer}{titlepage}

\htitle{A Hansl Primer}

\gsubtitle{The scripting language of gretl in \pageref{LastPage} minutes}

{\large \sffamily
Allin Cottrell\\
Department of Economics\\
Wake Forest University\\

\vspace{20pt}
Riccardo (Jack) Lucchetti\\
Dipartimento di Scienze Economiche e Sociali\\
Universit� Politecnica delle Marche\\

\vspace{20pt}
\input date
}

\end{center}
\clearpage

%% end titlepage, start license page

\thispagestyle{empty}

\pdfbookmark[1]{License}{license}

\vspace*{2in}

Permission is granted to copy, distribute and/or modify this document
under the terms of the \emph{GNU Free Documentation License}, Version
1.1 or any later version published by the Free Software Foundation
(see \url{http://www.gnu.org/licenses/fdl.html}).

\cleardoublepage

%% end license page, start table of contents
\pdfbookmark[1]{Table of contents}{contents}

\pagenumbering{roman}
\pagestyle{headings}

\tableofcontents

\clearpage
\pagenumbering{arabic}
%\setcounter{chapter}{-1}

\chapter{Introduction}
%\addcontentsline{toc}{chapter}{Introduction}

\section*{What hansl is and what it is not}

Hansl is a recursive acronym: it stands for ``Hansl's A Neat Scripting
Language''. You might therefore expect something very general in
purpose. Not really.  Hansl was born as the scripting language for the
econometrics program \app{gretl} and its role is unlikely to change.
As a consequence, hansl should not be viewed as a fully fledged
programming language such as C, Fortran, Perl or Python. Not because
it lacks anything to be considered as such,\footnote{Hansl is in fact
  Turing-complete.} but because its aim is different. Hansl should be
considered as a special-purpose or domain-specific language, designed
to make an econometrician's life easier. Hence it incorporates a
series of conventions and choices that may irritate purists and have
some marginal impact on raw performance, but that we, as professional
econometricians, consider ``nice to have''.  This makes hansl somewhat
different from plain matrix-oriented interpreted languages, such as
the Matlab/Octave family, Ox and so on.

On the other hand, hansl is by no means just a tool to automate rote
tasks. It has several features which support advanced work: structured
programming, recursion, complex data structures, and so on.  As for
style, the language which hansl most resembles is probably that of the
bash shell.

\section*{The intent and structure of this document}

The intended readers of this document are those who already know how
to write code, and are familiar with the associated do-s and don't-s.
Such people may wish to add hansl to their toolbox, alongside
languages like C or FORTRAN, or programs such as R, Ox, Matlab, Stata
or Gauss, some of which they are already confident with. Here,
therefore, the focus is not so much on ``How do I do this?'', but
rather on ``How do I do this \emph{in hansl}?''.

As a consequence, this document aims as making the reader a reasonably
proficient hansl user in a (relatively) short time; however, not all
the features of hansl are illustrated; for those, interested readers
should consult the \GCR{} and \GUG{}.

This guide is divided into two parts. Part~\ref{part:hp-nodata}
(``Without a dataset'') concentrates on hansl as a pure
matrix-oriented programming language. Part~\ref{part:hp-data} (``With
a dataset'') exploits the fact that hansl scripts are run through
gretl, which has very nice facilities for handling statistical
datasets (interactively if necessary). This provides hansl with a
series of extra constructs and features which make it extremely easy
to write hansl scripts to perform all sorts of statistical procedures.

In order to use hansl, you will need a working installation of
gretl. We assume you have one. If you don't, please refer to chapter 1
of \GUG.

\section*{Other resources}

If you are serious about learning hansl then after working through
this primer---or in the process of doing so---you'll want to take a
look at the following additional resources.
\begin{itemize}
\item The \GCR. This contains a complete listing of the commands and
  built-in functions available in hansl, with a full account of their
  syntax and options. Examples of usage are provided in some
  instances. This is available in an ``online'' version for handy
  reference as well as in PDF, both accessible via the \textsf{Help} menu
  in the gretl GUI.
\item The \textit{Gretl User's Guide}. Chapters 10 to 16, in
  particular, go into more detail on some of the programming topics
  discussed here (data types, loops, the definition and use of
  functions). In addition Part II of the \textit{Guide}, on
  Econometric Methods, gives many examples of hansl usage. The
  \textit{Guide} is available via gretl's \textsf{Help} menu; the
  latest version can also be found online at
  \url{http://sourceforge.net/projects/gretl/files/manual/}.
\item Sample scripts. The gretl package comes with a large number of
  sample or practice scripts, which can be found under the menu item
  \textsf{/File/Script files/Practice file}. Many of these are simple
  replication exercises for textbook problems but you will find some
  more interesting examples under the \textsf{Gretl} tab.
\item Function packages. Relatively ambitious examples of hansl coding
  can be found in the various contributed ``function packages''. You
  can access these packages via the gretl menu item
  \textsf{/File/Function files/On server}. Right-click on a package
  and select \textsf{View code} to examine the hansl functions.
\item The gretl-users mailing list. Most well-considered
  questions get answered quite quickly and fully. See
  \url{http://lists.wfu.edu/mailman/listinfo/gretl-users}.
\end{itemize}


\part{Without a dataset}
\label{part:hp-nodata}
\chapter{Hello, world!}
\label{chap:hello}

Начнем с проверенной временем программы «Hello, world» ---
обязательного первого шага в любом языке программирования. На самом
деле в Hansl все очень просто:
\begin{code}
  # Первый пример
  print "Hello, world!"
\end{code}

Есть несколько способов запустить приведенный выше сценарий: вы можете
поместить его в текстовый файл \texttt{first\_ex.inp} и пусть gretl
выполнит его прямо из командной строки с помощью команды
\begin{code}
  gretlcli -b first_ex.inp
\end{code}
или же вы можете просто скопировать содержимое скрипта в окно
редактора сеанса gretl с графическим интерфейсом и щелкнуть по значку
«шестеренки». Решать вам: используйте тот способ, который вам больше
нравится.

С синтаксической точки зрения позвольте обратить ваше внимание на
следующие моменты:
\begin{enumerate}
\item Строка, начинающаяся с решетки (\texttt{\#}), является
  комментарием: если встречается знак решетки, все от этой точки до
  конца текущей строки рассматривается как комментарий и игнорируется
  программой.
\item Следующая строка содержит команду \emph{command} (\cmd{print})
  (в данном случае: «вывод»), за которой следует аргумент; это
  довольно типично для Hansl: многие работы выполняются с помощью
  различных команд.
\item У Hansl нет явного указателя конца команды, такого как
  ``\texttt{;}'' символ в семействе языков C (C++, Java, C\#, \ldots)
  или GAUSS; вместо этого он использует символ новой строки как
  неявный терминатор. Итак, в конце команды вы \emph{должны} вставить
  новую строку; и наоборот, \emph{нельзя} вставлять новую строку в
  середине команды. То есть можно НЕ без специальных мер: если вам
  нужно разбить команду на более чем одну строку для удобства чтения,
  вы можете использовать символ ``\textbackslash'' (обратная косая
  черта), который заставляет gretl игнорировать следующий разрыв
  строки.
 \end{enumerate}

 Также обратите внимание, что команда печати \cmd{print} автоматически
 добавляет разрыв строки и не распознает ``escape''
 последовательности, такие как ``\verb|\n|''; такие последовательности
 выводятся буквально. Команда printf может использоваться для большего
 контроля над выводом; см. главу \ref{chap:formatting}.

Давайте теперь рассмотрим простой вариант этого же сценария:
\begin{code}
  /*
    Второй пример
  */
  string foo = "Hello, world"
  print foo
\end{code}

В этом примере наш первый сценарий написан с использованием условия,
принятого в языке программирования C: все, что находится между
``\verb|/*|'' и ``\verb|*/|'', игнорируется \footnote{Каждый тип
  комментария может быть перекрыт другим:
	\begin{itemize}
        \item Если \texttt{/*} следует за \texttt{\#} в данной строке,
          которая еще не прочтена программой в режиме игнорирования,
          то в \texttt{/*} нет ничего особенного, это просто часть
          комментария в стиле \texttt{\#}.
        \item Если \texttt{\#} появляется, когда мы уже находимся в
          режиме комментариев, это просто часть комментария.
\end{itemize}}. Комментарии этого типа не поддерживаются.

Далее у нас есть строка
\begin{code}
  string foo = "Hello, world"
\end{code}
В этой строке мы присваиваем значение ``\texttt{Hello, world}''
переменной с именем \texttt{foo}. Обратите внимание на то, что
\begin{enumerate}
\item оператором присваивания является знак равенства (\texttt{=}).
\item имя переменной (ее идентификатор) должно соответствовать
  следующим условиям: идентификаторы не могут быть длиннее 31 символа
  и должны быть в формате ASCII. Они также должны начинаться с буквы,
  и могут содержать только буквы, цифры и нижнее
  подчеркивание.\footnote{Фактически поддерживается лишь одно
    исключение из этого правила: идентификаторы могут содержать одну
    греческую букву. Подробнее в гл.~\ref{chap:greeks}.}
  Идентификаторы в Hansl чувствительны к регистру, поэтому
  \texttt{foo}, \texttt{Foo} и \texttt{FOO} --- это три разных
  имени. Конечно, некоторые слова уже зарезервированы и поэтому не
  могут использоваться в качестве идентификаторов (однако почти все
  зарезервированные слова содержат только строчные буквы и символы).
\item Разделителем строки является двойная кавычка (\verb|"|). 
\end{enumerate}

В Hansl переменная должна быть одного из следующих типов:
\texttt{scalar}, \texttt{series}, \texttt{matrix}, \texttt{list},
\texttt{string}, \texttt{bundle} or \texttt{array}. Как мы только что
определили, строковые переменные используются для хранения
последовательностей буквенно-цифровых символов. Остальные мы будем
вводить постепенно; например, следующая глава будет посвещена
матрицам.

Читатель мог заметить, что строка 
\begin{code}
  string foo = "Hello, world"
\end{code}
неявно выполняет две задачи: объявляет \texttt{foo} как переменную
типа \texttt{string} и одновременно присваивает значение
\texttt{foo}. Первый компонент строго не требуется. В большинстве
случаев Gretl может самостоятельно определить тип данных для вновь
введенной переменной, и строка \verb|foo = "Hello, world"| (без
спецификатора типа) работает правильно.  Однако пользователям, во всем
любящих порядок (который, как известно, приводит к более разборчивому
и удобному в коду), рекомендовано использовать спецификатор типа по
крайней мере первый раз, когда вводится новая переменная.
  
В следующем примере мы будем использовать переменную \texttt{scalar}
скалярного типа:
\begin{code}
  scalar x = 42
  print x
\end{code}
Скаляр \texttt{scalar} --- это число с плавающей запятой двойной
точности, поэтому \texttt{42} совпадает с \texttt{42.0} или
\texttt{4.20000E+01}. Обратите внимание, что в Hansl нет специального
типа переменной для целых или комплексных чисел.

Следует объяснить важную деталь: в отличие от большинства других
матрично-ориентированных языков, используемых в сообществах
эконометристов, Hansl строго типизирован. Это означает, что нельзя
присвоить значение одного типа для переменной, которая уже была
объявлена как переменная другого типа. Например, возникнет следующая
ошибка:
\begin{code}
  string a = "zoo"
  a = 3.14 # нет-нет!
\end{code}
Если вы попытаетесь запустить приведенный выше пример, вы получите
данную ошибку. Однако допустимо уничтожить исходную переменную с
помощью команды удаления \cmd{delete}, а затем повторно присвоить ее
другому типу:
\begin{code}
  scalar X = 3.1415
  delete X
  string X = "apple pie"
\end{code}

Здесь нет ``type-casting'' как в C, но возможны некоторые
автоматические преобразования типов (подробнее об этом поговорим
позже).

Многие команды могут содержать более одного аргумента, как в:
\begin{code}
  set echo off
  set messages off

  scalar x = 42
  string foo = "not bad"
  print x foo 
\end{code}
В этом примере одна функция вывода \texttt{print} используется для
демонстрации значений двух переменных; в более общем виде за словом
\texttt{print} может следовать любое количество аргументов. Другое
отличие от предыдущего примера состоит в том, что здесь используются
две команды set. Подробное описание команды \texttt{set} отвлечет нас
от темы главы на долгое время; достаточно сказать, что эта команда
используется для установки значения различных «переменных состояния»,
влияющих на поведение программы; здесь она используется для того,
чтобы заглушить нежелательный вывод описания действия. См. Справочник
по командам Gretl для получения дополнительной информации о
команде\texttt{set}.

% There was a reference here to a {chap:settings}, but it has not
% been written at this point

Команда \cmd{eval} полезна, когда вы хотите увидеть результат
выражения без присваивания какого-либо имени или типа данной
переменной. Следующая команда
\begin{code}
  eval 2+3*4
\end{code}
выведет на экран число 14. Эта команда наиболее полезна при запуске
gretl как калькулятора, но ее можно использовать и в сценарии Hansl
для проверки выражений, как в следующем (довольно упрощенном) примере:
\begin{code}
  scalar a = 1
  scalar b = -1
  # это должно быть 0
  eval a+b
\end{code}

\section{Действия со скалярными величинами}

Алгебраические операции работают очевидным образом, а классические
алгебраические операторы имеют свои традиционные правила приоритета:
знак вставки (\verb|^|) используется для возведения в
степень. Например,
\begin{code}
  scalar phi = exp(-0.5 * (x-m)^2 / s2) / sqrt(2 * $pi * s2)
\end{code}

здесь мы предполагаем, что\texttt{x}, \texttt{m} и \texttt{s2} --- уже
существующие скалярные величины. Данный пример содержит два
примечательных момента:
\begin{itemize}
\item Использование функций \cmd{exp} (экспонента) и \cmd{sqrt}
  (кв.корень) само собой разумеется: Hansl обладает достаточно широким
  набором таких функций. См. Справочник по командам для полного списка
  функций.
\item Использование \verb|$pi| для постоянного $\pi$. Хотя
  пользовательские идентификаторы должны начинаться с буквы,
  встроенные идентификаторы для внутренних объектов обычно имеют
  префикс «доллар»; они известны как аксессоры \emph{accessors}(в
  основном, переменные только для чтения). Большинство средств доступа
  определены в контексте открытого набора данных (см. часть
  ~\ref{part:hp-data}), но некоторые из них представляют собой заранее
  определенные константы, такие как $\pi$. Опять же, см. Справочник по
  командам для полного списка функций.
\end{itemize}

Hansl не имеет определенного типа Boolean, но скаляры могут
использоваться для хранения истинного / ложного значения. Отсюда
следует, что вы также можете использовать логические операторы
\emph{and} – и (\verb|&&|), \emph{or} – или (\verb+||+), \emph{not} –
не (\verb|!|) для скалярных величин, как в следующем примере:
\begin{code}
  a = 1
  b = 0
  c = !(a && b) 
\end{code}
В приведенном выше примере \texttt{c} будет равно 1 (истина),
поскольку \verb|(a && b)| --- ложь, а восклицательный знак ---
оператор отрицания. Обратите внимание, что 0 оценивается как ложь, а
все остальное (не обязательно 1) оценивается как истина.  Некоторые
конструкции взяты из семейства языков C. В частности, одна из них ---
оператор приращения:
\begin{code}
  a = 5
  b = a++
  print a b
\end{code}
вторая строка эквивалентна \texttt{a++}, за которой следует a ++, что,
в свою очередь, является сокращением для \texttt{a = a+1}, поэтому
выполнение приведенного выше кода приведет к результату, где b = 5, и
a=6. Вычитание также поддерживается программой; однако префиксные
операторы не поддерживаются. Другое заимствование C это измененное
присвоение, как в случае \texttt{a += b} (эквивалентно \texttt{a = a +
  b}); несколько других подобных операторов также доступны, например
\texttt{-=}, \texttt{*=} и другие. Подробности: см. Справку по
командам Gretl для полного списка функций.

Внутреннее представление отсутствующего значения --- \texttt{NaN} («не
число», not a number), как определено в стандарте IEEE 754. Программа
обычно выводит NaN, если мы пытаемся вычислить такие величины, как
квадратный корень или логарифм отрицательного числа. Возможно заранее
установить значение «не существует» напрямую, используя буквы
\texttt{NA}. Дополнительные функции \cmd{missing} и \cmd{ok} могут
использоваться для определения того, что скаляр \texttt{NA}, «не
существует». В следующем примере переменной с именем \texttt{test}
присваивается нулевое значение:
\begin{code}
  scalar not_really = NA
  scalar test = ok(not_really)
\end{code}
Обратите внимание на то, что вы не можете проверять на равенство
\texttt{NA} никакие переменные, как в:
\begin{code}
  if x == NA ... # неверно!
\end{code}
потому что отсутствующее значение считается неопределенным и,
следовательно, ничему не равным. Этот последний пример, несмотря на
то, что он ошибочен, иллюстрирует один важный момент: оператор
проверки на равенство в Hansl --- это двойной знак равенства,
``\texttt{==}'' (в отличие от простого ``\texttt{=}'', указывающего на
присвоение).

\section{Действия со строками}

Большая часть предыдущего раздела с очевидными изменениями применима к
строкам: вы можете совершать действия над строками, используя
операторы и/или функции. Репертуар функций Hansl для работы со
строками предлагает все стандартные возможности, которые можно
ожидать, такие как \cmd{toupper}, \cmd{tolower}, \cmd{strlen}, etc., а
также несколько более специализированных. Опять же, см. Справочник по
командам Gretl для полного списка функций.  Чтобы получить доступ к
желаемой части строки, вы можете использовать функцию
\cmd{substr}\footnote{На самом деле, есть более продвинутый метод,
  который использует тот же синтаксис, что и для матриц (см. Главу
  \ref{chap:matrices}): \cmd{substr(s, 3, 5)} функционально
  эквивалентен \cmd{s[3:5]}} как в
\begin{code}
  string s = "endogenous"
  string pet = substr(s, 3, 5)
\end{code}
что приведет к присвоению значения \texttt{dog} переменной
\texttt{pet}. Ниже приведены некоторые полезные операторы для строк:
\begin{itemize}
\item оператор тильда \verb|~| для соединения двух или более строк,
  как в \footnote{На некоторых клавиатурах нет символа тильды
    \verb|~|. В редакторе скриптов gretl знак тильды можно вставить
    через Юникод: введите Ctrl-Shift-U, а затем 7e.}
  \begin{code}
    string s1 = "sweet"
    string s2 = "Home, " ~ s1 ~ " home."
  \end{code}
\item подобный оператор \verb|~=| который действует как оператор
  присваивания с изменением угла наклона (так что \verb|a ~= "_ij"|
  эквивалентно \verb|a = a ~ "_ij"|);
\item оператор смещения \texttt{+}, который возвращает подстроку
  предыдущего элемента, начиная с данного символа смещения. Пустая
  строка выводится, если смещение больше, чем длина строки, о которой
  идет речь.
\end{itemize}

Примечательный момент: строки могут быть (почти) произвольно длинными;
кроме того, они могут содержать специальные символы, такие как перенос
строки и табуляция. Поэтому можно использовать Hansl для выполнения
сложных операций с текстовыми файлами, если загрузить их в память в
виде очень длинной строки, а затем с этим работать; заинтересованные
читатели могут обратиться к функциям \cmd{readfile}, \cmd{getline},
\cmd{strsub} и \cmd{regsub} в справочнике по командам
\GCR.\footnote{Мы не утверждаем, что Hansl будет предпочтительным
  инструментом для обработки текста в целом. Тем не менее, упомянутые
  здесь функции могут быть полезны для таких задач, как
  предварительная обработка файлов данных в виде простого текста,
  который не соответствует требованиям для прямого импортирования в
  gretl.}

Для создания сложных строк наиболее гибким инструментом является
функция \cmd{sprintf}. Ее использование подробно описано в
главе~\ref{chap:formatting}.

% Finally, it is quite common to use \emph{string
%   substitution} in hansl sripts; however, this is another topic that
% deserves special treatment so we defer its description to section
% \ref{sec:stringsub}.

%%% Local Variables: 
%%% mode: latex
%%% TeX-master: "hansl-primer"
%%% End: 

\chapter{Matrices}
\label{chap:matrices}

Matrices are one- or two-dimensional arrays of double-precision
floating-point numbers. Hansl users who are accustomed to other matrix
languages should note that multi-index objects are not
supported. Matrices have rows and columns, and that's it.

\section{Matrix indexing}
\label{sec:mat-index}

Individual matrix elements are accessed through the \verb|[r,c]|
syntax, where indexing starts at 1. For example, \texttt{X[3,4]}
indicates the element of $X$ on the third row, fourth column. For
example,
\begin{code}
  matrix X = zeros(2,3)
  X[2,1] = 4
  print X
\end{code}
produces
\begin{code}
X (2 x 3)

  0   0   0 
  4   0   0 
\end{code}

Here are some more advanced ways to access matrix elements:
\begin{enumerate}
\item In case the matrix has only one row (column), the column (row)
  specification can be omitted, as in \texttt{x[3]}.
\item Including the comma but omitting the row or column specification
  means ``take them all'', as in \texttt{x[4,]} (fourth row, all columns).
\item For square matrices, the special syntax \texttt{x[diag]} can be
  used to access the diagonal.
\item Consecutive rows or columns can be specified via the colon
  (\texttt{:}) character, as in \texttt{x[,2:4]} (columns 2 to 4).
  But note that, unlike some other matrix languages, the syntax
  \texttt{[m:n]} is illegal if $m>n$.
\item It is possible to use a vector to hold indices to a matrix. E.g.\
  if $e = [2,3,6]$, then \texttt{X[,e]} contains the second, third and
  sixth columns of $X$.
\end{enumerate}
Moreover, matrices can be empty (zero rows and columns). 

In the example above, the matrix \texttt{X} was constructed using
the function \texttt{zeros()}, whose meaning should be obvious, but
matrix elements can also be specified directly, as in
\begin{code}
scalar a = 2*3
matrix A = { 1, 2, 3 ; 4, 5, a }
\end{code}
The matrix is defined by rows; the elements on each row are separated
by commas and rows are separated by semicolons.  The whole expression
must be wrapped in braces.  Spaces within the braces are not
significant. The above expression defines a $2\times3$ matrix.

Note that each element should be a numerical value, the name of a
scalar variable, or an expression that evaluates to a scalar. In the
example above the scalar \texttt{a} was first assigned a value and
then used in matrix construction. (Also note, in passing, that
\texttt{a} and \texttt{A} are two separate identifiers, due to
case-sensitivity.)

\section{Matrix operations}
\label{sec:mat-op}

Matrix sum, difference and product are obtained via \texttt{+},
\texttt{-} and \texttt{*}, respectively. The prime operator
(\texttt{'}) can act as a unary operator, in which case it transposes
the preceding matrix, or as a binary operator, in which case it acts
as in ordinary matrix algebra, multiplying the transpose of the first
matrix into the second one. Errors are flagged if conformability is a
problem. For example:
\begin{code}
  matrix a = {11, 22 ; 33, 44}  # a is square 2 x 2
  matrix b = {1,2,3; 3,2,1}     # b is 2 x 3

  matrix c = a'         # c is the transpose of a
  matrix d = a*b        # d is a 2x3 matrix equal to a times b

  matrix gina = b'd     # valid: gina is 3x3
  matrix lina = d + b   # valid: lina is 2x3

  /* -- these would generate errors if uncommented ----- */

  # pina = a + b  # sum non-conformability
  # rina = d * b  # product non-conformability
\end{code}

Other noteworthy matrix operators include \texttt{\^} (matrix power),
\texttt{**} (Kronecker product), and the ``concatenation'' operators,
\verb|~| (horizontal) and \texttt{|} (vertical). Readers are invited
to try them out by running the following code
\begin{code}
matrix A = {2,1;0,1}
matrix B = {1,1;1,0}

matrix KP = A ** B
matrix PWR = A^3 
matrix HC = A ~ B
matrix VC = A | B

print A B KP PWR HC VC
\end{code}
Note, in particular, that $A^3 = A \cdot A \cdot A$, which is different
from what you get by computing the cubes of each element of $A$
separately.

Hansl also supports matrix left- and right-``division'', via the
\verb'\' and \verb'/' operators, respectively. The expression
\verb|A\b| solves $Ax = b$ for the unknown $x$. $A$ is assumed to be
an $m \times n$ matrix with full column rank. If $A$ is square the
method is LU decomposition. If $m > n$ the QR decomposition is used to
find the least squares solution. In most cases, this is numerically
more robust and more efficient than inverting $A$ explicitly.

Element-by-element operations are supported by the so-called ``dot''
operators, which are obtained by putting a dot (``\texttt{.}'') before
the corresponding operator. For example, the code
\begin{code}
A = {1,2; 3,4}
B = {-1,0; 1,-1}
eval A * B
eval A .* B
\end{code}
produces
\begin{code}
   1   -2 
   1   -4 

  -1    0 
   3   -4 
\end{code}

It's easy to verify that the first operation performed is regular
matrix multiplication $A \cdot B$, whereas the second one is the
Hadamard (element-by-element) product $A \odot B$. In fact, dot
operators are more general and powerful than shown in the example
above; see the chapter on matrices in \GUG{} for details.

Dot and concatenation operators are less rigid than ordinary matrix
operations in terms of conformability requirements: in most cases
hansl will try to do ``the obvious thing''. For example, a common
idiom in hansl is \texttt{Y = X ./ w}, where $X$ is an $n \times k$
matrix and $w$ is an $n \times 1$ vector. The result $Y$ is an $n
\times k$ matrix in which each row of $X$ is divided by the
corresponding element of $w$. In proper matrix notation, this
operation should be written as
\[
  Y = \langle w \rangle^{-1} X,
\]
where the $\langle \cdot \rangle$ indicates a diagonal
matrix. Translating literally the above expression would imply
creating a diagonal matrix out of $w$ and then inverting it, which is
computationally much more expensive than using the dot operation. A
detailed discussion is provided in \GUG.

Hansl provides a reasonably comprehensive set of matrix functions,
that is, functions that produce and/or operate on matrices. For a
full list, see the \GCR, but a basic ``survival kit'' is provided
in Table~\ref{tab:essential-matfuncs}.  Moreover, most scalar
functions, such as \texttt{abs(), log()} etc., will operate on a
matrix element-by-element.

\begin{table}[htbp]
  \centering
  \begin{tabular}{rp{0.6\textwidth}}
    \textbf{Function(s)} & \textbf{Purpose} \\
    \hline
    \texttt{rows(X), cols(X)} & return the number of rows and columns
    of $X$, respectively \\
    \texttt{zeros(r,c), ones(r,c)} & produce matrices with $r$ rows
    and $c$ columns, filled with zeros and ones, respectively \\
    \texttt{mshape(X,r,c)} & rearrange the elements of $X$ into a
    matrix with $r$ rows and $c$ columns \\
    \texttt{I(n)} & identity matrix of size $n$ \\
    \texttt{seq(a,b)} & generate a row vector containing integers form
    $a$ to $b$ \\
    \texttt{inv(A)} & invert, if possible, the matrix $A$ \\
    \texttt{maxc(A), minc(A), meanc(A)} & return a row vector
    with the max, min, means of each column of $A$, respectively\\
    \texttt{maxr(A), minr(A), meanr(A)} & return a column vector
    with the max, min, means of each row of $A$, respectively\\
    \texttt{mnormal(r,c), muniform(r,c)} & generate $r \times c$
    matrices filled with standard Gaussian and uniform pseudo-random
    numbers, respectively \\
    \hline
  \end{tabular}
  \caption{Essential set of hansl matrix functions}
  \label{tab:essential-matfuncs}
\end{table}

The following piece of code is meant to provide a concise example of
all the features mentioned above.

\begin{code}
# example: OLS using matrices

# fix the sample size
scalar T = 256

# construct vector of coefficients by direct imputation
matrix beta = {1.5, 2.5, -0.5} # note: row vector

# construct the matrix of independent variables
matrix Z = mnormal(T, cols(beta)) # built-in functions

# now construct the dependent variable: note the
# usage of the "dot" and transpose operators

matrix y = {1.2} .+ Z*beta' + mnormal(T, 1)

# now do estimation
matrix X = 1 ~ Z  # concatenation operator
matrix beta_hat1 = inv(X'X) * (X'y) # OLS by hand
matrix beta_hat2 = mols(y, X)       # via the built-in function
matrix beta_hat3 = X\y              # via matrix division

print beta_hat1 beta_hat2 beta_hat3
\end{code}

\section{Matrix pointers}
\label{sec:mat-pointers}

Hansl uses the ``by value'' convention for passing parameters to
functions. That is, when a variable is passed to a function as an
argument, what the function actually gets is a \emph{copy} of the
variable, which means that the value of the variable at the caller
level is not modified by anything that goes on inside the function.
But the use of pointers allows a function and its caller to cooperate
such that an outer variable can be modified by the function.

This mechanism is used by some built-in matrix functions to provide
more than one ``return'' value. The primary result is always provided
by the return value proper but certain auxiliary values may be
retrieved via ``pointerized'' arguments; this usage is flagged by
prepending the ampersand symbol, ``\texttt{\&}'', to the name of the
argument variable.

The \texttt{eigensym} function, which performs the eigen-analysis of
symmetric matrices, is a case in point. In the example below the first
argument $A$ represents the input data, that is, the matrix
whose analysis is required. This variable will not be modified in any
way by the function call. The primary result is the vector of
eigenvalues of $A$, which is here assigned to the variable
\texttt{ev}. The (optional) second argument, \texttt{\&V} (which may
be read as ``the address of \texttt{V}''), is used to retrieve the
right eigenvectors of $A$. A variable named in this way must be
already declared, but it need not be of the right dimensions to
receive the result; it will be resized as needed.
\begin{code}
matrix A = {1,2 ; 2,5}
matrix V
matrix ev = eigensym(A, &V)
print A ev V
\end{code}

%%% Local Variables: 
%%% mode: latex
%%% TeX-master: "hansl-primer"
%%% End: 

\chapter{Nice-looking output}
\label{chap:formatting}

\section{Formatted output}
\label{sec:printf}

A common occurrence when you're writing a script---particularly when
you intend for the script to be used by others, and you'd like the
output to be reasonably self-explanatory---is that you want to output
something along the following lines:
\begin{code}
The coefficient on X is Y, with standard error Z
\end{code}
where \texttt{X}, \texttt{Y} and \texttt{Z} are placeholders for
values not known at the time of writing the script; they will be
filled out as the values of variables or expressions when the script
is run. Let's say that at run time the replacements in the sentence
above should come from variables named \texttt{vname} (a string),
\texttt{b} (a scalar value) and \texttt{se} (also a scalar value),
respectively.

Across the spectrum of programming languages there are basically two
ways of arranging for this. One way originates in the \textsf{C}
language and goes under the name \texttt{printf}. In this approach we
(a) replace the generic placeholders \texttt{X}, \texttt{Y} and
\texttt{Z} with more informative \textit{conversion specifiers}, and
(b) append the variables (or expressions) that are to be stuck into
the text, in order. Here's the hansl version:
\begin{code}
printf "The coefficient on %s is %g, with standard error %g\n", vname, b, se
\end{code}
The value of \texttt{vname} replaces the conversion specifier
``\texttt{\%s},'' and the values of \texttt{b} and \texttt{se} replace
the two ``\texttt{\%g}'' specifiers, left to right. In relation to
hansl, here are the basic points you need to know: ``\texttt{\%s}''
pairs with a string argument, and ``\texttt{\%g}'' pairs with a
numeric argument.

The \textsf{C}-derived \texttt{printf} (either in the form of a
function, or in the form of a command as shown above) is present in
most ``serious'' programming languages. It is extremely versatile, and
in its advanced forms affords the programmer fine control over
the output.

In some scripting languages, however, \texttt{printf} is reckoned
``too difficult'' for non-specialist users. In that case some sort of
substitute is typically offered. We're skeptical: ``simplified''
alternatives to \texttt{printf} can be quite confusing, and if at some
point you want fine control over the output, they either do not
support it, or support it only via some convoluted mechanism. A
typical alternative looks something like this (please note,
\texttt{display} is \textit{not} a hansl command, it's just
illustrative):
\begin{code}
display "The coefficient on ", vname, "is ", b, ", with standard error ", se, "\n"
\end{code}
That is, you break the string into pieces and intersperse the names of
the variables to be printed. The requirement to provide conversion
specifiers is replaced by a default automatic formatting of the
variables based on their type. By the same token, the command line
becomes peppered with multiple commas and quotation marks. If this looks
preferable to you, you are welcome to join one of the gretl mailing
lists and argue for its provision!

Anyway, to be a bit more precise about \cmd{printf}, its syntax goes
like this:
\begin{flushleft}
  \texttt{printf \emph{format}, \emph{arguments}}
\end{flushleft}
The \emph{format} is used to specify the precise way in which you want
the \emph{arguments} to be printed.

\subsection{The format string}
\label{sec:fmtstring}

In the general case the \cmd{printf} format must be an expression that
evaluates to a string, but in most cases will just be a \textit{string
  literal} (an alphanumeric sequence surrounded by double
quotes). However, some character sequences in the format have a
special meaning. As illustrated above, those beginning with a
percent sign (\texttt{\%}) are interpreted as placeholders for the
items contained in the argument list. In addition, special characters
such as the newline character are represented via a combination
beginning with a backslash (\verb|\|).

For example,
\begin{code}
printf "The square root of %d is (roughly) %6.4f.\n", 5, sqrt(5)
\end{code}
will print 
\begin{code}
The square root of 5 is (roughly) 2.2361.
\end{code}

Let's see how:
\begin{itemize}
\item The first special sequence is \verb|%d|: this indicates that we
  want an integer at that place in the output; since it is the
  leftmost ``percent'' expression, it is matched to the first
  argument, that is 5.
\item The second special sequence is \verb|%6.4f|, which stands for a
  decimal value with 4 digits after the decimal separator\footnote{The
    decimal separator is the dot in English, but may be different in
    other locales.} and at least 6 digits wide; this will be matched
  to the second argument. Note that arguments are separated by
  commas. Also note that the second argument is neither a scalar
  constant nor a scalar variable, but an expression that evaluates to
  a scalar.
\item The format string ends with the sequence \verb|\n|, which
  inserts a newline.
\end{itemize}

The conversion specifiers in the square-root example are relatively
fancy, but as we noted earlier \texttt{\%g} will work fine for
almost all numerical values in hansl. So we could have used the
simpler form:
\begin{code}
printf "The square root of %g is (roughly) %g.\n", 5, sqrt(5)
\end{code}
The effect of \texttt{\%g} is to print a number using up to 6
significant digits (but dropping trailing zeros); it automatically
switches to scientific notation if the number is very large or very
small. So the result here is
\begin{code}
The square root of 5 is (roughly) 2.23607.
\end{code}

The escape sequences \verb|\n| (newline), \verb|\t| (tab), \verb|\v|
(vertical tab) and \verb|\\| (literal backslash) are recognized. To
print a literal percent sign, use \verb|%%|.

Apart from those shown in the above example, recognized numeric
formats are \verb|%e|, \verb|%E|, \verb|%f|, \verb|%g|, \verb|%G| and
\verb|%x|, in each case with the various modifiers available in C. The
format \verb|%s| should be used for strings.
As in C, numerical values that form part of the format (width and or
precision) may be given directly as numbers, as in \verb|%10.4f|, or
they may be given as variables. In the latter case, one puts asterisks
into the format string and supplies corresponding arguments in
order. For example,

\begin{code}
  scalar width = 12 
  scalar precision = 6 
  printf "x = %*.*f\n", width, precision, x
\end{code}

If a matrix argument is given in association with a numeric format,
the entire matrix is printed using the specified format for each
element. A few more examples are given in table \ref{tab:printf-ex}.
\begin{table}[htbp]
  \centering
   {\small
    \begin{tabular}{p{0.45\textwidth}p{0.3\textwidth}}
      \textbf{Command} & \textbf{effect} \\
      \hline
      \verb|printf "%12.3f", $pi| & 3.142 \\
      \verb|printf "%12.7f", $pi| & 3.1415927 \\
      \verb|printf "%6s%12.5f%12.5f %d\n", "alpha",| \\
      \verb|   3.5, 9.1, 3| &
      \verb| alpha     3.50000     9.10000 3| \\
      \verb|printf "%6s%12.5f%12.5f\t%d\n", "beta",| \\
      \verb|   1.2345, 1123.432, %11| &
      \verb|  beta     1.23450  1123.43200 11| \\
      \verb|printf "%d, %10d, %04d\n", 1,2,3| & 
      \verb|1,          2, 0003| \\
      \verb|printf "%6.0f (%5.2f%%)\n", 32, 11.232| & \verb|32 (11.23%)| \\
      \hline
    \end{tabular}
  }
  \caption{Print format examples}
  \label{tab:printf-ex}
\end{table}

\subsection{Output to a string}
\label{sec:sprintf}

A closely related effect can be achieved via the \cmd{sprintf}
function: instead of being printed directly the result is stored in a
named string variable, as in
\begin{code}
  string G = sprintf("x = %*.*f\n", width, precision, x)
\end{code}
after which the variable \texttt{G} can be the object of further
processing.

\subsection{Output to a file}
\label{sec:outfile}

Hansl does not have a file or ``stream'' type as such, but the
\cmd{outfile} command can be used to divert output to a named text
file. To start such redirection you must give the name of a file; by
default a new file is created or an existing one overwritten but the
\option{append} can be used to append material to an existing file.
Only one file can be opened in this way at any given time. The
redirection of output continues until the command \cmd{end outfile} is
given; then output reverts to the default stream.

Here's an example of usage:
\begin{code}
  printf "One!\n"
  outfile "myfile.txt"
    printf "Two!\n"
  end outfile
  printf "Three!\n"
  outfile "myfile.txt" --append
    printf "Four!\n"
  end outfile
  printf "Five!\n"
\end{code}
After execution of the above the file \texttt{myfile.txt} will contain
the lines
\begin{code}
Two!
Four!  
\end{code}

Three special variants on the above are available. If you give the
keyword \texttt{null} in place of a real filename along with the write
option, the effect is to suppress all printed output until redirection
is ended. If either of the keywords \texttt{stdout} or \texttt{stderr}
are given in place of a regular filename the effect is to redirect
output to standard output or standard error output, respectively.

This command also supports a \option{quiet} option: its effect is to
turn off the echoing of commands and the printing of auxiliary
messages while output is redirected. It is equivalent to doing
\begin{code}
  set verbose off 
\end{code}
before invoking \cmd{outfile}, except that when redirection is ended
the prior value of the \texttt{verbose} state variable is restored.

\section{Graphics}

The primary graphing command in hansl is \texttt{gnuplot} which, as
the name suggests, in fact provides an interface to the
\textsf{gnuplot} program. It is used for plotting series in a dataset
(see part~\ref{part:hp-data}) or columns in a matrix. For an account
of this command (and some other more specialized ones, such as
\texttt{boxplot} and \texttt{qqplot}), see the \GCR.

%%% Local Variables: 
%%% mode: latex
%%% TeX-master: "hansl-primer"
%%% End: 

\chapter{Структурированные типы данных}
\label{chap:structypes}

В Hansl есть два типа «структурированного типа данных»: ассоциативные
массивы, называемые пакетами, и массивы в буквальном смысле слова
(\emph{bundles and arrays}). Грубо говоря, основное различие между
ними состоит в том, что в связке вы можете собрать вместе переменные
разных типов, в то время как массивы могут содержать лишь один тип
переменной.

\section{Пакеты}
\label{sec:bundles}

Пакеты --- это ассоциативные массивы, то есть общие контейнеры для
любого набора данных Hansl (включая другие пакеты), в которых каждый
элемент идентифицируется строкой. Пользователи Python называют это
словарями; в C ++ и Java они называются картами; в Perl они известны
как хеши. Мы называем их пакетами, \emph{bundles}. Каждый элемент,
помещенный в пакет, связан с ключом, который можно использовать для
его восстановления в дальнейшем.  Для того, чтобы использовать пакет,
вы сначала либо «объявляете» его, как действующий:
%
\begin{code}
bundle foo
\end{code}
%
или как пустой, используя \texttt{null}:
%
\begin{code}
bundle foo = null
\end{code}
%
Эти две формулировки эквивалентны в том смысле, что они создают пустой
пакет. Разница заключается в том, что второй вариант можно
использовать повторно. Если пакет с именем \texttt{foo} уже
существует, результатом будет его очищение. Первый вариант может
использоваться только один раз в данном сеансе gretl; задать уже
существующую переменную не получится.  Чтобы добавить объект в пакет,
вы назначаете значение слева в данном составе: имя пакета с
последующим ключом. Наиболее распространенный способ присоединить ключ
к имени пакета с точкой, как в
\begin{code}
  foo.matrix1 = m
\end{code}
где добавляется объект с именем \texttt{m} (предположительно матрица),
чтобы связать \texttt{foo} с ключом \texttt{matrix1}. Ключ должен
соответствовать правилам определения имени переменной gretl (максимум
31 символ, начинается с буквы и состоит лишь из букв, цифр или нижнего
подчеркивания).  Альтернативный способ добиться того же эффекта ---
указать ключ в виде строкового литерала в кавычках, заключенного в
квадратные скобки, как в quoted string literal enclosed in square
brackets, as in
\begin{code}
  foo["matrix1"] = m
\end{code}
При использовании более сложного синтаксиса ключи не обязательно
должны быть написаны как имена переменных --- например, они могут
содержать пробелы, но их длина по-прежнему ограничена 31 символом.
Чтобы исключить элемент из пакета, снова используйте имя пакета, за
которым следует ключ, как в

\begin{code}
matrix bm = foo.matrix1
# или используйте более длинный вариант 
matrix m = foo["matrix1"]
\end{code}
Обратите внимание, что ключ, определяющий объект в данном пакете,
обязательно уникален. Если вы повторно используете существующий ключ в
новом назначении, результатом будет замена объекта, который был ранее
сохранен под данный ключ. Необязательно, чтобы тип заменяемого объекта
совпадал с именем изначального объекта.  Более быстрый способ, впервые
представленный в gretl 2017b --- использовать функцию \cmd{defbundle},
как в

\begin{code}
  bundle b = defbundle("s", "Sample string", "m", I(3))
\end{code}
где каждый аргумент с нечетным номером должен оцениваться как строка
(ключ), а каждый аргумент с четным номером должен оцениваться как
объект данного типа, который может быть включен в пакет.  Обратите
внимание, что когда вы добавляете объект в пакет, на самом деле
происходит получение пакетом копии объекта. Внешний объект сохраняет
свою идентичность и не изменяется, если связанный объект заменяется
другим. Рассмотрим следующий фрагмент скрипта:

\begin{code}
bundle foo
matrix m = I(3)
foo.mykey = m
scalar x = 20
foo.mykey = x
\end{code}
После выполнения вышеуказанных команд в пакете \texttt{foo} не будет
матрицы под \texttt{mykey}, но исходная матрица \texttt{m} все еще в
отличном состоянии. Чтобы удалить объект из пакета, используйте
команду удаления с комбинацией пакет/ключ, как в
\begin{code}
delete foo.mykey
delete foo["quoted key"]
\end{code}
Данная команда уничтожает объект, связанный с ключом, и удаляет ключ
из хэш-таблицы\footnote{На самом деле пакеты gretl имеют форму
  хеш-таблицы \textsf{GLib}.}.  Помимо добавления, использования,
замены и удаления отдельных элементов, другие поддерживаемые операции
для пакетов включают объединение и вывод. Что касается объединения,
если определены пакеты \texttt{b1} и \texttt{b2}, можно написать

\begin{code}
bundle b3 = b1 + b2
\end{code}

чтобы создать новый пакет, который представляет собой объединение двух
других. Алгоритм такой: создать новый пакет (копия \texttt{b1}), затем
добавить любые элементы из \texttt{b2}, ключи которых еще не
присутствуют в новом пакете. Это означает, что объединение пакетов не
обязательно коммутативно (результат может зависеть от перестановки
элементов), если пакеты имеют одну или более общих ключевых строк.
Если \texttt{b} --- это пакет, и вы даете команду \texttt{print b},
далее вы получите список ключей пакета вместе с типами соответствующих
объектов, как в
\begin{code}
? print b
bundle b:
 x (scalar)
 mat (matrix)
 inside (bundle)
\end{code}

\subsection{Использование пакета}
\label{sec:bundle-usage}

Чтобы проиллюстрировать, как пакет сохраняет информацию, мы будем
использовать метод обыкновенных наименьших квадратов (OLS): в качестве
примера следующий код оценивает регрессию OLS и сохраняет все
результаты в пакет:

\begin{code}
/* предположим, что для y и X даны T x 1 и T x k матрицы */

bundle my_model = null               # инициализация
my_model.T = rows(X)                 # размер выборки
my_model.k = cols(X)                 # кол-во регрессоров
matrix e                             # содержит остатки
b = mols(y, X, &e)                   # запуск OLS через собств. функцию
s2 = meanc(e.^2)                     # оценка разброса выборки
matrix V = s2 .* invpd(X'X)          # вывод матрицы ковариаций

/* теперь сохраняем оцененные параметры в пакет */

my_model.betahat = b
my_model.s2 = s2
my_model.vcv = V
my_model.stderr = sqrt(diag(V))
\end{code}

Полученный таким образом пакет представляет собой контейнер, который
можно использовать для любых целей. Например, следующий фрагмент кода
показывает, как использовать пакет с той же структурой для выполнения
вневыборочного прогноза. Представьте, что $k=4$ и значение
$\mathbf{x}$ , для которого мы хотим прогноз $y$ , равно
\[
  \mathbf{x}' = [ 10 \quad 1  \quad -3 \quad 0.5 ]
\]
Прогнозные формулы будут иметь следующий вид:
\begin{eqnarray*}
  \hat{y}_f & = & \mathbf{x}'\hat{\beta} \\
  s_f & = & \sqrt{\hat{\sigma}^2 + \mathbf{x}'V(\hat{\beta})\mathbf{x}} \\
  CI & = & \hat{y}_f \pm 1.96 s_f 
\end{eqnarray*}
где $CI$ (приблизительный) 95-процентный доверительный
интервал. Приведенные выше формулы переводятся в
\begin{code}
  x = { 10, 1, -3, 0.5 }
  scalar ypred    = x * my_model.betahat
  scalar varpred  = my_model.s2 + qform(x, my_model.vcv)
  scalar sepred   = sqrt(varpred)
  matrix CI_95    = ypred + {-1, 1} .* (1.96*sepred)
  print ypred CI_95
\end{code}

\section{Массивы}
\label{sec:arrays}

Массив gretl --- это контейнер, который может содержать ноль или более
объектов определенного типа, индексированных как последовательные
целые числа, начиная с 1. Он одномерный. Этот тип реализован довольно
«общим» бэк-ендом. В массивы можно помещать следующие типы объектов:
строки, матрицы, пакеты и списки; один массив может содержать только
один из этих типов.

\subsection{Операции с массивами}

Следующее, как мы полагаем, не требует пояснений:

\begin{code}
strings S1 = array(3)
matrices M = array(4)
strings S2 = defarray("fish", "chips")
S1[1] = ":)"
S1[3] = ":("
M[2] = mnormal(2,2)
print S1
eval inv(M[2])
S = S1 + S2
print S
\end{code}

\texttt{Array()} принимает целочисленный аргумент для размера массива;
функция \texttt{defarray()} обозначает количество аргументов (один или
несколько), каждый из которых может быть именем переменной данного
типа или выражением, оценивающим объект данного типа. Соответствующий
результат будет следующим:

\begin{code}
Array of strings, length 3
[1] ":)"
[2] null
[3] ":("

     0.52696      0.28883 
    -0.15332     -0.68140 

Array of strings, length 5
[1] ":)"
[2] null
[3] ":("
[4] "fish"
[5] "chips"
\end{code}

Чтобы узнать количество элементов в массиве, вы можете использовать функцию
\texttt{nelem()}.

%%% Local Variables: 
%%% mode: latex
%%% TeX-master: "hansl-primer"
%%% End: 


\chapter{Численные методы}
\label{chap:numerical}

\section{Численная оптимизация}
\label{sec:hp-numopt}

В случае, когда эконометрист хочет использовать язык программирования,
такой как Hansl, а не просто полагаться на заранее подготовленные
подпрограммы, возникает необходимость применить какую-либо форму
численной оптимизации. Это может быть форма максимизации вероятности
или аналогичные методы индуктивной статистики. В другом варианте можно
использовать оптимизацию в более общем и абстрактном виде, например,
для решения задачи о выборе портфеля ценных бумаг или аналогичных
задач о распределении ресурсов.  Поскольку Hansl представляет собой
полный язык по Тьюрингу, в принципе любая числовая техника оптимизации
может программироваться в самом Hansl. Некоторые такие методы включены
в набор собственных функций программы, в интересах простоты ее
использования и эффективности. Они предназначены для решения наиболее
распространенных проблем, которые встречаются в экономике и
эконометрике, что неограниченно лишь оптимизацией дифференцируемых
функций.  В этой главе мы кратко рассмотрим, что конкретно предлагает
Hansl для решения задач следующего вида:
\[
\hat{\mathbf{x}} \equiv \argmax_{\mathbf{x} \in \Re^k} f(\mathbf{x}; \mathbf{a}),
\]
где $f(\mathbf{x}; \mathbf{a})$ --- функция от $\mathbf{x}$, форма
которой зависит от вектора параметров $\mathbf{a}$. Предполагается,
что целевая функция $f(\cdot)$ возвращает скаляру действительное
значение. В большинстве случаев также предполагается, что она
непрерывна и дифференцируема, хотя это необязательно. (Обратите
внимание, что хотя встроенные функции Hansl максимизируют заданную
целевую функцию, ее легко можно минимизировать, если просто
перевернуть знак $f(\cdot)$.)  Частный случай вышеизложенного
наступает, когда $\mathbf{x}$ --- это вектор параметров, а
$\mathbf{a}$ обозначает «данные». В этих случаях целевая функция
обычно представляет собой (логарифмическую) вероятность, а задача
заключается в ее оценке.  Для подобных случаев Hansl предлагает
несколько специальных конструкций, рассмотренных в
разделе~\ref{sec:est-blocks}. Здесь мы поговорим о более общих
проблемах. Тем не менее, важно отметить, что различия состоят только в
используемом синтаксисе Hansl; однако, математические алгоритмы,
которые использует gretl для решения задачи оптимизации, те же самые.
Читателю предлагается прочитать главу «Численные методы» Руководства
пользователя Gretl для полного представления о данной проблеме. Ниже
мы приведем лишь небольшой пример того, как это может выглядеть:

\begin{code}
function scalar Himmelblau(matrix x)
    /* extrema:
    f(3.0, 2.0) = 0.0, 
    f(-2.805118, 3.131312) = 0.0,
    f(-3.779310, -3.283186) = 0.0
    f(3.584428, -1.848126) = 0.0
    */
    scalar ret = (x[1]^2 + x[2] - 11)^2
    return -(ret + (x[1] + x[2]^2 - 7)^2)
end function

# ----------------------------------------------------------------------

set max_verbose 1

matrix theta1 = { 0, 0 }
y1 = BFGSmax(theta1, "Himmelblau(theta1)")
matrix theta2 = { 0, -1 }
y2 = NRmax(theta2, "Himmelblau(theta2)")

print y1 y2 theta1 theta2
\end{code}
Мы используем для иллюстрации классическую «неприятную» функцию из
литературы по численной оптимизации, а именно функцию Химмельблау,
которая имеет четыре различных минимума
$f(x, y) = (x^2+y-11)^2 + (x+y^2-7)^2$. Алгоритм выглядит следующим
образом:

\begin{enumerate}
\item Сначала мы определяем функцию для оптимизации: она должна
  возвращать скалярное значение и иметь среди своих аргументов вектор
  для оптимизации. В данном конкретном случае это единственный
  аргумент, но могли быть также и другие. Поскольку в этом случае мы
  минимизируем функцию, программа возвращает отрицательное значение
  функции Himmelblau.
\item Затем мы задаем \verb|max_verbose| как 1. Это еще один пример
  использования команды \cmd{set} ; сама команда предлагает: «давайте
  посмотрим, как проходят итерации», и по умолчанию она равна 0. С
  помощью команды \cmd{set} с соответствующими параметрами, можно
  управлять несколькими функциями процесса оптимизации, такими как
  числовые значения допусков, визуализация итераций и т. д.
\item Определяем $\theta_1 = [0, 0]$ как начальную точку.
\item Применяем функцию \cmd{BFGSmax}; она будет искать максимум с
  помощью техники BFGS. Ее базовый синтаксис выглядит так:
  \texttt{BFGSmax(arg1, arg2)}, где \texttt{arg1} --- вектор,
  содержащий переменную оптимизации, а \texttt{arg2} --- это строка,
  содержащая вызов функции, которую нужно максимизировать. BFGS
  подставляет значения $\theta_1$, пока максимум не будет
  достигнут. При успешном завершении вектор \texttt{theta1} будет
  содержать конечную точку. (Примечание: об этом сказано гораздо
  больше в Руководстве пользователя Gretl и Справочнике по командам
  Gretl.)
\item Затем мы решаем ту же проблему, но с другой отправной точкой и
  другой техникой оптимизации. Мы начинаем с $\theta_2 = [0, -1]$ и
  используем метод Ньютона – Рафсона вместо BFGS, применяя
  \cmd{NRmax()} вместо \cmd{BFGSmax()}. Синтаксис тот же.
\item Выводим результаты.
\end{enumerate}

Данный вывод содержится в ~\ref{tab:optim-output}. Обратите внимание,
что второй раз получается иной локальный оптимум, чем в первый
раз. Это следствие того, что мы начали тот же алгоритм с другой
отправной точки. В этом примере использовались числовые производные,
но вы можете использовать аналитически вычисленные производные для
обоих методов, если у вас есть функция Hansl для них; см. Руководство
пользователя Gretl для получения более подробной информации.

\begin{table}[ht]
  \begin{footnotesize}
\begin{scode}
? matrix theta1 = { 0, 0 }
Replaced matrix theta1
? y1 = BFGSmax(theta1, "Himmelblau(11, theta1)")
Iteration 1: Criterion = -170.000000000
Parameters:       0.0000      0.0000
Gradients:        14.000      22.000 (norm 0.00e+00)

Iteration 2: Criterion = -128.264504038 (steplength = 0.04)
Parameters:      0.56000     0.88000
Gradients:        33.298      39.556 (norm 5.17e+00)

...

--- FINAL VALUES: 
Criterion = -1.83015730011e-28 (steplength = 0.0016)
Parameters:       3.0000      2.0000
Gradients:    1.7231e-13 -3.7481e-13 (norm 7.96e-07)

Function evaluations: 39
Evaluations of gradient: 16
Replaced scalar y1 = -1.83016e-28
? matrix theta2 = { 0, -1 }
Replaced matrix theta2
? y2 = NRmax(theta2, "Himmelblau(11, theta2)")
Iteration 1: Criterion = -179.999876556 (steplength = 1)
Parameters:   1.0287e-05     -1.0000
Gradients:        12.000  2.8422e-06 (norm 7.95e-03)

Iteration 2: Criterion = -175.440691085 (steplength = 1)
Parameters:      0.25534     -1.0000
Gradients:        12.000  4.5475e-05 (norm 1.24e+00)

...

--- FINAL VALUES: 
Criterion = -3.77420797114e-22 (steplength = 1)
Parameters:       3.5844     -1.8481
Gradients:   -2.6649e-10  2.9536e-11 (norm 2.25e-05)

Gradient within tolerance (1e-07)
Replaced scalar y2 = -1.05814e-07
? print y1 y2 theta1 theta2

             y1 = -1.8301573e-28

             y2 = -1.0581385e-07

theta1 (1 x 2)

  3   2 

theta2 (1 x 2)

      3.5844      -1.8481 
\end{scode}
    
  \end{footnotesize}
  \caption{Результат максимизации}
  \label{tab:optim-output}
\end{table}
Язык Hansl предлагает вашему вниманию следующие методы оптимизации:
\begin{itemize}
\item BFGS через функцию \cmd{BFGSmax()}. В большинстве случаев это
  лучший компромисс между производительностью и
  надежностью. Предполагается, что функция максимизации
  дифференцируема и будет аппроксимировать кривую за счет
  использования изменений градиента между итерациями. Вы можете
  снабдить ее аналитически вычисленным градиентом для большей скорости
  и точности, но если его нет, то первые производные будут вычисляться
  численно.
\item Метод Ньютона --- Рафсона с помощью функции \cmd{NRmax()}. На
  самом деле это название немного вводит в заблуждение. Метод должен
  был называться чем-то вроде «на основе кривизны функции», поскольку
  он основан на итерациях
  \[
    x_{i+1} = -\lambda_i C(x_i)^{-1} g(x_i)
  \]
  где $g(x)$ градиент, а $C(x_i)$ некоторая мера кривизны функции для
  оптимизации; если $C(x)$ --- матрица Гессиана, то вы получите
  функцию Ньютона – Рафсона. Опять же, вы можете закодировать свои
  собственные функции для $g(\cdot)$ и $C(\cdot)$, но если вы этого не
  сделаете, то будут использоваться соответственно численные
  приближения градиента и гессиана. Другие популярные методы
  оптимизации, такие как BHHH и алгоритм подсчета очков, могут быть
  применены путем добавления в \cmd{NRmax()} соответствующей матрицы
  кривизны $C(\cdot)$. Этот метод очень эффективен, когда он работает,
  но довольно хрупок в использовании: например, если $C(x_i)$
  оказывается неотрицательно определенной на некоторой итерации,
  сходимость может стать проблематичной.
\item Методы без производных: единственный метод, который в настоящее
  время предлагает Hansl, --- это алгоритм имитации отжига (simulated
  annealing) через функцию \cmd{simann()}, но реализация алгоритма
  Нелдера – Мида (также известного как метод «амебы») лишь вопрос
  времени. Эти методы работают даже тогда, когда функция максимизации
  имеет некоторую форму разрыва или не везде дифференцируема; однако
  они могут работать очень медленно и перегружать процессор.
\end{itemize}

\section{Численное дифференцирование}
\label{sec:hp-numdiff}

Для численного дифференцирования возможно применить функцию
\texttt{fdjac}. Например:

\begin{code}
set echo off
set messages off

function scalar beta(scalar x, scalar a, scalar b)
    return x^(a-1) * (1-x)^(b-1)
end function

function scalar ad_beta(scalar x, scalar a, scalar b)
    scalar g = beta(x, a-1, b-1)
    f1 = (a-1) * (1-x)
    f2 = (b-1) * x
    return (f1 - f2) * g
end function

function scalar nd_beta(scalar x, scalar a, scalar b)
    matrix mx = {x}
    return fdjac(mx, beta(mx, a, b))
end function

a = 3.5
b = 2.5

loop for (x=0; x<=1; x+=0.1)
    printf "x = %3.1f; beta(x) = %7.5f, ", x, beta(x, a, b)
    A = ad_beta(x, a, b)
    N = nd_beta(x, a, b)
    printf "analytical der. = %8.5f, numerical der. = %8.5f\n", A, N
endloop
\end{code}

возвращает
\begin{code}
x = 0.0; beta(x) = 0.00000, analytical der. =  0.00000, numerical der. =  0.00000
x = 0.1; beta(x) = 0.00270, analytical der. =  0.06300, numerical der. =  0.06300
x = 0.2; beta(x) = 0.01280, analytical der. =  0.13600, numerical der. =  0.13600
x = 0.3; beta(x) = 0.02887, analytical der. =  0.17872, numerical der. =  0.17872
x = 0.4; beta(x) = 0.04703, analytical der. =  0.17636, numerical der. =  0.17636
x = 0.5; beta(x) = 0.06250, analytical der. =  0.12500, numerical der. =  0.12500
x = 0.6; beta(x) = 0.07055, analytical der. =  0.02939, numerical der. =  0.02939
x = 0.7; beta(x) = 0.06736, analytical der. = -0.09623, numerical der. = -0.09623
x = 0.8; beta(x) = 0.05120, analytical der. = -0.22400, numerical der. = -0.22400
x = 0.9; beta(x) = 0.02430, analytical der. = -0.29700, numerical der. = -0.29700
x = 1.0; beta(x) = 0.00000, analytical der. = -0.00000, numerical der. =       NA
\end{code}

Подробности об используемом алгоритме можно найти в Справочнике по
командам Gretl. Достаточно сказать, что здесь параметр
\texttt{fdjac\_quality} задан от 0 до 2. Значение по умолчанию --- 0,
который дает приближение прямой разницы: это самый быстрый алгоритм,
но иногда он может быть недостаточно точным. Значение 1 дает
двустороннюю разницу, а 2 использует экстраполяцию Ричардсона. По мере
того, как значение параметра повышается, растет и точность, но метод
значительно больше загружает процессор.

% \section{Random number generation}

% \begin{itemize}
% \item Mersenne Twister in its various incarnations
% \item Ziggurat vs Box--Muller
% \item Other distributions
% \end{itemize}

%%% Local Variables: 
%%% mode: latex
%%% TeX-master: "hansl-primer"
%%% End: 


\chapter{Control flow}
\label{chap:hp-ctrlflow}

The primary means for controlling the flow of execution in a hansl
script are the \cmd{if} statement (conditional execution), the
\cmd{loop} statement (repeated execution), the \cmd{catch} modifier
(which enables the trapping of errors that would otherwise halt
execution), and the \cmd{quit} command (which forces termination).

\section{The \cmd{if} statement}

Conditional execution in hansl uses the \cmd{if} keyword. Its fullest
usage is as follows
\begin{code}
if <condition>
   ...
elif <condition>
   ...
else 
   ...
endif  
\end{code}

Points to note:
\begin{itemize}
\item The \texttt{<condition>} can be any expression that evaluates to a
  scalar: 0 is interpreted as ``false'', non-zero is interpreted as
  ``true''; \texttt{NA} generates an error.
\item Following \cmd{if}, ``then'' is implicit; there is no \texttt{then}
  keyword as found in, e.g., Pascal or Basic.
\item The \cmd{elif} and \cmd{else} clauses are optional: the minimal
  form is just \texttt{if} \dots{} \texttt{endif}.
\item Conditional blocks of this sort can be nested up to a maximum
  depth of 1024.
\end{itemize}

Example:
\begin{code}
scalar x = 15

# --- simple if ----------------------------------
if x >= 100
   printf "%g is more than two digits long\n", x
endif

# --- if with else -------------------------------
if x >= 0
   printf "%g is non-negative\n", x
else
   printf "%g is negative\n", x
endif

# --- multiple branches --------------------------
if missing(x)
   printf "%g is missing\n", x
elif x < 0
   printf "%g is negative\n", x
elif floor(x) == x
   printf "%g is an integer\n", x
else
   printf "%g is a positive number with a fractional part\n", x
endif
\end{code}

Note, from the example above, that the \cmd{elif} keyword can be
repeated, making hansl's \cmd{if} statement a multi-way branch
statement. There is no separate \cmd{switch} or \cmd{case} statement
in hansl. With one or more \cmd{elif}s, hansl will execute the first
one for which the logical condition is satisfied and then jump to
\cmd{endif}.

\tip{Stata users, beware: hansl's \cmd{if} statement is fundamentally
  different from Stata's \texttt{if} option: the latter selects a
  subsample of observations for some action, while the former is used
  to decide if a group of statements should be executed or not;
  hansl's \cmd{if} is what Stata calls ``branching \texttt{if}''.}


\subsection{The ternary query operator}

Besides use of \texttt{if}, the ternary query operator, \texttt{?:},
can be used to perform conditional assignment on a more ``micro''
level. This has the form
\begin{code}
result = <condition> ? <value-if-true> : <value-if-false>
\end{code}

If \texttt{<condition>} evaluates as ``true'' (non-zero) then the
first following value is assigned to \texttt{result}, otherwise the
value after the colon is so assigned.\footnote{Some readers may find
  it helpful to note that the conditional assignment operator works in
  exactly the same way as the \texttt{=IF()} function in
  spreadsheets.}  This is obviously more compact than \texttt{if}
\dots{} \texttt{else} \dots{} \texttt{endif}. The following example
replicates the \cmd{abs} function by hand:
\begin{code}
scalar ax = x>=0 ? x : -x
\end{code}
Of course, in the above case it would have been much simpler to just
write \texttt{ax = abs(x)}. Consider, however, the following case,
which exploits the fact that the ternary operator can be nested:
\begin{code}
scalar days = (m==2) ? 28 : maxr(m.={4,6,9,11}) ? 30 : 31
\end{code}
This example deserves a few comments. We want to compute the number of
days in a month, coded in the variable \texttt{m}. The value we assign
to the scalar \texttt{days} comes from the following pathway.
\begin{enumerate}
\item First we check if the month is February (\texttt{m==2}); if so,
  we set \texttt{days} to 28 and we're done.\footnote{OK, we're ignoring
    leap years here.}
\item Otherwise, we compute a matrix of zeros and ones via the
  operation \verb|m.={4,6,9,11}| (note the use of the ``dot'' operator
  to perform an element-by element comparison---see section
  \ref{sec:mat-op}); if \texttt{m} equals any of the elements in the
  vector, the corresponding element of the result will be 1, and 0
  otherwise;
\item The \cmd{maxr} function gives the maximum of this vector, so
  we're checking whether \texttt{m} is any one of the four values
  corresponding to 30-day months.
\item Since the above evaluates to a scalar, we put the correct value
  into \texttt{days}.
\end{enumerate}

The ternary operator is more flexible than the ordinary \cmd{if}
statement. With \cmd{if}, the \texttt{<condition>} to be evaluated
must always come down to a scalar, but the query operator just
requires that the condition is of ``suitable'' type in light of the
types of the operands.  So, for example, suppose you have a square
matrix \texttt{A} and you want to switch the sign of the negative
elements of \texttt{A} on and above its diagonal. You could use a loop
(see below) and write a piece of code such as
\begin{code}
matrix A = mnormal(4,4)
matrix B = A

loop r = 1 .. rows(A)
  loop c = r .. cols(A)
     if A[r,c] < 0
       B[r,c] = -A[r,c]
     endif
  endloop
endloop
\end{code}

By using the ternary operator, you can achieve the same effect via a
considerably shorter (and faster) construct:
\begin{code}
matrix A = mnormal(4,4)
matrix B = upper(A.<0) ? -A : A
\end{code}

\tip{At this point some readers may be thinking ``Well, this may be as
  cool as you want, but it's way too complicated for me; I'll just use
  the traditional \cmd{if}''. Of course, there's nothing wrong with
  that, but in some cases the ternary assignment operator can lead to
  substantially faster code, and it becomes surprisingly natural when
  one gets used to it.}

\section{Loops}
\label{sec:hr-loops}

The basic hansl command for looping is (doh!) \cmd{loop}, and
takes the form
\begin{code}
loop <control-expression> <options>
    ...
endloop
\end{code}
In other words, the pair of statements \cmd{loop} and \cmd{endloop}
enclose the statements to repeat. Of course, loops can be nested.
Several variants of the \texttt{<control-expression>} for a loop are
supported, as follows:
\begin{enumerate}
\item unconditional loop
\item while loop
\item index loop
\item foreach loop
\item for loop.
\end{enumerate}
These variants are briefly described below.

\subsection{Unconditional loop}

This is the simplest variant. It takes the form
\begin{code}
loop <times>
   ...
endloop
\end{code}
where \texttt{<times>} is any expression that evaluates to a scalar,
namely the required number of iterations. This is only evaluated at
the beginning of the loop, so the number of iterations cannot be
changed from within the loop itself. Example:
\begin{code}
# triangular numbers
scalar n = 6
scalar count = 1
scalar x = 0
loop n
    scalar x += count
    count++
    print x
endloop
\end{code}
yields
\begin{code}
              x =  1.0000000
              x =  3.0000000
              x =  6.0000000
              x =  10.000000
              x =  15.000000
              x =  21.000000
\end{code}

Note the usage of the increment (\texttt{count++}) and of the
inflected assignment (\texttt{x += count}) operators.

\subsection{Index loop}

The unconditional loop is used quite rarely, as in most cases it is
useful to have a counter variable (\texttt{count} in the previous
example). This is easily accomplished via the \emph{index loop}, whose
syntax is
\begin{code}
loop <counter>=<min>..<max>
   ...
endloop
\end{code}
The limits \texttt{<min>} and \texttt{<max>} must evaluate to scalars;
they are automatically turned into integers if they have a fractional
part. The \texttt{<counter>} variable is started at \texttt{<min>} and
incremented by 1 on each iteration until it equals \texttt{<max>}.

The counter is ``read-only'' inside the loop. You can access either
its numerical value through the scalar \texttt{i} or use the accessor
\dollar{i}, which will perform \emph{string substitution}: inside the
loop, the hansl interpreter will substitute for the expression
\dollar{i} the string representation of the current value of the index
variable. An example should made this clearer: the following input
\begin{code}
scalar a_1 = 57
scalar a_2 = 85
scalar a_3 = 13

loop i=1..3
    print i a_$i
endloop
\end{code}
%$
has for output
\begin{code}
    i = 1.0000000
  a_1 = 57.000000
    i = 2.0000000
  a_2 = 85.000000
    i = 3.0000000
  a_3 = 13.000000
\end{code}

In the example above, at the first iteration the value of \texttt{i}
is 1, so the interpreter expands the expression \verb|a_$i| to
\verb|a_1|, finds that a scalar by that name exists, and prints
it. The same happens through the rest of the iterations. If one of the
automatically constructed identifiers had not been defined, execution
would have stopped with an error.

\subsection{While loop}

Here you have
\begin{code}
loop while <condition>
   ...
endloop
\end{code}
where \texttt{<condition>} should evaluate to a scalar, which is
re-evaluated at each iteration. Looping stops as soon as
\texttt{<condition>} becomes false (0). If \texttt{<condition>}
becomes \texttt{NA}, an error is flagged and execution stops.  By
default, \texttt{while} loops cannot exceed 100,000 iterations. This is
intended as a safeguard against potentially infinite loops. This
setting can be overridden if necessary by setting the
\dtk{loop_maxiter} state variable to a different value.
% (see chapter \ref{chap:settings}).

\subsection{Foreach loop}
\label{sec:loop-foreach}

In this case the syntax is
\begin{code}
loop foreach <counter> <catalogue>
   ...
endloop
\end{code}
where \texttt{<catalogue>} can be either a collection of
space-separated strings, or a variable of type \texttt{list} (see
section \ref{sec:lists}). The counter variable automatically takes on
the numerical values 1, 2, 3, and so on as execution proceeds, but its
string value (accessed by prepending a dollar sign) shadows the names
of the series in the list or the space-separated strings; this sort of
loop is designed for string substitution.

Here is an example in which the \texttt{<catalogue>} is a collection
of names of functions that return a scalar value when given a scalar
argument.
\begin{code}
scalar x = 1
loop foreach f sqrt exp ln
    scalar y = $f(x)
    print y
endloop
\end{code}
%$
This will produce
\begin{code}
              y =  1.0000000
              y =  2.7182818
              y =  0.0000000
\end{code}

\subsection{For loop}

The final form of loop control emulates the \cmd{for} statement in the
C programming language.  The syntax is \texttt{loop for}, followed by
three component expressions, separated by semicolons and surrounded by
parentheses, that is
\begin{code}
loop for (<init>; <cont>; <modifier>)
   ...
endloop
\end{code}

The three components are as follows:
\begin{enumerate}
\item Initialization (\texttt{<init>}): this must be an assignment
  statement, evaluated at the start of the loop.
\item Continuation condition (\texttt{<cont>}): this is evaluated at
  the top of each iteration (including the first).  If the expression
  evaluates as true (non-zero), iteration continues, otherwise it
  stops. 
\item Modifier (\texttt{<modifier>}): an expression which modifies the
  value of some variable.  This is evaluated prior to checking the
  continuation condition, on each iteration after the first.
\end{enumerate}

Here's an example, in which we find the square root of a number by
successive approximations:
\begin{code}
# find the square root of x iteratively via Newton's method
scalar x = 256
d = 1
loop for (y=(x+1)/2; abs(d) > 1.0e-7; y -= d/(2*y))
    d = y*y - x
    printf "y = %15.10f, d = %g\n", y, d
endloop

printf "sqrt(%g) = %g\n", x, y
\end{code}
Running the example gives
\begin{code}
y =  128.5000000000, d = 16256.3
y =   65.2461089494, d = 4001.05
y =   34.5848572866, d = 940.112
y =   20.9934703720, d = 184.726
y =   16.5938690915, d = 19.3565
y =   16.0106268314, d = 0.340172
y =   16.0000035267, d = 0.000112855
y =   16.0000000000, d = 1.23919e-11

Number of iterations: 8

sqrt(256) = 16
\end{code}
Be aware of the limited precision of floating-point arithmetic. For
example, the code snippet below will iterate forever on most platforms
because \texttt{x} will never equal \textit{exactly} 0.01, even though
it might seem that it should.
\begin{code}
loop for (x=1; x!=0.01; x=x*0.1)
    printf "x = .18g\n", x
endloop  
\end{code}
However, if you replace the condition \texttt{x!=0.01} with
\texttt{x>=0.01}, the code will run as (probably) intended.
 
\subsection{Loop options}

Three options can be given to the \cmd{loop} statement. One is
\option{verbose}. This has simply the effect of printing extra output
to trace progress of the loop; it has no other effect and the
semantics of the loop contents remain unchanged.

The \option{decr} option can be applied to an index loop to indicate
that the counter should be decremented by 1, not incremented, on each
iteration. Note that the default behavior with an index loop is that
the code is skipped altogether if the starting index value exceeds the
ending value.

The \option{progressive} option is mostly used as a quick and
efficient way to set up simulation studies. When this option is given,
a few commands (notably \cmd{print} and \cmd{store}) are given a
special, \emph{ad hoc} meaning. Please refer to \GUG\ for more
information.
 
\subsection{Breaking and continuing}
\label{sec:loop-break}

The \cmd{break} command makes it possible to break out of a loop if
necessary. Note that if you nest loops, \cmd{break} in the innermost
loop will interrupt that loop only and not the outer ones.  Here is an
example in which we use the \texttt{while} variant of the \cmd{loop}
statement to perform calculation of the square root in a manner
similar to the example above, using \cmd{break} to jump out of the
loop when the job is done.
\begin{code}
scalar x = 256
scalar y = 1
loop while 1
    d = y*y - x
    if abs(d) < 1.0e-7
        break
    else
        y -= d/(2*y)
        printf "y = %15.10f, d = %g\n", y, d
    endif
endloop

printf "sqrt(%g) = %g\n", x, y
\end{code}

The \cmd{continue} command can be used to short-circuit an iteration:
execution jumps from the line on which \cmd{continue} occurs to the
top of the loop. Iteration will then proceed if the continuation
condition is met. This can promote efficiency if a condition is met at
a certain point in an iteration such that the subsequent code becomes
irrelevant.

\section{The \cmd{catch} modifier}

Hansl offers a simple form of exception handling via the \cmd{catch}
keyword. This is not a command in its own right but can be used as a
prefix to most regular commands: the effect is to prevent termination
of a script if an error occurs in executing the command. If an error
does occur, this is registered in an internal error code which can be
accessed as \dollar{error} (a zero value indicating success). The
value of \dollar{error} should always be checked immediately after
using \texttt{catch}, and appropriate action taken if the command
failed. Here is a simple example:

\begin{code}
matrix a = floor(2*muniform(2,2))
catch ai = inv(a)
scalar err = $error
if err
    printf "The matrix\n%6.0f\nis singular!\n", a
else
    print ai
endif
\end{code}
%$

Note that the catch keyword cannot be used before \cmd{if}, \cmd{elif}
or \cmd{endif}. In addition, it should not be used on calls to
user-defined functions; it is intended for use only with gretl
commands and calls to ``built-in'' functions or operators. Suppose
you're writing a function package which includes some subsidiary
functionality which may fail under certain conditions, and you want to
prevent such failure from aborting execution. In that case you should
use \cmd{catch} \textit{within} the particular function in question,
and if an error condition is detected, signal this to the caller by
returning a suitable ``invalid'' value---say, \texttt{NA} (for a
function that returns a scalar) or an empty matrix. For example:

\begin{code}
function scalar may_fail (matrix *m)
  catch scalar x = ... # call to built-in procedure
  if $error
    x = NA
  endif
  return x
end function

function scalar caller (...)
  matrix m = ... # whatever
  scalar x = may_fail(&m)
  if na(x)
    print "Couldn't calculate x"
  else
    printf "Calculated x = %g\n", x
  endif
end function
\end{code}

What you should \textit{not} do here is apply catch to
\dtk{may_fail()}

\begin{code}
function scalar caller (...)
  matrix m = ... # whatever
  catch scalar x = may_fail(&m) # No, don't do this!
  ...
end function
\end{code}

as this is likely to leave gretl in a confused state.

\section{The \cmd{quit} statement}

When the \cmd{quit} statement is encountered in a hansl script,
execution stops. If the command-line program \app{gretlcli} is running
in batch mode, control returns to the operating system; if gretl is
running in interactive mode, gretl will wait for interactive input.

The \texttt{quit} command is rarely used in scripts since execution
automatically stops when script input is exhausted, but it could be
used in conjunction with \cmd{catch}. A script author could arrange
matters so that on encountering a certain error condition an
appropriate message is printed and the script is halted. Another use
for \texttt{quit} is in program development: if you want to inspect the
output of an initial portion of a complex script, the most convenient
solution may to insert a temporary ``quit'' at a suitable point.

%%% Local Variables: 
%%% mode: latex
%%% TeX-master: "hansl-primer"
%%% End: 

\chapter{User-written functions}

Hansl natively provides a reasonably wide array of pre-defined
functions for manipulating variables of all kinds; the previous
chapters contain several examples. However, it is also possible to
extend hansl's native capabilities by defining additional
functions.

Here's how a user-defined function looks like:
\begin{flushleft}
\texttt{function \emph{type} \emph{funcname}(\emph{parameters})}\\
   \quad \ldots\\
   \quad \texttt{\emph{function body}}\\
   \quad \ldots \\
\texttt{end function}
\end{flushleft}

The opening line of a function definition contains these elements, in
strict order:

\begin{enumerate}
\item The keyword \texttt{function}.
\item \texttt{\emph{type}}, which states the type of value returned by the
  function, if any.  This must be one of \texttt{void} (if the
  function does not return anything), \texttt{scalar},
  \texttt{series}, \texttt{matrix}, \texttt{list} or \texttt{string}.
\item \texttt{\emph{funcname}}, the unique identifier for the
  function.  Function names have a maximum length of 31 characters;
  they must start with a letter and can contain only letters, numerals
  and the underscore character. It cannot coincide with a native gretl
  command or function.
\item The functions's \texttt{\emph{parameters}}, in the form of a
  comma-separated list enclosed in parentheses.
\end{enumerate}

Function parameters can be of any of the types shown below.

\begin{center}
\begin{tabular}{ll}
  \multicolumn{1}{c}{Type} & 
  \multicolumn{1}{c}{Description} \\ [4pt]
  \texttt{bool}   & scalar variable acting as a Boolean switch \\
  \texttt{int}    & scalar variable acting as an integer  \\
  \texttt{scalar} & scalar variable \\
  \texttt{series} & data series (see section~\ref{sec:series})\\
  \texttt{list}   & named list of series  (see section~\ref{sec:lists})\\
  \texttt{matrix} & matrix or vector \\
  \texttt{string} & string variable or string literal \\
  \texttt{bundle} & all-purpose container
\end{tabular}
\end{center}

Each element in the listing of parameters must include two terms: a
type specifier, and the name by which the parameter shall be known
within the function.  An example follows:
%    
\begin{code}
function scalar myfunc (series y, list xvars, bool verbose)
\end{code}

In order to get a feel for how functions work, here's a simple
example:

\begin{code}
function scalar quasi_log(scalar x)
/* popular approximation to the natural logarithm
   via Pad� polynomials */

   if (x<0)
      scalar ret = NA
   else 
      scalar ret = 2*(x-1)/(x+1)
   endif

   return ret
end function

loop for (x=0.5; x<2; x+=0.1)
   printf "x = %4.2f; ln(x) = %g, approx = %g\n", x, ln(x), quasi_log(x)
end loop
\end{code}

The code above computes the rational function
\[
  f(x) = 2 \cdot \frac{x-1}{x+1} ,
\]
which provides a decent approximation to the natural logarithm in a
neighborhood of 1.

\begin{enumerate}
\item We begin by defining the function via the \cmd{function}
  keyword; the function definition will end with the \texttt{end
    function} marker below;
\item since the function is meant to return a scalar, we put the
  keyword \texttt{scalar} after \cmd{function};
\item between the round brackets, the arguments follow; in this case,
  we only have one, which we call \texttt{x} and is assumed to be a
  scalar;
\item on the next line, the function definition begins: this can be
  (almost) arbitrary hansl code; in this case, it includes a comment,
  and an \cmd{if} block;
\item the function ends with the \cmd{return} keyword, which exports
  the result;
\item the next lines provide a simple usage example; note that in the
  \cmd{printf} command, the two functions \cmd{ln()} and
  \cmd{quasi\_log()} are indistinguishable from a purely syntactic
  viewpoint, although the former is a native function and the
  latter is a user-defined one. 
\end{enumerate}

In many cases, you may end up writing several functions, which may be
quite long; in order to avoid cluttering your script with the function
definition, hansl provides the \cmd{include} command, so you can put all
your function definition in a separate file (or separate files if
necessary). For example, imagine you saved the \verb|quasi_log()|
function definition above in a separate file called
\verb|quasilog_def.inp|: the code above could be written more
compactly as
\begin{code}
include quasilog_def.inp

loop for (x=0.5; x<2; x+=0.1)
   printf "x = %4.2f; ln(x) = %g, approx = %g\n", x, ln(x), quasi_log(x)
end loop
\end{code}
Moreover, \cmd{include} commands can be nested.


\section{Parameter passing and return values}
\label{sec:params-returns}

Important points to remember:
\begin{itemize}
\item parameters are passed \emph{by value}; if you need a function to
  modify its arguments, you ought to use pointers (see below);
\item In hansl, you cannot use global variables
\item If you have to pass many parameters, you might want to wrap them
  into a bundle.
\end{itemize}

\subsection{Pointers}

Each of the type-specifiers, with the exception of \texttt{list} and
\texttt{string}, may be modified by prepending an asterisk to the
associated parameter name, as in
%    
\begin{code}
function scalar myfunc (series *y, scalar *b)
\end{code}

The meaning of this modification is related to the use of pointer
arguments in the C programming language.

For matrix arguments, this is a nice way to write faster functions, as
producing a copy of a large matrix can be quite
time-consuming. However, the \cmd{const} qualifier achieves the same
effect.

\section{Recursion}

Obligatory factorial example
\begin{code}
function scalar factorial(scalar n)
    if (n<0) || (n>floor(n))
        # filter out everything that isn't a 
        # non-negative integer
        return NA
    elif n==0
        return 1
    else
        return n*factorial(n-1)
    endif
end function

loop i=0..6 --quiet
    printf "%d! = %d\n", i, factorial(i)
end loop
\end{code}

Note: this is fun, but in practice, you'll be much better off using
the pre-cooked gamma function (or, better still, its logarithm).

%%% Local Variables: 
%%% mode: latex
%%% TeX-master: "hansl-primer"
%%% End: 


\part{With a dataset}
\label{part:hp-data}
\chapter{What is a dataset?}
\label{chap:dataset}

A dataset is a memory area designed to hold the data you want to work
on, if any. It may be thought of a big global variable, containing a
(possibly huge) matrix of data and a hefty collection of metadata.

\app{R} users may think that a dataset is similar to what you get when
you \texttt{attach} a data frame in \app{R}. Not really: in hansl, you
cannot have more than one dataset open at the same time. That's why we
talk about \emph{the} dataset.

When a dataset is present in memory (that is, ``open''), a number of
objects become available for your hansl script in a transparent and
convenient way. Of course, the data themselves: the columns of the
dataset matrix are called \emph{series}, which will be described in
section \ref{sec:series}; sometimes, you will want to organize one or
more series in a \emph{list} (section \ref{sec:lists}). Additionally,
you have the possibility of using, as read-only global variables, some
scalars or matrices, such as the number of observations, the number of
variables, the nature of your dataset (cross-sectional, time series or
panel), and so on. These are called \emph{accessors}, and will be
discussed in section \ref{sec:accessors}.

You can open a dataset by reading data from a disk file, via the
\cmd{open} command, or by creating one from scratch.

\section{Creating a dataset from scratch}

The primary commands in this context are \cmd{nulldata} and
\cmd{setobs}.  For example:
\begin{code}
set echo off
set messages off

set seed 443322           # initialize the random number generator
nulldata 240              # stipulate how long your series will be
setobs 12 1995:1          # define as monthly dataset, starting Jan 1995   
\end{code}

For more details see \GUG, and the \GCR\ for the \cmd{nulldata} and
\cmd{setobs} commands. The only important thing to say at this point,
however, is that you can resize your dataset and/or change some of its
characteristics, such as its periodicity, at nearly any point inside
your script if necessary.

Once your dataset is in place, you can start populating it with
series, either by reading them from files or by generating them via
appropriate commands and functions.

\section{Reading a dataset from a file}

The primary commands here are \cmd{open}, \cmd{append} and \cmd{join}.

The \cmd{open} command is what you'll want to use in most cases. It
handles transparently a wide variety of formats (native, CSV,
spreadsheet, data files produced by other packages such as
\textsf{Stata}, \textsf{Eviews}, \textsf{SPSS} and \textsf{SAS}) and
also takes care of setting up the dataset for you automatically.
\begin{code}
  open mydata.gdt    # native format
  open yourdata.dta  # Stata format
  open theirdata.xls # Excel format
\end{code}

The \cmd{open} command can also be used to read stuff off the
Internet, by using a URL instead of a filename, as in
\begin{code}
  open http://someserver.com/somedata.csv
\end{code}

The \textit{Gretl User's Guide} describes the requirements on plain
text data files of the ``CSV'' type for direct importation by
gretl. It also describes gretl's native data formats (XML-based and
binary).

The \cmd{append} and \cmd{join} commands can be used to add further
series from file to a previously opened dataset. The \cmd{join}
command is extremely flexible and has a chapter to itself in
\GUG.

\section{Saving datasets}

The \cmd{store} command is used to write the current dataset (or a
subset) out to file. Besides writing in gretl's native formats,
\cmd{store} can also be used to export data as CSV or in the format of
\textsf{R}. Series can be written out as matrices using the
\texttt{mwrite} function. If you have special requirements that are
not met by \cmd{store} or \cmd{mwrite} it is possible to use
\cmd{outfile} plus \cmd{printf} (see chapter~\ref{chap:formatting})
to gain full control over the way data are saved.


\section{The \cmd{smpl} command}

Once you have opened a dataset somehow, the \cmd{smpl} command allows
you to discard observations selectively, so that your series will
contain only the observations you want (automatically changing the
dimension of the dataset in the process). See chapter 4 in \GUG\ for
further information.\footnote{Users with a Stata background may find
  the hansl way of doing things a little disconcerting at first. In
  hansl, you first restrict your sample through the \cmd{smpl}
  command, which applies until further notice, then you do what you
  have to. There is no equivalent to Stata's \texttt{if} clause to
  commands.}

There are basically three variants to the \cmd{smpl} command:
\begin{enumerate}
\item Selecting a contiguous subset of observations: this will be
  mostly useful with time-series datasets. For example:
  \begin{code}
    smpl 4 122            # select observations for 4 to 122
    smpl 1984:1 2008:4    # the so-called "Great Moderation" period
    smpl 2008-01-01 ;     # form January 1st, 2008 onwards
  \end{code}
\item Selecting observations on the basis of some criterion: this is
  typically what you want with cross-sectional datasets. Example:
  \begin{code}
    smpl male == 1 --restrict                # males only
    smpl male == 1 && age < 30 --restrict    # just the young guys
    smpl employed --dummy                    # via a dummy variable
  \end{code}
  Note that, in this context, restrictions go ``on top of'' previous
  ones. In order to start from scratch, you either reset the full
  sample via \texttt{smpl full} or use the \option{replace} option
  along with \option{restrict}.
\item Restricting the active dataset to some observations so that a
  certain effect is achieved automatically: for example, drawing a
  random subsample, or ensuring that all rows that have missing
  observations are automatically excluded. This is achieved via the
  \option{no-missing}, \option{contiguous}, and \option{random}
  options.
\end{enumerate}

In the context of panel datasets, some extra qualifications have to be
made; see \GUG.

\section{Dataset accessors}
\label{sec:accessors}

Several characteristics of the current dataset can be determined by
reference to built-in accessor (``dollar'') variables. The main ones,
which all return scalar values, are shown in
Table~\ref{tab:dataset-accessors}.

\begin{table}[htbp]
  \centering
  \begin{tabular}{lp{0.7\textwidth}}
    \textbf{Accessor} & \textbf{Value returned} \\ \hline
    \verb|$datatype| & Coding for the type of dataset: 
    0 = no data; 1 = cross-sectional (undated); 2 = time-series;
    3 = panel \\
    \verb|$nobs| & The number of observations in the current 
    sample range \\
    \verb|$nvars| & The number of series (including the constant)\\
    \verb|$pd| & The data frequency (1 for cross-sectional, 4 for
    quarterly, and so on) \\
    \verb|$t1| & 1-based index of the first observation in the
    current sample \\
    \verb|$t2| & 1-based index of the last observation in the
    current sample \\
    \hline
  \end{tabular}
  \caption{The principal dataset accessors}
  \label{tab:dataset-accessors}
\end{table}

In addition there are a few more specialized accessors:
\dollar{obsdate}, \dollar{obsmajor}, \dollar{obsminor},
\dollar{obsmicro} and \dollar{unit}. These are specific to time-series
and/or panel data, and they all return series. See the \GCR{} for
details. 



%%% Local Variables: 
%%% mode: latex
%%% TeX-master: "hansl-primer"
%%% End: 

\chapter{Серии и списки}
\label{chap:series-etc}

Тогда как скаляры, матрицы и строки могут использоваться в сценарии
Hansl в любой момент; серии и списки по своей сути привязаны к набору
данных и поэтому могут использоваться только в том случае, если открыт
набор данных.

\section{Тип серии: \texttt{series}}
\label{sec:series}

Серии --- это то, что любой прикладной экономист назвал бы
«переменными», то есть повторяющиеся наблюдения в некотором заданном
количестве; набор данных --- это упорядоченный массив серий,
дополненный информацией, например, такой, как характер данных
(временной ряд, поперечное сечение или панель), описательные метки для
ряда и/или наблюдения, исходная информация и так далее.

Серии --- это базовый тип данных, от которого зависят все встроенные
оценочные команды gretl.  Серии, принадлежащие набору данных,
именуются стандартными идентификаторами Hansl (строки не более 31
символа, как описано выше). Что касается команд, которые используют
серии в качестве аргументов, они могут идентифицировать серии либо по
имени, либо по идентификатору, то есть по индексу серии в наборе
данных. Позиция 0 в наборе данных всегда является автоматической
«переменной», известной как const, которая представляет собой просто
столбец с единицами.

Идентификаторы фактических серий данных можно отобразить с помощью
команды \cmd{varlist}. (Но обратите внимание, что в функции вызова, в
отличие от команд, серии должны обозначаться только по имени.)
Подробное описание того, как работать с наборами данных, можно найти в
главе 4 Руководства пользователя Gretl.

Ниже приведены некоторые основные правила, касающиеся серий:

\begin{itemize}
\item Если \texttt{lngdp} принадлежит к временному ряду или набору
  данных панели, то синтаксис \texttt{lngdp(-1)} выдает первый лаг, а
  \texttt{lngdp(+1)} его первый лид.
\item Для доступа к отдельным элементам серии используйте квадратные
  скобки, заключающие что-то одно из следующего списка:
  \begin{itemize}
  \item прогрессивный (с 1) номер нужного вам наблюдения, как в
    \verb|lngdp[15]|
  \item соответствующий код даты в случае данных временного ряда, как
    в \verb|lngdp[2008:4]| (для 4-го квартала 2008 г.)
  \item соответствующую строку маркера наблюдения, если она есть в
    наборе данных, как например, \verb|GDP["USA"]|.
  \end{itemize}
\end{itemize}

Правила присвоения значений сериям такие же, как и для других
объектов, поэтому следующие примеры должно быть понятны:
\begin{code}
  series k = 3         # неявное преобразование из скаляра; постоянный ряд (серия)
  series x = normal()  # псевдо-rv через встроенную функцию
  series s = a/b       # поэлементная работа с существующими сериями

  series movavg = 0.5*(x + x(-1)) # с использованием лагов
  series y[2012:4] = x[2011:2]    # с использованием индивидуальных точек данных
  series x2000 = 100*x/x[2000:1]  # с построением индекса
\end{code}

\tip{В Hansl нет отдельных команд для создания и изменения
  серий. Другие популярные пакеты различают эти понятия, но мы все еще
  не нашли ответ, почему это различие может быть полезным.}

\subsection{Преобразование серий в матрицы и обратно}

Причина, по которой Hansl использует серии как отдельный и
определенный тип данных, отличный от матриц, уже ушла в историю. Тем
не менее, это тоже очень удобная функция. Операции, которые обычно
выполняются при помощи серий в прикладной работе, может быть неудобно
или сложно реализовать с использованием «сырых» матриц --- например,
вычисление опережения и запаздывания или регулярные и сезонные
различия; обработка пропущенных значений; добавление описательных
меток и т. д.

В любом случае, преобразовать данные в любом направлении между типами
серий и матриц не составит труда.

\begin{itemize}
\item Чтобы преобразовать ряды в матрицы, используйте синтаксис
  фигурных скобок, как в
  \begin{code}
    matrix MACRO = {outputgap, unemp, infl}
  \end{code}
  где также можно использовать списки; количество строк итоговой
  матрицы будет зависеть от вашей текущей выборки.
\item Чтобы превратить матрицы в серии, вы можете просто использовать
  столбцы матрицы, как в
  \begin{code}
    series y = my_matrix[,4]
  \end{code}
  Но имейте ввиду, что это сработает только в том случае, если
  количество строк в \texttt{my\_matrix} соответствует длине набора
  данных (или текущему диапазону выборки).
\end{itemize}
Также обратите внимание на то, что функции \cmd{lincomb} и
\cmd{filter} весьма полезны для создания серий и управления ими в
сложных случаях без необходимости преобразовывать данные в матричную
форму (что может быть накладно с точки зрения вычислений больших
наборов данных).

\subsection{Тернарный оператор с сериями}

Рассмотрим это задание:

\begin{code}
  worker_income = employed ? income : 0
\end{code}
Здесь мы предполагаем, что занятый \verb|employed| --- это фиктивный
ряд, кодирующий статус сотрудника. Его значение будет проверено при
каждом наблюдении в текущем диапазоне выборки и значение, присвоенное
\texttt{worker\_income} в этом наблюдении будет соответственно
определено. Следовательно, это же самое содержится в следующей более
подробной формулировке, где \dollar{t1} и \dollar{t2} --- аксессоры
для начала и конца диапазона выборки:

\begin{code}
series worker_income
loop i=$t1..$t2
    if employed[i]
        worker_income[i] = income[i]
    else
        worker_income[i] = 0
    endif
endloop
\end{code}

\section{Тип: список \texttt{list}}
\label{sec:lists}

На языке Hansl, список \textit{list} представляет собой массив целых
чисел, представляющих идентификационные номера набора серий. По этой
причине наиболее распространенные операции со списками --- это
операции с наборами серий, такие как добавление или удаление
элементов, объединение, пересечение и так далее. Однако, в отличие от
наборов, списки Hansl упорядочены, поэтому к отдельным членам списка
можно получить доступ через синтаксис [], как в\texttt{X[3]}.

Есть несколько способов присвоить значения списку. Самый простой вид
выражения, который работает в этом контексте --- это список серий,
разделенных пробелами, по имени или по номеру ID. Например,

\begin{code}
list xlist = 1 2 3 4
list reglist = income price 
\end{code}
Пустой список получается с помощью ключевого слова \texttt{null}, как
в
\begin{code}
list W = null  
\end{code}
или с помощью простого объявления. Еще несколько специальных форм
(например, с использованием подстановочных знаков) описаны в
Руководстве пользователя Gretl.

Основная идея состоит в том, чтобы использовать списки для группировки
под одним идентификатором одной или нескольких серий, которые
логически могут быть состыкованы вместе (например, в качестве
независимых переменных в модели). Так, например,

\begin{code}
list xlist = x1 x2 x3 x4
ols y 0 xlist
\end{code}
это способ определения регрессии OМНК, который также может быть
записан как
\begin{code}
ols y 0 x1 x2 x3 x4
\end{code}
Обратите внимание, что мы использовали здесь условие, упомянутое в
разделе \ref{sec:series}, согласно которому серию можно
идентифицировать по ID при использовании в качестве аргумента, если
ввести 0 вместо const.

Списки можно объединять, как в \texttt{list L3 = L1 L2} (где L1 и L2
--- имена существующих списков).  Однако это не обязательно то, что вы
хотите, поскольку полученный список может содержать дубликаты. Чаще
всего рекомендуется использовать следующие операции над наборами:

\begin{center}
  \begin{tabular}{rl}
    \textbf{Оператор} & \textbf{Значение} \\
    \hline
    \verb,||, & Объединение \\
    \verb|&&| & Пересечение \\
    \verb|-|  & Различие \\
    \hline
  \end{tabular}
\end{center}

Если \texttt{L1} и \texttt{L2} --- существующие списки, после
выполнения следующего фрагмента кода
\begin{code}
  list UL = L1 || L2 
  list IL = L1 && L2
  list DL = L1 - L2
\end{code}
список \texttt{UL} будет содержать все элементы \texttt{L1} плюс любые
элементы \texttt{L2}, которых еще нет в \texttt{L1}; \texttt{IL} будет
содержать все элементы, которые присутствуют как в L1, так и в L2, а
\texttt{DL} будет содержать все элементы L1, которых нет в L2.  Чтобы
добавить переменные к существующему списку, мы можем использовать
\textit{append} или \textit{prepend} и тот факт, что именованный
список создан пользователем «от руки». Например, предполагая, что
список \texttt{xlist} уже определен (возможно, как \texttt{null}), мы
сможем сделать следующее:

\begin{code}
list xlist = xlist 5 6 7
xlist = 9 10 xlist 11 12
\end{code}
Другой вариант для добавления или удаления условий из существующего
списка --- использовать \texttt{+=} или \texttt{-=}, соответственно,
как в

\begin{code}
xlist += cpi
zlist -= cpi
\end{code}
Хороший пример вышесказанного --- обычный скрипт: вы можете увидеть в
сценариях Hansl что-то вроде
\begin{code}
  list C -= const
  list C = const C
\end{code}
что гарантирует, что серия const будет включена (один раз) в список
\texttt{C} и будет первой.

\subsection{Преобразование списков в матрицы и обратно}

Как определено выше, идея преобразования из списка в матрицу может
быть реализована двумя способами. Вы можете превратить список в
матрицу (вектор), заполнив последнюю содержащимися в ней
идентификационными номерами из списка или же просто создаете матрицу,
столбцы которой содержат серии, на которые ссылаются
идентификаторы. Оба спобоса действенны (и потенциально полезны в
разных контекстах), поэтому Hansl позволяет вам действовать как
угодно.

Если вы назначите список матрице, как в

\begin{code}
  list L = moo foo boo zoo
  matrix A = L
\end{code}
матрица \texttt{A} будет содержать идентификационные номера четырех
серий в виде вектора-строки. Эта операция может быть осуществлена
обоими способами, так что заявление
\begin{code}
  list C = seq(7,10)
\end{code}
совершенно верно (при условии, конечно, что у вас есть не менее 10
серий в текущем открытом наборе данных).

Если вместо этого вы хотите создать матрицу данных из ряда,
принадлежащего данному списку, вы должны заключить название списка в
фигурные скобки, как в

\begin{code}
  matrix X = {L}
\end{code}

\subsection{Вариант цикла \texttt{foreach} со списками}

Списки могут использоваться в качестве «каталога» в варианте цикла
foreach (см. раздел \ref{sec:loop-foreach}). Это особенно удобно,
когда вам нужно выполнить какие-то операции с несколькими
сериями. Например, следующий синтаксис можно использовать для
вычисления и вывода среднего значения каждой из нескольких серий
\begin{code}
list X = age income experience
loop foreach i X
    printf "mean($i) = %g\n", mean($i)
endloop
\end{code}

%%% Local Variables: 
%%% mode: latex
%%% TeX-master: "hansl-primer"
%%% End: 

\chapter{Estimation methods}
\label{chap:estimation}

You can, of course, estimate econometric models via hansl without
having a dataset (in the sense in which we're using that term here) in
place---just as you might in \textsf{Matlab}, for instance. You'll
need \textit{data}, but these can be loaded in matrix form (see the
\texttt{mread} function in the \GCR), or generated artificially via
functions such as \texttt{mnormal} or \texttt{muniform}. You can roll
your own estimator using hansl's linear algebra primitives, and you
also have access to more specialized functions such as \texttt{mols}
(see section \ref{sec:mat-op}) and \texttt{mrls} (restricted least
squares) if you need them.

However, unless you need to use an estimation method which is not
currently supported by gretl, or have a strong desire to reinvent the
wheel, you will probably want to make use of the built-in estimation
commands available in hansl. These commands are series-oriented and
therefore require a dataset. They fall into two main categories:
``canned'' procedures, and generic tools that can be used to estimate
a wide variety of models based on common principles.

\section{Canned estimation procedures}
\label{sec:canned}

``Canned'' doesn't sound very appetizing these days but it's the term
that's commonly used. Basically it means two things, neither of them
in fact unappetizing.
\begin{itemize}
\item The user is presented with a fairly simple interface. A few
  inputs must be specified, and perhaps a few options selected, then
  the heavy lifting is done within the gretl library.
\item The algorithm is written in C, by experienced coders. It is
  therefore faster (possibly \textit{much} faster) than an
  implementation in an interpreted language such as hansl.
\end{itemize}

Such procedures share, more or less, the syntax
\begin{flushleft}
\texttt{\emph{commandname parameters options}}
\end{flushleft}
with a few exceptions (e.g.\ systems).

A crude categorization: 
\begin{description}
\item[Linear, single equation] \cmd{ols}, \cmd{tsls}, \cmd{ar1},
  \cmd{mpols}
\item[Linear, multi-equation] \cmd{system}, \cmd{var}, \cmd{vecm} 
\item[Nonlinear, single equation] \cmd{logit}, \cmd{probit},
  \cmd{poisson}, \cmd{negbin}, \cmd{tobit}, \cmd{intreg},
  \cmd{logistic}
\item[Panel] \cmd{panel}, \cmd{dpanel}
\item[Assorted] \cmd{arima}, \cmd{garch}, \cmd{heckit},
  \cmd{quantreg}, \cmd{lad}, \cmd{biprobit}, \cmd{duration}
\end{description}

Don't let names deceive you: for example, the \cmd{probit} command can
estimate ordered models, random-effect panel probit models, \dots{}

See the \GCR{} for details.

\subsection{Simultaneous systems}

The \cmd{system} block.

\section{Post-estimation accessors}
\label{sec:postest-accessors}

After having estimated a model, you can access most of the relevant
quantities via accessors.

\begin{itemize}
\item Generic: \dollar{coeff}, \dollar{vcv}, \dollar{uhat}
\item Model-specific: for example, \dollar{jbeta}, \dollar{h},
  \dollar{mnlprobs}
\end{itemize}

\subsection{Named models}

\begin{code}
diff y x
ADL <- ols y const y(-1) x(0 to -1)
ECM <- ols d_y const d_x y(-1) x(-1)
ssr_a = ADL.$ess
ssr_e = ECM.$ess # should be equal
\end{code}

\section{Generic estimation tools}
\label{sec:est-blocks}

These are useful if you want to write your own estimator. Here we give
a generic overview. See the relevant chapters in \GUG{} for a full explanation.

\begin{itemize}
\item \cmd{nls}
\item \cmd{mle}
\item \cmd{gmm}
\end{itemize}

\subsection{Formatting your output}

The \cmd{modprint} command.

\label{LastPage}


%%% Local Variables: 
%%% mode: latex
%%% TeX-master: "hansl-primer"
%%% End: 


%% \part{Further topics}
%% \label{part:hp-further}
%% \part{Further topics}

\chapter{The \texttt{set} command}
\label{chap:settings}

\chapter{Nice-looking output}
\label{chap:formatting}

\cmd{printf}, \cmd{sprintf} and the ``commands vs functions'' thing.

The \cmd{modprint} command.

Friendly relationship with \LaTeX{} and \app{gnuplot}

\chapter{Function packages}

\chapter{Going abroad}
``Foreign'' blocks.

\chapter{Tricks}
\section{String substitution}
\label{sec:stringsub}

String variables can be used in two ways in hansl scripting: the name
of the variable can be typed ``as is'', or it may be preceded by the
``at'' sign, \verb|@|. In the first variant the named string is
treated as a variable in its own right, while the second calls for
``string substitution''. Which of these variants is appropriate
depends on the context.

In the following contexts the names of string variables should be
given in plain form (without the ``at'' sign):

\begin{itemize}
\item When such a variable appears among the arguments to the
  commands \texttt{printf} or \texttt{sprintf}.
\item When such a variable is given as the argument to a function.
\item On the right-hand side of a \texttt{string} assignment.
\end{itemize}

Here is an illustration of the use of a named string argument with
\texttt{printf}:
%
\begin{code}
? string vstr = "variance"
Generated string vstr
? printf "vstr: %12s\n", vstr
vstr:     variance
\end{code}

String substitution, on the other hand, can be used in contexts where
a string variable is not acceptable as such. If gretl encounters
the symbol \verb|@| followed directly by the name of a string
variable, this notation is in effect treated as a ``macro'': the value
of the variable is sustituted literally into the command line before
the regular parsing of the command is carried out.

One common use of string substitution is when you want to construct
and use the name of a series programatically. For example, suppose you
want to create 10 random normal series named \texttt{norm1} to
\texttt{norm10}. This can be accomplished as follows.
%
\begin{code}
string sname = null
loop i=1..10
  sprintf sname "norm%d", i
  series @sname = normal()
endloop
\end{code}
%
Note that plain \texttt{sname} could not be used in the second line
within the loop: the effect would be to attempt to overwrite the
string variable named \texttt{sname} with a series of the same name,
hence generating an error. What we want is for the current
\textit{value} of \texttt{sname} to be dumped directly into the
command that defines a series, and the ``\verb|@|'' notation achieves
that.

Another typical use of string substitution is when you want the
options used with a particular command to vary depending on
some condition. For example,
%
\begin{code}
function void use_optstr (series y, list xlist, int verbose)
   string optstr = verbose ? "" : "--simple-print"
   ols y xlist @optstr 
end function

open data4-1
list X = const sqft
use_optstr(price, X, 1)
use_optstr(price, X, 0)
\end{code}

In the first call to the function \texttt{use\_optstr} the option
string \verb|--simple-print| will be appended to the \cmd{ols}
command; in the second call it will be omitted.

When printing the value of a string variable using the \texttt{print}
command, the plain variable name should generally be used, as in
%
\begin{code}
string s = "Just testing"
print s
\end{code}
%
The following variant is equivalent, though clumsy and not
recommended.
%
\begin{code}
string s = "Just testing"
print "@s"
\end{code}
%
But note that this next variant does something quite different.
%
\begin{code}
string s = "Just testing"
print @s
\end{code}
%
After string substitution, the command reads
%
\begin{code}
print Just testing
\end{code}
%
which is taken as a request to print the values of two variables,
\texttt{Just} and \texttt{testing}.

\subsection{Performance hint}

Warning: gretl attempts to speed up execution of assignment
expressions within loops by ``pre-compiling'' the expression.
However, if such an expression contains string substitution it cannot
be pre-compiled. So be sure to use the \verb|@| operator only when
strictly necessary if you want to avoid a performance penalty.

%%% Local Variables: 
%%% mode: latex
%%% TeX-master: "hansl-primer"
%%% End: 

% \clearpage
% \bibliographystyle{gretl}
\bibliography{gretl}

%%% Local Variables: 
%%% mode: latex
%%% TeX-master: "gretl-guide"
%%% End: 



\end{document}
