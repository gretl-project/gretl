\chapter{Persistent objects}
\label{persist}

%%% Work in progress, not really ready for translation yet.

\section{Named lists}
\label{named-lists}

Many \app{gretl} commands take one or more lists of variables as
arguments.  To make this easier to handle in the context of command
scripts, and in particular within user-defined functions, \app{gretl}
offers the possibility of \textit{named lists}.  

A named list is created using the keyword \texttt{list}, followed by
the name of the list, an equals sign, and either \texttt{null} (to
create an empty list) or one or more variables to be placed on the
list.  For example,

\begin{code}
list xlist = 1 2 3 4
list reglist = income price 
list empty_list = null
\end{code}

The name of the list must start with a letter, and must be composed
entirely of letters, numbers or the underscore character.  The maximum
length of the name is 31 characters; list names cannot contain
spaces.  When adding variables to a list, you can refer to them either
by name or by their ID numbers. 

Once a list has been created, it will be ``remembered' for the
duration of the \app{gretl} session.  Lists can be modified in two
ways.  To \textit{append variables} to an existing list, put the name
of the list to the right of the equals sign, before the variables to
be added, as in

\begin{code}
list xlist = xlist 5 6 7
\end{code}

In this example, if \texttt{xlist} originally contained variables 1,
2, 3 and 4, it will now contain variables 1 to 7.

To \textit{redefine} an existing list altogether, use the same syntax
as for creating a list.  For example

\begin{code}
list xlist = 1 2 3
list xlist = 4 5 6
\end{code}

After the second assignment, \texttt{xlist} contains just variables 4,
4 and 6.  

A named list can be used in any context where \app{gretl} expects a
list of variables.  FIXME need examples.

You can determine whether an unknown variable actually represents a list
using the function \texttt{islist()}.

\begin{code}
series xl1 = log(x1)
series xl2 = log(x2)
list xvars = xl1 xl2
genr isl_a = islist(xvars)
genr isl_b = islist(xl1)
\end{code}

The first \texttt{genr} command above will assign a value of 1 to
\texttt{isl\_a} since \texttt{xl} is in fact a named list.  The second
genr will assign 0 to \texttt{isl\_b} since \texttt{xl1} is a data
series, not a list.  

You can also determine the number of variables or elements in a list
using the function \texttt{nelem()}.

\begin{code}
list xlist = 1 2 3
genr nl = nelem(xlist)
\end{code}

The scalar \texttt{nl} will be assigned a value of 3 since
\texttt{xlist} contains 3 members.






    
%%% Local Variables: 
%%% mode: latex
%%% TeX-master: "gretl-guide"
%%% End: 

