\chapter{Cointegrazione e modelli Vector Error Correction}
\label{vecm-explanation}

\section{Il test di cointegrazione di Johansen}
\label{johansen-test}

Il test di Johansen per la cointegrazione deve tenere conto di quali
ipotesi vengono fatte a proposito dei termini deterministici, per cui
si possono individuare i ben noti ``cinque casi''. Una presentazione
approfondita dei cinque casi richiede una certa quantit� di algebra
matriciale, ma � possibile dare un'intuizione del problema per mezzo
di un semplice esempio.
    
Si consideri una serie $x_t$ che si comporta nel modo seguente
%      
\[ x_t = m + x_{t-1} + \varepsilon_t \]
%
dove $m$ � un numero reale e $\varepsilon_t$ � un processo ``white
noise''.  Come si pu� facilmente mostrare, $x_t$ � un ``random walk''
che fluttua intorno a un trend deterministico con pendenza $m$. Nel
caso particolare in cui $m$ = 0, il trend deterministico scompare e
$x_t$ � un puro random walk.
    
Si consideri ora un altro processo $y_t$, definito da
%      
\[ y_t = k + x_t + u_t \]
%
dove, ancora, $k$ � un numero reale e $u_t$ � un processo white noise.
Poich� $u_t$ � stazionario per definizione, $x_t$ e $y_t$ sono
cointegrate, ossia la loro differenza
%      
\[ z_t = y_t - x_t = k + u_t \]
%	
� un processo stazionario. Per $k$ = 0, $z_t$ � un semplice white
noise a media zero, mentre per $k$ $\ne$ 0 il processo $z_t$ � white
noise con media diversa da zero.
  
Dopo alcune semplici sostituzioni, le due equazioni precedenti possono
essere rappresentate congiuntamente come un sistema VAR(1)
%      
\[ \left[ \begin{array}{c} y_t \\ x_t \end{array} \right] = \left[
  \begin{array}{c} k + m \\ m \end{array} \right] + \left[
  \begin{array}{rr} 0 & 1 \\ 0 & 1 \end{array} \right] \left[
  \begin{array}{c} y_{t-1} \\ x_{t-1} \end{array} \right] + \left[
  \begin{array}{c} u_t + \varepsilon_t \\ \varepsilon_t \end{array}
\right] \]
%	
o in forma VECM
%      
\begin{eqnarray*}
  \left[  \begin{array}{c} \Delta y_t \\ \Delta x_t \end{array} \right]  & = & 
  \left[  \begin{array}{c} k + m \\ m \end{array} \right] +
  \left[  \begin{array}{rr} -1 & 1 \\ 0 & 0 \end{array} \right] 
  \left[  \begin{array}{c} y_{t-1} \\ x_{t-1} \end{array} \right] + 
  \left[  \begin{array}{c} u_t + \varepsilon_t \\ \varepsilon_t \end{array} \right] = \\
  & = & 
  \left[  \begin{array}{c} k + m \\ m \end{array} \right] +
  \left[  \begin{array}{r} -1 \\ 0 \end{array} \right]
  \left[  \begin{array}{rr} 1 & -1 \end{array} \right] 
  \left[  \begin{array}{c} y_{t-1} \\ x_{t-1} \end{array} \right] + 
  \left[  \begin{array}{c} u_t + \varepsilon_t \\ \varepsilon_t \end{array} \right] = \\
  & = &
  \mu_0 + \alpha \beta^{\prime} \left[  \begin{array}{c} y_{t-1} \\ x_{t-1} \end{array} \right] + \eta_t = 
  \mu_0 + \alpha z_{t-1} + \eta_t ,
\end{eqnarray*}
%	
dove $\beta$ � il vettore di cointegrazione e $\alpha$ � il vettore
dei ``loading'' o ``aggiustamenti''.
     
Possiamo ora considerare tre casi possibili:
    
\begin{enumerate}
\item $m \ne 0$: In questo caso $x_t$ ha un trend, come abbiamo appena
  visto; ne consegue che anche $y_t$ segue un trend lineare perch� in
  media si mantiene a una distanza di $k$ da $x_t$.  Il vettore
  $\mu_0$ non ha restrizioni. Questo � il caso predefinito per il
  comando \cmd{vecm} di gretl.
	
\item $m = 0$ e $k \ne 0$: In questo caso, $x_t$ non ha un trend, e di
  conseguenza neanche $y_t$.  Tuttavia, la distanza media tra $y_t$ e
  $x_t$ � diversa da zero. Il vettore $\mu_0$ � dato da
%	  
  \[
  \mu_0 = \left[ \begin{array}{c} k \\ 0 \end{array} \right]
  \]
%	    
  che � non nullo, quindi il VECM mostrato sopra ha un termine
  costante.  La costante, tuttavia � soggetta alla restrizione che il
  suo secondo elemento deve essere pari a 0. Pi� in generale,
  $\mu$\ensuremath{_{0}} � un multiplo del vettore $\alpha$. Si noti
  che il VECM potrebbe essere scritto anche come
%	  
  \[
  \left[ \begin{array}{c} \Delta y_t \\ \Delta x_t \end{array} \right]
  = \left[ \begin{array}{r} -1 \\ 0 \end{array} \right] \left[
    \begin{array}{rrr} 1 & -1 & -k \end{array} \right] \left[
    \begin{array}{c} y_{t-1} \\ x_{t-1} \\ 1 \end{array} \right] +
  \left[ \begin{array}{c} u_t + \varepsilon_t \\ \varepsilon_t
    \end{array} \right]
  \]
%	   
  che incorpora l'intercetta nel vettore di cointegrazione. Questo �
  il caso di ``costante vincolata''; pu� essere specificato nel
  comando \cmd{vecm} di gretl usando l'opzione \verb+--rc+.
	
\item $m = 0$ e $k = 0$: Questo caso � il pi� vincolante: chiaramente,
  n� $x_t$ n� $y_t$ hanno un trend, e la loro distanza media � zero.
  Anche il vettore $\mu_0$ vale 0, quindi questo caso pu� essere
  chiamato ``senza costante''.  Questo caso � specificato usando
  l'opzione \verb+--nc+ con \cmd{vecm}.
	
\end{enumerate}


Nella maggior parte dei casi, la scelta tra le tre possibilit� si basa
su un misto di osservazione empirica e di ragionamento economico. Se
le variabili in esame sembrano seguire un trend lineare, � opportuno
non imporre alcun vincolo all'intercetta. Altrimenti occorre chiedersi
se ha senso specificare una relazione di cointegrazione che includa
un'intercetta diversa da zero. Un esempio appropriato potrebbe essere
la relazione tra due tassi di interesse: in generale questi non hanno
un trend, ma il VAR potrebbe comunque avere un'intercetta perch� la
differenza tra i due (lo ``spread'' sui tassi d'interesse) potrebbe
essere stazionaria attorno a una media diversa da zero (ad esempio per
un premio di liquidit� o di rischio).
    
L'esempio precedente pu� essere generalizzato in tre direzioni:
    
\begin{enumerate}
\item Se si considera un VAR di ordine maggiore di 1, l'algebra si
  complica, ma le conclusioni sono identiche.
\item Se il VAR include pi� di due variabili endogene, il rango di
  cointegrazione $r$ pu� essere maggiore di 1. In questo caso $\alpha$
  � una matrice con $r$ colonne e il caso con la costante vincolata
  comporta che $\mu_0$ sia una combinazione lineare delle colonne di
  $\alpha$.
\item Se si include un trend lineare nel modello, la parte
  deterministica del VAR diventa $\mu_0$ + $\mu_1$. Il ragionamento �
  praticamente quello visto sopra, tranne per il fatto che
  l'attenzione � ora posta su $\mu_1$ invece che su $\mu_0$. La
  controparte del caso con ``costante vincolata'' discusso sopra � un
  caso con ``trend vincolato'', cos� che le relazioni di
  cointegrazione includono un trend, ma la differenza prima delle
  variabili in questione no. Nel caso di un trend non vincolato, il
  trend appare sia nelle relazioni di cointegrazione sia nelle
  differenze prime, il che corrisponde alla presenza di un trend
  quadratico nelle variabili (espresse in livelli). Questi due casi
  sono specificati rispettivamente con le opzioni \verb+--crt+ e
  \verb+--ct+ del comando \cmd{vecm}.
\end{enumerate}


%%% Local Variables: 
%%% mode: latex
%%% TeX-master: "gretl-guide-it"
%%% End: 

