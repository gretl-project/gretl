\chapter{Opzioni, argomenti e percorsi}
\label{optarg}

\section{Avvio di gretl}
\label{optarg1}


\cmd{gretl} (in MS Windows, \cmd{gretlw32.exe})\footnote{In Linux
  viene installato uno script ``wrapper'' chiamato \texttt{gretl}.
  Questo script controlla se sia impostata la variabile di ambiente
  \texttt{DISPLAY} e in caso positivo lancia la versione grafica del
  programma, altrimenti lancia la versione a riga di comando,
  \texttt{gretlcli}.}

--- Apre il programma e aspetta l'input dall'utente.
      
\cmd{gretl} \textsl{filedati}
      
--- Avvia il programma caricando nello spazio di lavoro il file di
dati indicato. Il file pu� essere in uno dei vari formati supportati
(si veda la \emph{Guida ai comandi di gretl}): il programma cercher�
di riconoscere il formato del file e di trattarlo in modo
appropriato. Si veda anche la Sezione \ref{path-search} per quanto
riguarda la ricerca dei percorsi.
      
\cmd{gretl --help} (o \cmd{gretl -h})
      
--- Mostra un breve riepilogo delle modalit� di uso ed esce.
      
\cmd{gretl --version} (o \cmd{gretl -v})
      
--- Mostra le informazioni sulla versione del programma ed esce.
      
\cmd{gretl --english} (o \cmd{gretl -e})
      
--- Forza l'utilizzo della versione inglese invece di quella italiana.
      
\cmd{gretl --run} \textsl{file} (o \cmd{gretl -r} \textsl{file})
      
--- Avvia il programma e apre una finestra che mostra il file di
comandi indicato, pronto per essere eseguito. Si veda la Sezione
\ref{path-search} per quanto riguarda la ricerca dei percorsi.
      
\cmd{gretl --db} \textsl{database} (o \cmd{gretl -d}
\textsl{database})
      
--- Avvia il programma e apre una finestra che mostra il database
indicato. Se i file del database (il file \texttt{.bin} e il relativo
file \texttt{.idx}, si veda la \emph{Guida ai comandi di gretl}) non
sono nella directory predefinita dei database, occorre specificare il
percorso completo dei file.
      
\cmd{gretl --dump} (o \cmd{gretl -c})
      
--- Scrive le informazioni di configurazione del programma in un file
di testo (il nome del file � mostrato nello standard output). Pu�
essere utile nella risoluzione di problemi di funzionamento.
      
\cmd{gretlw32 --debug} (o \cmd{gretlw32 -g})
      
--- (solo MS Windows) Apre una finestra di terminale per mostrare
eventuali messaggi inviati allo ``standard output'' o allo ``standard
error''. Questi messaggi, che di solito non sono visibili in Windows,
possono essere utili per risolvere problemi di funzionamento.
      
\section{Men� preferenze}
\label{guiprefs}

Varie funzionalit� di \app{gretl} sono configurabili nel men�
\textsf{Strumenti, Preferenze} menu.
      
Il men� contiene opzioni per la scelta del carattere a larghezza fissa
da usare nelle schermate di \app{gretl}, e, su alcune piattaforme, il
carattere usato per i men� e per gli altri messaggi.  Le altre opzioni
sono organizzate in cinque pagine, nel modo seguente.
      
Pagina \textbf{Generali}: qui � possibile configurare la directory
base per i file condivisi di \app{gretl}.  Inoltre, ci sono alcune
caselle di testo: selezionando ``Avvisa in caso di aggiornamenti di
gretl'', \app{gretl} cercher� di connettersi al server dei database
appena avviato; de-selezionando ``Mostra la barra degli strumenti di
gretl'', la barra degli strumenti sar� disabilitata.  Se
l'impostazione locale del proprio sistema non usa il punto
(``\texttt{.}'') come separatore delle cifre decimali, de-selezionando
``Usa le impostazioni locali per il punto decimale'', \app{gretl}
user� comunque il carattere punto. Selezionando ``Abilita comandi
shell'' sar� possibile invocare comandi della shell all'interno degli
script e nel terminale di \app{gretl} (questa funzionalit� �
disabilitata in modalit� predefinita per motivi di sicurezza).
      
Inoltre, in ``Generali'' � possibile scegliere l'algoritmo usato da
\app{gretl} per calcolare le stime dei minimi quadrati. La scelta
predefinita � la decomposizione di Cholesky, che � veloce,
relativamente economica in termini di requisiti di memoria, e accurata
per la maggior parte delle applicazioni. L'alternativa � la
decomposizione QR, che � computazionalmente pi� impegnativa e richiede
una maggiore quantit� di memoria, ma � pi� accurata.  � poco probabile
dover avere bisogno dell'accuratezza ulteriore fornita dalla
decomposizione QR, a meno che non si abbia a che fare con dati molto
mal-condizionati e si sia interessati a valori dei coefficienti e
degli errori standard precisi per pi� di 7 cifre
decimali\footnote{L'uso della decomposizione QR � attivabile anche
  impostando la variabile di ambiente \verb@GRETL_USE_QR@ a un
  qualsiasi valore non nullo.}.

Pagina \textbf{Database}: � possibile selezionare la directory in cui
iniziare a cercare i database in formato interno di \app{gretl} e
quella per i database in formato RATS 4; il nome host del server dei
database di \app{gretl} a cui connettersi, e l'indirizzo IP e il
numero di porta del proxy HTTP da usare per contattare il server dei
database, se necessario (se si � dietro a un firewall).
      
Pagina \textbf{Programmi}: � possibile specificare i nomi o i percorsi
per alcuni programmi esterni che possono essere chiamati da
\app{gretl} in alcuni casi. Si noti che la voce ``Comando per
compilare file {\TeX}'' pu� essere impostata come latex o pdflatex: se
si sceglie latex, le anteprime dei documenti {\TeX} saranno in formato
DVI; se si sceglie pdflatex, le anteprime saranno in formato PDF.
      
Pagina \textbf{HCCME}: contiene le preferenze riguardanti la stima
robusta della matrice di covarianza. Si veda la \emph{Guida ai comandi
  di gretl} per i dettagli.
      
Pagina \textbf{Manuali}: � possibile scegliere la lingua preferita per
la documentazione completa di \app{gretl} in formato PDF (al momento
sono supportati solo inglese e italiano). Se si usa la documentazione
inglese � possibile scegliere anche il formato della pagina, US letter
o A4.
      
Le impostazioni scelte usando il men� preferenze sono conservate da
una sessione di gretl all'altra. Su MS Windows vengono salvate nel
registro di Windows, mentre su altre piattaforme sono salvate in un
file di testo chiamato \texttt{.gretlrc} nella directory principale
dell'utente.


\section{Avvio di gretlcli}
\label{optarg2}

\cmd{gretlcli}

--- Apre il programma e attende l'input dall'utente.
      
\cmd{gretlcli} \textsl{filedati}

--- Avvia il programma caricando nello spazio di lavoro il file di
dati indicato. Il file pu� essere in uno qualsiasi dei formati
supportati da \app{gretl} (si veda la \emph{Guida ai comandi di gretl}
per i dettagli). Il programma cercher� di riconoscere il formato del
file e di trattarlo in modo appropriato. Si veda anche la Sezione
\ref{path-search} per quanto riguarda la ricerca dei percorsi.

\cmd{gretlcli --help} (o \cmd{gretlcli -h})

--- Mostra un breve riepilogo delle modalit� di uso ed esce.
      
\cmd{gretlcli --version} (o \cmd{gretlcli -v})

--- Mostra informazioni sulla versione del programma ed esce.
      
\cmd{gretlcli --english} (o \cmd{gretlcli -e})

--- Forza l'utilizzo della versione inglese invece di quella italiana.
      
\cmd{gretlcli --run} \textsl{file} (o \cmd{gretlcli -r} \textsl{file})

--- Esegue i comandi contenuti nel \textsl{file} e aspetta ulteriore
input dalla riga di comando.  Si veda la Sezione \ref{path-search} per
quanto riguarda la ricerca dei percorsi.

\cmd{gretlcli --batch} \textsl{file} (o \cmd{gretlcli -b}
\textsl{file})

--- Esegue i comandi contenuti nel \textsl{file} ed esce; tipicamente
questa opzione viene usata redirigendo i risultati del comando a un
file.  Si veda la Sezione \ref{path-search} per quanto riguarda la
ricerca dei percorsi.

Se si usano le opzioni \cmd{--run} e \cmd{--batch}, il file richiamato
deve indicare un file di dati da aprire. Ci� pu� essere fatto usando
il comando \cmd{open} nel file.


\section{Ricerca dei percorsi}
\label{path-search}

Quando si indica il nome di un file di dati o di comandi a \app{gretl}
o \app{gretlcli} sulla riga di comando, il file viene cercato nel modo
seguente:
      
\begin{enumerate}
\item ``Esattamente come viene specificato''.  Ossia, nella directory
  di lavoro attuale, o, se viene indicato un percorso completo, nella
  posizione indicata.
\item Nella directory gretl dell'utente (si veda la Tabella
  \ref{tab-path} per i valori predefiniti). Si noti che la la parola
  chiave \texttt{PERSONAL} viene interpretata da Windows a seconda
  dell'utente e della lingua del sistema: ad esempio su un sistema in
  lingua inglese, verr� interpretata come la directory ``My
  Documents''.
\item In qualsiasi directory immediatamente contenuta nella directory
  gretl dell'utente.
\item Nel caso di un file di dati, la ricerca continua nella directory
  principale dei dati di \app{gretl}.  Nel caso di un file di comandi,
  la ricerca continua nella directory principale dei file di
  comandi. Si veda la Tabella \ref{tab-path} per i valori predefiniti.
  \texttt{PREFIX} indica la directory base scelta al momento
  dell'installazione di \app{gretl}.
\item Nel caso dei file di dati, la ricerca continua in tutte le
  directory immediatamente contenute nella directory principale dei
  dati.
\end{enumerate}

\begin{table}[htbp]
  \caption{Impostazioni predefinite dei percorsi}
  \label{tab-path}
  \begin{center}
    \begin{tabular}{lll}
      \textit{�} & \textit{Linux} & \textit{MS Windows} \\ [4pt]
      Directory gretl dell'utente & \texttt{\$HOME/gretl} & 
      \verb@PERSONAL\gretl@ \\
      Directory di sistema dei dati & \texttt{PREFISSO/share/gretl/data} & 
      \verb@PREFIX\gretl\data@ \\
      Directory di sistema dei file di comandi & 
      \texttt{PREFIX/share/gretl/scripts} & \verb@PREFIX\gretl\scripts@ \\
    \end{tabular}
  \end{center}
\end{table}

Non � quindi necessario specificare il percorso completo per un file
di dati o di comandi, a meno che non si voglia fare a meno del
meccanismo di ricerca automatico (ci� � valido anche per
\app{gretlcli} quando si indica un nome di file come argomento per il
comando \cmd{open} o il comando \cmd{run})

Quando un file di comandi contiene l'istruzione di aprire un file di
dati, l'ordine di ricerca per il file � quello descritto sopra, con
l'eccezione che il file viene cercato anche nella directory che
contiene lo script, immediatamente dopo aver tentato di cercare il
file di dati ``esattamente come viene specificato''.
      

\subsection{MS Windows}
\label{MS-behave}

In MS Windows le informazioni di configurazione per \app{gretl} e
\app{gretlcli} sono scritte nel registro di Windows. Quando
\app{gretl} � installato per la prima volta, viene creata
un'appropriata serie di voci di registro; successivamente le
impostazioni possono essere modificate attraverso il men�
\textsf{Strumenti, Preferenze} di \app{gretl}. Se occorre eseguire
delle modifiche manuali dei valori, le voci di registro possono essere
trovate (usando il programma standard di Windows \app{regedit.exe})
sotto \verb@Software\gretl@ in \verb@HKEY_LOCAL_MACHINE@ (la directory
principale di \app{gretl} e i percorsi per i vari programmi ausiliari)
e in \verb@HKEY_CURRENT_USER@ (tutte le altre variabili
configurabili).
          

