\chapter{Variabili dipendenti discrete e censurate}
\label{chap:discr-models}

\section{Modelli logit e probit}
\label{sec:logit-probit}

Capita spesso che si voglia specificare e stimare un modello in cui la variabile
dipendente non � continua  ma discreta. Un esempio tipico � quello di un modello
in cui la variabile dipendente � lo stato lavorativo di un individuo (1 =
occupato, 0 = disoccupato). Un modo comodo per formalizzare questa situazione
consiste nel considerare la variabile $y_i$ come una variabile aleatoria di
Bernoulli e analizzarne la distribuzione condizionata alle variabili esplicative
$x_i$, ossia
\begin{equation}
  \label{eq:qr-Bernoulli}
  y_i \left\{ 
    \begin{array}{ll} 1 & P_i \\ 0 & 1 - P_i \end{array}
    \right. 
\end{equation}
dove $P_i = P(y_i = 1 | x_i) $ � una funzione delle variabili esplicative $x_i$.

Nella maggior parte dei casi, la funzione $P_i$ � una funzione di ripartizione
$F$, applicata a una combinazione lineare delle $x_i$. Nel modello probit, si usa la
funzione di ripartizione normale, mentre il modello logit usa la funzione
logistica $\Lambda()$. Quindi si ha
\begin{eqnarray}
  \label{eq:qr-link}
  \textrm{probit} & \qquad & P_i = F(z_i) = \Phi(z_i)  \\
  \textrm{logit}  & \qquad & P_i = F(z_i) = \Lambda(z_i) = \frac{1}{1 + e^{-z_i}} \\
  & &z_i = \sum_{j=1}^k x_{ij} \beta_j
\end{eqnarray}
dove $z_i$ � chiamata la funzione \emph{indicatrice}. Si noti che in questo
caso, i coefficienti $\beta_j$ non possono essere interpretati come derivate
parziali di $E(y_i | x_i)$ rispetto a $x_{ij}$. Comunque, per un dato valore
di $x_i$ � possibile calcolare il vettore delle ``pendenze'', ossia
\[
  \mathrm{slope}_j(\bar{x}) = \left. \pder{F(z)}{x_j} \right|_{z =
    \bar{z}} ;
\]
\app{gretl} calcola automaticamente le pendenze assegnando a ogni variabile
esplicativa un valore pari alla sua media.

Un modo equivalente di formulare questo modello consiste nell'ipotizzare
l'esistenza di una variabile non osservata $y^*_i$ che pu� essere descritta dal
modello
\begin{equation}
  \label{eq:qr-latent}
  y^*_i = \sum_{j=1}^k x_{ij} \beta_j + \varepsilon_i = z_i  +
  \varepsilon_i ;
\end{equation}
si osserva $y_i = 1$ quando $y^*_i > 0$ e $y_i = 0$ altrimenti. Se si assume
$\varepsilon_i$ come normale, si ottiene il modello probit, mentre il modello
logit assume che la funzione di densit� di $\varepsilon_i$ sia
\[
  \lambda(\varepsilon_i) =
  \pder{\Lambda(\varepsilon_i)}{\varepsilon_i} =
  \frac{e^{-\varepsilon_i}}{(1 + e^{-\varepsilon_i})^2} .
\]

\app{gretl} stima sia il modello probit che quello logit col metodo della
massima verosimiglianza; poich� le equazioni degli score non hanno una soluzione
in forma chiusa, vengono usate procedure di ottimizzazione numerica. La maggior
parte delle volte queste richiedono poche iterazioni per raggiungere la
convergenza, ma � possibile visualizzare i dettagli dell'algoritmo di
massimizzazione usando l'opzione \texttt{--verbose}.

\begin{script}[htbp]
  \caption{Stima di semplici modelli logit e probit}
  \label{simple-QR}
\begin{code}
open greene19_1

logit GRADE const GPA TUCE PSI
probit GRADE const GPA TUCE PSI
\end{code}
\end{script}

Come esempio, riproduciamo i risultati esposti nel capitolo 21 di Greene (2000),
dove viene valutata l'efficacia di un programma per insegnare l'economia
osservando i miglioramenti nei voti degli studenti.
Eseguendo il codice contenuto nell'esempio \ref{simple-QR} si ottengono i seguenti risultati:
\begin{code}

Modello 1: Stime Logit usando le 32 osservazioni 1-32
Variabile dipendente: GRADE

      VARIABILE      COEFFICIENTE       ERRORE STD    STAT T       PENDENZA
                                                                  (alla media)
  const               -13,0213           4,93132      -2,641
  GPA                   2,82611          1,26294       2,238      0,533859   
  TUCE                  0,0951577        0,141554      0,672      0,0179755  
  PSI                   2,37869          1,06456       2,234      0,449339   

  Media di GRADE = 0,344
  Numero dei casi 'previsti correttamente' = 26 (81,2%)
  f(beta'x) nella media delle variabili indipendenti = 0,189
  Pseudo-R-quadro di McFadden = 0,374038
  Log-verosimiglianza = -12,8896
  Test del rapporto di verosimiglianza: Chi-quadro(3) = 15,4042 (p-value 0,001502)
  Criterio di informazione di Akaike (AIC) = 33,7793
  Criterio bayesiano di Schwarz (BIC) = 39,6422
  Criterio di Hannan-Quinn (HQC) = 35,7227

              Previsto
                0    1
  Effettivo 0  18    3
            1   3    8

Modello 2: Stime Probit usando le 32 osservazioni 1-32
Variabile dipendente: GRADE

      VARIABILE      COEFFICIENTE       ERRORE STD    STAT T       PENDENZA
                                                                  (alla media)
  const                -7,45232          2,54247      -2,931
  GPA                   1,62581          0,693883      2,343      0,533347   
  TUCE                  0,0517288        0,0838903     0,617      0,0169697  
  PSI                   1,42633          0,595038      2,397      0,467908   

  Media di GRADE = 0,344
  Numero dei casi 'previsti correttamente' = 26 (81,2%)
  f(beta'x) nella media delle variabili indipendenti = 0,328
  Pseudo-R-quadro di McFadden = 0,377478
  Log-verosimiglianza = -12,8188
  Test del rapporto di verosimiglianza: Chi-quadro(3) = 15,5459 (p-value 0,001405)
  Criterio di informazione di Akaike (AIC) = 33,6376
  Criterio bayesiano di Schwarz (BIC) = 39,5006
  Criterio di Hannan-Quinn (HQC) = 35,581

              Previsto
                0    1
  Effettivo 0  18    3
            1   3    8

\end{code}

In questo contesto, la funzione accessoria \texttt{$\$$uhat} assume un
significato speciale: produce i residui generalizzati come sono definiti in
Gourieroux, che possono essere interpretati come stimatori non distorti dei
disturbi latenti $\varepsilon_t$. Questi sono definiti come
\begin{equation}
  \label{eq:QR-genres}
  u_i = \left\{
    \begin{array}{ll}
      y_i - \hat{P}_i & \textrm{for the logit model} \\
      y_i\cdot \frac{\phi(\hat{z}_i)}{\Phi(\hat{z}_i)} - 
      ( 1 - y_i ) \cdot \frac{\phi(\hat{z}_i)}{1 - \Phi(\hat{z}_i)}
      & \textrm{for the probit model} \\
    \end{array}
    \right.
\end{equation}

Tra l'altro, i residui generalizzati sono spesso usati a scopo diagnostico; ad
esempio, � molto facile costruire un test per variabili omesse
equivalente al test LM usato tipicamente nel contesto della regressione lineare:
l'esempio \ref{QR-add} mostra come eseguire un test per l'aggiunta di una
variabile.

\begin{script}[htbp]
  \caption{Test per l'aggiunta di una variabile in un modello probit}
  \label{QR-add}
\begin{code}
open greene19_1

probit GRADE const GPA PSI
series u = $uhat 

ols u const GPA PSI TUCE -q
printf "Test per l'aggiunta della variabile TUCE:\n"
printf "Rsq * T = %g (p. val. = %g)\n", $trsq, pvalue(X,1,$trsq) 
\end{code}

\end{script}

\subsection{Modelli ordinati}
\label{sec:ordered}

Questi modelli sono semplici variazioni sui normali modelli logit/probit,
utilizzati di solito nei casi in cui la variabile dipendente assume valori
discreti e ordinati, non necessariamente quantitativi. Ad esempio, questi
modelli possono essere applicati per analizzare casi in cui la variabile
dipendente � un giudizio qualitativo come ``Buono'', ``Medio'', ``Scarso''.
Ipotizzando di avere $p$ categorie, la probabilit� che l'individuo $i$ ricada
nella $j$-esima categoria � dato da
\begin{equation}
  \label{eq:QR-ordered}
  P(y_i = j | x_i) = \left\{
    \begin{array}{ll}
      F(z_i + \mu_0) & \textrm{for } j = 0 \\
      F(z_i + \mu_j) -  F(z_i + \mu_{j-1}) & \textrm{for } 0 < j < p \\
      1 -  F(z_i + \mu_{p-1}) & \textrm{for } j = p 
    \end{array}
    \right.
\end{equation}
I parametri ignoti $\mu_j$ sono chiamati ``punti di taglio'' e sono stimati
insieme ai $\beta$'. Ai fini dell'identificazione, $\mu_0$ � ipotizzato pari a
0.  In termini della variabile non osservata $y^*_i$, il modello pu� essere
espresso in modo equivalente come
$P(y_i = j | x_i) = P(\mu_{j-1} \le y^*_i < \mu_j)$. 

\begin{script}[htbp]
  \caption{Modello probit ordinato}
  \label{ex:oprobit}
\begin{code}
open pension.gdt
series pctstck = pctstck/50
discrete pctstck
probit pctstck const choice age educ female black married finc25 finc35 \
  finc50 finc75 finc100 finc101 wealth89 prftshr
\end{code}
\end{script}

Per applicare questi modelli, la variabile dipendente deve essere marcata come
discreta e il suo valore minimo deve essere pari a 0. L'esempio \ref{ex:oprobit}
riproduce la stima proposta nel capitolo 15 di Wooldridge (2002a). Si noti che
\app{gretl} non fornisce un comando separato per i modelli ordinati: i comandi
\texttt{logit} e \texttt{probit} stimano automaticamente le versioni ordinate se
la variabile dipendente non � binaria (a patto che sia stata marcata in
precedenza come discreta).

Dopo aver stimato modelli ordinati, la variabile \texttt{$\$$uhat} contiene i
residui generalizzati, come avviene per i modelli binari; in pi�, la variabile
\texttt{$\$$yhat} contiene $\hat{z}_i$, cos� � possibile calcolare una stima non
distorta della variabile latente $y^*_i$ semplicemente facendo l'addizione delle
due.

\section{Il modello Tobit}
\label{sec:tobit}

Il modello Tobit viene usato quando la variabile dipendente di un modello �
censurata: si ipotizzi che una variabile latente $y^*_i$ possa essere descritta
come
\[
  y^*_i = \sum_{j=1}^k x_{ij} \beta_j + \varepsilon_i ,
\]
dove $\varepsilon_i \sim N(0,\sigma^2)$. Se le $y^*_i$ fossero osservabili, i
parametri del modello potrebbero essere stimati con i minimi quadrati ordinari.
Al contrario, si supponga di poter osservare $y_i$, definita
come\footnote{Stiamo assumendo che i dati siano censurati per quanto riguarda
  i valori inferiori a zero. I casi di censura per valori maggiori di zero,
  o in corrispondenza di valori diversi da zero, possono essere trattati facilmente
  ridefinendo la variabile dipendente. Il caso pi� generale di censura da due
  lati non � contemplato automaticamente da \app{gretl}, ma � possibile stimare
  tali modelli usando il comando \texttt{mle} (si veda il capitolo
  \ref{chap:mle}).}
\begin{equation}
  \label{eq:tobit}
  y_i \left\{ 
    \begin{array}{ll} 
      y^*_i & \mathrm{for} \quad y^*_i > 0 \\ 
      0 & \mathrm{for} \quad y^*_i \le 0 
    \end{array}
    \right. 
\end{equation}
In questo caso, regredire $y_i$ sulle $x_i$ non produce stime consistenti dei
parametri $\beta$, perch� la media condizionale $E(y_i|x_i)$ non � pari a $\sum_{j=1}^k x_{ij}
\beta_j$. Come si pu� dimostrare, nemmeno restringere il campione alle
osservazioni diverse da zero non produrrebbe stime consistenti. La soluzione sta
nello stimare i parametri con la massima verosimiglianza. La sintassi �
semplicemente
\begin{code}
  tobit depvar indvars
\end{code}

Come al solito, � possibile visualizzare i progressi dell'algoritmo di
massimizzazione usando l'opzione \texttt{--verbose} e \texttt{$\$$uhat}
contiene i residui generalizzati.

Un'importante differenza tra lo stimatore Tobit e quello OLS � che le
conseguenze della non-normalit� del termine di disturbo sono molto pi� severe,
visto che la non-normalit� implica la non consistenza per lo stimatore Tobit.
Per questo motivo, fra i risultati del modello Tobit viene mostrato anche il
test di normalit� di Chesher-Irish.

\subsection{Modello Tobit generalizzato}
\label{sec:heckit}

Nel cosiddetto modello ``Tobit II'', esistono due variabili latenti:
\begin{eqnarray}
  \label{eq:heckit1}
  y^*_i & = & \sum_{j=1}^k x_{ij} \beta_j + \varepsilon_i \\
  \label{eq:heckit2}
  s^*_i & = & \sum_{j=1}^p z_{ij} \gamma_j + \eta_i 
\end{eqnarray}
e la regola di osservazione � data da
\begin{equation}
  \label{eq:tobitII}
  y_i \left\{ 
    \begin{array}{ll} 
      y^*_i & \mathrm{for} \quad s^*_i > 0 \\ 
      0 & \mathrm{for} \quad s^*_i \le 0 
    \end{array}
    \right. 
\end{equation}

Una delle applicazioni pi� popolari di questo modello in econometria prevede
un'equazione dei salari e un'equazione della partecipazione alla forza lavoro:
viene osservato solo il salario delle persone occupate. Se $y^*_i$ e $s^*_i$ fossero
(condizionalmente) indipendenti, non ci sarebbe motivo per non usare lo
stimatore OLS per stimare l'equazione (\ref{eq:heckit1}); ma in altri casi, lo
stimatore OLS non produce stime consistenti dei parametri $\beta_j$.

Uno stimatore molto usato � il cosidetto stimatore \emph{Heckit}, che prende il
nome da Heckman (1979). La procedura pu� essere schematizzata nel modo seguente:
per prima cosa viene stimato un modello probit per l'equazione (\ref{eq:heckit2});
quindi, vengono inseriti i residui generalizzati nell'equazione (\ref{eq:heckit1})
per correggere gli effetti della selezione del campione.

L'esempio \ref{ex:heckit} mostra due stime dal dataset usato in
Mroz (1987): la prima replica la tabella 22.7 di Greene (2003), mentre la
seconda replica la tabella 17.1 di Wooldridge (2002a). Si noti che lo script
usato, \texttt{heckit.inp}, fa parte degli script di esempio forniti con \app{gretl}.

\begin{script}[htbp]
  \caption{Modello Tobit generalizzato}
  \label{ex:heckit}
\begin{code}
open mroz.gdt
include heckit.inp

genr EXP2 = AX^2
genr WA2 = WA^2
genr KIDS = (KL6+K618)>0

# Greene's specification

list X = const AX EXP2 WE CIT
list Z = const WA WA2 FAMINC KIDS WE

heckit(WW,X,LFP,Z)

# Wooldridge's specification

series NWINC = FAMINC - WW*WHRS
series lww = log(WW)
list X = const WE AX EXP2
list Z = X NWINC WA KL6 K618

heckit(lww,X,LFP,Z)
\end{code}
\end{script}

%\section{Count data}
%\label{sec:poisson}

%also include example script for negative binomial (done in Vebeek
%example files).



%%% Local Variables: 
%%% mode: latex
%%% TeX-master: "gretl-guide"
%%% End: 
