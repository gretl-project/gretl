\chapter{Criteri di selezione dei modelli}
\label{select-criteria}

\section{Introduzione}
\label{select-intro}

In alcuni contesti, l'econometrico deve scegliere tra modelli alternativi
basandosi su test di ipotesi formali. Ad esempio, si pu� scegliere un
modello pi� generale rispetto ad uno pi� ristretto, se la restrizione in
questione pu� essere formulata sotto forma di ipotesi nulla testabile e
l'ipotesi nulla viene rifiutata da un apposito test.

In altri contesti si ha bisogno invece di un criterio di selezione dei modelli
che tenga conto da una parte dell'accuratezza dell'adattamento ai dati, o della
verosimiglianza del modello, e dall'altra parte della sua parsimonia. �
necessario mantenere questo equilibrio perch� l'aggiunta di variabili a un
modello pu� solo aumentare la sua capacit� di adattamento o la sua
verosimiglianza, ma � possibile che ci� avvenga anche se le variabili aggiuntive
non sono veramente rilevanti per il processo che ha generato i dati.

Il pi� famoso tra questi criteri di selezione, per modelli lineari stimati con i
minimi quadrati, � l'$R^2$ corretto,
%
\[
\bar{R}^2 = 1 - \frac{{\rm SSR} / (n-k)}{{\rm TSS} / (n-1)}
\]
%
dove $n$ � il numero di osservazioni nel campione, $k$ denota il numero di
parametri stimati, SSR e TSS denotano rispettivamente la somma dei quadrati dei
residui e la somma dei quadrati della variabile dipendente. Confrontata con il
classico coefficiente di determinazione $R^2$
%
\[
R^2 = 1 - \frac{{\rm SSR}}{{\rm TSS}}
\]
%
la versione ``corretta'' penalizza l'inclusione di variabili aggiuntive, a
parit� di altre condizioni.

\section{Criteri di informazione}
\label{select-aic}

Un criterio pi� generale, che segue un'impostazione simile, � il ``criterio di
informazione di Akaike'' (AIC) del 1974. La formulazione originale di questa
misura �
%
\begin{equation}
\label{aic-orig}
{\rm AIC} = -2 \ell(\hat{\theta}) + 2k
\end{equation}
%
dove $\ell(\hat{\theta})$ rappresenta la massima log-verosimiglianza come
funzione del vettore delle stime dei parametri, $\hat{\theta}$, e $k$
(come sopra) indica il numero di ``parametri indipendenti all'interno del
modello''. In questa formulazione, con AIC correlato negativamente alla
verosimiglianza e positivamente al numero dei parametri, il ricercatore mira a
minimizzare il suo valore.

L'AIC pu� generare confusione, dal momento che sono diffuse varie versioni per
il suo calcolo; ad esempio, Davidson e MacKinnon (2004) ne presentano una versione
semplificata,
%
\[
{\rm AIC} = \ell(\hat{\theta}) - k
\]
%
che vale quanto l'originale moltiplicata per $-2$: in questo caso, ovviamente,
si cercher� di massimizzare l'AIC.

Nel caso di modelli stimati con i minimi quadrati, la log-verosimiglianza pu�
essere scritta come
%
\begin{equation}
\label{ols-loglik}
\ell(\hat{\theta}) = -\frac{n}{2}(1 + \log 2\pi - \log n)
 - \frac{n}{2} \log {\rm SSR}
\end{equation}
%
Sostituendo (\ref{ols-loglik}) in (\ref{aic-orig}) otteniamo
%
\[
{\rm AIC} = n(1 + \log 2\pi - \log n) + n\log {\rm SSR} + 2k
\]
%
che pu� essere scritta anche come
%
\begin{equation}
\label{full-aic-alt}
{\rm AIC} = n\log \left( \frac{\rm SSR}{n} \right) + 2k + 
  n(1 + \log 2\pi)
\end{equation}
%

Alcuni autori semplificano la formula nel caso di modelli stimati con i minimi
quadrati. Ad esempio William Greene scrive
%
\begin{equation}
\label{aic-greene}
{\rm AIC} = \log \left( \frac{\rm SSR}{n} \right) + \frac{2k}{n}
\end{equation}
%
Questa variante pu� essere derivata da (\ref{full-aic-alt}) dividendo per
$n$ e sottraendo la costante $1 + \log 2\pi$.  Ossia, chiamando 
AIC$_G$ la versione proposta da Greene, abbiamo
%
\[
{\rm AIC}_G = \frac{1}{n} {\rm AIC} - (1 + \log 2\pi)
\]
%

Infine, Ramanathan offre un'altra variante:
%
\[
{\rm AIC}_R = \left( \frac{\rm SSR}{n} \right) e^{2k/n}
\]
%
che � l'esponenziale della versione di Greene.  

All'inizio, gretl usava la versione di Ramanathan, ma a partire dalla versione
1.3.1 del programma, viene usata la formula originale di Akaike
(\ref{aic-orig}), e pi� specificamente (\ref{full-aic-alt}) per i modelli
stimati con i minimi quadrati.

\vspace{1ex}

Anche se il criterio di Akaike � progettato per favorire la parsimonia, non lo
fa in modo eccessivo. Ad esempio, se abbiamo due modelli annidati con
rispettivamente $k-1$ e $k$ parametri, e se l'ipotesi nulla che il parametro
$k$ valga 0 � vera, per grandi campioni l'AIC tender� comunque a far preferire
il modello meno parsimonioso in circa il 16 per cento dei casi (si veda
Davidson e MacKinnon, 2004, capitolo 15).

Un criterio alternativo all'AIC che non risente di questo problema � il
``Criterio di informazione Bayesiana'' (BIC) di Schwarz (1978). Il BIC pu�
essere scritto (in modo simile alla formulazione di Akaike di AIC) come
%
\[
{\rm BIC} = -2 \ell(\hat{\theta}) + k \log n
\]
Il prodotto di $k$ per $\log n$ nel BIC significa che la penalizzazione per
l'aggiunta di parametri addizionali aumenta con l'ampiezza campionaria. Ci�
assicura che, asintoticamente, un modello troppo esteso non verr� mai scelto al
posto di un modello parsimonioso ma correttamente specificato.

Un'altra alternativa all'AIC che tende a favorire modelli pi� parsimoniosi � il
criterio di Hannan--Quinn, o HQC (Hannan e Quinn, 1979). Volendo essere coerenti
con la formulazione usata finora, pu� essere scritto nel modo seguente:
%
\[
{\rm HQC} = -2 \ell(\hat{\theta}) + 2k \log \log n
\]
%
Il calcolo di Hannan--Quinn si basa sulla regola del logaritmo iterato (si noti
che l'ultimo termine � il logaritmo del logaritmo dell'ampiezza campionaria).
Gli autori affermano che questa procedura fornisce un ``procedura di stima
consistente in senso forte per l'ordine di una autoregressione'', e che
``confrontata con altre procedure consistenti in senso forte, questa sottostimer�
l'ordine in modo minore''.

\vspace{1ex}

Gretl mostra AIC, BIC e HQC (calcolati nel modo spiegato sopra) per la maggior
parte dei modelli. Quando si interpretano questi valori occorre sempre
ricordarsi se sono calcolati in modo da essere massimizzati o minimizzati. In
gretl essi sono sempre calcolati per essere minimizzati: valori minori sono
da preferire.





