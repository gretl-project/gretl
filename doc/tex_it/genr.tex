\chapter{Funzioni speciali in genr}
\label{chap-genr}

\section{Introduzione}
\label{genr-intro}

Il comando \verb+genr+ offre un modo flessibile per definire nuove
variabili. Il comando � documentato nel \emph{XXX}, mentre questo
capitolo offre una discussione pi� approfondita di alcune delle
funzioni speciali disponibili con \verb+genr+ e di alcune
particolarit� del comando.
    

\section{Filtri per serie storiche}
\label{genr-filter}

Un tipo di funzione specializzata di \verb+genr+ � il filtro per le
serie storiche. Ne esistono di due tipi al momento: il filtro di
Hodrick--Prescott e quello passa banda di Baxter--King.  Sono
utilizzabili rispettivamente con le funzioni \verb+hpfilt()+ e
\verb+bkfilt()+, che richiedono come argomento il nome della variabile
da processare.
    
\subsection{Il filtro di Hodrick--Prescott}
\label{hodrick-prescott}

Da scrivere.

\subsection{Il filtro di Baxter e King}
\label{baxter-king}

Si consideri la rappresentazione spettrale di una serie storica $y_t$:
%	
\[ y_t = \int_{-\pi}^{\pi} e^{i\omega} \mathrm{d} Z(\omega) \]
%
se volessimo estrarre solo la componente di \emph{y\ensuremath{_{t}}}
che si trova tra le frequenze $\underline{\omega}$ e
$\overline{\omega}$ potremmo applicare un filtro passa banda:
%	
\[ c^*_t = \int_{-\pi}^{\pi} F^*(\omega) e^{i\omega} \mathrm{d}
Z(\omega) \] 
%
dove $F^*(\omega) = 1$ per $\underline{\omega} < |\omega| <
\overline{\omega}$ e 0 altrove. Ci� implicherebbe, nel dominio
temporale, applicare alla serie un filtro con un numero infinito di
coefficienti, cosa non desiderabile. Il filtro passa banda di Baxter e
King applica a $y_t$ un polinomio finito nell'operatore di ritardo
$A(L)$:
%	
\[ c_t = A(L) y_t \]
%
dove $A(L)$ � definito come
%	
\[ A(L) = \sum_{i=-k}^{k} a_i L^i \]

I coefficienti $a_i$ sono scelti in modo che $F(\omega) =
A(e^{i\omega})A(e^{-i\omega})$ sia la migliore approssimazione di
$F^*(\omega)$ per un dato $k$. Chiaramente, maggiore � $k$, migliore �
l'approssimazione, ma poich� occorre scartare $2k$ osservazioni, di
solito si cerca un compromesso.  Inoltre, il filtro ha altre propriet�
teoriche interessanti, tra cui quella che $a(1) = 0$, quindi una serie
con una sola radice unitaria � resa stazionaria con l'applicazione del
filtro.

In pratica, il filtro � usato di solito con dati mensili o trimestrali
per estrarne la componente di ``ciclo economico'', ossia la componente
tra 6 e 36 trimestri. I valori usuali per $k$ sono 8 o 12 (o forse di
pi� per serie mensili).  I valori predefiniti per i limiti di
frequenza sono 8 e 32, mentre il valore predefinito per l'ordine di
approssimazione, $k$, � 8.  � possibile impostare questi valori usando
il comando \cmd{set}.  La parola chiave per impostare i limiti di
frequenza � \verb+bkbp_limits+, mentre quella per $k$ � \verb+bkbp_k+.
Quindi ad esempio, se si stanno usando dati mensili e si vuole
impostare i limiti di frequenza tra 18 e 96, e $k$ a 24, si pu�
eseguire

\begin{code}
	set bkbp_limits 18 96
	set bkbp_k 24
\end{code}

Questi valori resteranno in vigore per le chiamate alla funzione
\verb+bkfilt+ finch� non saranno modificati da un altro uso di
\verb+set+.
      

\section{Ricampionamento e bootstrapping}
\label{genr-resample}

Un'altra funzione particolare � il ricampionamento, con reimmissione,
di una serie. Da scrivere.
    

\section{Gestione dei valori mancanti}
\label{genr-missing}

Sono disponibili quattro funzioni speciali per gestire i valori
mancanti.  La funzione booleana \verb+missing()+ richiede come unico
argomento il nome di una variabile e produce una serie con valore 1
per ogni osservazione in cui la variabile indicata ha un valore
mancante, 0 altrove (ossia dove la variabile indicata ha un valore
valido). La funzione \verb+ok()+ � il complemento di \verb+missing+,
ossia una scorciatoia per \verb+!missing+ (dove \verb+!+ � l'operatore
booleano NOT).  Ad esempio, � possibile contare i valori mancanti
della variabile \verb+x+ usando

\begin{code}
      genr nmanc_x = sum(missing(x))
\end{code}

La funzione \verb+zeromiss()+, che richiede anch'essa come unico
argomento il nome di una serie, produce una serie in cui tutti i
valori zero sono trasformati in valori mancanti. Occorre usarla con
attenzione (di solito non bisogna confondere valori mancanti col
valore zero), ma pu� essere utile in alcuni casi: ad esempio, �
possibile determinare la prima osservazione valida di una variabile
\verb+x+ usando

\begin{code}
        genr time
        genr x0 = min(zeromiss(time * ok(x)))
\end{code}


La funzione \verb+misszero()+ compie l'operazione opposta di
\verb+zeromiss+, ossia converte tutti i valori mancanti in zero.  

Pu� essere utile chiarire la propagazione dei valori mancanti
all'interno delle formule di \verb+genr+. La regola generale � che
nelle operazioni aritmetiche che coinvolgono due variabili, se una
delle variabili ha un valore mancante in corrispondenza
dell'osservazione $t$, anche la serie risultante avr� un valore
mancante in $t$. L'unica eccezione a questa regola � la
moltiplicazione per zero: zero moltiplicato per un valore mancante
produce sempre zero (visto che matematicamente il risultato � zero a
prescindere dal valore dell'altro fattore).
    

\section{Recupero di variabili interne}
\label{genr-internal}

Il comando \verb+genr+ fornisce un modo per recuperare vari valori
calcolati dal programma nel corso della stima dei modelli o della
verifica di ipotesi. Le variabili che possono essere richiamate in
questo modo sono elencate nella XXX; qui ci occupiamo in particolare
delle variabili speciali \verb+$test+ e \verb+$pvalue+.

Queste variabili contengono, rispettivamente, il valore dell'ultima
statistica test calcolata durante l'ultimo uso esplicito di un comando
di test e il p-value per quella statistica test. Se non � stato
eseguito alcun comando di test, le variabili contengono il codice di
valore mancante. I ``comandi espliciti di test'' che funzionano in
questo modo sono i seguenti: \cmd{add} (test congiunto per la
significativit� di variabili aggiunte a un modello); \cmd{adf} (test
di Dickey--Fuller aumentato, si veda oltre); \cmd{arch} (test per
ARCH); \cmd{chow} (test Chow per break strutturale); \cmd{coeffsum}
(test per la somma dei coefficienti specificati); \cmd{cusum} (la
statistica \emph{t} di Harvey--Collier); \cmd{kpss} (il test di
stazionariet� KPSS, p-value non disponibile); \cmd{lmtest} (si veda
oltre); \cmd{meantest} (test per la differenza delle medie);
\cmd{omit} (test congiunto per la significativit� delle variabili
omesse da un modello); \cmd{reset} (test RESET di Ramsey);
\cmd{restrict} (vincolo lineare generale); \cmd{runs} (test delle
successioni per la casualit�); \cmd{testuhat} (test per la normalit�
dei residui) e \cmd{vartest} (test per la differenza delle varianze).
Nella maggior parte dei casi, vengono salvati valori sia in
\verb+$test+ che in \verb+$pvalue+; l'eccezione � il test KPSS, per
cui non � disponibile il p-value.
    
Un punto da tenere in considerazione a questo proposito � che le
variabili interne \verb+$test+ e \verb+$pvalue+ vengono sovrascritte
ogni volta che viene eseguito uno dei test elencati sopra. Se si
intende referenziare questi valori durante una sequenza di comandi
\app{gretl}, occorre farlo nel momento giusto.
    
Una questione correlata � che alcuni dei comandi di test generano di
solito pi� di una statistica test e pi� di un p-value: in questi casi
vengono salvati solo gli ultimi valori. Per controllare in modo
preciso quali valori vengono recuperati da \verb+$test+ e
\verb+$pvalue+ occorre formulare il comando di test in modo che il
risultato non sia ambiguo. Questa nota vale in particolare per i
comandi \verb+adf+ e \verb+lmtest+.

\begin{itemize}
\item Di solito, il comando \cmd{adf} genera tre varianti del test
  Dickey--Fuller: una basata su una regressione che include una
  costante, una che include costante e trend lineare, e una che
  include costante e trend quadratico. Se si intende estrarre valori
  da \verb+$test+ o \verb+$pvalue+ dopo aver usato questo comando, �
  possibile selezionare la variante per cui verranno salvati i valori,
  usando una delle opzioni \verb+--nc+, \verb+--c+, \verb+--ct+ o
  \verb+--ctt+ con il comando \verb+adf+.
\item Di solito, il comando \cmd{lmtest} (che deve seguire una
  regressione OLS) esegue vari test diagnostici sulla regressione in
  questione. Per controllare cosa viene salvato in \verb+$test+ e
  \verb+$pvalue+ occorre limitare il test usando una delle opzioni
  \verb+--logs+, \verb+--autocorr+, \verb+--squares+ 
  o \verb+--white+.
\end{itemize}

Un aiuto all'uso dei valori immagazzinati in \verb+$test+ e
\verb+$pvalue+ � dato dal fatto che il tipo di test a cui si
riferiscono questi valori viene scritto nell'etichetta descrittiva
della variabile generata. Per controllare di aver recuperato il valore
corretto, � possibile leggere l'etichetta con il comando \cmd{label}
(il cui unico argomento � il nome della variabile). La seguente
sessione interattiva illustra la procedura.
    
\begin{code}
      ? adf 4 x1 --c

      Test Dickey-Fuller aumentati, ordine 4, per x1
      ampiezza campionaria 59
      ipotesi nulla di radice unitaria: a = 1

        test con costante
        modello: (1 - L)y = b0 + (a-1)*y(-1) + ... + e
        valore stimato di (a - 1): -0.216889
        statistica test: t = -1.83491
        p-value asintotico 0.3638

      P-value basati su MacKinnon (JAE, 1996)
      ? genr pv = $pvalue
      Generato lo scalare pv (ID 13) = 0.363844
      ? label pv    
      pv=Dickey-Fuller pvalue (scalar)
\end{code}


%%% Local Variables: 
%%% mode: latex
%%% TeX-master: "gretl-guide-it"
%%% End: 

