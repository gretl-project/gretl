\documentclass[oneside]{book}
\usepackage[T1]{fontenc}
\usepackage{url,verbatim,fancyvrb,appendix}
\usepackage{pifont}
\usepackage[latin1]{inputenc}
\usepackage[pdftex]{graphicx}
\usepackage{color,gretl}
\usepackage[italian]{babel}
\usepackage[compat2,a4paper,hmargin=1.1in,top=.5in]{geometry}
\usepackage[pdftex,hyperfootnotes=false]{hyperref}
\usepackage{dcolumn,amsmath}

%% \pdfimageresolution=120
\hypersetup{pdftitle={Guida all'uso di Gretl},
            pdfsubject={GNU Regression, Econometrics and Time Series Library},
            pdfauthor={Allin Cottrell},
            colorlinks=true,
            linkcolor=blue,
            urlcolor=red,
            bookmarks=true,
            bookmarksnumbered=true,
            plainpages=false
}

% ------------- custom commands -------------------------------------------
\newcommand{\LogLik}{\ensuremath\ell}
\newcommand{\stackunder}[2]{\ensuremath\mathrel{\mathop{#2}\limits_{#1}}}
\newcommand{\pder}[2]{\frac{\ensuremath\partial #1}{\partial #2}}
\newcommand{\convp}{\stackrel{\mathrm{p}}{\longrightarrow}}
\newcommand{\convd}{\stackrel{\mathrm{d}}{\longrightarrow}}
\renewcommand{\vec}[1]{{\rm vec}(#1)}
% -------------------------------------------------------------------------

\begin{document}

\renewcommand{\scriptname}{Esempio}
\renewcommand{\GCR}{\textit{Guida ai comandi di gretl}}
\renewcommand{\GUG}{\textit{Guida all'uso di gretl}}

\VerbatimFootnotes

\setlength{\parindent}{0pt}
\setlength{\parskip}{1ex}
\setcounter{tocdepth}{1}

%% titlepage

\thispagestyle{empty}

\begin{center}
\pdfbookmark[1]{Guida all'uso di gretl}{titlepage}

\gtitle{Guida all'uso di gretl}

\gsubtitle{Gnu Regression, Econometrics and Time-series library}

{\large \sffamily 
Allin Cottrell\\
Department of Economics\\
Wake Forest university\\

\vspace{20pt}
Riccardo ``Jack'' Lucchetti\\
Dipartimento di Economia\\
Universit� Politecnica delle Marche\\

\vspace{20pt}
Cristian Rigamonti\\
(traduzione italiana)\\

\vspace{20pt}
\input date
}

\end{center}
\clearpage

%% end titlepage, begin license page

\thispagestyle{empty}

\pdfbookmark[1]{Licenza}{license}

\vspace*{2in}

� garantito il permesso di copiare, distribuire e/o modificare questo
documento seguendo i termini della \emph{Licenza per Documentazione
  Libera GNU}, Versione 1.1 o ogni versione successiva pubblicata
dalla Free Software Foundation (si veda
\url{http://www.gnu.org/licenses/fdl.html}).

\clearpage

%% end license page, start table of contents
\pdfbookmark[1]{Indice}{contents}

\pagenumbering{roman}
\pagestyle{headings}

\tableofcontents

\clearpage
\pagenumbering{arabic}

\chapter{Introduction}
\label{intro}


\section{Features at a glance}
\label{features}

\app{Gretl} is an econometrics package, including a shared library, a
command-line client program and a graphical user interface.
    
\begin{description}
\item[User-friendly] \app{Gretl} offers an intuitive user interface;
  it is very easy to get up and running with econometric analysis.
  Thanks to its association with the econometrics textbooks by Ramu
  Ramanathan, Jeffrey Wooldridge, and James Stock and Mark Watson, the
  package offers many practice data files and command scripts.  These
  are well annotated and accessible. Two other useful resources for gretl
  users are the available documentation and the
  \href{http://gretl.sourceforge.net/lists.html}{gretl-users} mailing
  list.
\item[Flexible] You can choose your preferred point on the spectrum
  from interactive point-and-click to batch processing, and can easily
  combine these approaches.
\item[Cross-platform] \app{Gretl}'s ``home'' platform is Linux but it
  is also available for MS Windows and Mac OS X, and should work on
  any unix-like system that has the appropriate basic libraries (see
  Appendix~\ref{app-technote}).
\item[Open source] The full source code for \app{gretl} is available
  to anyone who wants to critique it, patch it, or extend it.
  See Appendix~\ref{app-technote}.
\item[Sophisticated] \app{Gretl} offers a full range of least-squares
  based estimators, either for single equations and for systems,
  including vector autoregressions and vector error correction models.
  Several specific maximum likelihood estimators (e.g.\ probit, ARIMA,
  GARCH) are also provided natively; more advanced estimation methods
  can be implemented by the user via generic maximum likelihood or
  nonlinear GMM.
\item[Extendible] Users can enhance \app{gretl} by writing their own
  functions and procedures in \app{gretl}'s scripting language, which
  includes a reasonably wide range of matrix functions.
\item[Accurate] \app{Gretl} has been thoroughly tested on several
  benchmarks, among which the NIST reference datasets. See
  Appendix~\ref{app-accuracy}.
\item[Internet ready] \app{Gretl} can access and fetch databases from
  a server at Wake Forest University.  The MS Windows version comes
  with an updater program which will detect when a new version is
  available and offer the option of auto-updating.
\item[International] \app{Gretl} will produce its output in English,
  French, Italian, Spanish, Polish or German, depending on your
  computer's native language setting.
\end{description}


\section{Acknowledgements}
\label{ack}

The \app{gretl} code base originally derived from the program
\app{ESL} (``Econometrics Software Library''), written by Professor
Ramu Ramanathan of the University of California, San Diego.  
We are much in debt to Professor Ramanathan for making this code
available under the GNU General Public Licence and for helping to
steer \app{gretl}'s early development.

We are also grateful to the authors of several econometrics textbooks
for permission to package for \app{gretl} various datasets associated
with their texts.  This list currently includes William Greene, author
of \emph{Econometric Analysis}; Jeffrey Wooldridge (\emph{Introductory
  Econometrics: A Modern Approach}); James Stock and Mark Watson
(\emph{Introduction to Econometrics}); Damodar Gujarati (\emph{Basic
  Econometrics}); and Russell Davidson and James MacKinnon
(\emph{Econometric Theory and Methods}).  

GARCH estimation in \app{gretl} is based on code deposited in the
archive of the \emph{Journal of Applied Econometrics} by Professors
Fiorentini, Calzolari and Panattoni, and the code to generate
\emph{p}-values for Dickey--Fuller tests is due to James MacKinnon.  In
each case we are grateful to the authors for permission to use their
work.

With regard to the internationalization of \app{gretl}, thanks go to
Ignacio D�az-Emparanza (Spanish), Michel Robitaille and Florent
Bresson (French) , Cristian Rigamonti (Italian), Tadeusz Kufel and
Pawel Kufel (Polish), and Markus Hahn and Sven Schreiber (German).

\app{Gretl} has benefitted greatly from the work of numerous
developers of free, open-source software: for specifics please see
Appendix~\ref{app-technote}.  Our thanks are due to Richard Stallman
of the Free Software Foundation, for his support of free software in
general and for agreeing to ``adopt'' \app{gretl} as a GNU program in
particular.

Many users of \app{gretl} have submitted useful suggestions and bug
reports.  In this connection particular thanks are due to Ignacio
D�az-Emparanza, Tadeusz Kufel, Pawel Kufel, Alan Isaac, Cri Rigamonti,
Sven Schreiber, Talha Yalta, and Dirk Eddelbuettel, who maintains the
\app{gretl} package for Debian GNU/Linux.


\section{Installing the programs}
\label{install}

\subsection{Linux}
\label{linux-install}

On the Linux\footnote{In this manual we use ``Linux'' as shorthand to
  refer to the GNU/Linux operating system.  What is said herein about
  Linux mostly applies to other unix-type systems too, though some
  local modifications may be needed.} platform you have the choice of
compiling the \app{gretl} code yourself or making use of a pre-built
package. Ready-to-run packages are available in \app{rpm} format
(suitable for Red Hat Linux and related systems) and also \app{deb}
format (Debian GNU/Linux).  If you prefer to compile your own (or are
using a unix system for which pre-built packages are not available)
here is what to do.
      
\begin{enumerate}
\item Download the latest \app{gretl} source package from
  \href{http://gretl.sourceforge.net/}{gretl.sourceforge.net}.
\item Unzip and untar the package.  On a system with the GNU utilities
  available, the command would be \cmd{tar xvfz gretl-N.tar.gz}
  (replace \cmd{N} with the specific version number of the file you
  downloaded at step 1).
\item Change directory to the gretl source directory created at step 2
  (e.g.\ \verb+gretl-1.1.5+).
          
\item The basic routine is then
            
\begin{code}
./configure 
make 
make check
make install
\end{code}
However, you should probably read the \verb+INSTALL+ file first,
and/or do
\begin{code}
./configure --help
\end{code}
first to see what options are available. One option you way wish to
tweak is \cmd{--prefix}.  By default the installation goes under
\verb+/usr/local+ but you can change this.  For example
\begin{code}
./configure --prefix=/usr
\end{code}
will put everything under the \verb+/usr+ tree.  In the event that a
required library is not found on your system, so that the configure
process fails, please see Appendix~\ref{app-technote}.
On most systems, the \texttt{make install} command requires you to
have administrative privileges. Hence, either you log in as
\texttt{root} before launching  \texttt{make install} or, more
correctly, you may want to use the \texttt{sudo} utility. 
\end{enumerate}

\app{Gretl} offers support for the \app{gnome} desktop.  To take
advantage of this you should compile the program yourself (as
described above).  If you want to suppress the \app{gnome}-specific
features you can pass the option \verb+--without-gnome+ to
\cmd{configure}.


\subsection{MS Windows}
\label{windows-install}

The MS Windows version comes as a self-extracting executable.
Installation is just a matter of downloading \verb+gretl_install.exe+
and running this program. You will be prompted for a location to
install the package (the default is \verb+c:\userdata\gretl+).


\subsection{Updating}
\label{updating}

If your computer is connected to the Internet, then on start-up
\app{gretl} can query its home website at Wake Forest University to
see if any program updates are available; if so, a window will open up
informing you of that fact.  If you want to activate this feature,
check the box marked ``Tell me about gretl updates'' under
\app{gretl}'s ``Tools, Preferences, General'' menu.
      
The MS Windows version of the program goes a step further: it tells
you that you can update \app{gretl} automatically if you wish.  To do
this, follow the instructions in the popup window: close \app{gretl}
then run the program titled ``gretl updater'' (you should find this
along with the main \app{gretl} program item, under the Programs
heading in the Windows Start menu). Once the updater has completed its
work you may restart \app{gretl}.
      
%%% Local Variables: 
%%% mode: latex
%%% TeX-master: "gretl-guide"
%%% End: 


\part{Uso del programma}

\chapter{Puesta en marcha}
\label{c-getting-started}

\section{Ejecutar una regresi�n}
\label{starting-regression}

Esta parte introductoria est� enfocada principalmente hacia el
programa de cliente gr�fico; para m�s detalles del programa de l�nea
de instrucciones, \app{gretlcli}, consultar XXX.

Aunque es posible abrir \app{gretl} tecleando el nombre de un archivo
de datos como argumento, por el momento no vamos a probar este modo;
simplemente ejecutamos el programa.\footnote{Por comodidad este manual
  se refiere al programa de cliente gr�fico simplemente como
  \app{gretl}. N�tese, sin embargo, que el nombre especifico del
  programa cambia seg�n la plataforma del ordenador. En Linux se llama
  \verb+gretl_x11+ mientras que en MS Windows se llama
  \verb+gretlw32.exe+. En sistemas Linux tambi�n se instala un lote de
  instrucciones de envoltura (wrapper script) llamado \verb+gretl+.}
  
Aparecer� una ventana principal (que puede tener alguna informaci�n
sobre los datos pero que al principio est� vac�a) y varios men�s,
algunos de los cuales est�n desactivados al principio.

�Qu� podemos hacer a partir de este punto? Se puede consultar los
archivos (o bases) de datos que vienen con el programa, abrir un
archivo de datos, crear un nuevo archivo de datos, leer los apartados
de la ayuda, o abrir alg�n lote de instrucciones. Por ahora vamos a
revisar los archivos de datos que acompa�an al programa. Desde el men�
Archivo elegir ``Abrir datos, archivo de muestra, Ramanathan...''. Se
deber�a abrir una segunda ventana con una lista de archivos de datos
que vienen con el paquete (v�ase la Figura \ref{fig-datafiles}). Los
archivos est�n numerados siguiendo los cap�tulos del libro de
Ramanathan (2002), el cual tambi�n contiene los puntos relacionados
con el an�lisis de estos datos. A�n en el caso de que no dispongamos
de este texto, los datos en s� son muy �tiles para realizar ejercicios
pr�cticos.

\begin{figure}[htbp]
\begin{center}
  \includegraphics[scale=0.5]{figures/datafiles}
\end{center}
\caption{Ventana de archivo de datos para ejercicios pr�cticos}
\label{fig-datafiles}
\end{figure}

Si seleccionamos una fila en esta ventana y pulsamos ``Informaci�n'',
se abre el archivo de cabecera (header file) de los datos en cuesti�n,
que contiene informaci�n sobre el origen y las definiciones de las
variables. Para abrir alg�n archivo que nos interese, pulsar
``Abrir'', o simplemente hacer doble clic sobre el nombre del archivo.
Por el momento, abriremos el archivo \verb+data3-6+.

\tip{En ventanas \app{gretl} con listas, un doble clic sobre una l�nea
  lanza una acci�n asociada por defecto a esta entrada de la lista:
  por ejemplo, mostrar valores de una serie de datos, abrir un
  archivo.}
  
Este archivo contiene datos pertenecientes a un ``problema'' cl�sico
de econometr�a: la funci�n de consumo. Ahora, la ventana de datos
deber�a mostrar el nombre del archivo de datos actual, el rango total
de los datos y el rango de la muestra, as� como los nombres de las
variables junto con sus notas descriptivas - v�ase la Figura
\ref{fig-mainwin}.

\begin{figure}[htbp]
\begin{center}
  \includegraphics[scale=0.5]{figures/mainwin}
\end{center}
\caption{Ventana principal, con un archivo de datos para ejercicios
  abierto}
\label{fig-mainwin}
\end{figure}

Bien, �cu�l es el siguiente paso? \app{gretl} intenta que las
diferentes opciones del men� sean bastantes expl�citas.  Primero,
hagamos una breve visita al men� Modelo; en Secci�n \ref{menus} se
realiza un breve recorrido a lo largo de los men�s de la ventana
principal.

El men� Modelo de \app{gretl} ofrece numerosos m�todos de estimaci�n
econom�trica. El m�s sencillo y est�ndar es el de M�nimos Cuadrados
Ordinarios (MCO). Si se selecciona MCO, aparece un cuadro de di�logo
en el que hay que \emph{especificar el modelo} - v�ase la Figura
\ref{fig-selector}.

\begin{figure}[htbp]
\begin{center}
  \includegraphics[scale=0.5]{figures/selector}
\end{center}
\caption{Cuadro de di�logo de especificaci�n del modelo}
\label{fig-selector}
\end{figure}

Para seleccionar la variable dependiente, resaltamos la variable
deseada en la lista que aparece a la izquierda y pulsamos el bot�n
``Elegir'' que apunta hacia la caja de la variable dependiente. Si
marcamos la casilla ``Selecci�n por defecto'', esta variable ser�
preseleccionada como dependiente la siguiente vez que abramos el
cuadro de di�logo de especificaci�n de modelo.  Un atajo: un doble
clic sobre una variable dentro de la lista de la izquierda, hace que
�sta sea la variable dependiente por defecto. Para seleccionar las
variables independientes, primero las resaltamos en la lista de la
izquierda, y a continuaci�n pulsamos el bot�n ``A�adir'' (o el bot�n
derecho del rat�n con el cursor sobre la variable seleccionada). Para
seleccionar varias variables en la lista, arrastramos el rat�n sobre
ellas; para seleccionar diferentes variables que no son contiguas,
apretamos la tecla \verb+Ctrl+ y manteni�ndola as�, hacemos clic sobre
las variables deseadas.

Para ejecutar una regresi�n con el consumo como variable dependiente,
y la renta como la independiente, hacemos clic sobre \verb+Ct+ para
ponerla en el espacio de ``Variable dependiente'', y luego a�adimos
\verb+Yt+ a la lista de variables Independientes.

\section{Resultados de la estimaci�n}
\label{est-output}

Una vez que hayamos especificado un modelo, aparecer� una ventana
mostrando los resultados de la regresi�n. La informaci�n que presenta
es bastante comprensible y est� escrita en un formato est�ndar.
(Figura \ref{fig-modelwin}).

\begin{figure}[htbp]
\begin{center}
  \includegraphics[scale=0.5]{figures/modelwin}
\end{center}
\caption{Ventana de resultados del modelo}
\label{fig-modelwin}
\end{figure}

La ventana de resultados contiene men�s que nos permiten inspeccionar
o hacer gr�ficos de los valores ajustados y de los residuos, y adem�s
podemos ejecutar varios programas de diagn�sticos sobre el modelo.

Para la mayor�a de los modelos, tambi�n existe la opci�n de reimprimir
los resultados de la regresi�n en formato LaTeX. Podemos imprimir los
resultados en formato tabular (similar a lo que hay en la ventana
output, pero bien escritos) o como una ecuaci�n, en una p�gina. Para
cada una de estas opciones podemos elegir entre una vista preliminar
de la composici�n final, o guardar los resultados en un archivo para
su posterior incorporaci�n en un documento LaTeX. La vista preliminar
requiere que tengamos un sistema TeX funcionando en nuestro ordenador.

Para exportar los resultados de \app{gretl} a un procesador de textos,
se puede copiar y pegar desde una ventana de resultados, utilizando su
men� \textsf{Editar}, al programa deseado. Muchas (pero no todas) las
ventanas de \app{gretl} ofrecen la opci�n de copiar como RTF (``Rich
Text Format'' de Microsoft) o como documento LaTeX. Si se quiere pegar
el contenido en un procesador de texto, la opci�n RTF quiz� sea la
mejor porque conserva el formato tabular de los resultados.
\footnote{T�ngase en cuenta que al copiar como RTF en MS Windows,
  Windows s�lo permitir� que se pegue en aquellas aplicaciones que
  ``entiendan'' el formato RTF. Por lo tanto, se puede pegar en MS
  Word, pero no en \app{notepad}.}
  
Alternativamente, podemos guardar los resultados en un archivo de
texto y luego importar este archivo desde el programa que estemos
utilizando .  Al terminar una sesi�n de gretl, tenemos la opci�n de
guardar todos los resultados de la sesi�n en un �nico archivo.

T�ngase en cuenta que tanto en el escritorio \app{gnome} como bajo MS
Windows, el men� \textsf{Archivo} incluye una instrucci�n para enviar
los resultados directamente a una impresora.

\tip{Al pegar o exportar los resultados de \app{gretl} como texto a un
  procesador de textos, conviene seleccionar una fuente monoespaciada
  o estilo m�quina de escribir (e.g. Courier) para conservar el
  formato tabular del output. Selecci�nese un tama�o peque�o de fuente
  (un Courier de 10 puntos ser� suficiente) para evitar que las l�neas
  de resultados se partan incorrectamente.}
  

\section{Men�s de la ventana principal}
\label{menus}

Si se lee la barra del men� de la ventana principal, empezando por la
izquierda hacia la derecha, se encuentran los siguientes men�s:
Archivo, Utilidades, Sesi�n, Datos, Muestra, Variable, Modelo y Ayuda.

\begin{center}
  \includegraphics[scale=0.5]{figures/menubar}
\end{center}

\begin{itemize}
\item \textsf{Men� Archivo}

\begin{itemize}
\item \textsf{Abrir datos}: Abre un archivo de datos propio de
  \app{gretl} o importa desde otro formato. V�ase el [?].
\item \textsf{Guardar datos}: Guarda el archivo de datos propio de
  \app{gretl} abierto en este momento.
\item \textsf{Guardar datos como}: Escribe el conjunto de datos en
  formato propio, con las opciones de utilizar gzip para comprimir los
  datos o guardar los datos en formato binario. V�ase el [?].
\item \textsf{Exportar datos}: Escribe el conjunto de datos en formato
  CSV (comma-separated values --- valores separados por comas), o en
  los formatos de GNU R o GNU Octave. V�ase el [?] as� como el [?].
\item \textsf{Cerrar conjunto de datos}: Cierra el conjunto de datos
  con que se est� trabajando en la memoria. Generalmente no tenemos
  que hacer esto (ya que al abrir un archivo de datos nuevo, se cierra
  de manera autom�tica el viejo) pero es de utilidad en algunas
  ocasiones (v�ase [?]).
\item \textsf{Revisar bases de datos}: V�anse [?] y [?].
\item \textsf{Crear conjunto de datos}: Inicia la hoja de c�lculo
  incorporada en el programa, para a�adir datos de forma manual. V�ase
  [?].
\item \textsf{Guardar el �ltimo gr�fico}: Guarda el gr�fico m�s
  reciente.
\item \textsf{Ver historial de instrucciones}: Abre una ventana donde
  puede verse el historial de todas las instrucciones ejecutadas hasta
  el momento.
\item \textsf{Abrir archivo de instrucciones}: Abre un lote de
  instrucciones de \app{gretl}, bien uno ya creado por el usuario, o
  bien uno de los archivos de ejercicios pr�cticos que acompa�an al
  programa. Si se desea crear un lote de instrucciones desde cero,
  �sese la siguiente opci�n: \textsf{Nuevo archivo de instrucciones}.
\item \textsf{Preferencias}: Especifica la ubicaci�n de varios
  archivos a los que \app{gretl} necesita acceder. Selecciona la
  fuente en que \app{gretl} mostrar� los textos generados. Activa o
  desactiva el ``modo experto''. (Si se activa este modo, se suprimen
  varios avisos.) Activa o desactiva los mensajes de \app{gretl} sobre
  la disponibilidad de las nuevas versiones del programa. Configura o
  quita/pone la barra de herramientas en la ventana principal.
\item \textsf{Salir}: Salir del programa. Si no se ha activado el modo
  experto, se mostrar� un aviso para que se guarde cualquier trabajo
  no guardado.
\end{itemize}

\item \textsf{Men� de Utilidades}
\begin{itemize}
\item \textsf{Tablas estad�sticas}: Consultar los valores cr�ticos de
  las distribuciones m�s frecuentes (normal o gausiana, \emph{t},
  chi-cuadrado, \emph{F} y Durbin-Watson).
\item \textsf{Buscador de valores p}: Abre una ventana que nos permite
  consultar valores p de distribuciones gausianas, \emph{t},
  chi-cuadrado, \emph{F} o gamma. V�ase tambi�n la instrucci�n
  \cmd{pvalue}.
\item \textsf{Calculadora de estad�sticos de contraste}: Calcula los
  estad�sticos de contraste y valores p para una amplia gama de
  contrastes de hip�tesis comunes (media poblacional, varianzas y
  proporciones; diferencias de medias, de varianzas y de
  proporciones). Los estad�sticos muestrales relevantes han de estar
  disponibles con anterioridad para su incorporaci�n en el cuadro de
  di�logo. Para algunos contrastes sencillos, que toman como entrada
  series de datos en vez de estad�sticos muestrales previamente
  calculados, v�ase ``Diferencia de medias'' y ``Diferencia de
  varianzas'' en el men� de Datos.
\item \textsf{Consola Gretl}: Abre una ventana tipo ``consola'' dentro
  de la cual podemos escribir instrucciones (en lugar de apuntar y
  hacer clic), del mismo modo que en el programa de l�nea de
  instrucciones, \app{gretlcli}.
\item \textsf{Iniciar GNU R}: Inicia una sesi�n de \app{R} (si est�
  instalado en el sistema), y carga una copia del conjunto de datos
  que est� abierto en \app{gretl}.
\end{itemize}

\item \textsf{Men� de Sesi�n}
\begin{itemize}
\item \textsf{Vista de iconos}: Abre una ventana que muestra la sesi�n
  actual de \app{gretl} en forma de un conjunto de iconos. Para m�s
  detalles v�ase [?].
\item \textsf{A�adir el �ltimo gr�fico}: Captura el gr�fico m�s
  reciente en forma de icono de sesi�n, para poder acceder a �l y
  manipularlo.
\item \textsf{Abrir}: Abre un archivo de sesi�n previamente guardado.
\item \textsf{Guardar}: Guarda la sesi�n actual en un archivo.
\item \textsf{Guardar como}: Guarda la sesi�n actual en el archivo
  deseado.
\end{itemize}


\item \textsf{Men� de Datos}
\begin{itemize}
\item \textsf{Mostrar valores}: Muestra una ventana con una simple
  vista (no editable) de los valores de las variables (todas o una
  parte seleccionada).
\item \textsf{Editar valores}: Muestra una ventana con una hoja de
  c�lculo en la cual podemos introducir cambios, a�adir nuevas
  variables, y extender el n�mero de observaciones. (La matriz de
  datos tiene que ser rectangular, con un n�mero id�ntico de
  observaciones para cada serie.)
\item \textsf{Gr�ficos}: Aqu� se puede elegir entre trazar un gr�fico
  de series temporales, un diagrama de dispersi�n X-Y regular, un
  gr�fico X-Y de impulsos (barras verticales), un gr�fico X-Y ``con
  separaci�n de factores'' (es decir, con los puntos en diferentes
  colores dependiendo del valor de una variable ficticia) y gr�ficos
  de caja (boxplots). Muestra un cuadro de di�logo donde se pueden
  especificar las variables a trazar. La manera m�s sencilla de
  rellenarlo es refiri�ndose a las variables con sus n�meros de
  identificaci�n (se encuentran en la primera columna de la izquierda
  en la ventana principal de datos). Por lo tanto, si hemos elegido la
  opci�n de diagrama de dispersi�n, poniendo ``2 3'' como los valores
  de los datos, se dibuja la variable 2 (aqu�, el consumo) contra la
  variable 3 (renta). La �ltima variable referenciada ser� situada en
  el eje \emph{x}. El programa utiliza gnuplot para hacer los gr�ficos
  (excepto en la opci�n gr�ficos de caja).
\item \textsf{Gr�ficos bivariantes m�ltiples}: Muestra una colecci�n
  de (hasta un m�ximo de seis) diagramas, bien como una variable en el
  eje \emph{y} contra varias variables en el eje \emph{x}, o como
  varias variables en el eje \emph{y} contra uno en el eje \emph{x}.
  Puede ser �til para realizar un an�lisis preliminar de los datos.
\item \textsf{Leer informaci�n}, \textsf{Editar informaci�n}: ``Leer
  informaci�n'' simplemente muestra la informaci�n de cabecera del
  archivo actual; ``Editar informaci�n'' permite cambiar esta
  informaci�n (en Linux, si tenemos los permisos para hacerlo).
\item \textsf{Estad�sticos principales}: Muestra un conjunto de
  estad�sticos descriptivos bastante completo respecto de todas las
  variables incluidas en el conjunto de datos.
\item \textsf{Matriz de correlaci�n}: Muestra los coeficientes de
  correlaci�n de cada par de variables incluidas en el conjunto de
  datos.
\item \textsf{Diferencia de medias}: calcula el estad�stico \emph{t}
  para la hip�tesis nula de que las medias poblacionales sean iguales
  para dos variables seleccionadas y muestra su valor p.
\item \textsf{Diferencia de varianzas}: Calcula el estad�stico
  \emph{F} para la hip�tesis nula de que las varianzas poblacionales
  sean iguales para dos variables seleccionadas y muestra su valor p.
\item \textsf{A�adir variables}: Ofrece un submen� con las
  transformaciones de variables m�s t�picas (logaritmos, retardos,
  cuadrados, etc.)  las cuales pueden a�adirse al conjunto de datos.
  Tambi�n se da la opci�n de a�adir variables aleatorias, y (para
  datos de series temporales) a�adir variables ficticias estacionales
  (por ejemplo, variables ficticias trimestrales para datos
  trimestrales). Incluye una opci�n para poner una semilla en el
  generador de n�meros pseudoaleatorios del programa.
\item \textsf{Actualizar ventana}: A veces las instrucciones de
  \app{gretl} generan variables nuevas. La ``actualizaci�n'' asegura
  que el listado de variables visibles en la ventana principal de
  datos est� sincronizado con el estado interno del programa.
\end{itemize}


\item \textsf{Men� de muestra}
\begin{itemize}
\item \textsf{Establecer rango}: Establecer un punto de partida y/o
  terminaci�n diferente para la muestra actual, dentro del rango de
  los datos disponibles.
\item \textsf{Restaurar el rango completo}: \emph{\textsl{idem}}.
\item \textsf{Establecer frecuencia, observaci�n inicial}: Impone una
  interpretaci�n particular de los datos en t�rminos de frecuencia y
  observaci�n inicial. Est� pensado principalmente para datos de
  panel; v�ase [?].
\item \textsf{Definir a partir de v. ficticia}: Dada una variable
  ficticia (indicador) con valor 0 o 1, se eliminan todas las
  observaciones de la muestra actual donde la variable ficticia tenga
  el valor 0.
\item \textsf{Restringir a partir de criterio}: Similar al anterior
  excepto que no necesitamos una variable predefinida: escribimos una
  expresi�n booleana (por ejemplo \verb+sqft > 1400+) y la muestra se
  restringe a las observaciones que cumplen este requisito. V�ase la
  ayuda para \cmd{genr} para detalles sobre los operadores booleanos
  que se pueden utilizar.
\item \textsf{Quitar todas las obs. con valores perdidos}: Elimina de
  la muestra actual todas las observaciones para las cuales hay, por
  lo menos, una variable que tiene un valor perdido (v�ase [?]).
\item \textsf{Contar valores perdidos}: Emite un informe sobre las
  observaciones en las que faltan valores de los datos. Puede ser �til
  para examinar conjuntos de datos de panel, en los cuales suele
  faltar alg�n que otro valor.
\item \textsf{A�adir marcadores de caja}: Pregunta por el nombre de un
  archivo de texto que contenga marcadores de caja (etiquetas
  asociadas a observaciones individuales) y a�ade esta informaci�n al
  conjunto de datos. V�ase el [?].
\item \textsf{Reestructurar panel}: Abre un cuadro de di�logo que nos
  permite determinar la interpretaci�n de un conjunto de datos panel,
  bien como series temporales apiladas o como muestras transversales
  apiladas (v�ase [?]).
\end{itemize}


\item \textsf{Men� de variable}: La mayor�a de las opciones que
  encontramos aqu� act�an sobre las variables una por una. Podemos
  elegir la variable ``activa'' seleccion�ndola y pulsando sobre ella
  en la ventana principal de datos. La mayor�a de las opciones no
  necesitan explicaci�n. N�tese que es posible renombrar una variable
  y editar su descripci�n.  Tambi�n es posible ``Definir una nueva
  variable'' mediante una formula (por ejemplo, con una funci�n de
  alguna/s variable/s ya existentes). Para conocer la sintaxis de
  estas f�rmulas, cons�ltese el t�rmino \cmd{genr} en la ayuda
  \emph{en l�nea} o v�ase la instrucci�n \cmd{genr}.  Un ejemplo
  sencillo ser�a:
          
\begin{code}
  foo = x1 * x2
\end{code}

esto crear�a una nueva variable \verb+foo+ como el producto de las
variables \verb+x1+ y \verb+x2+ ya existentes. En estas f�rmulas hay
que referenciar las variables por su nombre, y no por su n�mero.


\item \textsf{Men� de modelo}: Ya comentado en el cap�tulo
  \ref{c-getting-started}. Para m�s detalles sobre los estimadores
  ofrecidos en este men�, cons�ltese [?], el XXX a continuaci�n, y/o
  la ayuda \emph{en l�nea} sobre ``estimaci�n''.

\item \textsf{Men� de ayuda}: Incluye detalles sobre la sintaxis
  requerida en los cuadros de di�logo.(Nota del traductor: el fichero
  de ayuda en Espa�ol suele actualizarse con retraso, por eso, a
  partir de la versi�n gretl-1.0.9 se ha a�adido en la ventana de
  ayuda una nueva opci�n de men� que muestra la ayuda en Ingl�s,
  normalmente m�s actualizada y que mostrar� tambi�n la ayuda sobre
  las nuevas instrucciones a�adidas en la �ltima versi�n))

\end{itemize}

\section{La barra de herramientas de gretl}
\label{toolbar}

La barra de herramientas est� situada en la esquina inferior-izquierda
de la ventana principal.

\begin{center}
  \includegraphics[scale=0.5]{figures/toolbar}
\end{center}

Los iconos tienen las siguientes funciones, empezando de izquierda a
derecha:

\begin{enumerate}
\item Lanzar calculadora. Es una funci�n c�moda si queremos un r�pido
  acceso a una calculadora mientras trabajamos en \app{gretl}. El
  programa utilizado por defecto es \verb+calc.exe+ en MS Windows, o
  \verb+xcalc+ en el sistema de ventanas X . Es posible cambiar el
  programa desde el men� ``Archivo, Preferencias, General'', y submen�
  ``Barra de herramientas''.
\item Lanzar editor o procesador de textos. Por defecto
  \verb+winword.exe+ en MS Windows, y \verb+emacs+ en X. Se puede
  configurar de la misma manera que la calculadora.
\item Abrir consola \app{gretl}. Un acceso r�pido a la entrada del
  men� ``Consola \app{gretl}'' (ver m�s arriba Secci�n \ref{menus}).
\item Vista de iconos de sesi�n.
\item Abrir el sitio web de \app{gretl} en nuestro navegador de
  Internet. Esto s�lo funcionar� en el caso de que el ordenador est�
  conectado a Internet y el navegador est� configurado de manera
  correcta.
\item Abrir la versi�n actual de este manual en formato PDF. Requiere
  una conexi�n con Internet; adem�s el navegador ha de tener la
  capacidad de procesar archivos PDF.
\item Abrir la ayuda sobre la sintaxis de lotes de instrucciones (es
  decir, un listado detallando todas las instrucciones disponibles).
\item Abrir un cuadro de di�logo para definir un gr�fico.
\item Grabar el �ltimo gr�fico mostrado, para modificarlo.  (Este
  bot�n no se muestra en Linux, donde se puede grabar el gr�fico
  mediante su men� desplegable)
\item Abrir una ventana mostrando los conjuntos de datos asociados al
  libro \emph{Introductory Econometrics} de Ramanathan
  (alternativamente puede configurarse este bot�n para que muestre los
  conjuntos de datos de Wooldridge (2002)- v�ase el [?]).
\end{enumerate}

Para desactivar la barra de herramientas, ir a al men� ``Archivo,
preferencias, general'', y desmarcar la casilla de verificaci�n
``Mostrar barra de herramientas \app{gretl}''.


\chapter{Modos de trabajo}
\label{c-modes}

\section{Lotes de instrucciones}
\label{scripts}

A medida que vayamos ejecutando instrucciones en \app{gretl},
utilizando el GUI y rellenando cuadros de di�logo, estas instrucciones
quedan registradas en forma de un ``lote''. Podemos modificar o
reejecutar estos lotes de instrucciones mediante \app{gretl} o el
cliente de l�nea de instrucciones, \app{gretlcli}.

Para visualizar el estado actual del lote durante una sesi�n
\app{gretl}, se selecciona ``Ver historial de instrucciones '' dentro
del men� Archivo. Este archivo contiene el historial, llamado
\verb+session.inp+, y se sobreescribe cada vez que empezamos una nueva
sesi�n. Para conservarlo, hay que guardarlo con otro nombre. Ser� m�s
f�cil encontrar los archivos de lotes utilizando el selector de
archivos GUI, si se nombran con la extensi�n ``\verb+.inp+''.

Para abrir un lote instrucciones escrito de manera independiente,
�sese la opci�n ``Archivo, Abrir archivo de instrucciones, archivo de
usuario'' en el men�; para crear un lote desde cero, se utiliza la
opci�n ``Archivo, Nuevo archivo de instrucciones''. En ambos casos se
abrir� una ventana para escribir las instrucciones (v�ase la Figura
\ref{fig-scriptwin}).

\begin{figure}[htbp]
\begin{center}
  \includegraphics[scale=0.5]{figures/scriptwin}
\end{center}
\caption{Ventana de lotes de instrucciones, editando un archivo de instrucciones}
\label{fig-scriptwin}
\end{figure}

La barra de herramientas en la esquina inferior-izquierda de la
ventana principal ofrece las siguientes opciones (de izquierda a
derecha): (1) Guardar el archivo; (2) Guardar como; (3) Ejecutar las
instrucciones en el archivo; (4) Copiar el texto seleccionado; (5)
Pegar el texto seleccionado; (6) Buscar y reemplazar; (7) Deshacer la
�ltima acci�n de Pegar o Reemplazar; (8) Ayuda (si colocamos el cursor
sobre una palabra de la instrucci�n y pulsamos el signo de
interrogaci�n, se muestra la ayuda sobre esta instrucci�n); (9) Cerrar
la ventana.

Estas funciones (y adem�s la opci�n de Imprimir) se encuentran tambi�n
en los men�s ``Archivo'' y ``Editar'' en la parte superior de la
ventana de instrucciones.

Al pulsar el icono Ejecutar o al elegir la opci�n del men� ``Archivo,
Ejecutar'' los resultados se dirigen a una sola ventana, desde donde
es posible editarlos, guardarlos o copiarlos al portapapeles.

Para conocer m�s a fondo las posibilidades de los lotes de
instrucciones, cons�ltese la Ayuda sobre ``Sintaxis de gui�n de
instrucciones'' de \app{gretl}, o alternativamente, iniciar el
programa de l�nea de instrucciones \app{gretlcli} y consultar su
ayuda. Una tercera posibilidad es leer el [?] de este manual. Adem�s,
el paquete \app{gretl} incluye m�s de 70 lotes de ``ejercicios''. La
mayor�a de ellos corresponden al libro de Ramanathan (2002), pero
independientemente de ello tambi�n pueden utilizarse como una
introducci�n al modo de escribir lotes de instrucciones en \app{gretl}
as� como a varios temas de teor�a econom�trica. Se puede acceder a los
ejercicios desde ``Archivo, Abrir archivo de instrucciones, Archivo de
pr�cticas ''. All� hay un listado de los ficheros junto con una breve
descripci�n de los puntos que estos ilustran y los datos que utilizan.
Abra cualquier archivo y ejec�telo (``Archivo, Ejecutar'' en la
ventana de lotes de instrucciones resultante) para ver los resultados.

N�tese que es posible separar las instrucciones largas en un lote de
instrucciones en dos o m�s l�neas utilizando la barra inversa como
car�cter de continuidad.

Es posible, si as� lo desea el usuario, usar los controles GUI y los
lotes de modo simultaneo, seg�n cual sea el m�todo m�s apropiado en
cada momento. Hay dos sugerencias a continuaci�n.


\begin{itemize}
\item Abrir un archivo de datos en el GUI. Explorar los datos, generar
  los gr�ficos, ejecutar las regresiones, hacer los contrastes. Luego
  abrir el historial de instrucciones, quitar cualquier instrucci�n
  redundante y guardarlo con un nombre diferente. Ejecutar el lote de
  instrucciones para generar un �nico archivo que contendr� un
  historial conciso de nuestro trabajo.
\item Empezar creando un nuevo archivo de lotes. Teclear cualquier
  instrucci�n que sea necesaria para poner en marcha las
  transformaciones de los datos (v�ase la instrucci�n \cmd{genr} en el
  [?] m�s adelante). Normalmente, este tipo de tareas son m�s f�ciles
  de realizar si se han pensado las instrucciones con antelaci�n y no
  pulsando sobre la marcha. Luego guardar y ejecutar el lote: la
  ventana de datos del GUI se actualizar� en consecuencia.  Ahora
  podemos manipular los datos mediante el GUI. Para revisitar los
  datos m�s tarde, abrir y ejecutar en primer lugar el lote
  ``preliminar''.
\end{itemize}

Hay otra opci�n para hacer m�s c�moda la tarea. En el men� Archivo de
\app{gretl} se encuentra la opci�n ``Consola \app{gretl}''.  Esta abre
una ventana donde se pueden teclear las instrucciones y ejecutarlas
una por una (con la tecla Retorno) de manera interactiva.
Esencialmente, es el mismo modo de operaci�n que \app{gretlcli},
excepto que (a) el GUI es actualizado bas�ndose en las instrucciones
ejecutadas desde la consola, haciendo posible el trabajo en cualquier
entorno, y que (b) por ahora, la tarea de simulaciones de Monte Carlo
de \app{gretl} (v�ase [?]) no est� disponible en este modo.


\section{El concepto de sesi�n}
\label{session}

\app{Gretl} ofrece la idea de una ``sesi�n'' como un modo de
seguimiento de los trabajos realizados y la posibilidad de volver a
ellos m�s tarde. Se encuentra en un estado experimental (y en la
actualidad hay m�s posibilidades de encontrar errores aqu� que en el
resto del programa); el autor est� interesado en conocer las opiniones
de los usuarios a este respecto.

La idea inicial es ofrecer un peque�o espacio con iconos, que contenga
varios objetos pertenecientes a la sesi�n de trabajo actual (v�ase la
Figura \ref{fig-session}). Despu�s, se pueden a�adir objetos
(representados por iconos) a este espacio. Si se guarda la sesi�n,
estos objetos a�adidos estar�n disponibles cuando se vuelva a abrir la
sesi�n.


\begin{figure}[htbp]
\begin{center}
  \includegraphics[scale=0.5]{figures/session}
\end{center}
\caption{Vista de iconos: a los iconos por defecto, se ha a�adido 
  un modelo y un gr�fico
}
\label{fig-session}
\end{figure}

Al iniciar \app{gretl}, abriendo un conjunto de datos, y seleccionando
``Vista de iconos'' del men� Session, se visualizan los iconos b�sicos
que aparecen por defecto: estos son una manera r�pida de acceder al
lote de instrucciones (``Sesi�n''), informaci�n sobre los datos (si la
hay), la matriz de correlaci�n y estad�sticos pricipales. Todos estos
se activan mediante un doble clic sobre el icono deseado. El icono de
``Conjunto de datos'' es un poco m�s complejo: al hacer doble clic se
abre una hoja de c�lculo con los datos incorporados, pero tambi�n se
despliega un men� con otras opciones si se pulsa el bot�n derecho del
rat�n.

\tip{En muchas ventanas de \app{gretl}, al hacer clic con el bot�n
  derecho del rat�n se despliega un men� con una lista de tareas
  habituales.}
  
Es posible a�adir dos tipos de objetos a la ventana de vista de
iconos: modelos y gr�ficos.

Para a�adir un modelo, primero hay que estimarlo utilizando el men� de
Modelo. Luego se pulsa la opci�n Archivo en la ventana de modelo y se
selecciona ``Guardar a sesi�n como icono...'' o `` Guardar como icono
y cerrar''. Se puede atajar a la segunda opci�n pulsando la tecla
\verb+S+ sobre la ventana de modelo.

Para a�adir un gr�fico, primero hay que crearlo (en el men� Datos,
``Gr�ficos'', o mediante una de las otras instrucciones de \app{gretl}
para generar gr�ficos), luego se selecciona ``A�adir �ltimo gr�fico''
desde el men� Sesi�n o se pulsa el rat�n sobre el bot�n con el dibujo
de una peque�a c�mara fotogr�fica en la barra de herramientas (v�ase
[?]).

Una vez a�adido el modelo o gr�fico, su icono aparecer� en la ventana
de vista de iconos. Haciendo doble clic sobre el icono se vuelve a
mostrar el objeto, mientras que pulsando el bot�n derecho del rat�n se
despliega un men� que permite mostrar o eliminar el objeto.  Este men�
desplegable tambi�n ofrece la posibilidad de editar los gr�ficos.

Si se crean modelos o gr�ficos y tenemos la intenci�n de volver a
utilizarlos m�s adelante, entonces conviene seleccionar ``Guardar
como'' del men� Sesi�n antes de salir de \app{gretl}, dando un nombre
a la sesi�n. Para volver a abrir la sesi�n,


\begin{itemize}
\item Iniciar \app{gretl} y volver a abrir el archivo de la sesi�n
  desde la opci�n ``Abrir'' en el men� Sesi�n, o
\item Desde una l�nea de instrucciones, teclear \cmd{\app{gretl} -r}
  \textsl{archivo de sesi�n}, donde \textsl{archivo de sesi�n} es el
  nombre bajo el cual se hab�a guardado la sesi�n.
\end{itemize}


\chapter{Data files}
\label{datafiles}



\section{Native format}
\label{native-format}

\app{gretl} has its own format for data files.  Most users will
probably not want to read or write such files outside of \app{gretl}
itself, but occasionally this may be useful and full details on the
file formats are given in [?].

\section{Other data file formats}
\label{other-formats}


\app{gretl} will read various other data formats.
    
\begin{itemize}
\item Plain text (ASCII) files.  These can be brought in using
  \app{gretl}'s ``File, Open Data, Import ASCII\dots{}'' menu item, or
  the \cmd{import} script command.  For details on what \app{gretl}
  expects of such files, see Section \ref{scratch}.
\item Comma-Separated Values (CSV) files.  These can be imported using
  \app{gretl}'s ``File, Open Data, Import CSV\dots{}'' menu item, or
  the \cmd{import} script command. See also Section \ref{scratch}.
\item Worksheets in the format of either MS \app{Excel} or
  \app{Gnumeric}.  These are also brought in using \app{gretl}'s
  ``File, Open Data, Import'' menu.  The requirements for such files
  are given in Section \ref{scratch}.
\item BOX1 format data.  Large amounts of micro data are available
  (for free) in this format via the
  \href{http://www.census.gov/ftp/pub/DES/www/welcome.html}{Data
    Extraction Service} of the US Bureau of the Census. BOX1 data may
  be imported using the ``File, Open Data, Import BOX\dots{}'' menu
  item or the \cmd{import -o} script command.
\end{itemize}

When you import data from the ASCII, CSV or BOX formats, \app{gretl}
opens a ``diagnostic'' window, reporting on its progress in reading
the data.  If you encounter a problem with ill-formatted data, the
messages in this window should give you a handle on fixing the
problem.

For the convenience of anyone wanting to carry out more complex data
analysis, \app{gretl} has a facility for writing out data in the
native formats of GNU R and GNU Octave (see [?]).  In the GUI client
this option is found under the ``File'' menu; in the command-line
client use the \cmd{store} command with the flag \cmd{-r} (R) or
\cmd{-m} (Octave).

\section{Binary databases}
\label{dbase}

For working with large amounts of data I have supplied \app{gretl}
with a database-handling routine.  A \emph{database}, as opposed to a
\emph{data file}, is not read directly into the program's workspace.
A database can contain series of mixed frequencies and sample ranges.
You open the database and select series to import into the working
dataset.  You can then save those series in a native format data file
if you wish. Databases can be accessed via \app{gretl}'s menu item
``File, Browse databases''.

For details on the format of \app{gretl} databases, see [?].

\subsection{Online access to databases}
\label{online-data}

As of version 0.40, \app{gretl} is able to access databases via the
internet.  Several databases are available from Wake Forest
University.  Your computer must be connected to the internet for this
option to work.  Please see the item on ``Online databases'' under
\app{gretl}'s Help menu.

\subsection{RATS 4 databases}
\label{RATS}

Thanks to Thomas Doan of \emph{Estima}, who provided me with the
specification of the database format used by RATS 4 (Regression
Analysis of Time Series), \app{gretl} can also handle such databases.
Well, actually, a subset of same: I have only worked on time-series
databases containing monthly and quarterly series.  My university has
the RATS G7 database containing data for the seven largest OECD
economies and \app{gretl} will read that OK.\tip{Visit the \app{gretl}
  \href{http://gretl.sourceforge.net/gretl_data.html}{data page} for
  details and updates on available data.}

\section{Creating a data file from scratch}
\label{scratch}

There are five ways to do this:
\begin{enumerate}
\item Find, or create using a text editor, a plain text data file and
  open it with \app{gretl}'s ``Import ASCII'' option.
	
\item Use your favorite spreadsheet to establish the data file, save
  it in Comma Separated Values format if necessary (this should not be
  necessary if the spreadsheet program is MS Excel or Gnumeric), then
  use one of \app{gretl}'s ``Import'' options (CSV, Excel or Gnumeric,
  as the case may be).
	
\item Use \app{gretl}'s built-in spreadsheet.
	
\item Select data series from a suitable database.
	
\item Use your favorite text editor or other software tools to a
  create data file in \app{gretl} format independently.
	
\end{enumerate}

Here are a few comments and details on these methods.

\subsection{Common points on imported data}


Options (1) and (2) involve using \app{gretl}'s ``import'' mechanism.
For \app{gretl} to read such data successfully, certain general
conditions must be satisfied:
\begin{itemize}
\item The first row must contain valid variable names.  A valid
  variable name is of 8 characters maximum; starts with a letter; and
  contains nothing but letters, numbers and the underscore character,
  \verb+_+.  (Longer variable names will be truncated to 8
  characters.)  Qualifications to the above: First, in the case of an
  ASCII or CSV import, if the file contains no row with variable names
  the program will automatically add names, \verb+v1+, \verb+v2+ and
  so on.  Second, by ``the first row'' is meant the first
  \emph{relevant} row.  In the case of ASCII and CSV imports, blank
  rows and rows beginning with a hash mark, \verb+#+, are ignored.  In
  the case of Excel and Gnumeric imports, you are presented with a
  dialog box where you can select an offset into the spreadsheet, so
  that \app{gretl} will ignore a specified number of rows and/or
  columns.
	  
\item Data values: these should constitute a rectangular block, with
  one variable per column (and one observation per row).  The number
  of variables (data columns) must match the number of variable names
  given. See also Section \ref{missing-data}.  Numeric data are
  expected, but in the case of importing from ASCII/CSV, the program
  offers limited handling of character (string) data: if a given
  column contains character data only, consecutive numeric codes are
  substituted for the strings, and once the import is complete a table
  is printed showing the correspondence between the strings and the
  codes.
	  
\item Dates (or observation labels): Optionally, the \emph{first}
  column may contain strings such as dates, or labels for
  cross-sectional observations.  Such strings have a maximum of 8
  characters (as with variable names, longer strings will be
  truncated).  A column of this sort should be headed with the string
  \verb+obs+ or \verb+date+, or the first row entry may be left
  blank

  For dates to be recognized as such, the date strings must adhere to
  one or other of a set of specific formats, as follows.  For
  \emph{annual} data: 4-digit years.  For \emph{quarterly} data: a
  4-digit year, followed by a separator (either a period, a colon, or
  the letter \verb+Q+), followed by a 1-digit quarter.  Examples:
  \verb+1997.1+, \verb+2002:3+, \verb+1947Q1+.  For \emph{monthly}
  data: a 4-digit year, followed by a period or a colon, followed by a
  two-digit month.  Examples: \verb+1997.01+, \verb+2002:10+.
	  
\end{itemize}

CSV files can use comma, space or tab as the column separator.  When
you use the ``Import CSV'' menu item you are prompted to specify the
separator.  In the case of ``Import ASCII'' the program attempts to
auto-detect the separator that was used.If you use a spreadsheet to
prepare your data you are able to carry out various transformations of
the ``raw'' data with ease (adding things up, taking percentages or
whatever): note, however, that you can also do this sort of thing
easily --- perhaps more easily --- within \app{gretl}, by using the
tools under the ``Data, Add variables'' menu and/or ``Variable, define
new variable''.

\subsection{Appending imported data}


You may wish to establish a \app{gretl} dataset piece by piece, by
incremental importation of data from other sources.  This is supported
via the ``File, Append data'' menu items.  \app{gretl} will check the
new data for conformability with the existing dataset and, if
everything seems OK, will merge the data.  You can add new variables
in this way, provided the data frequency matches that of the existing
dataset.  Or you can append new observations for data series that are
already present; in this case the variable names must match up
correctly.  Note that by default (that is, if you choose ``Open data''
rather than ``Append data''), opening a new data file closes the
current one.

\subsection{Using the built-in spreadsheet}


Under \app{gretl}'s ``File, Create data set'' menu you can choose the
sort of dataset you want to establish (e.g. quarterly time series,
cross-sectional).  You will then be prompted for starting and ending
dates (or observation numbers) and the name of the first variable to
add to the dataset. After supplying this information you will be faced
with a simple spreadsheet into which you can type data values.  In the
spreadsheet window, clicking the right mouse button will invoke a
popup menu which enables you to add a new variable (column), to add an
observation (append a row at the foot of the sheet), or to insert an
observation at the selected point (move the data down and insert a
blank row.)Once you have entered data into the spreadsheet you import
these into \app{gretl}'s workspace using the spreadsheet's ``Apply
changes'' button.

Please note that \app{gretl}'s spreadsheet is quite basic and has no
support for functions or formulas.  Data transformations are done via
the ``Data'' or ``Variable'' menus in the main \app{gretl} window.

\subsection{Selecting from a database}

Another alternative is to establish your dataset by selecting
variables from a database.  \app{gretl} comes with a database of US
macroeconomic time series and, as mentioned above, the program will
reads RATS 4 databases.

Begin with \app{gretl}'s ``File, Browse databases'' menu item. This
has three forks: ``gretl native'', ``RATS 4'' and ``on database
server''.  You should be able to find the file \verb+fedstl.bin+ in
the file selector that opens if you choose the ``gretl native'' option
--- this file, which contains a large collection of US macroeconomic
time series, is supplied with the distribution.You won't find anything
under ``RATS 4'' unless you have purchased RATS data.\footnote{See
  \href{http://www.estima.com/}{www.estima.com}} If you do possess
RATS data you should go into \app{gretl}'s ``File, Preferences,
General'' dialog, select the Databases tab, and fill in the correct
path to your RATS files.

If your computer is connected to the internet you should find several
databases (at Wake Forest University) under ``on database server''.
You can browse these remotely; you also have the option of installing
them onto your own computer.  The initial remote databases window has
an item showing, for each file, whether it is already installed
locally (and if so, if the local version is up to date with the
version at Wake Forest).

Assuming you have managed to open a database you can import selected
series into \app{gretl}'s workspace by using the ``Import'' menu item
in the database window, or via the popup menu that appears if you
click the right mouse button, or by dragging the series into the
program's main window.

\subsection{Creating a gretl data file independently}


It is possible to create a data file in one or other of \app{gretl}'s
own formats using a text editor or software tools such as \app{awk},
\app{sed} or \app{perl}.  This may be a good choice if you have large
amounts of data already in machine readable form. You will, of course,
need to study the \app{gretl} data formats (XML format or
``traditional'' format) as described in chapter \ref{datafiles}.

\subsection{Further note}



\app{gretl} has no problem compacting data series of relatively high
frequency (e.g. monthly) to a lower frequency (e.g. quarterly): you
are given a choice of method (average, sum, start of period, or end of
period).  But it has no way of converting lower frequency data to
higher.  Therefore if you want to import series of various different
frequencies from a database into \app{gretl} \emph{you must start by
  importing a series of the lowest frequency you intend to use.} This
will initialize your \app{gretl} dataset to the low frequency, and
higher frequency data can be imported subsequently (they will be
compacted automatically).  If you start with a high frequency series
you will not be able to import any series of lower frequency.

\section{Missing data values}
\label{missing-data}

These are represented internally as \verb+DBL_MAX+, the largest
floating-point number that can be represented on the system (which is
likely to be at least 10 to the power 300, and so should not be
confused with legitimate data values).  In a native-format data file
they should be represented as \verb+NA+. When importing CSV data
\app{gretl} accepts several common representations of missing values
including $-$999, the string \verb+NA+ (in upper or lower case), a
single dot, or simply a blank cell.  Blank cells should, of course, be
properly delimited, e.g. \verb+120.6,,5.38+, in which the middle value
is presumed missing.

As for handling of missing values in the course of statistical
analysis, \app{gretl} does the following:

\begin{itemize}
\item In calculating descriptive statistics (mean, standard deviation,
  etc.) under the \cmd{summary} command, missing values are simply
  skipped and the sample size adjusted appropriately.
\item In running regressions \app{gretl} first adjusts the beginning
  and end of the sample range, truncating the sample if need be.
  Missing values at the beginning of the sample are common in time
  series work due to the inclusion of lags, first differences and so
  on; missing values at the end of the range are not uncommon due to
  differential updating of series and possibly the inclusion of leads.
\end{itemize}

If \app{gretl} detects any missing values ``inside'' the (possibly
truncated) sample range for a regression, the result depends on the
character of the dataset and the estimator chosen.  In many cases, the
program will automatically skip the missing observations when
calculating the regression results.  In this situation a message is
printed stating how many observations were dropped.  On the other
hand, the skipping of missing observations is not supported for all
procedures: exceptions include all autoregressive estimators, system
estimators such as SUR, and nonlinear least squares.  In the case of
panel data, the skipping of missing observations is supported only if
their omission leaves a balanced panel. If missing observations are
found in cases where they are not supported, \app{gretl} gives an
error message and refuses to produce estimates.

In case missing values in the middle of a dataset present a problem,
the \cmd{misszero} function (use with care!) is provided under the
\cmd{genr} command. By doing \cmd{genr foo = misszero(bar)} you can
produce a series \cmd{foo} which is identical to \cmd{bar} except that
any missing values become zeros.  Then you can use carefully
constructed dummy variables to, in effect, drop the missing
observations from the regression while retaining the surrounding
sample range.\footnote{\cmd{genr} also offers the inverse function to
  \cmd{misszero}, namely \cmd{zeromiss}, which replaces zeros in a
  given series with the missing observation code.}

\section{Data file collections}
\label{collections}


If you're using \app{gretl} in a teaching context you may be
interested in adding a collection of data files and/or scripts that
relate specifically to your course, in such a way that students can
browse and access them easily.

This is quite easy as of \app{gretl} version 1.2.1.  There are three
ways to access such collections of files:

\begin{itemize}
\item For data files: select the menu item ``File, Open data, sample
  file'', or click on the folder icon on the \app{gretl} toolbar.
\item For script files: select the menu item ``File, Open command
  file, practice file''.
\end{itemize}

When a user selects one of the items:

\begin{itemize}
\item The data or script files included in the gretl distribution are
  automatically shown (this includes files relating to Ramanathan's
  \emph{Introductory Econometrics} and Greene's \emph{Econometric
    Analysis}).
\item The program looks for certain known collections of data files
  available as optional extras, for instance the datafiles from
  various econometrics textbooks (Wooldridge, Gujarati, Stock and
  Watson) and the Penn World Table (PWT 5.6).  (See
  \href{http://gretl.sourceforge.net/gretl_data.html}{the data page}
  at the gretl website for information on these collections.)  If the
  additional files are found, they are added to the selection windows.
\item The program then searches for valid file collections (not
  necessarily known in advance) in these places: the ``system'' data
  directory, the system script directory, the user directory, and all
  first-level subdirectories of these.  (For reference, typical values
  for these directories are shown in Table \ref{tab-colls}.)
\end{itemize}

\begin{table}[htbp]
  \begin{center}
    \caption{Typical locations for file collections}
    \label{tab-colls}
    \begin{tabular}{lll}
      � & Linux & MS Windows\\
        system data dir & \verb+/usr/share/gretl/data+ 
        & \verb+c:\userdata\gretl\data+\\
        system script dir & \verb+/usr/share/gretl/scripts+ 
        & \verb+c:\userdata\gretl\scripts+\\
        user dir & \verb+/home/me/gretl+ 
        & \verb+c:\userdata\gretl\user+\\
      \end{tabular}
    \end{center}
  \end{table}

  Any valid collections will be added to the selection windows. So
  what constitutes a valid file collection?  This comprises either a
  set of data files in \app{gretl} XML format (with the \verb+.gdt+
  suffix) or a set of script files containing gretl commands (with
  \verb+.inp+ suffix), in each case accompanied by a ``master file''
  or catalog.  The \app{gretl} distribution contains several example
  catalog files, for instance the file \verb+descriptions+ in the
  \verb+misc+ sub-directory of the \app{gretl} data directory and
  \verb+ps_descriptions+ in the \verb+misc+ sub-directory of the
  scripts directory.  If you are adding your own collection, data
  catalogs should be named \verb+descriptions+ and script catalogs
  should be be named \verb+ps_descriptions+.  In each case the catalog
  should be placed (along with the associated data or script files) in
  its own specific sub-directory (e.g.
  \verb+/usr/share/gretl/data/mydata+ or
  \verb+c:\userdata\gretl\data\mydata+).The syntax of the (plain text)
  description files is straightforward.  Here, for example, are the
  first few lines of gretl's ``misc'' data catalog:

\begin{code}
      # Gretl: various illustrative datafiles
      "arma","artificial data for ARMA script example"
      "ects_nls","Nonlinear least squares example"
      "hamilton","Prices and exchange rate, U.S. and Italy"
\end{code}

  The first line, which must start with a hash mark, contains a short
  name, here ``Gretl'', which will appear as the label for this
  collection's tab in the data browser window, followed by a colon,
  followed by an optional short description of the
  collection.

  Subsequent lines contain two elements, separated by a comma and
  wrapped in double quotation marks.  The first is a datafile name
  (leave off the \verb+.gdt+ suffix here) and the second is a short
  description of the content of that datafile.  There should be one
  such line for each datafile in the collection.

  A script catalog file looks very similar, except that there are
  three fields in the file lines: a filename (without its \verb+.inp+
  suffix), a brief description of the econometric point illustrated in
  the script, and a brief indication of the nature of the data used.
  Again, here are the first few lines of the supplied ``misc'' script
  catalog:

\begin{code}
      # Gretl: various sample scripts
      "arma.inp","ARMA modeling","artificial data"
      "ects_nls","Nonlinear least squares (Davidson)","artificial data"
      "leverage","Influential observations","artificial data"
      "longley","Multicollinearity","US employment"
\end{code}

  If you want to make your own data collection available to users,
  these are the steps:

  \begin{enumerate}
  \item Assemble the data, in whatever format is convenient.
  \item Convert the data to \app{gretl} format and save as \verb+gdt+
    files.  It is probably easiest to convert the data by importing
    them into the program from plain text, CSV, or a spreadsheet
    format (MS Excel or Gnumeric) then saving them. You may wish to
    add descriptions of the individual variables (the ``Variable, Edit
    attributes'' menu item), and add information on the source of the
    data (the ``Data, Edit info'' menu item).
  \item Write a descriptions file for the collection using a text
    editor.
  \item Put the datafiles plus the descriptions file in a subdirectory
    of the \app{gretl} data directory (or user directory).
  \item If the collection is to be distributed to other people,
    package the data files and catalog in some suitable manner, e.g.
    as a zipfile.
  \end{enumerate}

  If you assemble such a collection, and the data are not proprietary,
  I would encourage you to submit the collection for packaging as a
  \app{gretl} optional extra.

%%% Local Variables: 
%%% mode: latex
%%% TeX-master: "gretl-guide"
%%% End: 


\chapter{Funzioni speciali in genr}
\label{chap-genr}

\section{Introduzione}
\label{genr-intro}

Il comando \verb+genr+ offre un modo flessibile per definire nuove
variabili. Il comando � documentato nel \emph{XXX}, mentre questo
capitolo offre una discussione pi� approfondita di alcune delle
funzioni speciali disponibili con \verb+genr+ e di alcune
particolarit� del comando.
    

\section{Filtri per serie storiche}
\label{genr-filter}

Un tipo di funzione specializzata di \verb+genr+ � il filtro per le
serie storiche. Ne esistono di due tipi al momento: il filtro di
Hodrick--Prescott e quello passa banda di Baxter--King.  Sono
utilizzabili rispettivamente con le funzioni \verb+hpfilt()+ e
\verb+bkfilt()+, che richiedono come argomento il nome della variabile
da processare.
    
\subsection{Il filtro di Hodrick--Prescott}
\label{hodrick-prescott}

Da scrivere.

\subsection{Il filtro di Baxter e King}
\label{baxter-king}

Si consideri la rappresentazione spettrale di una serie storica $y_t$:
%	
\[ y_t = \int_{-\pi}^{\pi} e^{i\omega} \mathrm{d} Z(\omega) \]
%
se volessimo estrarre solo la componente di \emph{y\ensuremath{_{t}}}
che si trova tra le frequenze $\underline{\omega}$ e
$\overline{\omega}$ potremmo applicare un filtro passa banda:
%	
\[ c^*_t = \int_{-\pi}^{\pi} F^*(\omega) e^{i\omega} \mathrm{d}
Z(\omega) \] 
%
dove $F^*(\omega) = 1$ per $\underline{\omega} < |\omega| <
\overline{\omega}$ e 0 altrove. Ci� implicherebbe, nel dominio
temporale, applicare alla serie un filtro con un numero infinito di
coefficienti, cosa non desiderabile. Il filtro passa banda di Baxter e
King applica a $y_t$ un polinomio finito nell'operatore di ritardo
$A(L)$:
%	
\[ c_t = A(L) y_t \]
%
dove $A(L)$ � definito come
%	
\[ A(L) = \sum_{i=-k}^{k} a_i L^i \]

I coefficienti $a_i$ sono scelti in modo che $F(\omega) =
A(e^{i\omega})A(e^{-i\omega})$ sia la migliore approssimazione di
$F^*(\omega)$ per un dato $k$. Chiaramente, maggiore � $k$, migliore �
l'approssimazione, ma poich� occorre scartare $2k$ osservazioni, di
solito si cerca un compromesso.  Inoltre, il filtro ha altre propriet�
teoriche interessanti, tra cui quella che $a(1) = 0$, quindi una serie
con una sola radice unitaria � resa stazionaria con l'applicazione del
filtro.

In pratica, il filtro � usato di solito con dati mensili o trimestrali
per estrarne la componente di ``ciclo economico'', ossia la componente
tra 6 e 36 trimestri. I valori usuali per $k$ sono 8 o 12 (o forse di
pi� per serie mensili).  I valori predefiniti per i limiti di
frequenza sono 8 e 32, mentre il valore predefinito per l'ordine di
approssimazione, $k$, � 8.  � possibile impostare questi valori usando
il comando \cmd{set}.  La parola chiave per impostare i limiti di
frequenza � \verb+bkbp_limits+, mentre quella per $k$ � \verb+bkbp_k+.
Quindi ad esempio, se si stanno usando dati mensili e si vuole
impostare i limiti di frequenza tra 18 e 96, e $k$ a 24, si pu�
eseguire

\begin{code}
	set bkbp_limits 18 96
	set bkbp_k 24
\end{code}

Questi valori resteranno in vigore per le chiamate alla funzione
\verb+bkfilt+ finch� non saranno modificati da un altro uso di
\verb+set+.
      

\section{Ricampionamento e bootstrapping}
\label{genr-resample}

Un'altra funzione particolare � il ricampionamento, con reimmissione,
di una serie. Da scrivere.
    

\section{Gestione dei valori mancanti}
\label{genr-missing}

Sono disponibili quattro funzioni speciali per gestire i valori
mancanti.  La funzione booleana \verb+missing()+ richiede come unico
argomento il nome di una variabile e produce una serie con valore 1
per ogni osservazione in cui la variabile indicata ha un valore
mancante, 0 altrove (ossia dove la variabile indicata ha un valore
valido). La funzione \verb+ok()+ � il complemento di \verb+missing+,
ossia una scorciatoia per \verb+!missing+ (dove \verb+!+ � l'operatore
booleano NOT).  Ad esempio, � possibile contare i valori mancanti
della variabile \verb+x+ usando

\begin{code}
      genr nmanc_x = sum(missing(x))
\end{code}

La funzione \verb+zeromiss()+, che richiede anch'essa come unico
argomento il nome di una serie, produce una serie in cui tutti i
valori zero sono trasformati in valori mancanti. Occorre usarla con
attenzione (di solito non bisogna confondere valori mancanti col
valore zero), ma pu� essere utile in alcuni casi: ad esempio, �
possibile determinare la prima osservazione valida di una variabile
\verb+x+ usando

\begin{code}
        genr time
        genr x0 = min(zeromiss(time * ok(x)))
\end{code}


La funzione \verb+misszero()+ compie l'operazione opposta di
\verb+zeromiss+, ossia converte tutti i valori mancanti in zero.  

Pu� essere utile chiarire la propagazione dei valori mancanti
all'interno delle formule di \verb+genr+. La regola generale � che
nelle operazioni aritmetiche che coinvolgono due variabili, se una
delle variabili ha un valore mancante in corrispondenza
dell'osservazione $t$, anche la serie risultante avr� un valore
mancante in $t$. L'unica eccezione a questa regola � la
moltiplicazione per zero: zero moltiplicato per un valore mancante
produce sempre zero (visto che matematicamente il risultato � zero a
prescindere dal valore dell'altro fattore).
    

\section{Recupero di variabili interne}
\label{genr-internal}

Il comando \verb+genr+ fornisce un modo per recuperare vari valori
calcolati dal programma nel corso della stima dei modelli o della
verifica di ipotesi. Le variabili che possono essere richiamate in
questo modo sono elencate nella XXX; qui ci occupiamo in particolare
delle variabili speciali \verb+$test+ e \verb+$pvalue+.

Queste variabili contengono, rispettivamente, il valore dell'ultima
statistica test calcolata durante l'ultimo uso esplicito di un comando
di test e il p-value per quella statistica test. Se non � stato
eseguito alcun comando di test, le variabili contengono il codice di
valore mancante. I ``comandi espliciti di test'' che funzionano in
questo modo sono i seguenti: \cmd{add} (test congiunto per la
significativit� di variabili aggiunte a un modello); \cmd{adf} (test
di Dickey--Fuller aumentato, si veda oltre); \cmd{arch} (test per
ARCH); \cmd{chow} (test Chow per break strutturale); \cmd{coeffsum}
(test per la somma dei coefficienti specificati); \cmd{cusum} (la
statistica \emph{t} di Harvey--Collier); \cmd{kpss} (il test di
stazionariet� KPSS, p-value non disponibile); \cmd{lmtest} (si veda
oltre); \cmd{meantest} (test per la differenza delle medie);
\cmd{omit} (test congiunto per la significativit� delle variabili
omesse da un modello); \cmd{reset} (test RESET di Ramsey);
\cmd{restrict} (vincolo lineare generale); \cmd{runs} (test delle
successioni per la casualit�); \cmd{testuhat} (test per la normalit�
dei residui) e \cmd{vartest} (test per la differenza delle varianze).
Nella maggior parte dei casi, vengono salvati valori sia in
\verb+$test+ che in \verb+$pvalue+; l'eccezione � il test KPSS, per
cui non � disponibile il p-value.
    
Un punto da tenere in considerazione a questo proposito � che le
variabili interne \verb+$test+ e \verb+$pvalue+ vengono sovrascritte
ogni volta che viene eseguito uno dei test elencati sopra. Se si
intende referenziare questi valori durante una sequenza di comandi
\app{gretl}, occorre farlo nel momento giusto.
    
Una questione correlata � che alcuni dei comandi di test generano di
solito pi� di una statistica test e pi� di un p-value: in questi casi
vengono salvati solo gli ultimi valori. Per controllare in modo
preciso quali valori vengono recuperati da \verb+$test+ e
\verb+$pvalue+ occorre formulare il comando di test in modo che il
risultato non sia ambiguo. Questa nota vale in particolare per i
comandi \verb+adf+ e \verb+lmtest+.

\begin{itemize}
\item Di solito, il comando \cmd{adf} genera tre varianti del test
  Dickey--Fuller: una basata su una regressione che include una
  costante, una che include costante e trend lineare, e una che
  include costante e trend quadratico. Se si intende estrarre valori
  da \verb+$test+ o \verb+$pvalue+ dopo aver usato questo comando, �
  possibile selezionare la variante per cui verranno salvati i valori,
  usando una delle opzioni \verb+--nc+, \verb+--c+, \verb+--ct+ o
  \verb+--ctt+ con il comando \verb+adf+.
\item Di solito, il comando \cmd{lmtest} (che deve seguire una
  regressione OLS) esegue vari test diagnostici sulla regressione in
  questione. Per controllare cosa viene salvato in \verb+$test+ e
  \verb+$pvalue+ occorre limitare il test usando una delle opzioni
  \verb+--logs+, \verb+--autocorr+, \verb+--squares+ 
  o \verb+--white+.
\end{itemize}

Un aiuto all'uso dei valori immagazzinati in \verb+$test+ e
\verb+$pvalue+ � dato dal fatto che il tipo di test a cui si
riferiscono questi valori viene scritto nell'etichetta descrittiva
della variabile generata. Per controllare di aver recuperato il valore
corretto, � possibile leggere l'etichetta con il comando \cmd{label}
(il cui unico argomento � il nome della variabile). La seguente
sessione interattiva illustra la procedura.
    
\begin{code}
      ? adf 4 x1 --c

      Test Dickey-Fuller aumentati, ordine 4, per x1
      ampiezza campionaria 59
      ipotesi nulla di radice unitaria: a = 1

        test con costante
        modello: (1 - L)y = b0 + (a-1)*y(-1) + ... + e
        valore stimato di (a - 1): -0.216889
        statistica test: t = -1.83491
        p-value asintotico 0.3638

      P-value basati su MacKinnon (JAE, 1996)
      ? genr pv = $pvalue
      Generato lo scalare pv (ID 13) = 0.363844
      ? label pv    
      pv=Dickey-Fuller pvalue (scalar)
\end{code}


%%% Local Variables: 
%%% mode: latex
%%% TeX-master: "gretl-guide-it"
%%% End: 


\chapter{Modifica del campione}
\label{sampling}

\section{Introduzione}
\label{sample-intro}

Questo capitolo affronta alcune questioni correlate alla creazione di
sotto-campioni in un dataset.

� possibile definire un sotto-campione per un dataset in due modi
diversi, che chiameremo rispettivamente ``impostazione'' del campione
e ``restrizione'' del campione.
    
\section{Impostazione del campione}
\label{sample-set}

Per ``impostazione'' del campione, si intende la definizione di un
campione ottenuta indicando il punto iniziale e/o quello finale
dell'intervallo del campione. Questa modalit� � usata
tipicamente con le serie storiche; ad esempio se si hanno dati
trimestrali per l'intervallo da 1960:1 a 2003:4 e si vuole stimare una
regressione usando solo i dati degli anni '70, un comando adeguato �

\begin{code}
smpl 1970:1 1979:4
\end{code}
      
Oppure se si vuole riservare la parte finale delle osservazioni
disponibili per eseguire una previsione "fuori dal campione", si pu�
usare il comando

\begin{code}
smpl ; 2000:4
\end{code}
      
dove il punto e virgola significa ``mantenere inalterata
l'osservazione iniziale'' (e potrebbe essere usato in modo analogo al
posto del secondo parametro, indicando di mantenere inalterata
l'osservazione finale). Per ``inalterata'' in questo caso si intende
inalterata relativamente all'ultima impostazione eseguita con
\verb+smpl+, o relativamente all'intero dataset, se in precedenza non � ancora
stato definito alcun sotto-campione.  Ad esempio, dopo

\begin{code}
smpl 1970:1 2003:4
smpl ; 2000:4
\end{code}
      
l'intervallo del campione sar� da 1970:1 a 2000:4.

� possibile anche impostare l'intervallo del campione in modo incrementale o
relativo: in questo caso occorre indicare per il punto iniziale e finale uno
spostamento relativo, sotto forma di numero preceduto dal segno pi� o dal segno
meno (o un punto e virgola per indicare nessuna variazione). Ad esempio

\begin{code}
smpl +1 ;
\end{code}
      
sposter� in avanti di un'osservazione l'inizio del campione,
mantenendo inalterata la fine del campione, mentre

\begin{code}
smpl +2 -1
\end{code}

sposter� l'inizio del campione in avanti di due osservazioni e la fine
del campione indietro di una.

Una caratteristica importante dell'operazione di ``impostazione del campione''
descritta fin qui � che il sotto-campione creato risulta sempre composto da un
insieme di osservazioni contigue. La struttura del dataset rimane quindi
inalterata: se si lavora su una serie trimestrale, dopo aver impostato il
campione la serie rimarr� trimestrale.
    

\section{Restrizione del campione}
\label{sample-restrict}

Per ``restrizione'' del campione si intende la definizione di un
campione ottenuta selezionando le osservazioni in base a un criterio
Booleano (logico), o usando un generatore di numeri casuali. Questa
modalit� � usata tipicamente con dati di tipo cross-section o panel.

Si supponga di avere dei dati di tipo cross-section che descrivono il
genere, il reddito e altre caratteristiche di un gruppo di individui e
si vogliano analizzare solo le donne presenti nel campione. Se si
dispone di una variabile dummy \verb+genere+, che vale 1 per gli
uomini e 0 per le donne, si potrebbe ottenere questo risultato con

\begin{code}
smpl genere=0 --restrict
\end{code}
    
Oppure si supponga di voler limitare il campione di lavoro ai soli
individui con un reddito superiore ai 50.000 euro. Si potrebbe usare
\begin{code}
smpl reddito>50000 --restrict
\end{code}

Qui sorge un problema: eseguendo in sequenza i due comandi visti
sopra, cosa conterr� il sotto-campione? Tutti gli individui con
reddito superiore a 50.000 euro o solo le donne con reddito superiore
a 50.000 euro? La risposta corretta � la seconda: la seconda
restrizione si aggiunge alla prima, ossia la restrizione finale � il
prodotto logico della nuova restrizione e di tutte le restrizioni
precedenti.  Se si vuole applicare una nuova restrizione
indipendentemente da quelle applicate in precedenza, occorre prima
re-impostare il campione alla sua lunghezza originaria, usando
    
\begin{code}
smpl --full
\end{code}
%
In alternativa, � possibile aggiungere l'opzione \verb+replace+ al
comando \verb+smpl+:
%
\begin{code}
smpl income>50000 --restrict --replace
\end{code}

Questa opzione ha l'effetto di re-impostare automaticamente il
campione completo prima di applicare la nuova restrizione.

A differenza della semplice ``impostazione'' del campione, la
``restrizione'' del campione pu� produrre un insieme di osservazioni
non contigue nel dataset originale e pu� anche modificare la struttura
del dataset.

Questo fenomeno pu� essere osservato nel caso dei dati
panel: si supponga di avere un panel di cinque imprese (indicizzate
dalla variabile \verb+impresa+) osservate in ognuno degli anni
identificati dalla variabile \verb+anno+.  La restrizione
%
\begin{code}
smpl anno=1995 --restrict
\end{code}
%
produce un dataset che non � pi� di tipo panel, ma cross-section per
l'anno 1995.  In modo simile
%      
\begin{code}
smpl impresa=3 --restrict
\end{code}
%
produce un dataset di serie storiche per l'impresa numero 3.

Per questi motivi (possibile non-contiguit� nelle osservazioni,
possibile cambiamento nella struttura dei dati) gretl si comporta in
modo diverso a seconda che si operi una ``restrizione'' del campione o
una semplice ``impostazione'' di esso. Nel caso dell'impostazione, il
programma memorizza semplicemente le osservazioni iniziali e finali e
le usa come parametri per i vari comandi di stima dei modelli, di
calcolo delle statistiche ecc. Nel caso della restrizione, il
programma crea una copia ridotta del dataset e la tratta come un
semplice dataset di tipo cross-section non datato.\footnote{Con una
  eccezione: se si parte da un dataset panel bilanciato e la
  restrizione � tale da preservare la struttura di panel bilanciato
  (ad esempio perch� implica la cancellazione di tutte le osservazioni
  per una unit� cross-section), allora il dataset ridotto � ancora
  trattato come panel.}

Se si vuole re-imporre un'interpretazione di tipo ``serie storiche'' o
``panel'' al dataset ridotto, occorre usare il comando \cmd{setobs}, o
il comando dal men� ``Dati, Struttura dataset'', se appropriato.

Il fatto che una ``restrizione'' del campione comporti la creazione di
una copia ridotta del dataset originale pu� creare problemi quando il
dataset � molto grande (nell'ordine delle migliaia di osservazioni).
Se si usano simili dataset, la creazione della copia pu� causare
l'esaurimento della memoria del sistema durante il calcolo dei
risultati delle regressioni. � possibile aggirare il problema in
questo modo:

\begin{enumerate}
\item Aprire il dataset completo e imporre la restrizione sul
  campione.
\item Salvare su disco una copia del dataset ridotto.
\item Chiudere il dataset completo e aprire quello ridotto.
\item Procedere con l'analisi.
\end{enumerate}

\section{Campionamento casuale}
\label{sample-random}

Se si usano dataset molto grandi (o se si intende studiare le
propriet� di uno stimatore), pu� essere utile estrarre un campione
casuale dal dataset completo. � possibile farlo ad esempio con

\begin{code}
smpl 100 --random
\end{code}

che seleziona 100 osservazioni.  Se occorre che il campione sia
riproducibile, occorre per prima cosa impostare il seme del generatore
di numeri casuali, usando il comando \cmd{set}. Questo tipo di
campionamento � un esempio di ``restrizione'' del campione: viene
infatti generata una copia ridotta del dataset.

\section{I comandi del men� Campione}
\label{sample-menu}

Gli esempi visti finora hanno mostrato il comando testuale \cmd{set},
ma � possibile creare un sotto-campione usando i comandi del men�
``Campione'' nell'interfaccia grafica del programma.

I comandi del men� permettono di ottenere gli stessi risultati delle
varianti del comando testuale \verb+smpl+, con la seguente eccezione:
se si usa il comando ``Campione, Imposta in base a condizione...'' e
sul dataset � gi� stato impostato un sotto-campione, viene data la
possibilit� di preservare la restrizione gi� attiva o di sostituirla
(in modo analogo a quanto avviene invocando l'opzione \verb+replace+
descritta nella sezione~\ref{sample-restrict}).
    
%%% Local Variables: 
%%% mode: latex
%%% TeX-master: "gretl-guide-it"
%%% End: 


\chapter{Graphics}
\label{chap:graphs}

\section{Gnuplot graphs}
\label{gnuplot-graphs}

A separate program, \app{gnuplot}, is called to generate graphs.
Gnuplot is a very full-featured graphing program with myriad options.
It is available from \href{http://www.gnuplot.info/}{www.gnuplot.info}
(but note that a suitable copy of gnuplot is bundled with the packaged
versions of gretl for MS Windows and Mac OS X).  Gretl gives you
direct access, via a graphical interface, to a subset of gnuplot's
options and it tries to choose sensible values for you; it also allows
you to take complete control over graph details if you wish.

With a graph displayed, you can right-click on the graph window (or
use the ``hamburger'' toolbar button) for a pop-up menu with the
following options.

\begin{itemize}
\item \textsf{Save as PNG}: Save the graph in Portable Network
  Graphics format (the same format that you see on screen).
\item \textsf{Save as postscript (EPS)}: Save in encapsulated
  postscript format.
\item \textsf{Save as PDF}: Save in PDF format.
\item \textsf{Save as Windows metafile}: Save in Enhanced Metafile
  (EMF) format (with color and monochrome options).
\item \textsf{Copy to clipboard}: with color and monochrome options.
\item \textsf{Save to session as icon}: The graph will appear in
  iconic form when you select ``Icon view'' from the View menu.
\item \textsf{Zoom}: Lets you select an area within the graph for
  closer inspection (not available for all graphs).
\item \textsf{Display PDF}: view a PDF version of the graph.
\item \textsf{Edit}: Opens a controller for the plot which lets you
  adjust many aspects of its appearance.
\item \textsf{Close}: Closes the graph window.
\end{itemize}

If you select \textsf{Save as postscript} or \textsf{Save as PDF} you
get a dialog box that lets you adjust several aspects of the graph,
and also preview the result.

\subsection{Displaying data labels}
\label{plot-labels}

For simple X-Y scatter plots, some further options are available if
the dataset includes ``case markers'' (that is, labels identifying
each observation).\footnote{For an example of such a dataset, see the
  Ramanathan file \verb+data4-10+: this contains data on private
  school enrollment for the 50 states of the USA plus Washington, DC;
  the case markers are the two-letter codes for the states.} With a
scatter plot displayed, when you move the mouse pointer over a data
point its label is shown on the graph.  By default these labels are
transient: they do not appear in the printed or copied version of the
graph.  They can be removed by selecting ``Clear data labels'' from
the graph pop-up menu. If you want the labels to be affixed
permanently (so they will show up when the graph is printed or
copied), select the option ``Freeze data labels'' from the pop-up
menu; ``Clear data labels'' cancels this operation.  The other
label-related option, ``All data labels'', requests that case markers
be shown for all observations.  At present the display of case markers
is disabled for graphs containing more than 250 data points.


\subsection{GUI plot editor}
\label{plot-editor}

Selecting the \textsf{Edit} option in the graph popup menu opens an
editing dialog box, shown in Figure~\ref{fig-plot}.  Notice that there
are several tabs, allowing you to adjust many aspects of a graph's
appearance: font, title, axis scaling, line colors and types, and so
on.  You can also add lines or descriptive labels to a graph (under
the Lines and Labels tabs).  The ``Apply'' button applies your changes
without closing the editor; ``OK'' applies the changes and closes the
dialog.

\begin{figure}[htbp]
  \begin{center}
    \includegraphics[scale=0.6]{figures/plot_control}
  \end{center}
  \caption{gretl's plot controller}
  \label{fig-plot}
\end{figure}


\subsection{Publication-quality graphics: advanced options}
\label{plot-advanced}

The GUI plot editor has two limitations.  First, it cannot represent
all the myriad options that \app{gnuplot} offers. Users who are
sufficiently familiar with \app{gnuplot} to know what they're missing
in the plot editor presumably don't need much help from gretl,
so long as they can get hold of the \app{gnuplot} command file that
gretl has put together.  Second, even if the plot editor meets
your needs, in terms of fine-tuning the graph you see on screen, a few
details may need further work in order to get optimal results for
publication.

Either way, the first step in advanced tweaking of a graph is to get
access to the graph command file.

\begin{itemize}
\item In the graph display window, right-click and choose ``Save to
  session as icon''.
\item If it's not already open, open the icon view window---either
  via the menu item View/Icon view, or by clicking the ``session icon
  view'' button on the main-window toolbar.
\item Right-click on the icon representing the newly added graph and
  select ``Edit plot commands'' from the pop-up menu.
\item You get a window displaying the plot file
  (Figure~\ref{fig:plot-edit}).
\end{itemize}

\begin{figure}[htbp]
  \centering
  \includegraphics[scale=0.6]{figures/plotedit}
  \caption{Plot commands editor}
  \label{fig:plot-edit}
\end{figure}

Here are the basic things you can do in this window.  Obviously, you
can edit the file you just opened.  You can also send it for
processing by gnuplot, by clicking the ``Execute'' (cogwheel) icon in
the toolbar.  Or you can use the ``Save as'' button to save a copy for
editing and processing as you wish. And please note that the Help
button on the toolbar (a lifebelt in Figure~\ref{fig:plot-edit}) gives
you access to the \app{gnuplot} manual.

One relatively simple editorial job would be to set a chosen driver
(or ``terminal'' in gnuplot parlance) and output filename. For
example, to get PDF output you could insert lines like the following
at the top:

\begin{code}
# PDF, slightly amended (the default size is 5in x 3in)
set term pdfcairo font "Sans,6" size 5in,3.5in
set output 'mygraph1.pdf'
# or small size
set term pdfcairo font "Sans,5" size 3in,2in
set output 'mygraph2.pdf'
# or with size given in centimeters
set term pdfcairo font "Sans,6" size 6cm,4.2cm
set output 'mygraph3.pdf'
\end{code}

Or substitute \texttt{epscairo} for \texttt{pdfcairo} (and change the
filenames) if you want EPS output. However, such changes may be more
easily made via the \textsf{Save as PDF} and \textsf{Save as
  postscript} options in the plot menu.\footnote{A ``traditional''
  \texttt{postscript} terminal may also be available in gnuplot, with
  an \texttt{eps} option. The defaults in this case are quite
  different from \texttt{epscairo}, and to make use of the alternative
  you'll have to consult the \app{gnuplot} manual.}

The real payoff to editing the plot code can be obtained if you dive
into the details and employ \app{gnuplot} features that are not
accessible via gretl, and/or use one of the terminal types not
directly supported by gretl, such as \texttt{context} (ConTeXt),
\texttt{mp} (MetaPost), \texttt{lua} (Lua) or \texttt{pslatex}
(\LaTeX\ picture environment with PostScript specials). The
\texttt{lua} terminal with the \texttt{tikz} option is especially
useful for \LaTeX\ users, because it produces a \texttt{tikzpicture}
environment, which offers almost unlimited customization possibilities
(note that in order to use plots produced in this way you'll also need
the \texttt{gnuplot-lua-tikz} \LaTeX\ package).

To find out more about \app{gnuplot} visit
\href{https://gnuplot.sourceforge.net/}{gnuplot.sourceforge.net}. This
site has documentation for the current version of the program in
various formats along with a large collection of demonstration plots.

\subsection{Additional tips}
\label{subsect-graph-tips}

To be written.  Line widths, enhanced text.  Show a ``before and
after'' example.  

\section{Plotting graphs from scripts}
\label{sec:plotenv}

When working with scripts, you may want to have a graph shown onto
your display or saved into a file. In fact, if in your usual workflow
you find yourself creating similar graphs over and over again, you
might want to consider the option of writing a script which automates
this process for you. gretl gives you two main tools for doing
this: one is a command called \cmd{gnuplot}, whose main use is to
create standard plot quickly. The other one is the \cmd{plot} command
block, which has a more elaborate syntax but offers you more control
on output.

\subsection{The \cmd{gnuplot} command}
\label{sec:gnuplot-cmd}

The \cmd{gnuplot} command is described at length in the \GCR\ and the
online help system. Here, we just summarize its main features:
basically, it consists of the \cmd{gnuplot} keyword, followed by a
list of items, telling the command \emph{what} you want plotted and a
list of options, telling it \emph{how} you want it plotted.

For example, the line
\begin{code}
gnuplot y1 y2 x   
\end{code}
will give you a basic XY plot of the two series \texttt{y1} and
\texttt{y2} on the vertical axis versus the series \texttt{x} on the
horizontal axis. In general, the arguments to the \cmd{gnuplot}
command is a list of series, the last of which goes on the x-axis,
while all the other ones go onto the y-axis. By default, the
\cmd{gnuplot} command gives you a scatterplot. If you just have one
variable on the y-axis, then gretl will also draw a the OLS
interpolation, if the fit is good enough.\footnote{The technical
  condition for this is that the two-tailed $p$-value for the slope
  coefficient should be under 10\%.}

Several aspects of the behavior described above can be modified. You
do this by appending options to the command. Most options can be
broadly grouped in three categories:
\begin{enumerate}
\item Plot styles: we support points (the default choice), lines,
  lines and points together, and impulses (vertical lines). 
\item Algorithm for the fitted line: here you can choose between
  linear, quadratic and cubic interpolation, but also more exotic
  choices, such as semi-log, inverse or loess (non-parametric). Of
  course, you can also turn this feature off.
\item Input and output: you can choose whether you want your graph on
  your computer screen (and possibly use the in-built graphical widget
  to further customize it --- see above, page \pageref{plot-editor}),
  or rather save it to a file. We support several graphical formats,
  among which PNG and PDF, to make it easy to incorporate your
  plots into text documents.
\end{enumerate}

Listing~\ref{awm-plots} shows examples of some traditional plots in macroeconomics,
using time series from the ``area-wide model'' dataset produced by the 
European Central Bank, which is shipped with gretl in the file \texttt{AWM.gdt}.
\texttt{PCR} is aggregate private real consumption and \texttt{YER} is real GDP.

\begin{script}
  \caption{Plotting macroeconomic data}
  \label{awm-plots}
\begin{scode}
open AWM.gdt --quiet

# --- consumption and income, different styles ------------

gnuplot PCR YER
gnuplot PCR YER --output=display
gnuplot PCR YER --output=display --time-series
gnuplot PCR YER --output=display --time-series --with-lines

# --- Phillips' curve, different fitted lines -------------

gnuplot INFQ URX --output=display
gnuplot INFQ URX --fit=none --output=display
gnuplot INFQ URX --fit=inverse --output=display
gnuplot INFQ URX --fit=loess --output=display
\end{scode}
\end{script}

The first command line in the listing
plots consumption against income as a kind of Keynesian 
consumption function. More precisely, it produces a simple scatter plot with
an automatically linear fitted line. If this is executed in the gretl console
the plot will be directly shown in a new window, but if this line is contained
in a script then instead a file with the plot commands will be saved for later
execution. The second example line changes this behavior for a script command
and forces the plot to be shown directly. 

The third line instead asks for a plot of the two variables as two separate
curves against time on the x-axis. Each observation point is drawn separately
with a certain symbol determined by gnuplot defaults. If you add the option
\cmd{--with-lines} the points will be connected with a continuous line and the
symbols omitted.

The second batch of examples demonstrate how the fitted line in the 
scatter plot can be controlled from gretl's side. The option \cmd{--fit=none}
overrides gnuplot's default to draw a line if it deems the fit to be ``good
enough''. The effect of \cmd{--fit=inverse} is to consider the variable on the
y-axis as a function of $1/X$ instead of $X$ and draw the corresponding
hyperbolic branch. For the workings of a Loess fit (locally-weighted polynomial 
regression) please refer to the documentation of the \cmd{loess} function.

For more detail, consult the \GCR.


\subsection{The \cmd{plot} command block}
\label{sec:plotblock}

The \cmd{plot} environment is a way to pass information to
\app{Gnuplot} in a more structured way, so that customization of basic
plots becomes easier. It has the following characteristics:

The block starts with the \cmd{plot} keyword, followed by a required
parameter: the name of a list, a single series or a matrix. This
parameter specifies the data to be plotted. The starting line may be
prefixed with the \verb|savename <-| apparatus to save a plot as an icon
in the GUI program. The block ends with \cmd{end plot}.

Inside the block you have zero or more lines of these types, identified 
by an initial keyword:
\begin{description}
\item[\normalfont \texttt{option}:] specify a single option (details below)
\item[\normalfont \texttt{options}:] specify multiple options on a single line; if
  more than one option is given on a line, the options should be
  separated by spaces.
\item[\normalfont \texttt{literal}:] a command to be passed to gnuplot literally 
\item[\normalfont \texttt{printf}:] a printf statement whose result will be passed
  to gnuplot literally; this allows the use of string variables
  without having to resort to \verb!@!-style string substitution.
\end{description}

The options available are basically those of the current \cmd{gnuplot} 
command, but with a few differences. For one thing you don't need the 
leading double-dash in an "option" (or "options") line. Besides that,
\begin{itemize}
\item You can't use the option \option{matrix=whatever} with \cmd{plot}:
  that possibility is handled by providing the name of a matrix on the
  initial \cmd{plot} line.
\item The \option{input=filename} option is not supported: use
  \cmd{gnuplot} for the case where you're supplying the entire plot
  specification yourself.
\item The several options pertaining to the presence and type of a
  fitted line, are replaced in \cmd{plot} by a single option \cmd{fit} which
  requires a parameter. Supported values for the parameter are: none,
  linear, quadratic, cubic, inverse, semilog and loess. Example:
\begin{code}
  option fit=quadratic
\end{code}
\end{itemize}

As with \cmd{gnuplot}, the default is to show a linear fit in an X-Y
scatter if it's significant at the 10 percent level.

Here's a simple example, the plot specification from the ``bandplot''
package, which shows how to achieve the same result via the
\cmd{gnuplot} command and a \cmd{plot} block, respectively---the
latter occupies a few more lines but is clearer

\begin{code}
   gnuplot 1 2 3 4 --with-lines --matrix=plotmat \
   --fit=none --output=display \
   { set linetype 3 lc rgb "#0000ff"; set title "@title"; \
     set nokey; set xlabel "@xname"; }
\end{code}

\begin{code}
   plot plotmat
     options with-lines fit=none
     literal set linetype 3 lc rgb "#0000ff"
     literal set nokey
     printf "set title \"%s\"", title
     printf "set xlabel \"%s\"", xname
   end plot --output=display
\end{code}

Note that \option{output=display} is appended to \cmd{end plot}; also
note that if you give a matrix to \cmd{plot} it's assumed you want to
plot all the columns. In addition, if you give a single series and the
dataset is time series, it's assumed you want a time-series plot.

\subsubsection{Example: Plotting an histogram together with a density}

\begin{script}[htbp]
  \scriptinfo{mroz-logwage}{Plotting the log wage from the Mroz example dataset}
\begin{scode}
set verbose off
open mroz87.gdt --quiet

series lWW = log(WW)
scalar m = mean(lWW)
scalar s = sd(lWW)

###
### prepare matrix with data for plot
###

# number of valid observations
scalar n = nobs(lWW)
# discretize log wage
scalar k = 4
series disc_lWW = round(lWW*k)/k
# get frequencies
matrix f = aggregate(null, disc_lWW)
# add density
phi = dnorm((f[,1] - m)/s) / (s*k)
# put columns together and add labels
plotmat = f[,2]./n ~ phi  ~ f[,1]
strings cnames = defarray("frequency", "density", "log wage")
cnameset(plotmat, cnames)

###
### create plot
###

plot plotmat
    # move legend
    literal set key outside rmargin
    # set line style
    literal set linetype 2 dashtype 2 linewidth 2
    # set histogram color
    literal set linetype 1 lc rgb "#777777"
    # set histogram style
    literal set style fill solid 0.25 border
    # set histogram width
    printf "set boxwidth %4.2f\n", 0.5/k
    options with-lines=2 with-boxes=1
end plot --output=display
\end{scode}
\end{script}

Listing \ref{ex:mroz-logwage} contains a slightly more elaborate
example: here we load the Mroz example dataset and calculate the log
of the individual's wage. Then, we match the histogram of a
discretized version of the same variable (obtained via the
\cmd{aggregate()} function) versus the theoretical density if data
were Gaussian.

There are a few points to note:
\begin{itemize}
\item The data for the plot are passed through a matrix in which we
  set column names via the \cmd{cnameset} function; those names are
  then automatically used by the \cmd{plot} environment.
\item In this example, we make extensive use of the \cmd{set literal}
  construct for refining the plot by passing instruction to
  \app{gnuplot}; the power of \app{gnuplot} is impossible to
  overstate. We encourage you to visit the ``demos'' version of
  \app{gnuplot}'s website (\url{http://gnuplot.sourceforge.net/}) and
  revel in amazement.
\item In the \cmd{plot} environment you can use all the quantities you
  have in your script. This is the way we calibrate the histogram
  width (try setting the scalar \verb|k| in the script to different
  values). Note that the \cmd{printf} command has a special meaning
  inside a \cmd{plot} environment.
\item The script displays the plot on your screen. If you want to save
  it to a file instead, replace \verb!--output=display! at the end
  with \texttt{-{}-output=\textsl{filename}}.
\item It's OK to insert comments in the \cmd{plot} environment;
  actually, it's a rather good idea to comment as much as possible (as
  always)!
\end{itemize}
The output from the script is shown in Figure \ref{fig:mroz-logwage}.


\begin{figure}[htbp]
  \centering
  \includegraphics{figures/Mroz-logwage}
  \caption{Output from listing \ref{ex:mroz-logwage}}
  \label{fig:mroz-logwage}
\end{figure}

\subsubsection{Example: Plotting Student's $t$ densities}

\begin{script}[htbp]
  \scriptinfo{student_plot}{Plotting $t$ densities for varying degrees of freedom}
\begin{scode}
set verbose off

function string tplot(scalar m)
    return sprintf("stud(x,%d) title \"t(%d)\"", m, m)
end function

matrix dfs = {2, 4, 16}

plot
    literal set xrange [-4.5:4.5]
    literal set yrange [0:0.45]
    literal Binv(p,q) = exp(lgamma(p+q)-lgamma(p)-lgamma(q))
    literal stud(x,m) = Binv(0.5*m,0.5)/sqrt(m)*(1.0+(x*x)/m)**(-0.5*(m+1.0))
    printf "plot %s, %s, %s", tplot(dfs[1]), tplot(dfs[2]), tplot(dfs[3])
end plot --output=display
\end{scode}
\end{script}

The power of the \cmd{printf} statement in a \cmd{plot} block becomes
apparent when used jointly with user-defined functions, as exemplified
in Listing \ref{ex:student_plot}, in which we create a plot showing
the density functions of Student's $t$ distribution for three
different settings of the ``degrees of freedom'' parameter (note that
plotting a $t$ density is very easy to do from the GUI: just go to the
\emph{Tools $>$ Distribution graphs} menu).

First we define a user function called \texttt{tplot}, which returns a
string with the ingredients to pass to the gnuplot \texttt{plot}
statement, as a function of a scalar parameter (the degrees of freedom
in our case). Next, this function is used within the \cmd{plot} block
to plot the appropriate density. Note that most of the 
statements to mathematically define the function to plot are
outsourced to gnuplot via the \cmd{literal} command.

The output from the script is shown in Figure \ref{fig:StudentPlot}.

\begin{figure}[htbp]
  \centering
  \includegraphics{figures/StudentPlot}
  \caption{Output from listing \ref{ex:student_plot}}
  \label{fig:StudentPlot}
\end{figure}


\pagebreak[4]

\section{Boxplots}
\label{sect-boxplots}

These plots (after Tukey and Chambers) display the distribution of a
variable. Its shape depends on a few quantities, defined as follows:

\begin{center}
\begin{tabular}{rl}
  $x_{\mathrm{min}}$ & sample minimum \\
  $Q_1$ & first quartile \\
  $m$ & median \\
  $\bar{x}$ & mean \\
  $Q_3$ & third quartile \\
  $x_{\mathrm{max}}$ & sample maximum\\
  $R = Q_3 - Q_1$ & interquartile range\\
\end{tabular}
\end{center}

The central box encloses the middle 50 percent of the data, i.e.\ goes
from $Q_1$ to $Q_3$; therefore, its height equals $R$.  A line
is drawn across the box at the median $m$ and a ``\texttt{+}'' sign
identifies the mean $\bar{x}$.

The length of the ``whiskers'' depends on the presence of
outliers. The top whisker extends from the top of the box up to a
maximum of 1.5 times the interquartile range, but can be shorter if
the sample maximum is lower than that value; that is, it reaches
$\min[x_{\mathrm{max}}, Q_3 + 1.5 R]$. Observations larger than
$Q_3 + 1.5 R$, if any, are considered outliers and represented
individually via dots.\footnote{To give you an intuitive idea, if a
  variable is normally distributed, the chances of picking an outlier
  by this definition are slightly below 0.7\%.} The bottom whisker
obeys the same logic, with obvious adjustments.
Figure~\ref{fig-boxplot} provides an example of all this by using the
variable \texttt{FAMINC} from the sample dataset \texttt{mroz87}.

\begin{figure}[htbp]
  \begin{flushleft}
    \hspace{1cm}
  \definecolor{c0000ff}{RGB}{0,0,255}
\definecolor{c00cc00}{RGB}{0,204,0}
\begin{tikzpicture}[y=0.80pt, x=0.80pt, yscale=-1.000000, xscale=1.000000, inner sep=0pt, outer sep=0pt]
\begin{scope}[cm={{1.5,0.0,0.0,-1.5,(0.0,336.0)}}]
    \path[draw=black,line join=miter,line cap=rect,miter limit=3.80,line
      width=0.200pt] (38.7500,13.3980) -- (41.8520,13.3980);
    \path[draw=black,line join=miter,line cap=rect,miter limit=3.80,line
      width=0.200pt] (38.7500,228.1990) -- (38.7500,8.5000);
    \path[cm={{1.0,0.0,0.0,-1.0,(29.76698,11.2373)}},fill=black,nonzero rule]
      (0.0000,0.0000) node[above right] (text18) {0};
    \path[draw=black,line join=miter,line cap=rect,miter limit=3.80,line
      width=0.200pt] (38.7500,56.4490) -- (41.8520,56.4490);
    \path[cm={{1.0,0.0,0.0,-1.0,(11.95058,54.28809)}},fill=black,nonzero rule]
      (0.0000,0.0000) node[above right] (text24) {20000};
    \path[draw=black,line join=miter,line cap=rect,miter limit=3.80,line
      width=0.200pt] (38.7500,99.5000) -- (41.8520,99.5000);
    \path[cm={{1.0,0.0,0.0,-1.0,(11.95058,97.33496)}},fill=black,nonzero rule]
      (0.0000,0.0000) node[above right] (text30) {40000};
    \path[draw=black,line join=miter,line cap=rect,miter limit=3.80,line
      width=0.200pt] (38.7500,142.5510) -- (41.8520,142.5510);
    \path[cm={{1.0,0.0,0.0,-1.0,(11.95058,140.38574)}},fill=black,nonzero rule]
      (0.0000,0.0000) node[above right] (text36) {60000};
    \path[draw=black,line join=miter,line cap=rect,miter limit=3.80,line
      width=0.200pt] (38.7500,185.6020) -- (41.8520,185.6020);
    \path[cm={{1.0,0.0,0.0,-1.0,(11.95058,183.43652)}},fill=black,nonzero rule]
      (0.0000,0.0000) node[above right] (text42) {80000};
    \path[draw=black,line join=miter,line cap=rect,miter limit=3.80,line
      width=0.200pt] (38.7500,228.1990) -- (38.7500,8.5000);
    \path[cm={{1.0,0.0,0.0,-1.0,(155,238)}},fill=black,nonzero rule]
      (0.0000,0.0000) node[above right] (text48) {FAMINC};
    \path[draw=c0000ff,line join=miter,line cap=rect,miter limit=3.80,line
      width=0.400pt] (173,16.6480) -- (173,46.6020);
    \path[draw=c0000ff,line join=miter,line cap=rect,miter limit=3.80,line
      width=0.400pt] (173,74.1990) -- (173,114.6020);
    \path[draw=c0000ff,line join=miter,line cap=rect,miter limit=3.80,line
      width=0.400pt] (150,74.1990) -- (196,74.1990) -- (196,46.6010)
      -- (150,46.6010) -- cycle;
    \path[draw=black,line join=miter,line cap=rect,miter limit=3.80,line
      width=0.200pt] (150,58.3520) -- (196,58.3520) -- (150,58.3520);
    \path[draw=c00cc00,line join=miter,line cap=rect,miter limit=3.80,line
      width=0.200pt] (170,63.1020) -- (176,63.1020);
    \path[draw=c00cc00,line join=miter,line cap=rect,miter limit=3.80,line
      width=0.200pt] (173,66.1020) -- (173,60.1020);
    \path[draw=black,fill=black,line join=miter,line cap=rect,miter
      limit=3.80,nonzero rule,line width=0.200pt] (173,115.6480) .. controls
      (173,114.6480) and (172,114.6480) .. (172,115.6480) .. controls
      (172,116.6480) and (173,116.6480) .. (173,115.6480);
    \path[draw=black,fill=black,line join=miter,line cap=rect,miter
      limit=3.80,nonzero rule,line width=0.200pt] (173,117.3980) .. controls
      (173,116.3980) and (172,116.3980) .. (172,117.3980) .. controls
      (172,118.3980) and (173,118.3980) .. (173,117.3980);
    \path[draw=black,fill=black,line join=miter,line cap=rect,miter
      limit=3.80,nonzero rule,line width=0.200pt] (173,118.4490) .. controls
      (173,117.4490) and (172,117.4490) .. (172,118.4490) .. controls
      (172,119.4490) and (173,119.4490) .. (173,118.4490);
    \path[draw=black,fill=black,line join=miter,line cap=rect,miter
      limit=3.80,nonzero rule,line width=0.200pt] (173,120.1990) .. controls
      (173,119.1990) and (172,119.1990) .. (172,120.1990) .. controls
      (172,121.1990) and (173,121.1990) .. (173,120.1990);
    \path[draw=black,fill=black,line join=miter,line cap=rect,miter
      limit=3.80,nonzero rule,line width=0.200pt] (173,121.6990) .. controls
      (173,120.6990) and (172,120.6990) .. (172,121.6990) .. controls
      (172,122.6990) and (173,122.6990) .. (173,121.6990);
    \path[draw=black,fill=black,line join=miter,line cap=rect,miter
      limit=3.80,nonzero rule,line width=0.200pt] (173,122.6480) .. controls
      (173,121.6480) and (172,121.6480) .. (172,122.6480) .. controls
      (172,123.6480) and (173,123.6480) .. (173,122.6480);
    \path[draw=black,fill=black,line join=miter,line cap=rect,miter
      limit=3.80,nonzero rule,line width=0.200pt] (173,123.0000) .. controls
      (173,122.0000) and (172,122.0000) .. (172,123.0000) .. controls
      (172,124.0000) and (173,124.0000) .. (173,123.0000);
    \path[draw=black,fill=black,line join=miter,line cap=rect,miter
      limit=3.80,nonzero rule,line width=0.200pt] (173,123.1990) .. controls
      (173,122.1990) and (172,122.1990) .. (172,123.1990) .. controls
      (172,124.1990) and (173,124.1990) .. (173,123.1990);
    \path[draw=black,fill=black,line join=miter,line cap=rect,miter
      limit=3.80,nonzero rule,line width=0.200pt] (173,123.1990) .. controls
      (173,122.1990) and (172,122.1990) .. (172,123.1990) .. controls
      (172,124.1990) and (173,124.1990) .. (173,123.1990);
    \path[draw=black,fill=black,line join=miter,line cap=rect,miter
      limit=3.80,nonzero rule,line width=0.200pt] (173,123.6020) .. controls
      (173,122.6020) and (172,122.6020) .. (172,123.6020) .. controls
      (172,124.6020) and (173,124.6020) .. (173,123.6020);
    \path[draw=black,fill=black,line join=miter,line cap=rect,miter
      limit=3.80,nonzero rule,line width=0.200pt] (173,126.6480) .. controls
      (173,125.6480) and (172,125.6480) .. (172,126.6480) .. controls
      (172,127.6480) and (173,127.6480) .. (173,126.6480);
    \path[draw=black,fill=black,line join=miter,line cap=rect,miter
      limit=3.80,nonzero rule,line width=0.200pt] (173,126.7500) .. controls
      (173,125.7500) and (172,125.7500) .. (172,126.7500) .. controls
      (172,127.7500) and (173,127.7500) .. (173,126.7500);
    \path[draw=black,fill=black,line join=miter,line cap=rect,miter
      limit=3.80,nonzero rule,line width=0.200pt] (173,131.8010) .. controls
      (173,130.8010) and (172,130.8010) .. (172,131.8010) .. controls
      (172,132.8010) and (173,132.8010) .. (173,131.8010);
    \path[draw=black,fill=black,line join=miter,line cap=rect,miter
      limit=3.80,nonzero rule,line width=0.200pt] (173,133.1020) .. controls
      (173,132.1020) and (172,132.1020) .. (172,133.1020) .. controls
      (172,134.1020) and (173,134.1020) .. (173,133.1020);
    \path[draw=black,fill=black,line join=miter,line cap=rect,miter
      limit=3.80,nonzero rule,line width=0.200pt] (173,134.1480) .. controls
      (173,133.1480) and (172,133.1480) .. (172,134.1480) .. controls
      (172,135.1480) and (173,135.1480) .. (173,134.1480);
    \path[draw=black,fill=black,line join=miter,line cap=rect,miter
      limit=3.80,nonzero rule,line width=0.200pt] (173,136.1020) .. controls
      (173,135.1020) and (172,135.1020) .. (172,136.1020) .. controls
      (172,137.1020) and (173,137.1020) .. (173,136.1020);
    \path[draw=black,fill=black,line join=miter,line cap=rect,miter
      limit=3.80,nonzero rule,line width=0.200pt] (173,136.7500) .. controls
      (173,135.7500) and (172,135.7500) .. (172,136.7500) .. controls
      (172,137.7500) and (173,137.7500) .. (173,136.7500);
    \path[draw=black,fill=black,line join=miter,line cap=rect,miter
      limit=3.80,nonzero rule,line width=0.200pt] (173,139.3520) .. controls
      (173,138.3520) and (172,138.3520) .. (172,139.3520) .. controls
      (172,140.3520) and (173,140.3520) .. (173,139.3520);
    \path[draw=black,fill=black,line join=miter,line cap=rect,miter
      limit=3.80,nonzero rule,line width=0.200pt] (173,147.0000) .. controls
      (173,146.0000) and (172,146.0000) .. (172,147.0000) .. controls
      (172,148.0000) and (173,148.0000) .. (173,147.0000);
    \path[draw=black,fill=black,line join=miter,line cap=rect,miter
      limit=3.80,nonzero rule,line width=0.200pt] (173,147.9490) .. controls
      (173,146.9490) and (172,146.9490) .. (172,147.9490) .. controls
      (172,148.9490) and (173,148.9490) .. (173,147.9490);
    \path[draw=black,fill=black,line join=miter,line cap=rect,miter
      limit=3.80,nonzero rule,line width=0.200pt] (173,149.4490) .. controls
      (173,148.4490) and (172,148.4490) .. (172,149.4490) .. controls
      (172,150.4490) and (173,150.4490) .. (173,149.4490);
    \path[draw=black,fill=black,line join=miter,line cap=rect,miter
      limit=3.80,nonzero rule,line width=0.200pt] (173,150.1020) .. controls
      (173,149.1020) and (172,149.1020) .. (172,150.1020) .. controls
      (172,151.1020) and (173,151.1020) .. (173,150.1020);
    \path[draw=black,fill=black,line join=miter,line cap=rect,miter
      limit=3.80,nonzero rule,line width=0.200pt] (173,156.1480) .. controls
      (173,155.1480) and (172,155.1480) .. (172,156.1480) .. controls
      (172,157.1480) and (173,157.1480) .. (173,156.1480);
    \path[draw=black,fill=black,line join=miter,line cap=rect,miter
      limit=3.80,nonzero rule,line width=0.200pt] (173,159.8520) .. controls
      (173,158.8520) and (172,158.8520) .. (172,159.8520) .. controls
      (172,160.8520) and (173,160.8520) .. (173,159.8520);
    \path[draw=black,fill=black,line join=miter,line cap=rect,miter
      limit=3.80,nonzero rule,line width=0.200pt] (173,171.8520) .. controls
      (173,170.8520) and (172,170.8520) .. (172,171.8520) .. controls
      (172,172.8520) and (173,172.8520) .. (173,171.8520);
    \path[draw=black,fill=black,line join=miter,line cap=rect,miter
      limit=3.80,nonzero rule,line width=0.200pt] (173,179.1480) .. controls
      (173,178.1480) and (172,178.1480) .. (172,179.1480) .. controls
      (172,180.1480) and (173,180.1480) .. (173,179.1480);
    \path[draw=black,fill=black,line join=miter,line cap=rect,miter
      limit=3.80,nonzero rule,line width=0.200pt] (173,185.1020) .. controls
      (173,184.1020) and (172,184.1020) .. (172,185.1020) .. controls
      (172,186.1020) and (173,186.1020) .. (173,185.1020);
    \path[draw=black,fill=black,line join=miter,line cap=rect,miter
      limit=3.80,nonzero rule,line width=0.200pt] (173,202.8520) .. controls
      (173,201.8520) and (172,201.8520) .. (172,202.8520) .. controls
      (172,203.8520) and (173,203.8520) .. (173,202.8520);
    \path[draw=black,fill=black,line join=miter,line cap=rect,miter
      limit=3.80,nonzero rule,line width=0.200pt] (173,208.8520) .. controls
      (173,207.8520) and (172,207.8520) .. (172,208.8520) .. controls
      (172,209.8520) and (173,209.8520) .. (173,208.8520);
    \path[draw=black,fill=black,line join=miter,line cap=rect,miter
      limit=3.80,nonzero rule,line width=0.200pt] (173,209.3980) .. controls
      (173,208.3980) and (172,208.3980) .. (172,209.3980) .. controls
      (172,210.3980) and (173,210.3980) .. (173,209.3980);
    \path[draw=black,fill=black,line join=miter,line cap=rect,miter
      limit=3.80,nonzero rule,line width=0.200pt] (173,220.0510) .. controls
      (173,219.0510) and (172,219.0510) .. (172,220.0510) .. controls
      (172,221.0510) and (173,221.0510) .. (173,220.0510);
    \path[draw=black,line join=miter,line cap=butt,miter limit=4.00,line
      width=0.544pt] (183,64.8980) -- (221,79.1760);
    \begin{scope}[cm={{1.0,0.37854,-0.37854,1.0,(0.0,252.0)}}]
      \path[draw=black,fill=black,line join=miter,line cap=butt,miter limit=4.00,even
        odd rule,line width=0.509pt] (98,-224) -- (100,-226) --
        (93,-224) -- (100,-221) -- cycle;
    \end{scope}
    \path[cm={{1.0,0.0,0.0,-1.0,(224,80)}},fill=black,nonzero rule]
      (0.0000,0.0000) node[above right] (text134) {$\bar{x}$};
    \path[cm={{1.0,0.0,0.0,-1.0,(132.5,76)}},fill=black,nonzero rule]
      (0.0000,0.0000) node[above right] (text142) {$Q_3$};
    \path[cm={{1.0,0.0,0.0,-1.0,(132.5,40)}},fill=black,nonzero rule]
      (0.0000,0.0000) node[above right] (text142) {$Q_1$};
    \path[cm={{1.0,0.0,0.0,-1.0,(132.5,57)}},fill=black,nonzero rule]
      (0.0000,0.0000) node[above right] (text142) {$m$};
    \path[cm={{1.0,0.0,0.0,-1.0,(178,12)}},fill=black,nonzero rule]
      (0.0000,0.0000) node[above right] (text146) {$x_{\mathrm{min}}$};
    \path[cm={{1.0,0.0,0.0,-1.0,(178,218)}},fill=black,nonzero rule]
      (0.0000,0.0000) node[above right] (text146) {$x_{\mathrm{max}}$};
    \path[cm={{1.0,0.0,0.0,-1.0,(110,165)}},fill=black,nonzero rule]
      (0.0000,0.0000) node[above right] (text152) {outliers};
    \path[draw=black,line join=miter,line cap=butt,miter limit=4.00,line
      width=0.626pt] (168,219.8910) .. controls (152,218.0820) and
      (162,185.3050) .. (159,174.6600) .. controls (159,172.9570) and
      (158,170.4340) .. (156,169.8520) .. controls (154,169.4100) and
      (153,168.6050) .. (151,168.0270) .. controls (153,167.4300) and
      (155,166.6450) .. (156,166.2070) .. controls (158,165.6250) and
      (160,163.0980) .. (160,161.3980) .. controls (162,150.7500) and
      (152,117.9770) .. (168,116.1680);
\end{scope}

\end{tikzpicture}

%%% Local Variables:
%%% mode: latex
%%% TeX-master: "../tex/gretl-guide"
%%% End:

  \end{flushleft}
  \caption{Sample boxplot}
  \label{fig-boxplot}
\end{figure}

In the case of boxplots with confidence intervals, dotted lines show
the limits of an approximate 90 percent confidence interval for the
median.  This is obtained by the bootstrap method, which can take a
while if the data series is very long. For details on constructing
boxplots, see the entry for \cmd{boxplot} in the \GCR\, or use the
\textsf{Help} button that appears when you select one of the boxplot
items under the menu item ``View, Graph specified vars'' in the main
gretl window.

\subsection{Factorized boxplots}

A nice feature which is quite useful for data visualization is the
conditional, or factorized boxplot.  This type of plot allows you to
examine the distribution of a variable conditional on the value of
some discrete factor.

As an example, we'll use one of the datasets supplied with
gretl, that is \cmd{rac3d}, which contains an example taken from
\cite{cameron-trivedi13} on the health conditions of 5190 people. The
script below compares the unconditional (marginal) distribution of the
number of illnesses in the past 2 weeks with the distribution of the
same variable, conditional on age classes.

\begin{scode}
open rac3d.gdt
# unconditional boxplot
boxplot ILLNESS --output=display
# create a discrete variable for age class: 
# 0 = below 20, 1 = between 20 and 39, etc
series age_class = floor(AGE/0.2)
# conditional boxplot
boxplot ILLNESS age_class --factorized --output=display
\end{scode}

After running the code above, you should see two graphs similar to
Figure \ref{fig:fact-boxplots}. By comparing the marginal plot to
the factorized one, the effect of age on the mean number of illnesses
is quite evident: by joining the green crosses you get what is
technically known as the conditional mean function, or regression
function if you prefer.

\begin{figure}[htbp]
  \centering
  \begin{tabular}{cc}
    \includegraphics[width=0.475\textwidth]{figures/uboxplot} & 
    \includegraphics[width=0.475\textwidth]{figures/fboxplot}
  \end{tabular}
  \caption{Conditional and unconditional distribution of illnesses}
  \label{fig:fact-boxplots}
\end{figure}

%%% Local Variables: 
%%% mode: latex
%%% TeX-master: "gretl-guide"
%%% End: 


\chapter{Discrete variables}
\label{chap:discrete}

When a variable can take only a finite, typically small, number of
values, then it is said to be \emph{discrete}. In gretl, variables of
the series type (only) can be marked as discrete. (When we speak of
``variables'' below this should be understood as referring to series.)
Some gretl commands act in a slightly different way when applied to
discrete variables; moreover, gretl provides a few commands that only
apply to discrete variables.  Specifically, the \texttt{dummify} and
\texttt{xtab} commands (see below) are available only for discrete
variables, while the \texttt{freq} (frequency distribution) command
produces different output for discrete variables.


\section{Declaring variables as discrete}
\label{discr-declare}

Gretl uses a simple heuristic to judge whether a given variable
should be treated as discrete, but you also have the option of
explicitly marking a variable as discrete, in which case the heuristic
check is bypassed.

The heuristic is as follows: First, are all the values of the variable
``reasonably round'', where this is taken to mean that they are all
integer multiples of 0.25?  If this criterion is met, we then ask
whether the variable takes on a ``fairly small'' set of distinct
values, where ``fairly small'' is defined as less than or equal to 8.
If both conditions are satisfied, the variable is automatically
considered discrete.

To mark a variable as discrete you have two options.
\begin{enumerate}
\item From the graphical interface, select ``Variable, Edit
  Attributes'' from the menu. A dialog box will appear and, if the
  variable seems suitable, you will see a tick box labeled ``Treat
  this variable as discrete''.  This dialog box can also be invoked
  via the context menu (right-click on a variable) or by pressing the
  F2 key.
\item From the command-line interface, via the \texttt{discrete}
  command. The command takes one or more arguments, which can be
  either variables or list of variables. For example:
\begin{code}
list xlist = x1 x2 x3
discrete z1 xlist z2
\end{code}
This syntax makes it possible to declare as discrete many
variables at once, which cannot presently be done via the graphical
interface. The switch \option{reverse} reverses the declaration of a
variable as discrete, or in other words marks it as continuous.
For example:
\begin{code}
discrete foo
# now foo is discrete
discrete foo --reverse
# now foo is continuous
\end{code}
\end{enumerate}

The command-line variant is more powerful, in that you can mark a
variable as discrete even if it does not seem to be suitable for
this treatment.

Note that marking a variable as discrete does not affect its content.
It is the user's responsibility to make sure that marking a variable
as discrete is a sensible thing to do.  Note that if you want to
recode a continuous variable into classes, you can use gretl's
arithmetical functionality, as in the following example:
\begin{code}
nulldata 100
# generate a series with mean 2 and variance 1
series x = normal() + 2
# split into 4 classes
series z = (x>0) + (x>2) + (x>4)
# now declare z as discrete
discrete z
\end{code}

Once a variable is marked as discrete, this setting is remembered when
you save the data file.

\section{Commands for discrete variables}
\label{discr-commands}

\subsection{The \texttt{dummify} command}
\label{discr-dummify}

The \texttt{dummify} command takes as argument a series $x$ and creates
dummy variables for each distinct value present in $x$, which must
have already been declared as discrete.  Example:
\begin{code}
open greene22_2
discrete Z5 # mark Z5 as discrete
dummify Z5
\end{code}

The effect of the above command is to generate 5 new dummy variables,
labeled \dtkttt{DZ5_1} through \dtkttt{DZ5_5}, which correspond to
the different values in \texttt{Z5}. Hence, the variable
\dtkttt{DZ5_4} is 1 if \texttt{Z5} equals 4 and 0 otherwise. This
functionality is also available through the graphical interface by
selecting the menu item ``Add, Dummies for selected discrete variables''.

The \texttt{dummify} command can also be used with the following
syntax:
\begin{code}
list dlist = dummify(x)
\end{code}
This not only creates the dummy variables, but also a named list (see
section~\ref{named-lists}) that can be used afterwards. The
following example computes summary statistics for the variable \texttt{Y} for
each value of \texttt{Z5}:
\begin{code}
open greene22_2
discrete Z5 # mark Z5 as discrete
list foo = dummify(Z5)
loop foreach i foo
  smpl $i --restrict --replace
  summary Y
endloop
smpl --full
\end{code}
% $

Since \texttt{dummify} generates a list, it can be used directly
in commands that call for a list as input, such as \texttt{ols}.  For
example:
\begin{code}
open greene22_2
discrete Z5 # mark Z5 as discrete
ols Y 0 dummify(Z5)
\end{code}

\subsection{The \texttt{freq} command}
\label{discr-freq}

The \texttt{freq} command displays absolute and relative frequencies
for a given variable. The way frequencies are counted depends on
whether the variable is continuous or discrete.  This command is also
available via the graphical interface by selecting the ``Variable,
Frequency distribution'' menu entry.

For discrete variables, frequencies are counted for each distinct
value that the variable takes. For continuous variables, values are
grouped into ``bins'' and then the frequencies are counted for each
bin. The number of bins, by default, is computed as a function of the
number of valid observations in the currently selected sample via the
rule shown in Table~\ref{tab:bins}. However, when the command is
invoked through the menu item ``Variable, Frequency Plot'', this
default can be overridden by the user.

\begin{table}[htbp]
  \centering
  \begin{tabular}{cc}
\hline
  Observations & Bins \\
\hline
  $8 \le n < 16$ & 5 \\
  $16 \le n < 50 $ & 7 \\
  $50 \le n \le 850 $ & $\lceil \sqrt{n} \rceil$  \\
  $n > 850 $ & 29 \\
\hline
\end{tabular}
\caption{Number of bins for various sample sizes}
\label{tab:bins}
\end{table}

For example, the following code
%
\begin{code}
open greene19_1
freq TUCE
discrete TUCE # mark TUCE as discrete
freq TUCE
\end{code}
%
yields
%
\begin{code}
Read datafile /usr/local/share/gretl/data/greene/greene19_1.gdt
periodicity: 1, maxobs: 32,
observations range: 1-32

Listing 5 variables:
  0) const    1) GPA      2) TUCE     3) PSI      4) GRADE  

? freq TUCE

Frequency distribution for TUCE, obs 1-32
number of bins = 7, mean = 21.9375, sd = 3.90151

       interval          midpt   frequency    rel.     cum.

          <  13.417     12.000        1      3.12%    3.12% *
    13.417 - 16.250     14.833        1      3.12%    6.25% *
    16.250 - 19.083     17.667        6     18.75%   25.00% ******
    19.083 - 21.917     20.500        6     18.75%   43.75% ******
    21.917 - 24.750     23.333        9     28.12%   71.88% **********
    24.750 - 27.583     26.167        7     21.88%   93.75% *******
          >= 27.583     29.000        2      6.25%  100.00% **

Test for null hypothesis of normal distribution:
Chi-square(2) = 1.872 with p-value 0.39211
? discrete TUCE # mark TUCE as discrete
? freq TUCE

Frequency distribution for TUCE, obs 1-32

          frequency    rel.     cum.

  12           1      3.12%    3.12% *
  14           1      3.12%    6.25% *
  17           3      9.38%   15.62% ***
  19           3      9.38%   25.00% ***
  20           2      6.25%   31.25% **
  21           4     12.50%   43.75% ****
  22           2      6.25%   50.00% **
  23           4     12.50%   62.50% ****
  24           3      9.38%   71.88% ***
  25           4     12.50%   84.38% ****
  26           2      6.25%   90.62% **
  27           1      3.12%   93.75% *
  28           1      3.12%   96.88% *
  29           1      3.12%  100.00% *

Test for null hypothesis of normal distribution:
Chi-square(2) = 1.872 with p-value 0.39211
\end{code}
%
As can be seen from the sample output, a Doornik--Hansen test for
normality is computed automatically.  This test is suppressed for
discrete variables where the number of distinct values is less than
10.

This command accepts two options: \option{quiet}, to avoid
generation of the histogram when invoked from the command line and
\option{gamma}, for replacing the normality test with Locke's
nonparametric test, whose null hypothesis is that the data follow a
Gamma distribution.

If the distinct values of a discrete variable need to be saved, the
\texttt{values()} matrix construct can be used (see chapter
\ref{chap:matrices}).

\subsection{The \texttt{xtab} command}
\label{discr-xtab}

The \texttt{xtab} command cab be invoked in either of the following
ways.  First,
%
\begin{code}
xtab ylist ; xlist
\end{code}
%
where \texttt{ylist} and \texttt{xlist} are lists of discrete
variables.  This produces cross-tabulations (two-way frequencies) of
each of the variables in \texttt{ylist} (by row) against each of the
variables in \texttt{xlist} (by column).  Or second,
%
\begin{code}
xtab xlist
\end{code}
%
In the second case a full set of cross-tabulations is generated; that
is, each variable in \texttt{xlist} is tabulated against each other
variable in the list.  In the graphical interface, this command is
represented by the ``Cross Tabulation'' item under the View menu,
which is active if at least two variables are selected.

Here is an example of use:
%
\begin{code}
open greene22_2
discrete Z* # mark Z1-Z8 as discrete
xtab Z1 Z4 ; Z5 Z6
\end{code}
which produces
\begin{code}
Cross-tabulation of Z1 (rows) against Z5 (columns)

       [   1][   2][   3][   4][   5]  TOT.
  
[   0]    20    91    75    93    36    315
[   1]    28    73    54    97    34    286

TOTAL     48   164   129   190    70    601

Pearson chi-square test = 5.48233 (4 df, p-value = 0.241287)

Cross-tabulation of Z1 (rows) against Z6 (columns)

       [   9][  12][  14][  16][  17][  18][  20]  TOT.
  
[   0]     4    36   106    70    52    45     2    315
[   1]     3     8    48    45    37    67    78    286

TOTAL      7    44   154   115    89   112    80    601

Pearson chi-square test = 123.177 (6 df, p-value = 3.50375e-24)

Cross-tabulation of Z4 (rows) against Z5 (columns)

       [   1][   2][   3][   4][   5]  TOT.
  
[   0]    17    60    35    45    14    171
[   1]    31   104    94   145    56    430

TOTAL     48   164   129   190    70    601

Pearson chi-square test = 11.1615 (4 df, p-value = 0.0248074)

Cross-tabulation of Z4 (rows) against Z6 (columns)

       [   9][  12][  14][  16][  17][  18][  20]  TOT.
  
[   0]     1     8    39    47    30    32    14    171
[   1]     6    36   115    68    59    80    66    430

TOTAL      7    44   154   115    89   112    80    601

Pearson chi-square test = 18.3426 (6 df, p-value = 0.0054306)
\end{code}

Pearson's $\chi^2$ test for independence is automatically displayed,
provided that all cells have expected frequencies under independence
greater than $10^{-7}$.  However, a common rule of thumb states that
this statistic is valid only if the expected frequency is 5 or
greater for at least 80 percent of the cells.  If this condition is not
met a warning is printed.

Additionally, the \option{row} or \option{column} options can be
given: in this case, the output displays row or column percentages,
respectively.

If you want to cut and paste the output of \texttt{xtab} to some other
program, e.g.\ a spreadsheet, you may want to use the \option{zeros}
option; this option causes cells with zero frequency to display the
number 0 instead of being empty.

%%% Local Variables: 
%%% mode: latex
%%% TeX-master: "gretl-guide"
%%% End: 

\chapter{Costrutti loop}
\label{looping}

\section{Introduzione}
\label{loop-intro}

Il comando \cmd{loop} permette di specificare un blocco di comandi
\app{gretl} da ripetere pi� volte. Questa funzionalit� � utile in
particolare per le simulazioni Monte Carlo, per il bootstrapping delle
statistiche test e per altre procedure di stima iterativa. La forma
generale di un loop, o ciclo, �:

\begin{code}
      loop espressione di controllo [ --progressive | --verbose | --quiet ]
         corpo del loop
      endloop
\end{code}

Sono disponibili cinque tipi di espressione di controllo, come si vede nella
sezione~\ref{loop-control}.

Non tutti i comandi di \app{gretl} sono disponibili all'interno di un loop: i
comandi accettabili in questo contesto sono mostrati nella
Tabella~\ref{tab:loopcmds}.

\begin{table}[htbp]
\caption{Comandi utilizzabili in un loop}
\label{tab:loopcmds}
\begin{center}
%% The following is generated automatically
\input ../tex/tabloopcmds.tex
\end{center}
\end{table}

In modalit� predefinita, il comando \cmd{genr} all'interno di un loop
opera in modo silenzioso (senza mostrare informazioni sulle variabili
generate). Per ottenere informazioni da \cmd{genr} � possibile
specificare l'opzione \verb+--verbose+ del comando \cmd{loop}. L'opzione
\verb+--quiet+ sopprime il consueto riepilogo del numero di iterazioni eseguite,
cosa desiderabile quando i loop sono annidati.

L'opzione \verb+--progressive+ del comando \cmd{loop} modifica il comportamento
dei comandi \cmd{ols}, \cmd{print} e \cmd{store} rendendoli pi� comodi
per l'uso in analisi di tipo Monte Carlo (si veda la
sezione~\ref{loop-progressive}).

Le sezioni che seguono descrivono le varie forme di espressioni di
controllo e forniscono alcuni esempi di uso dei loop.

\tip{Se occorre eseguire un'analisi Monte Carlo con molte migliaia di
  ripetizioni, possono verificarsi problemi di memoria e di tempo di
  calcolo. Un modo per minimizzare l'uso delle risorse di sistema
  consiste nell'eseguire lo script usando il programma a riga di
  comando, \app{gretlcli}, redirigendo i risultati su un file.}

\section{Varianti di controllo del loop}
\label{loop-control}

\subsection{Loop limitati}
\label{loop-count}

Il modo pi� semplice di controllare un loop consiste nello specificare
direttamente il numero di volte che il ciclo deve essere ripetuto, in
un cosiddetto ``loop limitato''.  Il numero di ripetizioni pu� essere
una costante numerica, ad esempio \verb+loop 1000+, o pu� essere letto
da una variabile, come in \verb+loop volte+.

Se il numero delle ripetizioni � indicato da una variabile, ad es.
\verb+volte+, la variabile dovrebbe essere uno scalare intero: se si
usa una serie, viene letto il primo valore, e se questo non � intero,
viene troncata la sua parte decimale. Si noti che \verb+volte+ viene
valutata solo una volta, quando il loop viene impostato.
      

\subsection{Loop di tipo while}
\label{loop-while}

Un secondo tipo di espressione di controllo consiste nell'uso del
comando \cmd{while} seguito da una disuguaglianza, in cui il termine a
sinistra � il nome di una variabile predefinita, mentre il lato destro
pu� essere una costante numerica o il nome di un'altra variabile
predefinita. Ad esempio:
        
\cmd{loop while differ > .00001} 

L'esecuzione del ciclo di comandi continuer� fintanto che la
condizione specificata rimane vera. Se il termine a destra della
disuguaglianza � una variabile, essa viene valutata all'inizio di ogni
nuova iterazione del ciclo.

\subsection{Loop con indice}
\label{loop-index}

Un terzo tipo di controllo di un loop utilizza la speciale variabile
indice \verb+i+.  In questo caso, vengono specificati valori iniziali
e finali per \verb+i+, che viene incrementata di uno ogni volta che
viene eseguito il ciclo.  La sintassi � la seguente: \cmd{loop
  i=1..20}.

La variabile indice pu� essere usata all'interno del corpo del loop,
in uno dei modi seguenti: � possibile accedere al valore di \verb+i+
(si veda l'esempio~\ref{loop-panel-script}), oppure � possibile usare
la sua rappresentazione come stringa \verb+$i+ (si veda
l'esempio~\ref{loop-string-script}).

I valori iniziale e finale per l'indice possono essere indicati in
forma numerica, o come riferimento a variabili predefinite.
Nell'ultimo caso, le variabili vengono valutate una volta, quando il
loop viene impostato. Inoltre, con dataset di serie storiche �
possibile indicare i valori iniziale e finale sotto forma di date, ad
esempio: \cmd{loop i=1950:1..1999:4}.
      

\subsection{Loop di tipo for each}
\label{loop-each}

Anche il terzo tipo di controllo usa la variabile interna \verb+i+,
che in questo caso pu� assumere valori solo all'interno di una lista
specifica di stringhe. Il loop � eseguito una volta per ogni stringa
presente nella lista, agevolando l'esecuzione di operazioni ripetitive
su un gruppo di variabili. Ecco un esempio:
      
\begin{code}
        loop foreach i mele pere pesche
          print "$i"
        endloop
\end{code}

Questo loop verr� eseguito tre volte, mostrando ``mele'', ``pere'' e
``pesche'' ad ogni iterazione.  

Per eseguire un loop su una lista di variabili contigue nel dataset, �
possibile indicare i nomi della prima e dell'ultima variabile nella
lista, separate da ``\verb+..+'', invece di dover indicare tutti i
nomi. Ad esempio, ipotizzando di avere 50 variabili \verb+AK+,
\verb+AL+, \dots{}, \verb+WY+, che contengono livelli di reddito per
gli stati degli USA, per stimare una regressione del reddito sul tempo
per ognuno degli stati, basta eseguire:
       
\begin{code}
       genr time
       loop foreach i AL..WY
          ols $i const time
       endloop
\end{code}


\subsection{Loop di tipo for}
\label{loop-for}

L'ultimo tipo di controllo usa una forma semplificata dell'istruzione
\cmd{for} del linguaggio di programmazione C. L'espressione di
controllo si compone di tre parti, separate da punto e virgola. La
prima parte specifica una condizione iniziale, espressa per mezzo di
una variabile di controllo; la seconda parte imposta una condizione di
continuazione (espressa in funzione della stessa variabile di
controllo), mentre la terza parte specifica un incremento (o un
decremento) per la variabile di controllo, da applicare ogni volta che
il ciclo viene eseguito. L'intera espressione deve essere racchiusa
tra parentesi.  Ad esempio:

\cmd{loop for (r=0.01; r<.991; r+=.01)}

La variabile \verb+r+ assumer� i valori 0.01, 0.02, \dots{}, 0.99 nel
giro di 99 iterazioni. Si noti che a causa della precisione limitata
dell'aritmetica in virgola mobile usata dal computer, pu� dover essere
necessario usare una condizione di continuazione come quella mostrata
sopra, \verb+r<.991+, invece della pi� ``naturale'' \verb+r<=.99+
(usando numeri in doppia precisione su un processore x86, quando ci si
aspetta che \verb+r+ valga 0.99, potrebbe in realt� valere
0.990000000000001).

Esistono altre regole per i tre componenti dell'espressione
di controllo: 

\begin{enumerate}
\item La condizione iniziale deve avere forma X1 = Y1, dove Y1 deve
  essere una costante numerica o una variabile predefinita.  Se la
  variabile X1 non esiste, viene creata automaticamente.
\item La condizione di continuazione deve avere la forma X1 operatore
  Y2, dove l'operatore pu� essere \verb+<+, \verb+>+, \verb+<=+ o
  \verb+>=+ e Y2 deve essere una costante numerica o una variabile
  predefinita (nel caso in cui sia una variabile, essa viene valutata
  ad ogni esecuzione del ciclo).
\item L'espressione che indica l'incremento o il decremento deve avere
  la forma X1 += DELTA, oppure X1 -= DELTA, dove DELTA � una costante
  numerica o una variabile predefinita (nel secondo caso, essa viene
  valutata solo una volta, quando il loop viene impostato)
\end{enumerate}


\section{La modalit� progressiva}
\label{loop-progressive}

Usando l'opzione \verb+--progressive+ nel comando loop, l'effetto dei
comandi \cmd{ols}, \cmd{print} e \cmd{store} � modificato nel modo
seguente.

\cmd{ols}: i risultati di ogni iterazione della regressione non
vengono mostrati. Al contrario, una volta terminato il loop si ottiene
un elenco dei seguenti valori: (a) il valore medio di ognuno dei
coefficienti stimati, calcolato su tutte le iterazioni; (b) la
deviazione standard relativa a questa media; (c) il valore medio
dell'errore standard stimato per ogni coefficiente; (d) la deviazione
standard degli errori standard stimati. Tutto ci� ha senso solo se
ogni iterazione del loop contiene un elemento di casualit�.

\cmd{print}: se si usa questo comando per mostrare il valore di una
variabile, questo non viene mostrato ad ogni iterazione. Al contrario,
alla fine del loop si ottiene il valore medio e la deviazione standard
della variabile, calcolata su tutte le iterazioni del ciclo. Questa
funzione � utile per le variabili che assumono un singolo valore per
ogni iterazione, ad esempio la somma dei quadrati degli errori di una
regressione.

\cmd{store}: questo comando scrive i valori delle variabili
specificate, ad ogni iterazione del loop, nel file indicato, ossia,
tiene traccia completa del valore delle variabili in tutte le
iterazioni. Ad esempio, si potrebbero salvare le stime dei
coefficienti per poi studiarne la distribuzione di frequenza. �
possibile usare il comando \cmd{store} solo una volta all'interno di
un loop.

\section{Esempi di loop}
\label{loop-examples}

\subsection{Esempio di procedura Monte Carlo}
\label{loop-mc-example}

Un semplice esempio di uso della modalit� ``progressiva'' per
realizzare una procedura Monte Carlo � mostrato in
esempio~\ref{monte-carlo-loop}.

\begin{script}[htbp]
  \caption{Un semplice loop di tipo Monte Carlo}
  \label{monte-carlo-loop}
\begin{scode}
nulldata 50
seed 547
genr x = 100 * uniform()
# Apre un loop "progressivo", da ripetere 100 volte
loop 100 --progressive
   genr u = 10 * normal()
   # Costruisce la variabile dipendente
   genr y = 10*x + u
   # Esegue la regressione OLS
   ols y const x
   # Definisce variabili per i coefficienti e R-quadro
   genr a = $coeff(const)
   genr b = $coeff(x)
   genr r2 = $rsq
   # Mostra le statistiche su queste variabili
   print a b r2
   # Salva i coefficienti in un file
   store coeffs.gdt a b
endloop
\end{scode}
\end{script}

Questo loop mostrer� le statistiche di riepilogo per le stime di "a",
"b" e $R^2$ lungo le 100 iterazioni. Dopo aver eseguito il loop, �
possibile aprire con \app{gretl} il file \verb+coeffs.gdt+, che
contiene le stime dei singoli coefficienti durante tutte le
iterazioni, ed esaminare nel dettaglio la distribuzione di frequenza
delle stime.

Il comando \cmd{nulldata} � utile per le procedure Monte Carlo: invece
di aprire un ``vero'' dataset, \cmd{nulldata 50} (ad esempio) apre un
finto dataset da 50 osservazioni, che contiene solo la costante e una
variabile indice. Successivamente � possibile aggiungervi variabili
usando il comando \cmd{genr}. Si veda il comando \cmd{set} per
informazioni su come generare numeri pseudo-casuali in modo
ripetibile.

\subsection{Minimi quadrati iterati}
\label{loop-ils-examples}

L'esempio~\ref{greene-ils-script} usa un loop di tipo ``while'' per
replicare la stima di una funzione di consumo non lineare nella forma
        
\[ C = \alpha + \beta Y^{\gamma} + \epsilon \]

presentata in Greene (2000, Esempio 11.3).  Questo script � compreso
nella distribuzione di \app{gretl} con il nome \verb+greene11_3.inp+;
� possibile aprirlo usando il comando del men� ``File, Comandi, File
di esempio, Greene...''.

L'opzione \verb+--print-final+ per il comando \cmd{ols} fa s� che non
vengano mostrati i risultati della regressione per ogni iterazione, ma
solo quelli dell'ultima iterazione del loop.

\begin{script}[htbp]
  \caption{Funzione di consumo non lineare}
  \label{greene-ils-script}
\begin{scode}
open greene11_3.gdt
# Esegue la regressione OLS iniziale
ols C 0 Y
genr essbak = $ess
genr essdiff = 1
genr beta = $coeff(Y)
genr gamma = 1
# Itera OLS finch� la somma dei quadrati degli errori converge
loop while essdiff > .00001
   # Genera le variabili linearizzate
   genr C0 = C + gamma * beta * Y^gamma * log(Y)
   genr x1 = Y^gamma
   genr x2 = beta * Y^gamma * log(Y)
   # Esegue la regressione OLS 
   ols C0 0 x1 x2 --print-final --no-df-corr --vcv
   genr beta = $coeff(x1)
   genr gamma = $coeff(x2)
   genr ess = $ess
   genr essdiff = abs(ess - essbak)/essbak
   genr essbak = ess
endloop 
# Mostra le stime dei parametri usando i "nomi giusti"
noecho
printf "alfa = %g\n", $coeff(0)
printf "beta  = %g\n", beta
printf "gamma = %g\n", gamma
\end{scode}
\end{script}

L'esempio~\ref{jack-arma} mostra come sia possibile usare
un loop per stimare un modello ARMA usando la regressione ``prodotto
esterno del gradiente'' (OPG - ``outer product of the gradient'')
discussa da Davidson e MacKinnon nel loro \emph{Estimation and
  Inference in Econometrics}.

\begin{script}[htbp]
  \caption{ARMA 1, 1}
  \label{jack-arma}
\begin{scode}
open armaloop.gdt

genr c = 0
genr a = 0.1
genr m = 0.1

series e = 1.0
genr de_c = e
genr de_a = e
genr de_m = e

genr crit = 1
loop while crit > 1.0e-9

   # Errori di previsione "one-step"
   genr e = y - c - a*y(-1) - m*e(-1)  

   # Log-verosimiglianza 
   genr loglik = -0.5 * sum(e^2)
   print loglik

   # Derivate parziali degli errori di previsione rispetto a c, a e m
   genr de_c = -1 - m * de_c(-1) 
   genr de_a = -y(-1) -m * de_a(-1)
   genr de_m = -e(-1) -m * de_m(-1)

   # Derivate parziali di l rispetto a c, a e m
   genr sc_c = -de_c * e
   genr sc_a = -de_a * e
   genr sc_m = -de_m * e

   # Regressione OPG
   ols const sc_c sc_a sc_m --print-final --no-df-corr --vcv

   # Aggiorna i parametri
   genr dc = $coeff(sc_c) 
   genr c = c + dc
   genr da = $coeff(sc_a) 
   genr a = a + da
   genr dm = $coeff(sc_m) 
   genr m = m + dm

   printf "  constant        = %.8g (gradient = %#.6g)\n", c, dc
   printf "  ar1 coefficient = %.8g (gradient = %#.6g)\n", a, da
   printf "  ma1 coefficient = %.8g (gradient = %#.6g)\n", m, dm

   genr crit = $T - $ess
   print crit
endloop

genr se_c = $stderr(sc_c)
genr se_a = $stderr(sc_a)
genr se_m = $stderr(sc_m)

noecho
print "
printf "constant = %.8g (se = %#.6g, t = %.4f)\n", c, se_c, c/se_c
printf "ar1 term = %.8g (se = %#.6g, t = %.4f)\n", a, se_a, a/se_a
printf "ma1 term = %.8g (se = %#.6g, t = %.4f)\n", m, se_m, m/se_m
\end{scode}
\end{script}


\subsection{Esempi di loop con indice}

L'esempio~\ref{loop-panel-script} mostra un loop con indice, in cui il
comando \cmd{smpl} contiene la variabile indice \verb+i+.  Si supponga
di avere un dataset di tipo panel, con osservazioni su alcuni ospedali
per gli anni dal 1991 al 2000 (dove l'anno dell'osservazione �
indicato da una variabile chiamata \verb+anno+).  Ad ogni iterazione,
restringiamo il campione a un certo anno e calcoliamo statistiche di
riepilogo sulla dimensione longitudinale (cross-section) per le
variabili da 1 a 4.

\begin{script}[htbp]
  \caption{Statistiche panel}
  \label{loop-panel-script}
\begin{scode}
open ospedali.gdt
loop i=1991..2000
  smpl (anno=i) --restrict --replace
  summary 1 2 3 4
endloop
\end{scode}
\end{script}

L'esempio~\ref{loop-string-script} illustra un loop indicizzato per
sostituire stringhe.

\begin{script}[htbp]
  \caption{Sostituzione di stringhe}
  \label{loop-string-script}
\begin{scode}
open bea.dat
loop i=1987..2001
  genr V = COMP$i
  genr TC = GOC$i - PBT$i
  genr C = TC - V
  ols PBT$i const TC V
endloop
\end{scode}
\end{script}

Alla prima iterazione, la variabile \verb+V+ verr� impostata a
\verb+COMP1987+ e la variabile dipendente per il comando \cmd{ols}
sar� \verb+PBT1987+.  All'iterazione successiva, \verb+V+ verr�
ridefinita come \verb+COMP1988+ e la variabile dipendente della
regressione sar� \verb+PBT1988+, e cos� via.

%%% Local Variables: 
%%% mode: latex
%%% TeX-master: "gretl-guide-it"
%%% End: 


\chapter{User-defined functions}
\label{chap:functions}

\section{Defining a function}
\label{func-define}

\app{Gretl} offers a mechanism for defining functions, which may be
called via the command line, in the context of a script, or (if
packaged appropriately, see section~\ref{sec:func-packages}) via the
program's graphical interface.

The syntax for defining a function looks like this:\footnote{The
  syntax given here differs from the standard prior to \app{gretl}
  version 1.8.4.  For reasons of backward compatibility the old syntax
  is still supported; see section~\ref{sec:old-func} for details.}

\begin{raggedright}
\texttt{function} \textsl{return-type} \textsl{function-name}
\texttt{(}\textsl{parameters}\texttt{)} \\
\qquad  \textsl{function body} \\
\texttt{end function}
\end{raggedright}

The opening line of a function definition contains these elements, in
strict order:

\begin{enumerate}
\item The keyword \texttt{function}.
\item \textsl{return-type}, which states the type of value returned by
  the function, if any.  This must be one of \texttt{void} (if the
  function does not return anything), \texttt{scalar},
  \texttt{series}, \texttt{matrix}, \texttt{list} or \texttt{string}.
\item \textsl{function-name}, the unique identifier for the
  function.  Names must start with a letter. They have a maximum
  length of 31 characters; if you type a longer name it will be
  truncated.  Function names cannot contain spaces.  You will get an
  error if you try to define a function having the same name as an
  existing \app{gretl} command.
\item The functions's \textsl{parameters}, in the form of a
  comma-separated list enclosed in parentheses.  This may be run
  into the function name, or separated by white space as shown.
\end{enumerate}

Function parameters can be of any of the types shown
below.\footnote{An additional parameter type is available for GUI use,
  namely \texttt{obs}; this is equivalent to \texttt{int} except for
  the way it is represented in the graphical interface for calling a
  function.}

\begin{center}
\begin{tabular}{ll}
\multicolumn{1}{c}{Type} & 
\multicolumn{1}{c}{Description} \\ [4pt]
\texttt{bool}   & scalar variable acting as a Boolean switch \\
\texttt{int}    & scalar variable acting as an integer  \\
\texttt{scalar} & scalar variable \\
\texttt{series} & data series \\
\texttt{list}   & named list of series \\
\texttt{matrix} & matrix or vector \\
\texttt{string} & string variable or string literal \\
\texttt{bundle} & all-purpose container (see section~\ref{sec:Bundles})
\end{tabular}
\end{center}

Each element in the listing of parameters must include two terms: a
type specifier, and the name by which the parameter shall be known
within the function.  An example follows:
%    
\begin{code}
function scalar myfunc (series y, list xvars, bool verbose)
\end{code}

Each of the type-specifiers, with the exception of \texttt{list} and
\texttt{string}, may be modified by prepending an asterisk to the
associated parameter name, as in
%    
\begin{code}
function scalar myfunc (series *y, scalar *b)
\end{code}

The meaning of this modification is explained below (see section
\ref{funscope}); it is related to the use of pointer arguments in the
C programming language.

\subsection{Function parameters: optional refinements}

Besides the required elements mentioned above, the specification of a
function parameter may include some additional fields, as follows:
\begin{itemize}
\item The \texttt{const} modifier.
\item For \texttt{scalar} or \texttt{int} parameters: minimum, maximum
  and default values; or for \texttt{bool} parameters, just a default
  value.
\item For optional pointer and list arguments (see
  section~\ref{func-details}), the special default value
  \texttt{null}.
\item For all parameters, a descriptive string.
\item For \texttt{int} parameters with minimum and maximum values
  specified, a set of strings to associate with the allowed numerical
  values (value labels).
\end{itemize}

The first two of these options may be useful in many contexts; the
last two may be helpful if a function is to be packaged for use in
the \app{gretl} GUI (but probably not otherwise). We now expand on
each of the options.

\begin{itemize}

\item The \texttt{const} modifier: must be given as a prefix to the
  basic parameter specification, as in
%    
\begin{code}
const matrix M
\end{code} 
%
This constitutes a promise that the corresponding argument will not be
modified within the function; \app{gretl} will flag an error if
the function attempts to modify the argument.

\item Minimum, maximum and default values for \texttt{scalar} or
  \texttt{int} types: These values should directly follow the name of
  the parameter, enclosed in square brackets and with the individual
  elements separated by colons.  For example, suppose we have an
  integer parameter \texttt{order} for which we wish to specify a
  minimum of 1, a maximum of 12, and a default of 4.  We can write
%    
\begin{code}
int order[1:12:4]
\end{code} 
%
If you wish to omit any of the three specifiers, leave the
corresponding field empty.  For example \texttt{[1::4]} would specify
a minimum of 1 and a default of 4 while leaving the maximum
unlimited.  

For a parameter of type \texttt{bool} (whose values are just zero or
non-zero), you can specify a default of 1 (true) or 0 (false), as in
%    
\begin{code}
bool verbose[0]
\end{code} 

\item Descriptive string: This will show up as an aid to the user if
  the function is packaged (see section~\ref{sec:func-packages} below)
  and called via \app{gretl}'s graphical interface.  The string should
  be enclosed in double quotes and separated from the preceding
  elements of the parameter specification with a space, as in
%
\begin{code}
series y "dependent variable"
\end{code} 

\item Value labels: These may be used only with \texttt{int}
  parameters for which minimum and maximum values have been specified,
  so there is a fixed number of admissible values, and the number of
  labels must match the number of values. They will show up in the
  graphical interface in the form of a drop-down list, making the
  function writer's intent clearer when an integer argument represents
  a categorical selection. A set of value labels must be enclosed in
  braces, and the individual labels must be enclosed in double quotes
  and separated by commas or spaces.  For example:
%
\begin{code}
int case[1:3:1] {"Fixed effects", "Between model", "Random effects"}
\end{code} 

\end{itemize}

If two or more of the trailing optional fields are given in a
parameter specification, they must be given in the order shown above:
min--max--default, description, value labels. Note that there is no
facility for ``escaping'' characters within descriptive strings or
value labels; these may contain spaces but they cannot contain the
double-quote character.  

Here is an example of a well-formed function specification using all
the elements mentioned above:
%
\begin{code}
function matrix myfunc (series y "dependent variable",
                        list X "regressors",
                        int p[0::1] "lag order",
                        int c[1:2:1] "criterion" {"AIC", "BIC"},
                        bool quiet[0])
\end{code} 

One advantage of specifying default values for parameters, where
applicable, is that in script or command-line mode users may omit
trailing arguments that have defaults. For example, \texttt{myfunc}
above could be invoked with just two arguments, corresponding to
\texttt{y} and \texttt{X}; implicitly \texttt{p} = 1, \texttt{c} = 1
and \texttt{quiet} is false.

\subsection{Functions taking no parameters}

You may define a function that has no parameters (these are called
``routines'' in some programming languages).  In this case,  
use the keyword \texttt{void} in place of the listing of parameters:
%    
\begin{code}
function matrix myfunc2 (void)
\end{code}


\subsection{The function body}
   
The \textsl{function body} is composed of \app{gretl} commands, or
calls to user-defined functions (that is, function calls may be
nested).  A function may call itself (that is, functions may be
recursive). While the function body may contain function calls, it may
not contain function definitions.  That is, you cannot define a
function inside another function.  For further details, see
section~\ref{func-details}.


\section{Calling a function}
\label{func-call}

A user function is called by typing its name followed by zero or more
arguments enclosed in parentheses.  If there are two or more arguments
these should be separated by commas.  

There are automatic checks in place to ensure that the number of
arguments given in a function call matches the number of parameters,
and that the types of the given arguments match the types specified in
the definition of the function.  An error is flagged if either of
these conditions is violated.  One qualification: allowance is made
for omitting arguments at the end of the list, provided that default
values are specified in the function definition.  To be precise, the
check is that the number of arguments is at least equal to the number
of \textit{required} parameters, and is no greater than the total
number of parameters.

A scalar, series or matrix argument to a function may be given either
as the name of a pre-existing variable or as an expression which
evaluates to a variable of the appropriate type.  Scalar arguments may
also be given as numerical values.  List arguments must be specified
by name.

The following trivial example illustrates a function call that
correctly matches the function definition.
    
\begin{code}
# function definition
function scalar ols_ess(series y, list xvars)
  ols y 0 xvars --quiet
  scalar myess = $ess
  printf "ESS = %g\n", myess
  return myess
end function
# main script
open data4-1
list xlist = 2 3 4
# function call (the return value is ignored here)
ols_ess(price, xlist)
\end{code}

The function call gives two arguments: the first is a data series
specified by name and the second is a named list of regressors.  Note
that while the function offers the variable \verb+myess+ as a return
value, it is ignored by the caller in this instance.  (As a side note
here, if you want a function to calculate some value having to do with
a regression, but are not interested in the full results of the
regression, you may wish to use the \option{quiet} flag with the
estimation command as shown above.)
    
A second example shows how to write a function call that assigns
a return value to a variable in the caller:
    
\begin{code}
# function definition
function series get_uhat(series y, list xvars)
  ols y 0 xvars --quiet
  series uh = $uhat
  return uh
end function
# main script
open data4-1
list xlist = 2 3 4
# function call
series resid = get_uhat(price, xlist)
\end{code}

\section{Deleting a function}
\label{func-del}

If you have defined a function and subsequently wish to clear it out
of memory, you can do so using the keywords \texttt{delete} or
\texttt{clear}, as in

\begin{code}
function myfunc delete
function get_uhat clear
\end{code}

Note, however, that if \texttt{myfunc} is already a defined function,
providing a new definition automatically overwrites the previous one,
so it should rarely be necessary to delete functions explicitly.

\section{Function programming details}
\label{func-details}

\subsection{Variables versus pointers}
\label{funscope}

Series, scalar, and matrix arguments to functions can be passed in two
ways: ``as they are'', or as pointers. For example, consider the
following:
\begin{code}
function series triple1(series x)
  return 3*x
end function
  
function series triple2(series *x)
  return 3*x
end function
\end{code}

These two functions are nearly identical (and yield the same result);
the only difference is that you need to feed a series into
\texttt{triple1}, as in \texttt{triple1(myseries)}, while
\texttt{triple2} must be supplied a \emph{pointer} to a series, as in
\texttt{triple2(\&myseries)}. 

Why make the distinction? There are two main reasons for doing so:
modularity and performance.

By modularity we mean the insulation of a function from the rest of
the script which calls it.  One of the many benefits of this approach
is that your functions are easily reusable in other contexts.  To
achieve modularity, \emph{variables created within a function are
  local to that function, and are destroyed when the function exits},
unless they are made available as return values and these values are
``picked up'' or assigned by the caller.
    
In addition, functions do not have access to variables in ``outer
scope'' (that is, variables that exist in the script from which the
function is called) except insofar as these are explicitly passed to
the function as arguments.

By default, when a variable is passed to a function as an argument,
what the function actually ``gets'' is a \emph{copy} of the outer
variable, which means that the value of the outer variable is not
modified by anything that goes on inside the function.  But the use of
pointers allows a function and its caller to cooperate such that
an outer variable can be modified by the function.  In effect, this
allows a function to ``return'' more than one value (although only one
variable can be returned directly --- see below).  The parameter in
question is marked with a prefix of \texttt{*} in the function
definition, and the corresponding argument is marked with the
complementary prefix \verb+&+ in the caller.  For example,
%
\begin{code}
function series get_uhat_and_ess(series y, list xvars, scalar *ess)
  ols y 0 xvars --quiet
  ess = $ess
  series uh = $uhat
  return uh
end function
# main script
open data4-1
list xlist = 2 3 4
# function call
scalar SSR
series resid = get_uhat_and_ess(price, xlist, &SSR)
\end{code}
%
In the above, we may say that the function is given the \emph{address}
of the scalar variable \texttt{SSR}, and it assigns a value to that
variable (under the local name \texttt{ess}).  (For anyone used to
programming in C: note that it is not necessary, or even possible, to
``dereference'' the variable in question within the function using the
\texttt{*} operator.  Unadorned use of the name of the variable is
sufficient to access the variable in outer scope.)

An ``address'' parameter of this sort can be used as a means of
offering optional information to the caller.  (That is, the
corresponding argument is not strictly needed, but will be used if
present).  In that case the parameter should be given a default value
of \texttt{null} and the the function should test to see if the caller
supplied a corresponding argument or not, using the built-in function
\texttt{isnull()}.  For example, here is the simple function shown
above, modified to make the filling out of the \texttt{ess} value
optional.
%
\begin{code}
function series get_uhat_and_ess(series y, list xvars, scalar *ess[null])
  ols y 0 xvars --quiet
  if !isnull(ess) 
     ess = $ess
  endif
  return $uhat
end function
\end{code}
%
If the caller does not care to get the \texttt{ess} value, it can
use \texttt{null} in place of a real argument:
%
\begin{code}
series resid = get_uhat_and_ess(price, xlist, null)
\end{code}
%
Alternatively, trailing function arguments that have default values 
may be omitted, so the following would also be a valid call:
%
\begin{code}
series resid = get_uhat_and_ess(price, xlist)
\end{code}

Pointer arguments may also be useful for optimizing performance: even if
a variable is not modified inside the function, it may be a good idea
to pass it as a pointer if it occupies a lot of memory. Otherwise, the
time \app{gretl} spends transcribing the value of the variable to the
local copy may be non-negligible, compared to the time the function
spends doing the job it was written for.

Example \ref{ex:perf-pointers} takes this to the extreme.  We define
two functions which return the number of rows of a matrix (a pretty
fast operation).  One function gets a matrix as argument, the other
one a pointer to a matrix.  The two functions are evaluated on a
matrix with 2000 rows and 2000 columns; on a typical system,
floating-point numbers take 8 bytes of memory, so the space occupied
by the matrix is roughly 32 megabytes.

Running the code in example \ref{ex:perf-pointers} will produce output
similar to the following (the actual numbers depend on the
machine you're running the example on):
\begin{code}
Elapsed time: 
	without pointers (copy) = 3.66 seconds,
	with pointers (no copy) = 0.01 seconds.
\end{code}

\begin{script}[htbp]
  \caption{Performance comparison: values versus pointer}
  \label{ex:perf-pointers}
  \begin{scode}
function scalar a(matrix X)
  return rows(X)
end function

function scalar b(const matrix *X)
  return rows(X)
end function

nulldata 10
set echo off
set messages off
X = zeros(2000,2000)
r = 0

set stopwatch
loop 100
  r = a(X)
endloop
fa = $stopwatch

set stopwatch
loop 100
  r = b(&X)
endloop
fb = $stopwatch

printf "Elapsed time:\n\
\twithout pointers (copy) = %g seconds,\n\
\twith pointers (no copy) = %g seconds.\n", fa, fb 
\end{scode}
%$
\end{script}

If a pointer argument is used for this sort of purpose --- and the
object to which the pointer points is not modified by the function ---
it is a good idea to signal this to the user by adding the
\texttt{const} qualifier, as shown for function \texttt{b} in
Example~\ref{ex:perf-pointers}.  When a pointer argument is qualified
in this way, any attempt to modify the object within the function will
generate an error.

One limitation on the use of pointer-type arguments should be noted:
you cannot supply a given variable as a pointer argument more than
once in any given function call. For example, suppose we have a
function that takes two matrix-pointer arguments,
\begin{code}
function scalar pointfunc (matrix *a, matrix *b)
\end{code}
And suppose we have two matrices, \texttt{x} and \texttt{y}, at the
caller level.  The call
\begin{code}
pointfunc(&x, &y)
\end{code}
is OK, but the call
\begin{code}
pointfunc(&x, &x) # will not work
\end{code}
will generate an error.

\subsection{List arguments}

The use of a named list as an argument to a function gives a means of
supplying a function with a set of variables whose number is unknown
when the function is written --- for example, sets of regressors or
instruments.  Within the function, the list can be passed on to
commands such as \texttt{ols}.  

A list argument can also be ``unpacked'' using a \texttt{foreach} loop
construct, but this requires some care.  For example, suppose you have
a list \texttt{X} and want to calculate the standard deviation of each
variable in the list.  You can do:
%
\begin{code}
loop foreach i X
   scalar sd_$i = sd(X.$i)
endloop
\end{code}

\textit{Please note}: a special piece of syntax is needed in this
context.  If we wanted to perform the above task on a list in a
regular script (not inside a function), we could do
%
\begin{code}
loop foreach i X
   scalar sd_$i = sd($i)
endloop
\end{code}
%
where \dollar{i} gets the name of the variable at position $i$ in the
list, and \verb|sd($i)| gets its standard deviation.  But inside a
function, working on a list supplied as an argument, if we want to
reference an individual variable in the list we must use the syntax
\textsl{listname.varname}.  Hence in the example above we write
\verb|sd(X.$i)|.

This is necessary to avoid possible collisions between the name-space
of the function and the name-space of the caller script.  For example,
suppose we have a function that takes a list argument, and that
defines a local variable called \texttt{y}.  Now suppose that this
function is passed a list containing a variable named \texttt{y}.  If
the two name-spaces were not separated either we'd get an error, or
the external variable \texttt{y} would be silently over-written by the
local one.  It is important, therefore, that list-argument variables
should not be ``visible'' by name within functions.  To ``get hold
of'' such variables you need to use the form of identification just
mentioned: the name of the list, followed by a dot, followed by the
name of the variable.

\paragraph{Constancy of list arguments} When a named list of
variables is passed to a function, the function is actually provided
with a copy of the list.  The function may modify this copy (for
instance, adding or removing members), but the original list at the
level of the caller is not modified.

\paragraph{Optional list arguments} If a list argument to a function is
optional, this should be indicated by appending a default value of
\texttt{null}, as in
%
\begin{code}
function scalar myfunc (scalar y, list X[null])
\end{code}
%
In that case, if the caller gives \texttt{null} as the list argument
(or simply omits the last argument) the named list \texttt{X} inside the
function will be empty.  This possibility can be detected using the
\texttt{nelem()} function, which returns 0 for an empty list.

\subsection{String arguments}

String arguments can be used, for example, to provide flexibility in
the naming of variables created within a function.  In the following
example the function \texttt{mavg} returns a list containing two
moving averages constructed from an input series, with the names of
the newly created variables governed by the string argument.
%
\begin{code}
function list mavg (series y, string vname)
   series @vname_2 = (y+y(-1)) / 2
   series @vname_4 = (y+y(-1)+y(-2)+y(-3)) / 4
   list retlist = @vname_2 @vname_4
   return retlist
end function

open data9-9
list malist = mavg(nocars, "nocars")
print malist --byobs
\end{code}
%
The last line of the script will print two variables named
\verb|nocars_2| and \verb|nocars_4|.  For details on the handling of
named strings, see chapter~\ref{chap-persist}.

If a string argument is considered optional, it may be given a
\texttt{null} default value, as in
%
\begin{code}
function scalar foo (series y, string vname[null])
\end{code}

\subsection{Retrieving the names of arguments}

The variables given as arguments to a function are known inside the
function by the names of the corresponding parameters.  For example,
within the function whose signature is
%
\begin{code}
function void somefun (series y)
\end{code}
%
we have the series known as \texttt{y}.  It may be useful, however, to
be able to determine the names of the variables provided as arguments.
This can be done using the function \texttt{argname}, which takes the
name of a function parameter as its single argument and returns a
string.  Here is a simple illustration:
%
\begin{code}
function void namefun (series y)
  printf "the series given as 'y' was named %s\n", argname(y)
end function

open data9-7
namefun(QNC)
\end{code}
%
This produces the output
%
\begin{code}
the series given as 'y' was named QNC
\end{code}

Please note that this will not always work: the arguments given
to functions may be anonymous variables, created on the fly, as in
\texttt{somefun(log(QNC))} or \texttt{somefun(CPI/100)}.  In that case
the \textsf{argname} function fails to return a string.  Function
writers who wish to make use of this facility should check the return
from \texttt{argname} using the \texttt{isstring()} function, which
returns 1 when given the name of a string variable, 0 otherwise.

\subsection{Return values}

Functions can return nothing (just printing a result, perhaps), or
they can return a single variable --- a scalar, series, list, matrix,
string, or bundle (see section~\ref{sec:Bundles}).  The return value,
if any, is specified via a statement within the function body
beginning with the keyword \texttt{return}, followed by either the
name of a variable (which must be of the type announced on the first
line of the function definition) or an expression which produces a
value of the correct type.

Having a function return a list or bundle is a way of permitting the
``return'' of more than one variable.  For example, you can define
several series inside a function and package them as a list; in this
case they are not destroyed when the function exits.  Here is a simple
example, which also illustrates the possibility of setting the
descriptive labels for variables generated in a function.
%    
\begin{code}
function list make_cubes (list xlist)
   list cubes = null
   loop foreach i xlist --quiet
      series $i3 = (xlist.$i)^3
      setinfo $i3 -d "cube of $i"
      list cubes += $i3
   endloop
   return cubes
end function

open data4-1
list xlist = price sqft
list cubelist = make_cubes(xlist)
print xlist cubelist --byobs
labels
\end{code}
%$

A \texttt{return} statement causes the function to return (exit) at
the point where it appears within the body of the function.  A
function may also exit when (a) the end of the function code is
reached (in the case of a function with no return value), (b) a
\app{gretl} error occurs, or (c) a \verb+funcerr+ statement is
reached.

The \verb+funcerr+ keyword, which may be followed by a string enclosed
in double quotes, causes a function to exit with an error flagged.  If
a string is provided, this is printed on exit, otherwise a generic
error message is printed.  This mechanism enables the author of a
function to pre-empt an ordinary execution error and/or offer a more
specific and helpful error message.  For example,
%
\begin{code}
if nelem(xlist) = 0
   funcerr "xlist must not be empty"
endif
\end{code}

A function may contain more than one \texttt{return} statement, as in
%
\begin{code}
function scalar multi (bool s)
   if s
      return 1000
   else
      return 10
   endif
end function
\end{code}
%
However, it is recommended programming practice to have a single
return point from a function unless this is very inconvenient.  The
simple example above would be better written as
%
\begin{code}
function scalar multi (bool s)
   return s ? 1000 : 10
end function
\end{code}
    

\subsection{Error checking}

When gretl first reads and ``compiles'' a function definition there is
minimal error-checking: the only checks are that the function name is
acceptable, and, so far as the body is concerned, that you are not
trying to define a function inside a function (see Section
\ref{func-define}). Otherwise, if the function body contains invalid
commands this will become apparent only when the function is called
and its commands are executed.

\subsection{Debugging}

The usual mechanism whereby \app{gretl} echoes commands and reports on
the creation of new variables is by default suppressed when a function
is being executed.  If you want more verbose output from a particular
function you can use either or both of the following commands within
the function:
%
\begin{code}
set echo on
set messages on
\end{code}

Alternatively, you can achieve this effect for all functions via
the command \texttt{set debug 1}.  Usually when you set the value of a
state variable using the \texttt{set} command, the effect applies only
to the current level of function execution.  For instance, if you do
\texttt{set messages on} within function \texttt{f1}, which in turn
calls function \texttt{f2}, then messages will be printed for
\texttt{f1} but not \texttt{f2}.  The debug variable, however, acts
globally; all functions become verbose regardless of their level.

Further, you can do \texttt{set debug 2}: in addition to command echo
and the printing of messages, this is equivalent to setting
\verb|max_verbose| (which produces verbose output from the BFGS
maximizer) at all levels of function execution.

\section{Function packages}
\label{sec:func-packages}

Since \app{gretl} 1.6.0 there has been a mechanism to package
functions and make them available to other users of \app{gretl}.  Here
is a walk-through of the process.

\subsection{Load a function in memory}

There are several ways to load a function:

\begin{itemize}
\item If you have a script file containing function definitions, open
  that file and run it.
\item Create a script file from scratch.  Include at least one
  function definition, and run the script.
\item Open the GUI console and type a function definition
  interactively.  This method is not particularly recommended; you are
  probably better composing a function non-interactively.
\end{itemize}

For example, suppose you decide to package a function that returns the
percentage change of a time series. Open a script file and type
\begin{code}
function series pc(series y "Series to process")
  return 100 * diff(y)/y(-1)
end function
\end{code}
In this case, we have appended a string to the function argument, as
explained in section \ref{func-define}, so as to make our interface
more informative.  This is not obligatory: if you omit the descriptive
string, \app{gretl} will supply a predefined one.

\begin{figure}[htbp]
  \centering
  \includegraphics[scale=0.5]{figures/func_check}
  \caption{Output of function check}
  \label{fig:func_check}
\end{figure}

Now run your function. You may want to make sure it works properly by
running a few tests. For example, you may open the console and type

\begin{code}
genr x = uniform()
genr dpcx = pc(x)
print x dpcx --byobs
\end{code}

You should see something similar to figure \ref{fig:func_check}. The
function seems to work ok.  Once your function is debugged, you
may proceed to the next stage.

\subsection{Create a package}

We first present the mechanism for creating a function package via
\app{gretl}'s graphical interface. This can also be done via the
command line, which offers some additional functionality for package
authors; an explanation is given later in this section.

Start the GUI program and take a look at the ``File, Function files'' menu.
This menu contains four items: ``On local machine'', ``On server'', ``Edit
package'', ``New package''.

Select ``New package''.  (This will produce an error message unless at
least one user-defined function is currently loaded in memory --- see
the previous point.)  In the first dialog you get to select:

\begin{itemize}
\item A public function to package.
\item Zero or more ``private'' helper functions.
\end{itemize}

Public functions are directly available to users; private functions are
part of the ``behind the scenes'' mechanism in a function package.

\begin{figure}[htbp]
  \centering
  \includegraphics[scale=0.5]{figures/package_editor}
  \caption{The package editor window}
  \label{fig:package_editor}
\end{figure}

On clicking ``OK'' a second dialog should appear (see
Figure~\ref{fig:package_editor}), where you get to enter the package
information (author, version, date, and a short description).  You can
also enter help text for the public interface.  You have a further
chance to edit the code of the function(s) to be packaged, by clicking
on ``Edit function code''.  (If the package contains more than one
function, a drop-down selector will be shown.)  And you get to add a
sample script that exercises your package.  This will be helpful for
potential users, and also for testing.  A sample script is required if
you want to upload the package to the gretl server (for which a
check-box is supplied).

You won't need it right now, but the button labeled ``Save as script''
allows you to ``reverse engineer'' a function package, writing out a
script that contains all the relevant function definitions.

Clicking ``Save'' in this dialog leads you to a File Save dialog.  All
being well, this should be pointing towards a directory named
\texttt{functions}, either under the \app{gretl} system directory (if
you have write permission on that) or the \app{gretl} user directory.
This is the recommended place to save function package files, since
that is where the program will look in the special routine for opening
such files (see below).

Needless to say, the menu command ``File, Function files, Edit package''
allows you to make changes to a local function package.

\vspace{6pt}

A word on the file you just saved.  By default, it will have a
\texttt{.gfn} extension.  This is a ``function package'' file: unlike
an ordinary \app{gretl} script file, it is an XML file containing both
the function code and the extra information entered in the packager.
Hackers might wish to write such a file from scratch rather than using
the GUI packager, but most people are likely to find it awkward.  Note
that XML-special characters in the function code have to be escaped,
e.g.\ \texttt{\&} must be represented as \texttt{\&amp;}.  Also, some
elements of the function syntax differ from the standard script
representation: the parameters and return values (if any) are
represented in XML.  Basically, the function is pre-parsed, and ready
for fast loading using \textsf{libxml}.

\vspace{6pt}

\subsection{Load a package}

Why package functions in this way?  To see what's on offer so far, try
the next phase of the walk-through.

Close gretl, then re-open it.  Now go to ``File, Function files, On
local machine''. If the previous stage above has gone OK, you should
see the file you packaged and saved, with its short description.  If
you click on ``Info'' you get a window with all the information gretl
has gleaned from the function package.  If you click on the ``View
code'' icon in the toolbar of this new window, you get a script view
window showing the actual function code. Now, back to the ``Function
packages'' window, if you click on the package's name, the relevant
functions are loaded into \app{gretl}'s workspace, ready to be called
by clicking on the ``Call'' button.

After loading the function(s) from the package, open the GUI console.
Try typing \texttt{help foo}, replacing \texttt{foo} with the name of
the public interface from the loaded function package: if any help text
was provided for the function, it should be presented.

In a similar way, you can browse and load the function packages
available on the \app{gretl} server, by selecting ``File, Function
files, On server''.

Once your package is installed on your local machine, you can use the
function it contains via the graphical interface as described above,
or by using the CLI, namely in a script or through the console. In the
latter case, you load the function via the \texttt{include} command,
specifying the package file as the argument, complete with the
\texttt{.gfn} extension.

\begin{figure}[htbp]
  \centering
  \includegraphics[scale=0.5]{figures/function_call}
  \caption{Using your package}
  \label{fig:function_call}
\end{figure}

To continue with our example, load the file \texttt{np.gdt} (supplied
with \app{gretl} among the sample datasets). Suppose you want to
compute the rate of change for the variable \texttt{iprod} via your
new function and store the result in a series named \texttt{foo}.

Go to ``File, Function files, On local machine''.  You will be shown a
list of the installed packages, including the one you have just
created. If you select it and click on ``Execute'' (or double-click on
the name of the function package), a window similar to the one shown
in figure \ref{fig:function_call} will appear.  Notice that the
description string ``Series to process'', supplied with the function
definition, appears to the left of the top series chooser.

Click ``Ok'' and the series \texttt{foo} will be generated (see figure
\ref{fig:iprod_pc}).  You may have to go to ``Data, Refresh data'' in
order to have your new variable show up in the main window variable
list (or just press the ``r'' key).

\begin{figure}[htbp]
  \centering
  \includegraphics[scale=0.5]{figures/iprod_pc}
  \caption{Percent change in industrial production}
  \label{fig:iprod_pc}
\end{figure}

Alternatively, the same could have been accomplished by the script
\begin{code}
include pc.gfn
open np
foo = pc(iprod)
\end{code}

\subsection{Creating a package via the command line}

The mechanism described above, for creating function packages using
the GUI, is likely to be convenient for small to medium-sized packages
but may be too cumbersome for ambitious packages that include a large
hierarchy of private functions. To facilitate the building of such
packages \app{gretl} offers the \texttt{makepkg} command.

To use \texttt{makepkg} you create three files: a driver script that
loads all the functions you want to package and invokes
\texttt{makepkg}; a small, plain-text specification file that contains
the required package details (author, version, etc.); and (in the
simplest case) a plain text help file.  You run the driver script and
\app{gretl} writes the package (\texttt{.gfn}) file.

We first illustrate with a simple notional package. We have a gretl
script file named \texttt{foo.inp} that contains a function,
\texttt{foo}, that we want to package. Our driver script would then
look like this
%
\begin{code}
include foo.inp
makepkg foo.gfn
\end{code}
%
Note that the \texttt{makepkg} command takes one argument, the name of
the package file to be created. The package \emph{specification file}
should have the same basename but the extension \texttt{.spec}. In
this case \app{gretl} will therefore look for \texttt{foo.spec}. It
should look something like this:
%
\begin{code}
# foo.spec
author = A. U. Thor
version = 1.0
date = 2011-02-01
description = Does something with time series
public = foo 
help = foohelp.txt
sample-script = example.inp
min-version = 1.9.3
data-requirement = needs-time-series-data
\end{code}

As you can see, the format of each line in this file is \texttt{key =
  value}, with two qualifications: blank lines are permitted (and
ignored, as are comment lines that start with \verb|#|). 

All the fields included in the above example are required, with the
exception of \texttt{data-requirement}, though the order in which they
appear is immaterial. Here's a run-down of the basic fields:

\begin{itemize}
\item \texttt{author}: the name(s) of the author(s). Accented or other
  non-ASCII characters should be given as UTF-8.
\item \texttt{version}: the version number of the package, which should
  be limited to two integers separated by a period.
\item \texttt{date}: the release date of the current verson of the
  package, in ISO 8601 format: YYYY-MM-DD.
\item \texttt{description}: a brief description of the functionality
  offered by the package. This will be displayed in the GUI function
  packages window so it should be just one short line.
\item \texttt{public}: the listing of public functions.
\item \texttt{help}: the name of a plain text (UTF-8) file containing
  help; all packages must provide help.
\item \texttt{sample-script}: the name of a sample script that
 illustrates use of the package; all packages must supply a
 sample script.
\item \texttt{min-version}: the minimum version of gretl required
 for the package to work correctly. If you're unsure about this,
 the conservative thing is to give the current \app{gretl} version.
\end{itemize}

The \texttt{public} field indicates which function or functions are to
be made directly available to users (as opposed to private ``helper''
functions).  In the example above there is just one public
function. Note that any functions in memory when \texttt{makepkg} is
invoked, other than those designated as public, are assumed to be
private functions that should also be included in the package. That
is, the list of private functions (if any) is implicit.

The \texttt{data-requirement} field should be specified if the package
requires time-series or panel data, or alternatively if no dataset is
required.  If the \texttt{data-requirement} field is omitted, the
assumption is that the package needs a dataset in place, but it
doesn't matter what kind; if the packaged functions do not use any
series or lists this requirement can be explicitly relaxed.  Valid
values for this field are:

\begin{center}
\begin{tabular}{ll}
\texttt{needs-time-series-data} & (any time-series data OK) \\ 
\texttt{needs-qm-data} & (must be quarterly or monthly) \\ 
\texttt{needs-panel-data} & (must be a panel) \\
\texttt{no-data-ok} & (no dataset is needed) \\
\end{tabular}
\end{center}

For a more complex example, let's look at the \app{gig}
(GARCH-in-gretl) package.  The driver script for building \app{gig}
looks something like this:
%
\begin{code}
set echo off
set messages off
include gig_mle.inp
include gig_setup.inp
include gig_estimate.inp
include gig_printout.inp
include gig_plot.inp
makepkg gig.gfn
\end{code}

In this case the functions to be packaged (of which there are many)
are distributed across several script files, each of which is the
target of an \texttt{include} command.  The \texttt{set} commands at
the top are included to cut down on the verbosity of the output.

The content of \texttt{gig.spec} is as follows:
%
\begin{code}
author = Riccardo "Jack" Lucchetti and Stefano Balietti
version = 2.0
date = 2010-12-21
description = An assortment of univariate GARCH models
public = GUI_gig \
    gig_setup gig_set_dist gig_set_pq gig_set_vQR \
    gig_print gig_estimate \
    gig_plot gig_dplot \
    gig_bundle_print GUI_gig_plot

gui-main = GUI_gig
bundle-print = gig_bundle_print
bundle-plot = GUI_gig_plot
help = gig.pdf
sample-script = examples/example1.inp
min-version = 1.9.3
data-requirement = needs-time-series-data
\end{code}

Note that backslash continuation can be used for the elements of the
\texttt{public} function listing.

In addition to the fields shown in the simple example above,
\texttt{gig.spec} includes three optional fields: \texttt{gui-main},
\texttt{bundle-print} and \texttt{bundle-plot}. These keywords are
used to designate certain functions as playing a special role in the
\app{gretl} graphical interface. A function picked out in this way
must be in the \texttt{public} list and must satisfy certain
further requirements.  
%
\begin{itemize}
\item \texttt{gui-main}: this specifies a function as the one which
  will be presented automatically to GUI users (instead of users'
  being faced with a choice of interfaces). This makes sense only for
  packages that have multiple public functions. In addition, the
  \texttt{gui-main} function must return a \texttt{bundle} (see
  section~\ref{sec:Bundles}).
\item \texttt{bundle-print}: this picks out a function that should be
  used to print the contents of a bundle returned by the
  \texttt{gui-main} function. It must take a pointer-to-bundle as its
  first argument. The second argument, if present, should be an
  \texttt{int} switch, with two or more valid values, that controls the
  printing in some way. Any further arguments must have default values
  specified so that they can be omitted.
\item \texttt{bundle-plot}: selects a function for the role of
  producing a plot or graph based on the contents of a returned
  bundle. The requirements on this function are as for 
  \texttt{bundle-print}.
\end{itemize}

The ``GUI special'' tags support a user-friendly mode of operation.
On a successful call to \texttt{gui-main}, \app{gretl} opens a 
window displaying the contents of the returned bundle (formatted
via \texttt{bundle-print}). Menus in this window give the user the
option of saving the entire bundle (in which case it's represented
as an icon in the ``icon view'' window) or of extracting specific
elements from the bundle (series or matrices, for example). 

If the package has a \texttt{bundle-plot} function, the bundle window
also has a \textsf{Graph} menu. In \app{gig}, for example, the
\texttt{bundle-plot} function has this signature:
%
\begin{code}
function void GUI_gig_plot(bundle *model, int ptype[0:1:0] \
                           "Plot type" {"Time series", "Density"})
\end{code}

The \texttt{ptype} switch is used to choose between a time-series plot
of the residual and its conditional variance, and a kernel density
plot of the innovation against the theoretical distribution it is
supposed to follow. The use of the value-labels \texttt{Time series}
and \texttt{Density} means that the \textsf{Graph} menu will display
these two choices.

One other feature of the \app{gig} spec file is noteworthy: the
\texttt{help} field specifies \texttt{gig.pdf}, documentation in PDF
format. Unlike plain-text help, this cannot be rolled into the
\texttt{gfn} (XML) file produced by the \texttt{makepkg} command;
rather, both \texttt{gig.gfn} and \texttt{gig.pdf} are packaged into a
zip archive for distribution. This represents a form of package
which is new in \app{gretl} 1.9.4. More details will be made
available before long. 

\section{Memo: updating old-style functions}
\label{sec:old-func}

As mentioned at the start of this chapter, different rules were in
force for defining functions prior to \app{gretl} 1.8.4. While the old
syntax is still supported to date, this may not always be the
case. But it is straightforward to convert a function to the new
style. The only thing that must be changed for compatibility with the
new syntax is the declaration of the function's return
type. Previously this was placed inline in the \texttt{return}
statement, whereas now it is placed right after the \texttt{function}
keyword. For example:
%
\begin{code}
# old style
function triple (series x)
  y = 3*x
  return series y # note the "series" here
end function

# new style
function series triple (series x)
  y = 3*x
  return y
end function
\end{code}

Note also that the role of the \texttt{return} statement has changed
(and its use has become more flexible):

\begin{itemize}
\item The \texttt{return} statement now causes the function to return
  directly, and you can have more than one such statement, wrapped in
  conditionals. Before there could only be one \texttt{return}
  statement, and its role was just to specify the type available for
  assignment by the caller.
\item The final element in the \texttt{return} statement can now be an
  expression that evaluates to a value of the advertised return type;
  before, it had to be the name of a pre-defined variable.
\end{itemize}


%%% Local Variables: 
%%% mode: latex
%%% TeX-master: "gretl-guide"
%%% End: 


\chapter{Persistent objects}
\label{persist}

%%% Work in progress, not really ready for translation yet.

\section{Named lists}
\label{named-lists}

Many \app{gretl} commands take one or more lists of variables as
arguments.  To make this easier to handle in the context of command
scripts, and in particular within user-defined functions, \app{gretl}
offers the possibility of \textit{named lists}.  

A named list is created using the keyword \texttt{list}, followed by
the name of the list, an equals sign, and either \texttt{null} (to
create an empty list) or one or more variables to be placed on the
list.  For example,

\begin{code}
list xlist = 1 2 3 4
list reglist = income price 
list empty_list = null
\end{code}

The name of the list must start with a letter, and must be composed
entirely of letters, numbers or the underscore character.  The maximum
length of the name is 31 characters; list names cannot contain
spaces.  When adding variables to a list, you can refer to them either
by name or by their ID numbers. 

Once a list has been created, it will be ``remembered' for the
duration of the \app{gretl} session.  Lists can be modified in two
ways.  To \textit{append variables} to an existing list, put the name
of the list to the right of the equals sign, before the variables to
be added, as in

\begin{code}
list xlist = xlist 5 6 7
\end{code}

In this example, if \texttt{xlist} originally contained variables 1,
2, 3 and 4, it will now contain variables 1 to 7.

To \textit{redefine} an existing list altogether, use the same syntax
as for creating a list.  For example

\begin{code}
list xlist = 1 2 3
list xlist = 4 5 6
\end{code}

After the second assignment, \texttt{xlist} contains just variables 4,
4 and 6.  

A named list can be used in any context where \app{gretl} expects a
list of variables.  FIXME need examples.

You can determine whether an unknown variable actually represents a list
using the function \texttt{islist()}.

\begin{code}
series xl1 = log(x1)
series xl2 = log(x2)
list xvars = xl1 xl2
genr isl_a = islist(xvars)
genr isl_b = islist(xl1)
\end{code}

The first \texttt{genr} command above will assign a value of 1 to
\texttt{isl\_a} since \texttt{xl} is in fact a named list.  The second
genr will assign 0 to \texttt{isl\_b} since \texttt{xl1} is a data
series, not a list.  

You can also determine the number of variables or elements in a list
using the function \texttt{nelem()}.

\begin{code}
list xlist = 1 2 3
genr nl = nelem(xlist)
\end{code}

The scalar \texttt{nl} will be assigned a value of 3 since
\texttt{xlist} contains 3 members.






    
%%% Local Variables: 
%%% mode: latex
%%% TeX-master: "gretl-guide"
%%% End: 


\chapter{Matrix manipulation}
\label{chap:matrices}

Together with the other two basic types of data (series and scalars),
gretl offers a quite comprehensive array of matrix methods. This
chapter illustrates the peculiarities of matrix syntax and discusses
briefly some of the more advanced matrix functions. For a full listing
of matrix functions and a comprehensive account of their syntax,
please refer to the \GCR.

In this chapter we're concerned with real matrices; most of the points
made here also apply to complex matrices but see the following chapter
for additional specifics on the complex case.

\section{Creating matrices}
\label{sec:matrix-create}

Matrices can be created using any of these methods:

\begin{enumerate}
\item By direct specification of the scalar values that compose the
  matrix---either in numerical form, or by reference to pre-existing
  scalar variables, or using computed values.
\item By providing a list of data series.
\item By providing a \textit{named list} of series.
\item Via a suitable expression that references existing matrices
  and/or scalars, or via some special functions.
\end{enumerate}

To specify a matrix \textit{directly in terms of scalars}, the syntax
is, for example:

\begin{code}
matrix A = {1, 2, 3 ; 4, 5, 6}
\end{code}

The matrix is defined by rows; the elements on each row are separated
by commas and the rows are separated by semi-colons.  The whole
expression must be wrapped in braces.  Spaces within the braces are
not significant.  The above expression defines a $2\times3$ matrix.
Each element should be a numerical value, the name of a scalar
variable, or an expression that evaluates to a scalar.  Directly after
the closing brace you can append a single quote (\texttt{'}) to obtain
the transpose.

To specify a matrix \textit{in terms of data series} the syntax is,
for example,
%
\begin{code}
matrix A = {x1, x2, x3}
\end{code}
%
where the names of the series are separated by commas.  Besides names
of existing series, you can use expressions that yield a series
result.  For example, given a series \texttt{x} you could do
%
\begin{code}
matrix A = {x, x^2}
\end{code}
%
Each series occupies a column (and there can only be one series per
column).  The semicolon cannot be used as a row separator in this
case: if you want the series arranged in rows, append the transpose
symbol.  The range of data values included in the matrix depends on
the current setting of the sample range.

Instead of giving an explicit list of series, you may instead provide
the \textit{name of a saved list} (see
Chapter~\ref{chap:lists-strings}), as in
%
\begin{code}
list xlist = x1 x2 x3
matrix A = {xlist}
\end{code}
%
When you provide a named list, the data series are by default placed
in columns, as is natural in an econometric context: if you want them
in rows, append the transpose symbol.

As a special case of constructing a matrix from a list of series,
you can say
%
\begin{code}
matrix A = {dataset}
\end{code}
%
This builds a matrix using all the series in the current dataset,
apart from the constant (series 0).  When this dummy list is used, it
must be the sole element in the matrix definition \texttt{\{...\}}.  You
can, however, create a matrix that includes the constant along with
all other series using horizontal concatenation (see below), as in
%
\begin{code}
matrix A = {const}~{dataset}
\end{code}
%

By default, when you build a matrix from series that include missing
values any observations (rows) that contain \texttt{NA}s are skipped.
This behavior can be modified via the command \texttt{set
  skip\_missing off}.  In that case missing values are represented in
the matrix as NaN (``Not a Number'').  Note that as per the IEEE
standard, arithmetic operations involving one or more NaNs always
produce a NaN result. Alternatively, you can take greater control over
the observations (data rows) that are included in the matrix using the
``set'' variable \texttt{matrix\_mask}, as in
%
\begin{code}
set matrix_mask msk
\end{code}
%
where \texttt{msk} is the name of a series.  Subsequent commands that
form matrices from series or lists will include only observations
for which \texttt{msk} has non-zero (and non-missing) values. You
can remove this mask via the command \texttt{set matrix\_mask null}.

\tip{Names of matrices must satisfy the same requirements as names of
  gretl variables in general: the name can be no longer than 31
  characters, must start with a letter, and must be composed of
  nothing but letters, numbers and the underscore character.}

\section{Empty matrices}
\label{sec:emptymatrix}

The syntax 
%
\begin{code}
matrix A = {}
\end{code}
%
creates an empty matrix---a matrix with zero rows and zero columns.

The main purpose of this construct is to allow the user to define a
starting point for subsequent concatenation operations.  For instance,
if \texttt{X} is an already defined matrix of any size, the commands
%
\begin{code}
  matrix A = {}
  matrix B = A ~ X
\end{code}
%
produce a matrix \texttt{B} that is identical to \texttt{X}.

\begin{table}[htbp]
\centering
\begin{tabular}{lc@{\hspace{5em}}lc}
\textit{Function} & \textit{Return value} & \textit{Function} & \textit{Return value} \\ [4pt]
  \texttt{A', transp(A)} & \texttt{A} & \texttt{rows(A)} & 0 \\
  \texttt{cols(A)} & 0 &
  \texttt{rank(A)} & 0 \\
  \texttt{det(A)} & \texttt{NA} &
  \texttt{ldet(A)} & \texttt{NA} \\
  \texttt{tr(A)} & \texttt{NA} &
  \texttt{onenorm(A)} & \texttt{NA} \\
  \texttt{infnorm(A)} & \texttt{NA} &
  \texttt{rcond(A)} & \texttt{NA} \\
  \texttt{diag(A)} & \verb|{}| &
  \texttt{vec(A)} & \verb|{}| \\
  \texttt{vech(A)} & \verb|{}| &
  \texttt{unvech(A)} & \verb|{}| \\
\end{tabular}
\caption{Valid functions on an empty matrix, \texttt{A}}
\label{tab:empty-matrix-funcs}
\end{table}

From an algebraic point of view, one can make sense of the idea of an
empty matrix in terms of vector spaces: if a matrix is an ordered set
of vectors, then \verb|A={}| is the empty set.  As a consequence,
operations involving addition and multiplications don't have any clear
meaning (arguably, they have none at all), but operations involving
the cardinality of this set (that is, the dimension of the space
spanned by \texttt{A}) are meaningful.

Legal function calls with an empty matrix argument are listed in Table
\ref{tab:empty-matrix-funcs}---other functions generate an error if an
empty matrix is given as an argument.  In line with the above
interpretation, the functions \texttt{I}, \texttt{ones},
\texttt{zeros}, \texttt{mnormal} and \texttt{muniform} return an empty
matrix when or more of the arguments is 0, as well as the function
\texttt{nullspace} when its argument is of full column rank.

\section{Selecting submatrices}
\label{sec:matrix-sub}

You can select submatrices of a given matrix using the syntax

\hspace{1em} \texttt{A[}\textsl{rows},\textsl{cols}\texttt{]}

where \textsl{rows} can take any of these forms:

\begin{center}
\begin{tabular}{lll}
1. & empty & selects all rows \\
2. & a single integer & selects the single specified row \\
3. & two integers separated by a colon & selects a range of rows \\
4. & the name of a matrix & selects the specified rows \\
\end{tabular}
\end{center}

With regard to option 2, the integer value can be given numerically,
as the name of an existing scalar variable, or as an expression that
evaluates to a scalar.  With option 4, the index matrix given in the
\textsl{rows} field must be either $p\times 1$ or $1\times p$, and
should contain integer values in the range 1 to $n$, where $n$ is the
number of rows in the matrix from which the selection is to be made.

The \textsl{cols} specification works in the same way, \textit{mutatis
  mutandis}.  Here are some examples.
%
\begin{code}
matrix B = A[1,]
matrix B = A[2:3,3:5]
matrix B = A[2,2]
matrix idx = {1, 2, 6}
matrix B = A[idx,]
\end{code}
%
The first example selects row 1 from matrix \texttt{A}; the second
selects a $2\times 3$ submatrix; the third selects a scalar; and
the fourth selects rows 1, 2, and 6 from matrix \texttt{A}.

If the matrix in question is $n\times 1$ or $1\times m$, it is
OK to give just one index specifier and omit the comma. For example,
\texttt{A[2]} selects the second element of \texttt{A} if \texttt{A}
is a vector. Otherwise the comma is mandatory.

In addition there are some predefined index specifications,
represented by the keywords \texttt{diag}, \texttt{lower},
\texttt{upper}, \texttt{real}, \texttt{imag} and \texttt{end}. With
the exception of \texttt{end}, these keywords imply specific row
\textit{and} column selections, and therefore cannot be combined with
any additional, comma-separated term.

\begin{itemize}
\item The \texttt{diag} specification selects the principal diagonal
  of a matrix.
\item \texttt{lower} and \texttt{upper} select, respectively, the
  elements of a matrix below and those above the principal diagonal.
\item \texttt{real} and \texttt{imag} are specific to complex matrices
  and are described in chapter~\ref{chap:complex}.
\item \texttt{end} selects the last element in a given row or column.
  It can be employed in arithmetical expressions, so for example
  \texttt{end-1} accesses the second-last element in a row or column.
\end{itemize}

Submatrix selections cane be used on either the right-hand side of a
matrix-generating formula or the left.  Here is an example of use of a
selection on the right, to extract a $2\times 2$ submatrix $B$ from a
$3\times 3$ matrix $A$, then the lower triangle of $A$:
%
\begin{code}
matrix A = {1, 2, 3; 4, 5, 6; 7, 8, 9}
matrix B = A[1:2,2:3]
matrix C = A[lower]
\end{code}
%
And here are examples of selection on the left.  The second line below
writes a $2\times 2$ identity matrix into the bottom right corner of the
$3\times 3$ matrix $A$.  The fourth line replaces the diagonal of $A$ 
with 1s.
%
\begin{code}
matrix A = {1, 2, 3; 4, 5, 6; 7, 8, 9}
matrix A[2:3,2:3] = I(2)
matrix d = {1, 1, 1}
matrix A[diag] = d
\end{code}

When the \texttt{lower} and \texttt{upper} selections are used on the
right, they yield a vector holding the elements in their scope. The
ordering of the elements is column-major in both cases, as illustrated
below for the $4 \times 4$ case.
\begin{center}
  \begin{tabular}{cccc}
    $d$ & 1 & 2 & 4 \\
    1 & $d$ & 3 & 5 \\
    2 & 4 & $d$ & 6 \\
    3 & 5 & 6 & $d$
  \end{tabular}
\end{center}
This means that \texttt{lower} and \texttt{upper} do not produce the
same result for symmetric matrices bigger than $3 \times 3$, which may
seem unfortunate, but it gives the user a degree of flexibility in
respect of the ordering of the elements. Suppose you have a
non-symmetric matrix $M$ and you'd like to extract the infradiagonal
elements in \textit{row}-major order: \texttt{(M')[upper]} will do the
job.

When \texttt{lower} and \texttt{upper} are used on the left, the
replacement must be either (a) a vector of length equal to the number
of elements in the selection or (b) a scalar value. In case (a) the
elements of the target matrix are filled in column-major order; in
case (b) they are all set using the scalar.

One possible use of these tools is taking (say) a lower triangular
matrix and rendering it symmetric by copying the elements from beneath
the diagonal to above. The way to get this right (assuming you have a
lower triangular matrix \texttt{L}) is
\begin{code}
L[upper] = (L')[upper]  # note: not L[upper] = L[lower]
\end{code}

\section{Deleting rows or columns}
\label{sec:neg-indices}

A variant of submatrix notation is available for convenience in
dropping specified rows and/or columns from a matrix, namely giving
negative values for the indices. Here is a simple example,
%
\begin{code}
matrix A = {1, 2, 3; 4, 5, 6; 7, 8, 9}
matrix B = A[-2,-3]
\end{code}
%
which creates \texttt{B} as a $2\times 2$ matrix which drops row 2 and
column 3 from \texttt{A}. Negative indices can also be given in the
form of an index vector:
%
\begin{code}
matrix rdrop = {-1,-3,-5}
matrix B = A[rdrop,]
\end{code}
%
In this case \texttt{B} is formed by dropping rows 1, 3 and 5 from
\texttt{A} (which must have at least 5 rows), but retaining the column
dimension of \texttt{A}.

There are two limitations on the use of negative indices. First, the
\texttt{from:to} range syntax described in the previous section is not
available, but you can use the \texttt{seq} function to achieve an
equivalent effect, as in
%
\begin{code}
matrix A = muniform(1, 10)
matrix B = A[,-seq(3,7)]
\end{code}
%
where \texttt{B} drops columns 3 to 7 from \texttt{A}. Second, use of
negative indices is valid only on the right-hand side of a matrix
calculation; there is no ``negative index'' equivalent of assignment
to a submatrix, as in
%
\begin{code}
A[1:3,] = ones(3, cols(A))
\end{code}

\section{Matrix operators}
\label{matrix-op}

The following binary operators are available for matrices:

\begin{center}
\begin{tabular}{ll}
\texttt{+}  & addition \\
\texttt{-}  & subtraction \\
\texttt{*}  & ordinary matrix multiplication \\
\texttt{'}  & pre-multiplication by transpose \\
\verb|\|    & matrix ``left division'' (see below) \\
\texttt{/}  & matrix ``right division'' (see below) \\
\verb+~+    & column-wise concatenation \\
\verb+|+    & row-wise concatenation \\
\texttt{**} & Kronecker product \\
\texttt{==}  & test for equality \\
\texttt{!=} & test for inequality
\end{tabular}
\end{center}

In addition, the following operators (``dot'' operators) apply on an
element-by-element basis:

\begin{center}
\begin{tabular}{ccccccccccc}
\texttt{.+}  &  \texttt{.-}  &
\texttt{.*}  &  \texttt{./}  &  \verb+.^+  &
\texttt{.=}  &  \texttt{.>}  &  \texttt{.<} &
\texttt{.>=}  &  \texttt{.<=} & \texttt{.!=}
\end{tabular}
\end{center}

Here are explanations of the less obvious cases. 

For matrix addition and subtraction, in general the two matrices have
to be of the same dimensions but an exception to this rule is granted
if one of the operands is a $1\times 1$ matrix or scalar.  The scalar
is implicitly promoted to the status of a matrix of the correct
dimensions, all of whose elements are equal to the given scalar value.
For example, if $A$ is an $m \times n$ matrix and $k$ a scalar, then
the commands
%
\begin{code}
matrix C = A + k
matrix D = A - k
\end{code}
%
both produce $m \times n$ matrices, with elements $c_{ij} = 
a_{ij} + k$ and $d_{ij} = a_{ij} - k$ respectively.

By ``pre-multiplication by transpose'' we mean, for example, that 
%
\begin{code}
matrix C = X'Y
\end{code}
%
produces the product of $X$-transpose and $Y$.  In effect, the
expression \texttt{X'Y} is shorthand for \texttt{X'*Y}, which is also
valid syntax. In the special case $X = Y$, however, the two are not
exactly equivalent. The former expression uses a specialized algorithm
with two advantages: it is more efficient computationally, and ensures
that the result is free of machine precision artifacts that may render
it numerically non-symmetric. This, however, is unlikely to be an
issue unless your $X$ matrix is rather large (at least several
hundreds rows/columns).

In matrix ``left division'', the statement 
%
\begin{code}
matrix X = A \ B
\end{code}
%
is interpreted as a request to find the matrix $X$ that solves $AX=B$,
so $A$ and $B$ must have the same number of rows. If $A$ is a square
matrix, this is in principle equivalent to $A^{-1}B$, which fails if
$A$ is singular; the numerical method employed here is the LU
decomposition.  If $A$ is a $T \times k$ matrix with $T > k$, then $X$
is the least-squares solution, $X = (A'A)^{-1}A'B$, which fails if
$A'A$ is singular; the numerical method is the QR decomposition.
Otherwise, the operation fails.

For matrix ``right division'', as in \texttt{X = A / B}, $X$ is the
matrix that solves $XB = A$, so $A$ and $B$ must have the same number
of columns. If $B$ is non-singular this is in principle equivalent to
$AB^{-1}$, otherwise $X$ is the least-squares solution.

In ``dot'' operations a binary operation is applied element by
element; the result of this operation is obvious if the matrices are
of the same size. However, there are several other cases where such
operators may be applied.  For example, if we write
%
\begin{code}
matrix C = A .- B
\end{code}
% 
then the result $C$ depends on the dimensions of $A$ and $B$.  Let $A$
be an $m \times n$ matrix and let $B$ be $p \times q$; the result is
as follows:
\begin{center}
  \begin{tabular}{ll}
    \textit{Case} & \textit{Result} \\[4pt]
    Dimensions match ($m=p$ and $n=q$) & 
    $c_{ij} = a_{ij} -  b_{ij}$ \\ 
    $A$ is a column vector; rows match ($m=p$; $n=1$) &
    $c_{ij} = a_{i} - b_{ij}$ \\ 
    $B$ is a column vector; rows match ($m=p$; $q=1$) &
    $c_{ij} = a_{ij} - b_{i}$ \\ 
    $A$ is a row vector; columns match ($m=1$; $n=q$) &
    $c_{ij} = a_{j} - b_{ij}$ \\ 
    $B$ is a row vector; columns match ($m=p$; $q=1$) &
    $c_{ij} = a_{ij} - b_{j}$ \\ 
    $A$ is a column vector; $B$ is a row vector ($n=1$; $p=1$) &
    $c_{ij} = a_{i} - b_{j}$ \\ 
    $A$ is a row vector; $B$ is a column vector ($m=1$; $q=1$) &
    $c_{ij} = a_{j} - b_{i}$ \\ 
    $A$ is a scalar ($m=1$ and $n=1$) &
    $c_{ij} = a - b_{ij}$ \\ 
    $B$ is a scalar ($p=1$ and $q=1$) &
    $c_{ij} = a_{ij} - b$ \\ 
  \end{tabular}
\end{center}
%
If none of the above conditions are satisfied the result is undefined
and an error is flagged.

Note that this convention makes it unnecessary, in most cases, to use
diagonal matrices to perform transformations by means of ordinary
matrix multiplication: if $Y = XV$, where $V$ is diagonal, it is
computationally much more convenient to obtain $Y$ via the instruction
%
\begin{code}
matrix Y = X .* v
\end{code}
%
where \texttt{v} is a row vector containing the diagonal of $V$.

\subsection{Concatenation}

In \textit{column-wise concatenation} of an $m\times n$ matrix $A$ and
an $m\times p$ matrix $B$, the result is an $m\times (n+p)$ matrix.
That is,
%
\begin{code}
matrix C = A ~ B
\end{code}
% 
produces $C = \left[ \begin{array}{cc} A & B \end{array} \right]$.

\textit{Row-wise concatenation} of an $m\times n$ matrix $A$ and
an $p\times n$ matrix $B$ produces an $(m+p) \times n$ matrix.
That is,
%
\begin{code}
matrix C = A | B
\end{code}
% 
produces $C = \left[ \begin{array}{cc} A \\ B \end{array} \right]$.

In general the two matrix operands must be conformable, as indicated
above. But as a special case for convenience, a scalar or $1 \times 1$
matrix can be concatenated with a matrix of any size; the single value
is automatically repeated, if necessary, to form a row or column
vector of length conformable with the matrix in question. So for
example
%
\begin{code}
matrix A = {1, 2, 3}
matrix B = A | 1
\end{code}
%
produces $B = \left[ \begin{array}{ccc} 1 & 2 & 3 \\
  1 & 1 & 1 \end{array} \right]$.

\section{Matrix--scalar operators}
\label{matrix-scalar-op}

For matrix $A$ and scalar $k$, the operators shown in
Table~\ref{tab:matrix-scalar-ops} are available.  (Addition and
subtraction were discussed in section~\ref{matrix-op} but we include
them in the table for completeness.)  In addition, for square $A$ and
scalar $x$, \verb|B = A^x| produces a matrix $B$ which is $A$ raised
to the power $x$, but only if either of two conditions are
satisfied. First, if $x$ is a non-negative integer then Golub and Van
Loan's ``Binary Powering'' Algorithm 11.2.2 is used---see
\cite{golub96}---and $A$ can then be a generic square matrix. Second,
if $A$ is positive semidefinite the power is computed via its
eigen-decomposition and $x$ can be a real number, subject to the
constraint that $x$ can be negative only if $A$ is invertible.

\begin{table}[htbp]
\centering
\begin{tabular}{ll}
\textit{Expression} & \textit{Effect} \\[4pt]
\texttt{matrix B = A * k} & $b_{ij} = k a_{ij}$ \\
\texttt{matrix B = A / k} & $b_{ij} = a_{ij} / k$ \\
\texttt{matrix B = k / A} & $b_{ij} = k / a_{ij}$ \\
\texttt{matrix B = A + k} & $b_{ij} = a_{ij} + k$ \\
\texttt{matrix B = A - k} & $b_{ij} = a_{ij} - k$ \\
\texttt{matrix B = k - A} & $b_{ij} = k - a_{ij}$ \\
\texttt{matrix B = A \% k} & $b_{ij} = a_{ij} \mbox{ modulo } k$ \\
\end{tabular}
\caption{Matrix--scalar operators}
\label{tab:matrix-scalar-ops}
\end{table}


\section{Matrix functions}
\label{sec:matrix-func}

Most of the functions available for scalars and series also apply to
matrices on an element-by-element basis. This is the case for
\texttt{log}, \texttt{exp}, \texttt{sqrt}, \texttt{sin} and many
others. For example, if a matrix \texttt{A} is already defined, then
%
\begin{code}
matrix B = sqrt(A)
\end{code}
%
generates a matrix such that $b_{ij} = \sqrt{a_{ij}}$.  All such
functions require a single matrix as argument, or an expression which
evaluates to a single matrix.\footnote{Note that to find the ``matrix
  square root'' you need the \texttt{cholesky} function (see below).
  And since the \texttt{exp} function computes the exponential element
  by element, it does \emph{not} return the matrix exponential unless
  the matrix is diagonal. To get the matrix exponential, use
  \texttt{mexp}.}

In this section, we review some aspects of functions that apply
specifically to matrices. A full account of each function is available
in the \GCR.

\newlength{\cwid}
\setlength{\cwid}{0.1\textwidth}

\begin{table}[htbp]
\centering
\input matfuncs.tex
\caption{Matrix functions by category}
\label{tab:matrix_funcs_cat}
\end{table}

\subsection{Matrix reshaping}
\label{matrix-mshape}

In addition to the methods discussed in sections
\ref{sec:matrix-create} and \ref{sec:matrix-sub}, a matrix can also be
created by re-arranging the elements of a pre-existing matrix. This is
accomplished via the \texttt{mshape} function. It takes three
arguments: the input matrix, $A$, and the rows and columns of the
target matrix, $r$ and $c$ respectively.  Elements are read from $A$
and written to the target in column-major order.  If $A$ contains
fewer elements than $n = r \times c$, they are repeated cyclically; if
$A$ has more elements, only the first $n$ are used.

For example:
\begin{code}
matrix a = mnormal(2,3)
a
matrix b = mshape(a,3,1)
b
matrix b = mshape(a,5,2)
b
\end{code}
produces
\begin{code}
?   a
a

      1.2323      0.99714     -0.39078
     0.54363      0.43928     -0.48467

?   matrix b = mshape(a,3,1)
Generated matrix b
?   b
b

      1.2323
     0.54363
     0.99714

?   matrix b = mshape(a,5,2)
Replaced matrix b
?   b
b

      1.2323     -0.48467
     0.54363       1.2323
     0.99714      0.54363
     0.43928      0.99714
    -0.39078      0.43928
\end{code}

\subsection{Multiple returns and the \texttt{null} keyword}
\label{matrix-multiples}

The functions listed below take one or more matrices as arguments and
compute one or more matrices.

\begin{center}
\begin{tabular}{ll}
\texttt{eigensym} & Eigen-analysis of symmetric matrix \\
\texttt{eigen}    & Eigen-analysis of general matrix \\
\texttt{mols}     & Matrix OLS \\
\texttt{qrdecomp} & QR decomposition \\
\texttt{svd}      & Singular value decomposition (SVD) 
\end{tabular}
\end{center}

The ``main'' result of the function is always returned as the result
proper. Auxiliary returns, if wanted, are retrieved using pre-existing
matrices, passed to the function in ``pointer'' form
\ref{sec:funscope}). If such values are not needed, the pointer may be
substituted with the keyword \texttt{null}.

The syntax for \texttt{qrdecomp} and \texttt{eigensym} is of the form
%
\begin{code}
matrix B = func(A, &C)
\end{code}
%
The first argument, \texttt{A}, represents the input data, that is,
the matrix whose decomposition or analysis is required.  The second
argument must be either the name of an existing matrix preceded by
\verb+&+ (to indicate the ``address'' of the matrix in question), in
which case an auxiliary result is written to that matrix, or the
keyword \texttt{null}, in which case the auxiliary result is not
produced.

In case a non-null second argument is given, the specified matrix will
be over-written with the auxiliary result.  (It is not required that
the existing matrix be of the right dimensions to receive the result.)

The function \texttt{eigensym} computes the eigenvalues, and
optionally the right eigenvectors, of a symmetric $n \times n$ matrix.
The eigenvalues are returned directly in a column vector of length
$n$; if the eigenvectors are required, they are returned in an $n
\times n$ matrix.  For example:
%
\begin{code}
matrix V = {}
matrix E = eigensym(M, &V)
matrix E = eigensym(M, null)
\end{code}
%
In the first case \texttt{E} holds the eigenvalues of \texttt{M} and
\texttt{V} holds the eigenvectors.  In the second, \texttt{E} holds
the eigenvalues but the eigenvectors are not computed.

The function \texttt{eigen} computes the eigenvalues, and optionally
the right and/or left eigenvectors, of a general $n \times n$
matrix. Following the input matrix argument there are two slots for
matrix-addresses, the first to retrieve the right eigenvectors and the
second for the left.  Calls to this function should therefore conform
to one of the following patterns.
\begin{code}
# get the eigenvalues only
matrix E = eigen(M)

# get the right eigenvectors as well
matrix V = {}
matrix E = eigen(M, &V)

# get both sets of eigenvectors
matrix V = {}
matrix W = {}
matrix E = eigen(M, &V, &W)

# get the left eigenvectors but not the right
matrix W = {}
matrix E = eigen(M, null, &W)
\end{code}

The eigenvalues are returned directly in a complex $n$-vector. If the
eigenvectors are wanted they are returned in a $n \times n$ complex
matrix.

The \texttt{qrdecomp} function computes the QR decomposition of an $m
\times n$ matrix $A$: $A = QR$, where $Q$ is an $m \times n$
orthogonal matrix and $R$ is an $n \times n$ upper triangular matrix.
The matrix $Q$ is returned directly, while $R$ can be retrieved via
the second argument.  Here are two examples:
%
\begin{code}
matrix R
matrix Q = qrdecomp(M, &R)
matrix Q = qrdecomp(M, null)
\end{code}
%
In the first example, the triangular $R$ is saved as \texttt{R}; in
the second, $R$ is discarded.  The first line above shows an example
of a ``simple declaration'' of a matrix: \texttt{R} is declared to be
a matrix but is not given any explicit value. The result is that it is
automatically initialized as an empty matrix (see
Section~\ref{sec:emptymatrix}).

The syntax for \texttt{svd} is
%
\begin{code}
matrix B = func(A, &C, &D)
\end{code}
%
The function \texttt{svd} computes all or part of the singular value
decomposition of the real $m \times n$ matrix $A$. Let $k =
\mbox{min}(m, n)$.  The decomposition is
\[
A = U \Sigma V'
\]
where $U$ is an $m \times k$ orthogonal matrix, $\Sigma$ is an $k
\times k$ diagonal matrix, and $V$ is an $k \times n$ orthogonal
matrix.\footnote{This is not the only definition of the SVD: some
  writers define $U$ as $m \times m$, $\Sigma$ as $m \times n$ (with
  $k$ non-zero diagonal elements) and $V$ as $n \times n$.} The
diagonal elements of $\Sigma$ are the singular values of $A$; they are
real and non-negative, and are returned in descending order.  The
first $k$ columns of $U$ and $V$ are the left and right singular
vectors of $A$.

The \texttt{svd} function returns the singular values, in a vector of
length $k$.  The left and/or right singular vectors may be obtained by
supplying non-null values for the second and/or third arguments
respectively.  For example:
%
\begin{code}
matrix s = svd(A, &U, &V)
matrix s = svd(A, null, null)
matrix s = svd(A, null, &V)
\end{code}
%
In the first case both sets of singular vectors are obtained, in the
second case only the singular values are obtained; and in the third,
the right singular vectors are obtained but $U$ is not computed.
\emph{Please note}: when the third argument is non-null, it is
actually $V'$ that is provided.  To reconstitute the original matrix
from its SVD, one can do:
%
\begin{code}
matrix s = svd(A, &U, &V)
matrix B = (U.*s)*V
\end{code}
%

Finally, the syntax for \texttt{mols} is
%
\begin{code}
matrix B = mols(Y, X, &U)
\end{code}
%
This function returns the OLS estimates obtained by regressing the $T
\times n$ matrix $Y$ on the $T \times k$ matrix $X$, that is, a $k
\times n$ matrix holding $(X'X)^{-1} X'Y$. The Cholesky decomposition
is used. The matrix $U$, if not \texttt{null}, is used to store the
residuals.

\subsection{Reading and writing matrices from/to text files}
\label{sec:matrix-csv}

The two functions \texttt{mread} and \texttt{mwrite} can be used for
basic matrix input/output. This can be useful to enable gretl to
exchange data with other programs.

The \texttt{mread} function accepts one string parameter: the name of
the (plain text) file from which the matrix is to be read.  The file
in question may start with any number of comment lines, defined as
lines that start with the hash mark, ``\texttt{\#}''; such lines are
ignored.  Beyond that, the content must conform to the following
rules:
%
\begin{enumerate}
\item The first non-comment line must contain two integers, separated
  by a space or a tab, indicating the number of rows and columns,
  respectively.
\item The columns must be separated by spaces or tab characters.
\item The decimal separator must be the dot ``\texttt{.}'' character.
\end{enumerate}

Should an error occur (such as the file being badly formatted or
inaccessible), an empty matrix (see section~\ref{sec:emptymatrix}) is
returned.

The complementary function \texttt{mwrite} produces text files
formatted as described above.  The column separator is the tab
character, so import into spreadsheets should be straightforward.
Usage is illustrated in example \ref{ex:matrix-files}.  Matrices stored via
the \texttt{mwrite} command can be easily read by other programs; the
following table summarizes the appropriate commands for reading a
matrix \texttt{A} from a file called \texttt{a.mat} in some
widely-used programs.\footnote{Matlab users may find the Octave
  example helpful, since the two programs are mostly compatible with
  one another.} Note that the Python example requires that the
\texttt{numpy} module is loaded.

\begin{center}
  \begin{tabular}{rl}
    \textbf{Program} & \textbf{Sample code} \\
    \hline
    GAUSS  & \verb|tmp[] = load a.mat;| \\
    & \verb|A = reshape(tmp[3:rows(tmp)],tmp[1],tmp[2]);| \\
    Octave & \verb|fd = fopen("a.mat");| \\
    & \verb|[r,c] = fscanf(fd, "%d %d", "C");| \\
    & \verb|A = reshape(fscanf(fd, "%g", r*c),c,r)';| \\
    & \verb|fclose(fd);| \\
    Ox     & \verb|decl A = loadmat("a.mat");| \\
    R      & \verb|A <- as.matrix(read.table("a.mat", skip=1))| \\
    Python & \verb|A = numpy.loadtxt('a.mat', skiprows=1)| \\
    Julia  & \verb|A = readdlm("a.mat", skipstart=1)| \\
  \hline
\end{tabular}
\end{center}


\begin{script}[htbp]
  \scriptinfo{matrix-files}{Matrix input/output via text files}
  \begin{scode}
nulldata 64
scalar n = 3
string f1 = "a.csv"
string f2 = "b.csv"

matrix a = mnormal(n,n)
matrix b = inv(a)

err = mwrite(a, f1)

if err != 0
  fprintf "Failed to write %s\n", f1
else
  err = mwrite(b, f2)
endif 

if err != 0
  fprintf "Failed to write %s\n", f2
else
  c = mread(f1)
  d = mread(f2)
  a = c*d
  printf "The following matrix should be an identity matrix\n"
  print a
endif
  \end{scode}
\end{script}

Optionally, the \texttt{mwrite} and \texttt{mread} functions can use
gzip compression: this is invoked if the name of the matrix file has
the suffix ``\texttt{.gz}.'' In this case the elements of the matrix
are written in a single column. Note, however, that compression should
not be applied when writing matrices for reading by third-party
software unless you are sure that the software can handle compressed
data.

\section{Matrix accessors}
\label{matrix-accessors}

In addition to the matrix functions discussed above,
various ``accessor'' strings allow you to create copies of internal
matrices associated with models previously estimated.
These are set out in Table~\ref{tab:matrix-accessors}.

\begin{table}[htbp]
\centering
\begin{tabular}{ll}
  \dollar{coeff}  & matrix of estimated coefficients \\
  \dollar{compan} & companion matrix (after VAR or VECM estimation) \\
  \dollar{jalpha} & matrix $\alpha$ (loadings) from Johansen's procedure \\
  \dollar{jbeta}  & matrix $\beta$ (cointegration vectors) from
  Johansen's procedure \\
  \dollar{jvbeta} & covariance matrix for the unrestricted elements of 
  $\beta$ from Johansen's procedure \\
  \dollar{rho}    & autoregressive coefficients for error process \\
  \dollar{sigma}  & residual covariance matrix \\
  \dollar{stderr} & matrix of estimated standard errors \\
  \dollar{uhat}   & matrix of residuals \\
  \dollar{vcv}    & covariance matrix of parameter estimates \\
  \dollar{vma}    & VMA matrices in stacked form (see section
    \ref{sec:var-estim}) \\
  \dollar{yhat}   & matrix of fitted values 
\end{tabular}
\caption{Matrix accessors for model data}
\label{tab:matrix-accessors}
\end{table}

Many of the accessors in Table~\ref{tab:matrix-accessors} behave
somewhat differently depending on the sort of model that is
referenced, as follows:

\begin{itemize}
\item Single-equation models: \dollar{sigma} gets a scalar (the
  standard error of the regression); \dollar{coeff} and
  \dollar{stderr} get column vectors; \dollar{uhat} and
  \dollar{yhat} get series.
\item System estimators: \dollar{sigma} gets the cross-equation
  residual covariance matrix; \dollar{uhat} and \dollar{yhat} get
  matrices with one column per equation.  The format of \dollar{coeff}
  and \dollar{stderr} depends on the nature of the system: for VARs
  and VECMs (where the matrix of regressors is the same for all
  equations) these return matrices with one column per equation, but
  for other system estimators they return a big column vector.
\item VARs and VECMs: \dollar{vcv} is not available, but 
  $X'X^{-1}$ (where $X$ is the common matrix of regressors) is
  available as \dollar{xtxinv}, 
  such that for VARs and VECMs (without restrictions on $\alpha$) a vcv
  equivalent can be easily and efficiently constructed as 
  \dollar{sigma} ** \dollar{xtxinv}.
\end{itemize}

If the accessors are given without any prefix, they retrieve results
from the last model estimated, if any.  Alternatively, they may be
prefixed with the name of a saved model plus a period (\texttt{.}), in
which case they retrieve results from the specified model.  Here are
some examples:
%
\begin{code}
matrix u = $uhat
matrix b = m1.$coeff
matrix v2 = m1.$vcv[1:2,1:2]
\end{code}
%$
The first command grabs the residuals from the last model; the second
grabs the coefficient vector from model \texttt{m1}; and the third
(which uses the mechanism of submatrix selection described above)
grabs a portion of the covariance matrix from model \texttt{m1}.

If the model in question a VAR or VECM (only) \dollar{compan} and
\dollar{vma} return the companion matrix and the VMA matrices in
stacked form, respectively (see section \ref{sec:var-estim} for
details).  After a vector error correction model is estimated via
Johansen's procedure, the matrices \dollar{jalpha} and \dollar{jbeta}
are also available. These have a number of columns equal to the chosen
cointegration rank; therefore, the product
\begin{code}
matrix Pi = $jalpha * $jbeta'
\end{code}
returns the reduced-rank estimate of $A(1)$. Since $\beta$ is
automatically identified via the Phillips normalization (see section
\ref{sec:johansen-ident}), its unrestricted elements do have a proper
covariance matrix, which can be retrieved through the
\dollar{jvbeta} accessor.

\section{Namespace issues}
\label{matrix-namespace}

Matrices share a common namespace with data series and scalar
variables.  In other words, no two objects of any of these types can
have the same name.  It is an error to attempt to change the type of
an existing variable, for example:
%
\begin{code}
scalar x = 3
matrix x = ones(2,2) # wrong!
\end{code}
%
It is possible, however, to delete or rename an existing variable then
reuse the name for a variable of a different type:
\begin{code}
scalar x = 3
delete x
matrix x = ones(2,2) # OK
\end{code}


\section{Creating a data series from a matrix}
\label{matrix-create-series}

Section~\ref{sec:matrix-create} above describes how to create a matrix
from a data series or set of series.  You may sometimes wish to go in
the opposite direction, that is, to copy values from a matrix 
into a regular data series.  The syntax for this operation is
%
\begin{textcode}
series \textsl{sname} = \textsl{mspec}
\end{textcode}
%
where \ttsl{sname} is the name of the series to create and
\ttsl{mspec} is the name of the matrix to copy from, possibly followed
by a matrix selection expression.  Here are two examples.
%
\begin{code}
series s = x
series u1 = U[,1]
\end{code}
%
It is assumed that \texttt{x} and \texttt{U} are pre-existing
matrices.  In the second example the series \texttt{u1} is formed from
the first column of the matrix \texttt{U}.

For this operation to work, the matrix (or matrix selection) must be a
vector with length equal to either the full length of the current
dataset, $n$, or the length of the current sample range, $n^{\prime}$.
If $n^{\prime} < n$ then only $n^{\prime}$ elements are drawn from the
matrix; if the matrix or selection comprises $n$ elements, the
$n^{\prime}$ values starting at element $t_1$ are used, where $t_1$
represents the starting observation of the sample range.  Any values
in the series that are not assigned from the matrix are set to the
missing code.


\section{Matrices and lists}
\label{matrix-and-list}

To facilitate the manipulation of named lists of series (see
Chapter~\ref{chap:lists-strings}), it is possible to convert between
matrices and lists.  In section~\ref{sec:matrix-create} above we mentioned
the facility for creating a matrix from a list of variables, as in
%
\begin{code}
matrix M = { listname }
\end{code}
%
That formulation, with the name of the list enclosed in braces, builds
a matrix whose columns hold the series referenced in the list.
What we are now describing is a different matter: if we say
%
\begin{code}
matrix M = listname
\end{code}
%
(without the braces), we get a row vector whose elements are
the ID numbers of the series in the list.  This special case
of matrix generation cannot be embedded in a compound
expression.  The syntax must be as shown above, namely simple
assignment of a list to a matrix.

To go in the other direction, you can include a matrix on the
right-hand side of an expression that defines a list, as in
%
\begin{code}
list Xl = M
\end{code}
%
where \texttt{M} is a matrix.  The matrix must be suitable for
conversion; that is, it must be a row or column vector containing
non-negative integer values, none of which exceeds the highest ID
number of a series in the current dataset.

Listing~\ref{ex:normalize-list} illustrates the use of this sort of
conversion to ``normalize'' a list, moving the constant (variable 0)
to first position.

\begin{script}[htbp]
  \scriptinfo{normalize-list}{Manipulating a list}
\begin{scode}
function void normalize_list (matrix *x)
  # If the matrix (representing a list) contains var 0,
  # but not in first position, move it to first position
  if (x[1] != 0)
     scalar k = cols(x)
     loop for (i=2; i<=k; i++)
        if (x[i] == 0)
            x[i] = x[1]
            x[1] = 0
            break
         endif
     endloop
  endif
end function

open data9-7
list Xl = 2 3 0 4
matrix x = Xl
normalize_list(&x)
list Xl = x
list Xl print
\end{scode}
\end{script}


\section{Deleting a matrix}
\label{matrix-delete}

To delete a matrix, just write
%
\begin{code}
delete M
\end{code}
%
where \texttt{M} is the name of the matrix to be deleted.

\section{Printing a matrix}

To print a matrix, the easiest way is to give the name of the matrix
in question on a line by itself, which is equivalent to using the
\cmd{print} command:
%
\begin{code}
matrix M = mnormal(100,2)
M
print M
\end{code}

You can get finer control on the formatting of output by using the
\cmd{printf} command, as illustrated in the interactive session below:
%
\begin{code}
? matrix Id = I(2)
 matrix Id = I(2)
Generated matrix Id
? print Id
 print Id
Id (2 x 2)

  1   0 
  0   1 

? printf "%10.3f", Id
     1.000     0.000
     0.000     1.000
\end{code}

For presentation purposes you may wish to give titles to the columns
of a matrix.  For this you can use the function \cmd{cnameset}. The
first argument to this function is a matrix and the second can take
any of three forms: a string that contains as many space-separated
substrings as the matrix has columns; an array of strings (see
Section~\ref{sec:arrays}) of the right length; or a named list of
series, whose names will be used as headings.  For example,
%
\begin{code}
? matrix M = mnormal(3,3)
? cnameset(M, "foo bar baz")
? print M
M (3 x 3)

         foo          bar          baz 
      1.7102     -0.76072     0.089406 
    -0.99780      -1.9003     -0.25123 
    -0.91762     -0.39237      -1.6114
\end{code}
%
Row names may be added via the function \cmd{rnameset}, which works in
the same way.

\section{Example: OLS using matrices}
\label{matrix-example}

Listing~\ref{ex:matrix-ols} shows how matrix methods can be used to
replicate gretl's built-in OLS functionality.

\begin{script}[htbp]
  \scriptinfo{matrix-ols}{OLS via matrix methods}
\begin{scode}
open data4-1
matrix X = { const, sqft }
matrix y = { price }
matrix b = invpd(X'X) * X'y
print "estimated coefficient vector"
b
matrix u = y - X*b
scalar SSR = u'u
scalar s2 = SSR / (rows(X) - rows(b))
matrix V = s2 * inv(X'X)
V
matrix se = sqrt(diag(V))
print "estimated standard errors"
se
# compare with built-in function
ols price const sqft --vcv
\end{scode}
\end{script}

%%% Local Variables: 
%%% mode: latex
%%% TeX-master: "gretl-guide"
%%% End: 

\chapter{Esempi svolti}
\label{chap:cheatsheet}

Questo capitolo spiega come eseguire alcuni compiti comuni (e altri meno comuni)
usando il linguaggio di scripting di \app{gretl}. Alcune delle tecniche mostrate,
ma non tutte, sono utilizzabili anche attraverso l'interfaccia grafica del
programma; sebbene questa possa sembrare pi� intuitiva e facile da utilizzare a
prima vista, incoraggiamo gli utenti a sfruttare le potenzialit� del linguaggio
di scripting di \app{gretl}, dopo aver preso confidenza col programma.

\section{Gestione dei dataset}

\subsection{Periodicit� ``strane''}

\emph{Problema:} si hanno dati rilevati ogni 3 minuti a partire dalle ore 9,
ossia ogni ora � suddivisa in 20 periodi.

\emph{Soluzione:}
\begin{code}
setobs 20 9:1 --special
\end{code}

\emph{Commento:} ora le funzioni come \texttt{sdiff()} (differenza ``stagionale'')
o i metodi di stima come l'ARIMA stagionale funzioneranno come ci si aspetta.

\subsection{Aiuto, i miei dati sono all'incontrario!}

\emph{Problema:} Gretl si aspetta che le serie storiche siano in ordine
cronologico (l'osservazione pi� recente per ultima), ma sono stati importati
da una fonte esterna dei dati ordinati in modo inverso (l'osservazione pi�
recente per prima).

\emph{Soluzione:}
\begin{code}
setobs 1 1 --cross-section
genr ordinamento = -obs
dataset sortby ordinamento
setobs 1 1950 --time-series
\end{code}

\emph{Commento:} la prima riga � richiesta solo se il dataset in uso �
interpretato come serie storiche: occorre rimuovere questo tipo di
interpretazione che (ovviamente) non permette di usare il comando
\texttt{dataset sortby}. Le due righe successive invertono l'ordine dei dati,
usando come variabile di ordinamento il negativo della variabile indice interna
\texttt{obs}. L'ultima riga � solo un esempio: imposta l'interpretazione del
dataset come serie storiche annuali che iniziano dall'anno 1950.

Se si ha un dataset ordinato correttamente per tutte le variabili tranne una, �
possibile correggere l'ordinamento di una sola variabile nel modo seguente:
\begin{code}
genr x = sortby(-obs, x)
\end{code}

\subsection{Eliminare osservazioni mancanti in modo selettivo}

\emph{Problema:} si ha un dataset con molte variabili e si vuole restringere il
campione a quelle osservazioni per cui non ci sono osservazioni mancanti per
nessuna delle variabili \texttt{x1}, \texttt{x2} e \texttt{x3}.

\begin{samepage}
\emph{Soluzione:}
\begin{code}
list X = x1 x2 x3
genr sel = ok(X)
smpl sel --restrict
\end{code}
\end{samepage}

\emph{Commento:} ora � possibile salvare il file con il comando \texttt{store} per
preservare una versione ristretta del dataset.

\subsection{Operazioni diverse a seconda dei valori di una variabile}

\emph{Problema:} si ha una variabile discreta \texttt{d} e si vuole eseguire
alcuni comandi (ad esempio, stimare un modello) suddividendo il campione a
seconda dei valori di \texttt{d}.

\emph{Soluzione:}
\begin{code}
matrix vd = values(d)
m = rows(vd)
loop for i=1..m
  scalar sel = vd[i]
  smpl (d=sel) --restrict --replace
  ols y const x
end loop
smpl full
\end{code}

\emph{Commento:} L'ingrediente principale � un loop, all'interno del quale �
possibile eseguire tutte le istruzioni volute per ogni valore di
\texttt{d}, a patto che siano istruzioni il cui uso � consentito all'interno di
un loop.

\section{Creazione e modifica di variabili}

\subsection{Generazione di una variabile dummy per una specifica osservazione}

\emph{Problema:} Generare $d_t = 0$ per tutte le osservazioni tranne una, per
cui vale $d_t = 1$.

\emph{Soluzione:}
\begin{code}
  genr d = (t="1984:2")
\end{code}

\emph{Commento:} La variabile interna \texttt{t} viene usata per riferirsi alle
osservazioni sotto forma di stringa, quindi se si ha un campione cross-section
si pu� usare semplicemente \texttt{d = (t="123")}; ovviamente, se il dataset
ha delle etichette per i dati si pu� usare l'etichetta corrispondente. Ad
esempio, se si apre il dataset \texttt{mrw.gdt}, fornito con \app{gretl}, si pu�
generare una variabile dummy per l'Italia usando
\begin{code}
  genr DIta = (t="Italy")
\end{code}

Si noti che questo metodo non richiede di creare uno script: � possibile inserire i
comandi appena visti usando il comando dell'interfaccia grafica
``Aggiungi/Definisci nuova variabile''.

\subsection{Generazione di un ARMA(1,1)}

\emph{Problema:} Generare $y_t = 0.9 y_{t-1} + \varepsilon_t - 0.5
\varepsilon_{t-1}$, con $\varepsilon_t \sim N\!I\!I\!D(0,1)$.

\emph{Soluzione:}
\begin{code}
alpha = 0.9
theta = -0.5
series e = normal()
series y = 0
series y = alpha * y(-1) + e + theta * e(-1)
\end{code}

\emph{Commento:} L'istruzione \texttt{series y = 0} � necessaria perch�
l'istruzione successiva valuta \texttt{y} ricorsivamente, quindi occorre
impostare \texttt{y[1]}. Si noti che occorre usare la parola chiave
\texttt{series}, invece di scrivere \texttt{genr y = 0} o semplicemente
\texttt{y = 0}, per assicurarsi che \texttt{y} sia una serie e non uno scalare.

\subsection{Assegnazione condizionale}

\emph{Problema:} Generare $y_t$ secondo la regola seguente:
\[
  y_t = \left\{ 
    \begin{array}{ll} 
      x_t & \mathrm{for} \quad d_t = 1 \\ 
      z_t & \mathrm{for} \quad d_t = 0 
    \end{array}
    \right. 
\]

\emph{Soluzione:}
\begin{code}
series y = d ? x : z
\end{code}

\emph{Commento:} ci sono varie alternative a quella presentata. La prima �
quella di forza bruta usando i loop. Un'altra, pi� efficiente ma ancora
subottimale, � quella di usare \verb|y = d*x + (1-d)*z|. L'operatore di
assegnazione condizionale ternario non solo rappresenta la soluzione
numericamente pi� efficiente, ma � anche quella di pi� semplice lettura, una volta
che si � abituati alla sua sintassi, che per alcuni lettori pu� ricordare quella
della funzione \texttt{=IF()} nei fogli di calcolo.

\section{Trucchi utili}
\label{sec:cheat-neat}

\subsection{Dummy di interazione}

\emph{Problema:} si vuole stimare il modello $y_i = \mathbf{x}_i
\beta_1 + \mathbf{z}_i \beta_2 + d_i \beta_3 + (d_i \cdot \mathbf{z}_i)
\beta_4 + \varepsilon_t$, dove $d_i$ � una variabile dummy, mentre
$\mathbf{x}_i$ e $\mathbf{z}_i$ sono vettori di variabili esplicative.

\emph{Soluzione:}
\begin{code}
list X = x1 x2 x3
list Z = z1 z2
list dZ = null
loop foreach i Z
  series d$i = d * $i
  list dZ = dZ d$i
end loop 

ols y X Z d dZ
\end{code} 
%$

\emph{Commento:} incredibile cosa si pu� fare con la sostituzione delle
stringhe, vero?

\subsection{Volatilit� percepita}

\emph{Problema:} avendo dati raccolti ogni minuto, si vuole calcolare la
``realized volatility'' per ogni ora come $RV_t = \frac{1}{60}
\sum_{\tau=1}^{60} y_{t:\tau}^2$. Il campione parte da 1:1.

\emph{Soluzione:}
\begin{code}
smpl full
genr time
genr minute = int(time/60) + 1
genr second = time % 60
setobs minute second --panel
genr rv = psd(y)^2
setobs 1 1
smpl second=1 --restrict
store foo rv
\end{code}

\emph{Commento:} qui facciamo credere a \app{gretl} che il dataset sia di tipo
panel, dove i minuti sono le ``unit�'' e i secondi sono il ``tempo''; in questo
modo possiamo utilizzare la funzione speciale \texttt{psd()} (panel standard deviation).
Quindi eliminiamo semplicemente tutte le osservazioni tranne una per minuto e
salviamo i dati risultanti (\texttt{store foo rv} significa ``salva nel file di dati
\texttt{foo.gdt} la serie \texttt{rv}'').


%%% Local Variables: 
%%% mode: latex
%%% TeX-master: "gretl-guide"
%%% End: 


\part{Metodi econometrici}

\chapter{Stima robusta della matrice di covarianza}
\label{chap-robust-vcv}

\section{Introduzione}
\label{vcv-intro}

Si consideri (ancora una volta) il modello di regressione lineare
%
\begin{equation}
\label{eq:ols-again}
y = X\beta + u
\end{equation}
%
dove $y$ e $u$ sono vettori di dimensione $T$, $X$ � una matrice
$T \times k$ di regressori, e $\beta$ � un vettore di parametri di
dimensione $k$. Come � noto, lo stimatore di $\beta$ dato dai minimi
quadrati ordinari (OLS) �
%
\begin{equation}
\label{eq:ols-betahat}
\hat{\beta} = (X'X)^{-1} X'y
\end{equation}
%
Se la condizione $E(u|X) = 0$ � soddisfatta, questo stimatore � non distorto;
sotto condizioni meno restrittive, lo stimatore � distorto ma consistente. �
semplice mostrare che quando lo stimatore OLS non � distorto (ossia quando
$E(\hat{\beta}-\beta) = 0$), la sua varianza �
%
\begin{equation}
\label{eq:ols-varb}
\mbox{Var}(\hat{\beta}) = 
  E\left((\hat{\beta}-\beta)(\hat{\beta}-\beta)'\right) 
  = (X'X)^{-1} X' \Omega X (X'X)^{-1}
\end{equation}
%
dove $\Omega = E(uu')$ � la matrice di covarianza dei termini di errore.

Sotto l'ipotesi che i termini di errore siano indipendenti e identicamente
distribuiti (iid), si pu� scrivere $\Omega = \sigma^2 I$, dove $\sigma^2$
� la varianza (comune) degli errori (e le covarianze sono zero). In questo caso,
la (\ref{eq:ols-varb}) si riduce alla ``classica'' formula,
%
\begin{equation}
\label{eq:ols-classical-varb}
\mbox{Var}(\hat{\beta}) = \sigma^2(X'X)^{-1}
\end{equation}

Se la condizione iid non � soddisfatta, ne derivano due conseguenze. Per prima
cosa � possibile costruire uno stimatore pi� efficiente di quello OLS, ad
esempio un qualche tipo di stimatore FGLS (Feasible Generalized Least Squares).
Inoltre, la semplice formula ``classica'' per la varianza dello stimatore dei
minimi quadrati non � pi� corretta, e quindi gli errori standard da essa
derivati (ossia le radici quadrate degli elementi sulla diagonale della matrice
definita dalla \ref{eq:ols-classical-varb}) non sono strumenti corretti per
l'inferenza statistica.

Nella storia recente dell'econometria ci sono due approcci principali al
problema rappresentato dagli errori non iid. L'approccio ``tradizionale''
consiste nell'usare uno stimatore FGLS. Ad esempio, se l'ipotesi iid viene
violata a causa di una dipendenza di tipo temporale tra i termini di errore, e
se si ha ragione di pensare che questo si possa modellare con un processo di
autocorrelazione del prim'ordine, si potrebbe utilizzare un metodo di stima
AR(1), come quello di Cochrane--Orcutt, o di Hildreth--Lu, o di Prais--Winsten.
Se il problema sta nel fatto che la varianza dell'errore non � costante tra le
osservazioni, si potrebbe stimare la varianza come funzione delle variabili
indipendenti e usare quindi i minimi quadrati ponderati, prendendo come pesi i
reciproci delle varianze stimate.

Mentre questi metodi sono tuttora utilizzati, un approccio alternativo sta
guadagnando favore: usare lo stimatore OLS ma calcolare gli errori standard (o
pi� in generale le matrici di covarianza) in modo che siano robusti rispetto
alle deviazioni dall'ipotesi iid. Questo approccio � spesso associato all'uso di
grandi dataset, abbastanza grandi da suggerire la validit� della propriet� di
consistenza (asintotica) dello stimatore OLS, ed � stato reso possibile anche
dalla disponibilit� di sempre maggiori potenze di calcolo: il calcolo degli
errori standard robusti e l'uso di grandi dataset erano compiti scoraggianti
fino a qualche tempo fa, ma ora non pongono alcun problema. Un punto a favore di
questo approccio  consiste nel fatto che, mentre la stima FGLS offre un
vantaggio in termini di efficienza, spesso richiede di fare delle ipotesi
statistiche aggiuntive, che potrebbero non essere giustificate, che potrebbe
essere difficile testare, e che potrebbero mettere in discussione la consistenza
dello stimatore; ad esempio, l'ipotesi di ``fattore comune'' che �
implicata dalle tradizionali ``correzioni'' per i termini di errore
autocorrelati.

\textit{Introduction to Econometrics} di James Stock e Mark Watson illustra
questo approccio in modo comprensibile agli studenti universitari: molti dei
dataset usati sono composti da migliaia o decine di migliaia di osservazioni, la
stima FGLS � poco considerata, mentre si pone l'enfasi sull'uso di errori
standard robusti (in effetti la discussione degli errori standard classici nel
caso di omoschedasticit� � confinata in un'appendice).

Pu� essere utile passare in rassegna le opzioni fornite da \app{gretl} per la
stima robusta della matrice di covarianza. Il primo punto da notare � che
\app{gretl} produce errori standard ``classici'' come risultato predefinito
(in tutti i casi tranne quello della stima GMM). In modalit� a riga di comando
(o negli script) � possibile ottenere gli errori standard robusti aggiungendo
l'opzione \verb|--robust| ai comandi di stima. Se si usa l'interfaccia grafica,
le finestre di dialogo per la specificazione dei modelli contengono una casella
``Errori standard robusti'', insieme a un pulsante ``Configura'' che viene
attivato se si seleziona la casella. Premendo il pulsante si ottiene una
finestra (raggiungibile anche attraverso il men� principale: Strumenti
$\rightarrow$ Preferenze $\rightarrow$ Generali $\rightarrow$ HCCME), da cui �
possibile scegliere tra diverse varianti di stima robusta, e anche rendere
predefinita la stima robusta.

Le specifiche opzioni disponibili dipendono dalla natura dei dati in esame
(cross-section, serie storiche o panel) e anche, in qualche misura, dalla scelta
dello stimatore (anche se finora si � parlato di errori standard robusti in
relazione allo stimatore OLS, questi possono essere usati anche con altri
stimatori). Le successive sezioni di questo capitolo presentano argomenti
caratteristici di ognuno dei tre tipi di dati appena ricordati. Dettagli
ulteriori riguardanti la stima della matrice di covarianza nel contesto GMM
si trovano nel capitolo~\ref{chap:gmm}.

Per concludere questa introduzione, ricordiamo ancora quello che gli ``errori
standard robusti'' possono e non possono garantire: possono fornire un'inferenza
statistica asintoticamente valida in modelli che sono correttamente specificati,
ma in cui gli errori non sono iid. Il termine ``asintotico'' significa che
questo approccio pu� essere poco utile su piccoli campioni. Il termine
``correttamente specificati'' significa che non si ha una bacchetta
magica: se il termine di errore � correlato con i regressori, le stime dei
parametri sono distorte e inconsistenti, gli errori standard robusti non possono
risolvere questo problema.

\section{Dati cross-section e HCCME}
\label{vcv-hccme}

Con dati cross-section, la causa pi� comune di violazione dell'ipotesi iid �
data dall'eteroschedasticit� (varianza non costante)\footnote{In alcuni contesti
speciali, il problema pu� essere invece l'autocorrelazione spaziale. Gretl
  non ha funzioni per gestire questo caso, che quindi verr� trascurato in questa
  trattazione.}. Il alcuni casi � possibile fare delle ipotesi plausibili sulla
forma specifica dell'eteroschedasticit� e quindi applicare una correzione ad
hoc, ma di solito non si sa con che tipo di eteroschedasticit� si ha a che fare.
Vogliamo quindi trovare uno stimatore della matrice di covarianza delle stime
dei parametri che mantenga la sua validit�, almeno dal punto di vista
asintotico, anche in caso di eteroschedasticit�. Che questo sia possibile non �
ovvio a priori, ma White (1980) ha mostrato che
%
\begin{equation}
\label{eq:ols-varb-h}
\widehat{\mbox{Var}}_{\rm h}(\hat{\beta}) = 
       (X'X)^{-1} X' \hat{\Omega} X (X'X)^{-1}
\end{equation}
%
fa al caso nostro (come al solito in statistica dobbiamo dire ``sotto alcune
condizioni'', ma in questo caso le condizioni non sono molto restrittive).
$\hat{\Omega}$ � una matrice diagonale i cui elementi diversi da zero possono
essere stimati usando i quadrati dei residui OLS. White ha chiamato la
(\ref{eq:ols-varb-h}) uno stimatore HCCME (heteroskedasticity-consistent covariance
matrix estimator).

Davidson e MacKinnon (2004, capitolo 5) offrono una discussione utile di
alcune varianti dello stimatore HCCME di White. Chiamano HC$_0$ la variante
originale della (\ref{eq:ols-varb-h}), in cui gli elementi diagonali di
$\hat{\Omega}$ sono stimati direttamente con i quadrati dei residui OLS,
$\hat{u}^2_t$ (gli errori standard associati sono chiamati spesso ``errori
standard di White''). Le varie estensioni dell'approccio di White hanno in
comune un punto: l'idea che i quadrati dei residui OLS siano probabilmente
``troppo piccoli'' in media. Questa idea � piuttosto intuitiva: le stime OLS dei
parametri, $\hat{\beta}$, per costruzione soddisfano il criterio che la somma
dei quadrati dei residui
%
\[
\sum \hat{u}^2_t = \sum \left( y_t - X_t \hat{\beta} \right)^2
\]
%
� minimizzata, dati $X$ e $y$.  Si supponga che $\hat{\beta} \neq
\beta$.  � quasi certo che sia cos�: anche se OLS non � distorto, sarebbe un
miracolo se i $\hat{\beta}$ calcolati da un campione finito fossero esattamente
uguali a $\beta$. Ma in questo caso la somma dei quadrati dei veri errori (non
osservati), $\sum u^2_t = \sum
(y_t - X_t \beta)^2$ � certamente maggiore di $\sum \hat{u}^2_t$.
Le varianti di HC$_0$ partono da questo punto nel modo seguente:
%
\begin{itemize}
\item HC$_1$: applica una correzione per gradi di libert�, moltiplicando la
  matrice HC$_0$ per $T/(T-k)$.
\item HC$_2$: invece di usare $\hat{u}^2_t$ per gli elementi diagonali di
  $\hat{\Omega}$, usa $\hat{u}^2_t/(1-h_t)$, dove $h_t =
  X_t(X'X)^{-1}X'_t$, il $t^{\rm esimo}$ elemento diagonale della matrice di
  proiezione, $P$, che ha la propriet� che $P\cdot y = \hat{y}$. La rilevanza di
  $h_t$ sta nel fatto che se la varianza di tutti gli $u_t$ �
  $\sigma^2$, il valore atteso di $\hat{u}^2_t$ � $\sigma^2(1-h_t)$, o in altre
  parole, il rapporto $\hat{u}^2_t/(1-h_t)$ ha un valore atteso di
  $\sigma^2$. Come mostrano Davidson e MacKinnon, $0\leq h_t <1$ per ogni
  $t$, quindi questa correzione non pu� ridurre gli elementi diagonali di
  $\hat{\Omega}$ e in generale li corregge verso l'alto.
\item HC$_3$: Usa $\hat{u}^2_t/(1-h_t)^2$.  Il fattore aggiuntivo
  $(1-h_t)$ nel denominatore, relativo a HC$_2$, pu� essere giustificato col
  fatto che le osservazioni con ampia varianza tendono a esercitare una grossa
  influenza sulle stime OLS, cos� che i corrispondenti residui tendono ad essere
  sottostimati. Si veda Davidson e MacKinnon per ulteriori dettagli.
\end{itemize}

I rispettivi meriti di queste varianti sono stati analizzati sia dal punto di
vista teorico che attraverso simulazioni, ma sfortunatamente non c'� un consenso
preciso su quale di esse sia ``la migliore''. Davidson e MacKinnon sostengono
che l'originale HC$_0$ probabilmente si comporta peggio delle altre varianti,
tuttavia gli ``errori standard di White'' sono citati pi� spesso delle altre
varianti pi� sofisticate e quindi per motivi di comparabilit�, HC$_0$ �
lo stimatore HCCME usato da \app{gretl} in modo predefinito.

Se si preferisce usare HC$_1$, HC$_2$ o HC$_3$, � possibile farlo in due modi.
In modalit� script, basta eseguire ad esempio
%
\begin{code}
set hc_version 2
\end{code}
%
Con l'interfaccia grafica, basta andare nella finestra di configurazione di
HCCME come mostrato sopra e impostare come predefinita una delle varianti.


\section{Serie storiche e matrici di covarianza HAC}
\label{vcv-hac}

L'eteroschedasticit� pu� essere un problema anche con le serie storiche, ma
raramente � l'unico, o il principale, problema.

Un tipo di eteroschedasticit� � comune nelle serie storiche macroeconomiche, ma
� abbastanza semplice da trattare: nel caso di serie con una forte tendenza,
come il prodotto interno lordo, il consumo o l'investimento aggregato, e simili,
alti valori della variabile sono probabilmente associati ad alta variabilit� in
termini assoluti. Il rimedio ovvio, usato da molti studi macroeconomici,
consiste nell'usare i logaritmi di queste serie, al posto dei livelli. A patto
che la variabilit� \textit{proporzionale} di queste serie rimanga abbastanza
costante nel tempo, la trasformazione logaritmica � efficace.

Altre forme di eteroschedasticit� possono sopravvivere alla trasformazione
logaritmica e richiedono un trattamento distinto dal calcolo degli errori
standard robusti. Ad esempio l'\textit{eteroschedasticit� autoregressiva
condizionale} riscontrabile ad esempio nelle serie dei prezzi di borsa, dove
grandi disturbi sul mercato possono causare periodi di aumento della volatilit�;
fenomeni come questo giustificano l'uso di specifiche strategie di stima, come
nei modelli GARCH (si veda il capitolo~\ref{chap:timeser}).

Nonostante tutto questo, � possibile che un certo grado di eteroschedasticit�
sia presente nelle serie storiche: il punto chiave � che nella maggior parte dei
casi, questa � probabilmente combinata con un certo grado di correlazione
seriale (autocorrelazione), e quindi richiede un trattamento speciale.
Nell'approccio di White, $\hat{\Omega}$, la matrice di covarianza stimata degli
$u_t$, rimane diagonale: le varianze,
$E(u^2_t)$, possono differire per $t$, ma le covarianze, $E(u_t u_s)$, sono
sempre zero. L'autocorrelazione nelle serie storiche implica che almeno alcuni
degli elementi fuori dalla diagonale di $\hat{\Omega}$ possono essere diversi da
zero. Questo introduce una complicazione evidente e un ulteriore termine da
tenere presente: le stime della matrice di covarianza che sono asintoticamente
valide anche in presenza di eteroschedasticit� e autocorrelazione nel processo
di errore vengono definite HAC (heteroskedasticity and autocorrelation
consistent).

Il tema della stima HAC � trattato in termini pi� tecnici nel capitolo~\ref{chap:gmm},
qui cerchiamo di fornire un'intuizione basilare. Iniziamo da un commento
generale: l'autocorrelazione dei residui non � tanto una propriet� dei dati,
quanto il sintomo di un modello inadeguato. I dati possono avere propriet�
persistenti nel tempo, ma se imponiamo un modello che non tiene conto
adeguatamente di questo aspetto, finiamo con avere disturbi autocorrelati. Al
contrario, spesso � possibile mitigare o addirittura eliminare il problema
dell'autocorrelazione includendo opportune variabili ritardate in un modello di
serie storiche, o in altre parole specificando meglio la dinamica del modello.
La stima HAC \textit{non} dovrebbe essere considerata il primo strumento per
affrontare l'autocorrelazione del termine di errore.

Detto questo, la ``ovvia'' estensione dello stimatore HCCME di White al caso di
errori autocorrelati sembra questa: stimare gli elementi fuori dalla diagonale
di $\hat{\Omega}$ (ossia le autocovarianze, $E(u_t u_s)$) usando, ancora una
volta, gli opportuni residui OLS: $\hat{\omega}_{ts} = \hat{u}_t \hat{u}_s$.
Questo approccio sembra giusto, ma richiede una correzione importante:
cerchiamo uno stimatore \textit{consistente}, che converga verso il vero
$\Omega$ quando l'ampiezza del campione tende a infinito. Campioni pi� ampi
permettono di stimare pi� elementi di $\omega_{ts}$ (ossia, per $t$ e $s$
pi� separati nel tempo), ma \textit{non} forniscono pi� informazione a proposito
delle coppie $\omega_{ts}$ pi� distanti nel tempo, visto che la massima separazione nel
tempo cresce anch'essa al crescere della dimensione del campione. Per assicurare
la consistenza, dobbiamo confinare la nostra attenzione ai processi che
esibiscono una dipendenza limitata nel tempo, o in altre parole interrompere il
calcolo dei valori $\hat{\omega}_{ts}$ a un certo valore massimo di
$p = t-s$ (dove $p$ � trattato come una funzione crescente dell'ampiezza
campionaria, $T$, anche se non � detto che cresca proporzionalmente a $T$).

La variante pi� semplice di questa idea consiste nel troncare il calcolo a un
certo ordine di ritardo finito $p$, che cresce ad esempio come $T^{1/4}$. Il
problema � che la matrice $\hat{\Omega}$ risultante potrebbe  non essere
definita positiva, ossia potremmo ritrovarci con delle varianze stimate
negative. Una soluzione a questo problema � offerta dallo stimatore di
Newey--West (Newey e West, 1987), che assegna pesi declinanti alle
autocovarianze campionarie, man mano che la separazione temporale aumenta.

Per capire questo punto pu� essere utile guardare pi� da vicino la
matrice di covarianza definita nella (\ref{eq:ols-varb-h}), ossia,
%
\[
(X'X)^{-1} (X' \hat{\Omega} X) (X'X)^{-1}
\]
%
Questo � noto come lo stimatore ``sandwich''. La fetta di pane �
$(X'X)^{-1}$, ossia una matrice $k \times k$, che � anche l'ingrediente
principale per il calcolo della classica  matrice di covarianza.
Il contenuto del sandwich �
%
\[
\begin{array}{ccccc}
\hat{\Sigma} & = & X' & \hat{\Omega} & X \\
{\scriptstyle (k \times k)} & &
{\scriptstyle (k \times T)} & {\scriptstyle (T \times T)} & 
  {\scriptstyle (T \times k)}
\end{array}
\]
%
Poich� $\Omega = E(uu')$, la matrice che si sta stimando pu� essere scritta
anche come
\[
\Sigma = E(X'u\,u'X)
\]
%
che esprime $\Sigma$ come la covarianza di lungo periodo del vettore casuale
$X'u$ di dimensione $k$.

Dal punto di vista computazionale, non � necessario salvare la matrice
$T \times T$ $\hat{\Omega}$, che pu� essere molto grande. Piuttosto, si pu�
calcolare il contenuto del sandwich per somma, come
%
\[
\hat{\Sigma} = \hat{\Gamma}(0) + \sum_{j=1}^p w_j 
  \left(\hat{\Gamma}(j) + \hat{\Gamma}'(j) \right)
\]
%
dove la matrice $k \times k$ di autocovarianza campionaria $\hat{\Gamma}(j)$,
per $j \geq 0$, � data da
\[
\hat{\Gamma}(j) = \frac{1}{T} \sum_{t=j+1}^T
  \hat{u}_t \hat{u}_{t-j}\, X'_t\, X_{t-j}
\]
e $w_j$ � il peso dato dall'autocovarianza al ritardo $j > 0$.

Rimangono due questioni. Come determiniamo esattamente la massima lunghezza del
ritardo (o ``larghezza di banda'') $p$ dello stimatore HAC? E come determiniamo
esattamente i pesi $w_j$? Torneremo presto sul (difficile) problema della
larghezza di banda, ma per quanto riguarda i pesi, \app{gretl} offre tre varianti.
Quella predefinita � il kernel di Bartlett, come � usato da
Newey e West. Questo stabilisce che
\[
w_j = \left\{ \begin{array}{cc}
     1 - \frac{j}{p+1} & j \leq p \\
     0 & j > p
     \end{array}
    \right.
\]
in  modo che i pesi declinino linearmente mentre $j$ aumenta. Le altre due
opzioni sono il kernel di Parzen e il kernel QS (Quadratic Spectral).
Per il kernel di Parzen,
\[
w_j = \left\{ \begin{array}{cc}
    1 - 6a_j^2 + 6a_j^3 & 0 \leq a_j \leq 0.5 \\
    2(1 - a_j)^3 & 0.5 < a_j \leq 1 \\
    0 & a_j > 1
    \end{array}
    \right.
\]
dove $a_j = j/(p+1)$, mentre per il kernel QS
\[
w_j = \frac{25}{12\pi^2 d_j^2} 
   \left(\frac{\sin{m_j}}{m_j} - \cos{m_j} \right)
\]
dove $d_j = j/p$ e $m_j = 6\pi d_i/5$.  

La figura~\ref{fig:kernels} mostra i pesi generati da questi kernel per
$p=4$ e $j$ che va da 1 a 9.

\begin{figure}[htbp]
\caption{Tre kernel per HAC}
\label{fig:kernels}
\centering
\includegraphics{figures/kernels}
\end{figure}

In \app{gretl} � possibile scegliere il kernel usando il comando \texttt{set}
col parametro \verb|hac_kernel|:
%
\begin{code}
set hac_kernel parzen
set hac_kernel qs
set hac_kernel bartlett
\end{code}

\subsection{Scelta della larghezza di banda HAC}
\label{sec:hac-bw}

La teoria asintotica sviluppata da Newey, West ed altri ci dice in termini
generali come la larghezza di banda HAC, $p$, deve crescere in relazione
all'ampiezza campionaria, $T$, ossia dice che $p$ dovrebbe crescere
proporzionalmente a qualche potenza frazionaria di $T$. Purtroppo questo non �
di molto aiuto quando nella pratica si ha a che fare con un dataset di ampiezza
fissa. Sono state suggerite varie regole pratiche, due delle quali sono
implementate da \app{gretl}. L'impostazione predefinita � $p = 0.75 T^{1/3}$,
come raccomandato da Stock e Watson (2003). Un'alternativa � $p =
4(T/100)^{2/9}$, come raccomandato in Wooldridge (2002b). In entrambi i casi si
prende la parte intera del risultato. Queste varianti sono chiamate
rispettivamente \texttt{nw1} e \texttt{nw2} nel contesto del comando \texttt{set} col parametro
\verb|hac_lag|. Ossia, � possibile impostare la versione data da
Wooldridge con il comando
%
\begin{code}
set hac_lag nw2
\end{code}
%
Come mostrato nella Tabella~\ref{tab:haclag} la scelta tra \texttt{nw1} e
\texttt{nw2} non causa rilevanti differenze.

\begin{table}[htbp]
  \centering
  \begin{tabular}{ccc}
    $T$ & $p$ (\texttt{nw1}) & $p$ (\texttt{nw2}) \\[4pt]
50& 	2& 	3 \\
100& 	3& 	4 \\
150& 	3& 	4 \\
200& 	4& 	4 \\
300& 	5& 	5 \\
400& 	5& 	5 \\
  \end{tabular}
\caption{Larghezza di banda HAC: confronto tra due regole pratiche}
\label{tab:haclag}
\end{table}

� anche possibile specificare un valore numerico fisso per $p$, come in
%
\begin{code}
set hac_lag 6
\end{code}
%
Inoltre � possibile impostare un valore diverso per il kernel QS (visto che
questo non deve essere necessariamente un valore intero).  Ad esempio:
%
\begin{code}
set qs_bandwidth 3.5
\end{code}


\subsection{Prewhitening e scelta della larghezza di banda basata sui dati}
\label{sec:hac-prewhiten}

Un approccio alternativo per trattare l'autocorrelazione dei residui consiste
nell'attaccare il problema da due fronti. L'intuizione alla base di questa
tecnica, nota come \emph{VAR prewhitening} (Andrews e Monahan, 1992) pu� essere
illustrata con un semplice esempio. Sia $x_t$ una serie di variabili casuali con
autocorrelazione del prim'ordine
%
\[
  x_t = \rho x_{t-1} + u_t
\]
%
Si pu� dimostrare che la varianza di lungo periodo di $x_t$ �
%
\[
  V_{LR}(x_t) = \frac{V_{LR}(u_t)}{(1-\rho)^2}
\]
%
Nella maggior parte dei casi, $u_t$ � meno autocorrelato di $x_t$,
quindi dovrebbe richiedere una minore larghezza di banda. La stima di
$V_{LR}(x_t)$ pu� quindi procedere in tre passi: (1) stimare $\rho$; (2)
ottenere una stima HAC di $\hat{u}_t = x_t - \hat{\rho} x_{t-1}$; (3)
dividere il risultato per $(1-\rho)^2$.

Applicare questo approccio al nostro problema implica stimare un'autoregressione
vettoriale (VAR) di ordine finito sulle variabili vettoriali
$\xi_t = X_t \hat{u}_t$. In generale, il VAR pu� essere di ordine qualsiasi, ma
nella maggior parte dei casi � sufficiente l'ordine 1; lo scopo non � quello di
produrre un modello preciso per $\xi_t$, ma solo quello di catturare la maggior parte
dell'autocorrelazione.  Quindi viene stimato il VAR seguente
%
\[
  \xi_t = A \xi_{t-1} + \varepsilon_t
\]
%
Una stima della matrice $X'\Omega X$ pu� essere ottenuta con
\[
  (I- \hat{A})^{-1} \hat{\Sigma}_{\varepsilon} (I- \hat{A}')^{-1}
\]
dove $\hat{\Sigma}_{\varepsilon}$ � uno stimatore HAC, applicato ai residui del
VAR.

In \app{gretl} � possibile usare il prewhitening con
%
\begin{code}
set hac_prewhiten on
\end{code}
%
Al momento non � possibile calcolare un VAR iniziale con un ordine diverso da 1.

Un ulteriore miglioramento di questo approccio consiste nello scegliere la
larghezza di banda in base ai dati. Intuitivamente, ha senso che la larghezza di
banda non tenga conto soltanto dell'ampiezza campionaria, ma anche delle
propriet� temporali dei dati (e anche del kernel scelto). Un metodo non
parametrico di scelta � stato proposto da Newey e West (1994) ed � spiegato
bene e in modo sintetico da Hall (2005). Questa opzione pu� essere abilitata in
gretl con il comando
%
\begin{code}
set hac_lag nw3
\end{code}
%
ed � abilitata in modo predefinito quando si seleziona il prewhitening, ma �
possibile modificarla utilizzando un valore numerico specifico per
\verb|hac_lag|.

Anche il metodo basato sui dati proposto da Newey--West non identifica univocamente
la larghezza di banda per una data ampiezza del campione. Il primo passo
consiste nel calcolare una serie di covarianze dei residui, e la lunghezza di
questa serie � una funzione dell'ampiezza campionaria, ma solo per un certo
multiplo scalare; ad esempio, � data da $O(T^{2/9})$ per il kernel di Bartlett.
\app{Gretl} usa un multiplo implicito pari a 1.


\section{Problemi speciali con dati panel}
\label{sec:vcv-panel}

Visto che i dati panel hanno sia caratteristiche di serie storiche sia
caratteristiche di dati cross-section, ci si pu� aspettare che in generale
la stima robusta della matrice di covarianza debba richiedere di gestire sia
l'eteroschedasticit� che l'autocorrelazione (l'approccio HAC). Inoltre ci sono
altre caratteristiche dei dati panel che richiedono attenzione particolare:
\begin{itemize}
\item La varianza del termine di errore pu� differire tra le unit�
  cross-section.
\item La covarianza degli errori tra le unit� pu� essere diversa da zero in ogni
  periodo temporale.
\item Se non si rimuove la variazione ``between'', gli errori possono esibire
  autocorrelazione, non nel senso classico delle serie storiche, ma nel senso
  che l'errore medio per l'unit� $i$ pu� essere diverso da quello per l'unit� $j$
  (questo � particolarmente rilevante quando il metodo di stima � pooled OLS).
\end{itemize}

\app{Gretl} al momento offre due stimatori robusti per la matrice di covarianza
da usare con dati panel, disponibili per modelli stimati con effetti fissi,
pooled OLS, e minimi quadrati a due stadi. Lo stimatore robusto predefinito �
quello suggerito da Arellano (2003), che � HAC a patto che il panel sia del tipo
``$n$ grande, $T$ piccolo'' (ossia si osservano molte unit� per pochi periodi).
Lo stimatore di Arellano �
\[
\hat{\Sigma}_{\rm A} = 
\left(X^{\prime}X\right)^{-1}
\left( \sum_{i=1}^n X_i^{\prime} \hat{u}_i 
    \hat{u}_i^{\prime} X_i \right)
\left(X^{\prime}X\right)^{-1}
\]
dove $X$ � la matrice dei regressori (con le medie di gruppo sottratte, nel caso
degli effetti fissi), $\hat{u}_i$ denota il vettore dei residui per l'unit� $i$,
e $n$ � il numero delle unit� cross-section. Cameron e Trivedi (2005) difendono
l'uso di questo stimatore, notando che il classico HCCME di White pu� produrre
errori standard artificialmente bassi in un contesto panel, perch� non tiene
conto dell'autocorrelazione.

Nei casi in cui l'autocorrelazione non � un problema, lo stimatore proposto da
Beck e Katz (1995) e discusso da Greene (2003, capitolo 13) pu� essere appropriato.
Questo stimatore, che tiene conto della correlazione contemporanea tra le unit�
e l'eteroschedasticit� per unit�, �
\[
\hat{\Sigma}_{\rm BK} = 
\left(X^{\prime}X\right)^{-1}
\left( \sum_{i=1}^n \sum_{j=1}^n \hat{\sigma}_{ij} X^{\prime}_iX_j \right)
\left(X^{\prime}X\right)^{-1}
\]
Le covarianze $\hat{\sigma}_{ij}$ sono stimate con
\[
\hat{\sigma}_{ij} = \frac{\hat{u}^{\prime}_i \hat{u}_j}{T}
\]
dove $T$ � la lunghezza della serie storica per ogni unit�. Beck e
Katz chiamano gli errori standard associati ``Panel-Corrected Standard
Errors'' (PCSE). Per usare questo stimatore in \app{gretl} basta eseguire
il comando
%
\begin{code}
set pcse on
\end{code}
%
Per reimpostare come predefinito lo stimatore di Arellano occorre eseguire
%
\begin{code}
set pcse off
\end{code}
%
Si noti che a prescindere dall'impostazione di \texttt{pcse}, lo stimatore
robusto non � usato a meno che non si aggiunga l'opzione \verb|--robust| ai
comandi di stima, o non si selezioni la casella ``Robusto'' nell'interfaccia
grafica.

%%% Local Variables: 
%%% mode: latex
%%% TeX-master: "gretl-guide"
%%% End: 

\chapter{Datos de Panel}
\label{chap-panel}


\section{Estructura de Panel}
\label{panel-structure}

Los datos de panel (una muestra combinada de datos de series
temporales y de secci�n cruzada) requieren un cuidado especial. He
aqu� algunas observaciones a tener en cuenta.

Consid�rese un conjunto de datos consistente en observaciones de
\emph{n} unidades de secci�n cruzada (pa�ses, provincias, personas,
etc.) durante \emph{T} periodos. Supongamos que cada observaci�n
contiene los valores de \emph{m} variables de inter�s. El conjunto de
datos est� formado entonces por \emph{mnT} valores.

Los datos deben de ordenarse ``por observaci�n'': cada fila representa
una observaci�n; cada columna contiene los valores de una variable en
particular. La matriz de datos tiene entonces \emph{nT} filas y
\emph{m} columnas. Esto deja abierta la cuesti�n de c�mo ordenar las
filas. Existen dos posibilidades.\footnote{Si no queremos diferenciar
  de manera conceptual o estad�stica entre variaciones muestrales y
  temporales, podemos ordenar las filas de modo arbitrario, pero esto
  es probablemente un derroche de datos.}

\begin{itemize}
\item Filas agrupadas por \emph{unidad}. Pi�nsese en la matriz de
  datos como si estuviera compuesta de \emph{n} bloques, cada uno con
  \emph{T} filas. El primer bloque de \emph{T} filas contiene las
  observaciones de la unidad 1 de la muestra para cada uno de los
  periodos; el siguiente bloque contiene las observaciones de la
  unidad 2 para todos los periodos; y as� sucesivamente. De hecho, la
  matriz de datos es un conjunto de datos de series temporales
  apilados verticalmente.
\item Filas agrupadas por \emph{periodo}. Pi�nsese en la matriz de
  datos como si estuviera compuesta por \emph{T} bloques, cada uno con
  \emph{n} filas. La primera de las \emph{n} filas contiene las
  observaciones de cada unidad muestral en el periodo 1; el siguiente
  bloque contiene las observaciones de todas las unidades en el
  periodo 2; y as� sucesivamente. La matriz de datos es un conjunto de
  datos de muestras de secci�n cruzada, apiladas verticalmente.
\end{itemize}

Puede utilizarse el esquema que resulte m�s conveniente. El primero es
quiz� m�s f�cil de mantener ordenado. Si se utiliza el segundo, hay
que asegurarse de que las unidades de secci�n cruzada aparezcan en el
mismo orden en cada uno de los bloques de datos de cada periodo.

En cualquiera de los dos casos se puede utilizar el campo frecuencia
en la l�nea \emph{observaciones} del archivo de cabecera de datos para
que el asunto resulte un poco m�s sencillo.

\begin{itemize}
\item \emph{Agrupados por unidades}: Establecer la frecuencia igual a
  \emph{T}. Supongamos que hay observaciones sobre 20 unidades durante
  5 periodos de tiempo. En este caso, la l�nea de observaciones m�s
  apropiada es la siguiente: \verb+5 1.1 20.5+ (l�ase: frecuencia 5,
  empezando con la observaci�n de la unidad 1, en el periodo 1, y
  finalizando con la observaci�n de la unidad 20, periodo 5).
  Entonces, por ejemplo, la observaci�n de la unidad 2 en el periodo 5
  puede ser referenciada como \verb+2.5+, y la correspondiente a la
  unidad 13 en periodo 1 como \verb+13.1+.
\item \emph{Agrupado por periodos}: Establecer la frecuencia igual a
  \emph{n}. En este caso, si hay observaciones sobre 20 unidades en
  cada uno de los 5 periodos, la l�nea de observaciones deber�a ser:
  \verb+20 1.01 5.20+ (l�ase: frecuencia 20, empezando con la
  observaci�n del periodo 1, unidad 01, y finalizando con la
  observaci�n del periodo 5, unidad 20). As�, nos referiremos a la
  observaci�n de la unidad 2, periodo 5 como \verb+5.02+.
\end{itemize}

Si se construye un conjunto de datos de panel utilizando un programa
de hoja de c�lculo para despu�s importar los datos a \app{gretl},
puede ser que el programa no reconozca, al principio, la clase
especial de los datos. Esto se puede arreglar mediante la instrucci�n
\cmd{setobs} (v�ase el \GCR) o la opci�n del men� GUI ``Muestra,
Seleccionar frecuencia, observaci�n inicial...)''.


\section{Variables ficticias}
\label{dummies}

En un estudio de panel puede que se desee construir variables
ficticias de uno o ambos tipos descritos a continuaci�n: (a) variables
ficticias como identificadores de las unidades muestrales, y (b)
variables ficticias como identificadores de los periodos de tiempo. El
primer m�todo puede utilizarse para permitir que el intercepto de la
regresi�n sea diferente en diferentes unidades, y el segundo para
permitir lo mismo en diferente periodos.

Hay dos opciones especiales para crear estas variables ficticias.  Se
encuentran dentro del men� ``Datos, A�adir variables'' en el GUI, o en
la instrucci�n \cmd{genr} en el modo lote de instrucciones, o
\app{gretlcli}.

\begin{enumerate}
\item ``variables ficticias peri�dicas'' (lote de instrucciones:
  \cmd{genr dummy}). Esta instrucci�n normalmente se utiliza para
  crear variables ficticias peri�dicas hasta la frecuencia de datos en
  los estudios de series temporales (por ejemplo un conjunto de
  variables ficticias trimestrales para ser utilizado en correcci�n
  estacional). No obstante, tambi�n funciona con datos de panel.
  N�tese que la interpretaci�n de las variables ficticias creadas
  mediante esta instrucci�n difiere dependiendo de si las filas de
  datos est�n agrupadas por unidad o por periodo. Si est�n agrupadas
  seg�n \emph{unidades} (frecuencia \emph{T}) las variables
  resultantes son \emph{variables ficticias peri�dicas} y habr� un
  n�mero \emph{T} de ellas. Por ejemplo, \verb+dummy_2+ tendr� el
  valor 1 en cada fila de datos correspondiente a una observaci�n del
  periodo 2, o 0 en caso contrario. Si est�n agrupadas seg�n
  \emph{periodos} (frecuencia \emph{n}) entonces se generaran \emph{n}
  \emph{variables ficticias unitarias}: \verb+dummy_2+ tendr� el valor
  1 en cada fila de datos asociada con la unidad muestral 2, o 0 en
  caso contrario.
\item ``Variables ficticias de panel'' (en modo consola \cmd{genr
    paneldum}). Esta instruccion crea todas las variables ficticias,
  de cada unidad y periodo, de golpe. Se supone que por defecto, las
  filas de datos est�n agrupadas por unidades. Las variables ficticias
  de cada unidad se denominan \verb+du_1+, \verb+du_2+ y as�
  sucesivamente, mientras que las variables ficticias peri�dicas se
  llaman \verb+dt_1+, \verb+dt_2+, etc. Es incorrecto utilizar la
  \verb+u+ (por unidad) y la \verb+t+ (por tiempo) en estos nombres si
  las filas de datos est�n agrupadas por periodos: su utilizaci�n
  correcta en este contexto se hace mediante \cmd{genr paneldum -o}
  (s�lo en modo lote de instrucciones).
\end{enumerate}

Si el conjunto de datos de panel contiene el a�o \verb+YEAR+ como una
de las variables, es posible crear un periodo ficticio para de escoger
alg�n a�o en particular como en este ejemplo \cmd{genr dum =
  (YEAR=1960)}. Tambi�n es posible crear variables ficticias
peri�dicas utilizando el operador de m�dulo, \verb+%+. Por
ejemplo, para crear una variable ficticia con valor 1 para la primera
observaci�n y cada treinta observaciones y 0 en lo dem�s casos, se
puede hacer lo siguiente

\begin{code}
  genr index genr dum = ((index-1)%30) = 0
\end{code}


\section{Uso de valores retardados con datos de panel}
\label{panel-lagged}

Si los periodos de tiempo est�n divididos en intervalos regulares,
quiz� queramos usar los valores retardados de las variables en una
regresi�n de panel. En este caso es preferible agrupar las filas de
datos por \emph{unidades} (series temporales apiladas).

Supongamos que creamos un retardo de la variable \verb+x1+, utilizando
\verb+genr x1_1 = x1(-1)+.  Los valores de esta variable ser�n en
general correctos, pero en los l�mites de los bloques de datos de cada
unidad son `` utilizables'': el valor ``previo'' no es realmente el
primer retardo de \verb+x1_1+, si no m�s bien la �ltima observaci�n de
\verb+x1+ para la unidad muestral previa.  \app{Gretl} marca estos
valores como ausentes.

Si hay que incluir un retardo de este tipo en una regresi�n, hay que
asegurarse de que la primera observaci�n de cada bloque o unidad no
est� incluida. Un modo de hacer esto es mediante M�nimos Cuadrados
Ponderados (\cmd{wls}) utilizando una variable ficticia apropiada como
ponderaci�n. Esta variable ficticia (vamos a denominarla \cmd{lagdum})
debe tener el valor 0 para las observaciones a descartar, y 1 en el
caso contrario. Es decir, es complementaria a una variable para el
periodo 1. De este modo, si hemos utilizado la instrucci�n\cmd{ genr
  dummy} podemos teclear \verb+genr lagdum = 1 - dummy_1+.  En caso de
que hubi�ramos utilizado \cmd{genr paneldum} ahora tendr�amos que
teclear \verb+genr lagdum = 1 - dt_1+.  De cualquier manera, la
siguiente instrucci�n ser�a

\verb+wls lagdum y const x1_1+ ...

para obtener una regresi�n combinada utilizando el primer retardo de
\verb+x1+, descartando todas las observaciones del periodo 1.

Otra opci�n es utilizar \cmd{smpl} con la marca \cmd{-o} y una
variable ficticia apropiada. El Ejemplo \ref{examp-pwt} muestra unas
instrucciones de ejemplo, suponiendo que cada bloque de datos de cada
unidad contiene 30 observaciones y queremos descartar la primera fila
de cada uno. Podemos entonces ejecutar las regresiones sobre el
conjunto de datos restringido sin tener que usar la instrucci�n
\cmd{wls}. Si se desea reutilizar el conjunto de datos restringido,
podemos guardarlo mediante la instrucci�n \cmd{store} (v�ase el \GCR).

\begin{script}[htbp]
  \caption{Retardos con datos de panel}
  \label{examp-panel-lags}
\begin{code}
  # crear la variable �ndice
  genr index 
  # crear dum = 0 para cada 30 observaciones 
  genr dum = ((index-1)%30) > 0
  # establecer la muestra por medio de esa variable ficticia
  smpl dum --dummy
  # crear de nuevo la estructura de observaciones, para 56 unidades
  setobs 29 1.01 56.29
\end{code}
\end{script}


\section{Estimaci�n combinada}
\label{pooled-est}

Llegados a este punto, podemos revelar que hay una instrucci�n de
estimaci�n con el prop�sito especial de ser utilizado con datos de
panel, la opci�n ``MCO combinados'' en el men� \textsf{Modelo}. Esta
instrucci�n s�lo est� disponible cuando se reconoce el conjunto de
datos como un panel. Para aprovechar esta opci�n, es preciso
especificar un modelo que no contenga ninguna variable ficticia para
representar unidades de secci�n cruzada. La rutina presenta
estimaciones sencillas de MCO combinadas, que tratan de igual manera
las variaciones de secci�n cruzada y de series temporales. Este modelo
puede que sea el apropiado o no. En el men� \textsf{Contrastes} en la
ventana de modelo, se encuentra una opci�n llamada ``Diagn�sticos de
panel'', la cual plantea el contraste de MCO combinados contra las
principales alternativas, es decir, los modelos de efectos fijos o de
efectos aleatorios.

El modelo de efectos fijos a�ade una variable ficticia a todas menos
una de las unidades de secci�n cruzada, permitiendo que var�e el
intercepto de la regresi�n en cada unidad. Se presenta un contraste
\emph{F} para la significaci�n conjunta de estas variables ficticias:
si el valor p para este contraste es peque�o, entonces se rechaza la
hip�tesis nula (de que un simple modelo combinado es adecuado) en
favor de un modelo de efectos fijos.

Por otro lado, el modelo de efectos aleatorios descompone la varianza
residual en dos partes, una parte espec�fica a la unidad de secci�n
cruzada o ``grupo'' y la otra espec�fica a una observaci�n en
particular. (Este estimador s�lo puede calcularse cuando el panel es
lo suficientemente ``amplio'', es decir, cuando el n�mero de unidades
de secci�n cruzada en el conjunto de datos excede el n�mero de
par�metros a estimar.) El contraste LM de Breusch-Pagan comprueba la
hip�tesis nula (una vez m�s, de que el estimador de MCO combinados es
adecuado) contra la alternativa de efectos aleatorios.

Cabe dentro de lo posible que el modelo MCO combinados sea rechazado
contra las dos alternativas de efectos fijos y aleatorios. Entonces la
pregunta es, �c�mo podemos valorar los m�ritos relativos de los
estimadores alternativos? El contraste de Hausman (tambi�n incluido en
el informe, siempre que el modelo de efectos aleatorios se pueda
estimar) intenta resolver este problema. El estimador de efectos
aleatorios es m�s eficiente que el estimador de efectos fijos, siempre
y cuando el error especifico a la unidad o grupo no est�
correlacionado con las variables independientes; si no es as�, el
estimador de efectos aleatorios es inconsistente, en cuyo caso es
preferible el estimador de efectos fijos. La hip�tesis nula para el
contraste de Hausman dice que el error especifico al grupo no esta tan
correlacionado (y por lo tanto es preferible el modelo de efectos
aleatorios). Por lo tanto, un valor p peque�o para este contraste
supone rechazar el modelo de efectos aleatorios en favor del modelo de
efectos fijos.

Para una discusi�n m�s rigurosa sobre este tema, v�ase Greene (2000),
cap�tulo 14.


\section{Ilustraci�n: La Tabla Mundial de Penn}
\label{PWT}

La Tabla Mundial de Penn (Penn World Table) (direcci�n
\href{http://pwt.econ.upenn.edu/}{pwt.econ.upenn.edu}) es un excelente
conjunto de datos macroecon�micos de panel, que incluye datos sobre
152 pa�ses entre los a�os 1950-1992. Los datos est�n disponibles en
formato \app{gretl}; v�ase el sitio web de datos de \app{gretl}
\url{http://gretl.sourceforge.net/gretl_data.html} (se puede descargar
gratuitamente, aunque no est� incluido en el paquete principal de
\app{gretl}).

El Ejemplo \ref{examp-pwt} de abajo abre \verb+pwt56_60_89.gdt+, un
conjunto parcial de la pwt que contiene datos sobre 120 pa�ses, entre
los a�os 1960-89, para 20 variables, sin que haya ninguna observaci�n
ausente (el conjunto de datos completo, que tambi�n est� incluido en
el paquete pwt para \app{gretl}, contiene muchas observaciones con
valores ausentes). El total de crecimiento del PIB real, entre
1960-89, se calcula para cada pa�s y se regresa contra el nivel real
del PIB en 1960, para ver si hay indicios de ``convergencia'' (es
decir, crecimiento m�s r�pido en los pa�ses que empezaron con el nivel
m�s bajo).

\begin{script}[htbp]
  \caption{Uso de la tabla mundial de Penn}
  \label{examp-pwt}
\begin{code}
  open pwt56_60_89.gdt 
  # para 1989 (�ltima observaci�n), el retardo 29 da 1960, 
  # la primera observaci�n 
  genr gdp60 = RGDPL(-29) 
  # encontrar el crecimiento total del PNB total durante 30 a�os
  genr gdpgro = (RGDPL - gdp60)/gdp60 
  # restringir la muestra a la secci�n cruzada de a�o 1989 
  smpl -r YEAR=1989 
  # �Hay convergencia?  �los pa�ses con una base menor, 
  # crecieron mas r�pido?   
  ols gdpgro const gdp60 
  # resultado: �No! Intentar la relaci�n inversa 
  genr gdp60inv = 1/gdp60 
  ols gdpgro const gdp60inv 
  # No otra vez.  �Intentar prescindir de Africa? 
  genr afdum = (CCODE = 1) genr
  afslope = afdum * gdp60 
  ols gdpgro const afdum gdp60 afslope
\end{code}
\end{script}

\chapter{Minimi quadrati non lineari}
\label{chap-nls}


\section{Introduzione ed esempi}
\label{nls-intro}

\app{gretl} supporta i minimi quadrati non
lineari (NLS - nonlinear least squares), usando una variante
dell'algoritmo Levenberg--Marquandt.  L'utente deve fornire la
specificazione della funzione di regressione, e, prima ancora di fare
questo, occorre ``dichiarare'' i parametri da stimare e fornire dei
valori iniziali. � anche possibile indicare analiticamente delle
derivate della funzione di regressione rispetto a ognuno dei
parametri. La tolleranza usata per fermare la procedura iterativa di
stima pu� essere impostata con il comando \cmd{set}.

La sintassi per la specificazione della funzione da stimare � la
stessa usata per il comando \cmd{genr}. Ecco due esempi, che includono
anche le derivate.

\begin{script}[htbp]
  \caption{Funzione di consumo da Greene}
  \label{nls-cons}
\begin{scode}
nls C = alfa + beta * Y^gamma
deriv alfa = 1
deriv beta = Y^gamma
deriv gamma = beta * Y^gamma * log(Y)
end nls
\end{scode}
\end{script}

\begin{script}[htbp]
  \caption{Funzione non lineare da Russell Davidson}
  \label{nls-ects}
\begin{scode}
nls y = alfa + beta * x1 + (1/beta) * x2
deriv alfa = 1
deriv beta = x1 - x2/(beta*beta)
end nls
\end{scode}
\end{script}

Si notino i comandi \cmd{nls} (che indica la funzione di regressione),
\cmd{deriv} (che indica la specificazione di una derivata) e \cmd{end
  nls}, che conclude la specificazione e avvia la stima. Se si
aggiunge l'opzione \cmd{--vcv} all'ultima riga, verr� mostrata la
matrice di covarianza delle stime dei parametri.

\section{Inizializzazione dei parametri}
\label{nls-param}

Prima di eseguire il comando \cmd{nls}, occorre definire dei valori iniziali per
i parametri della funzione di regressione. Per farlo, �
possibile usare il comando \cmd{genr} (o, nella versione grafica del
programma, il comando ``Variabile, Definisci nuova variabile'').  

In alcuni casi, in cui la funzione non lineare � una generalizzazione
di un modello lineare (o di una sua forma ristretta), pu� essere
conveniente eseguire un comando \cmd{ols} e inizializzare i parametri
a partire dalle stime OLS dei coefficienti. In relazione al primo
degli esempi visti sopra, si potrebbe usare:

\begin{code}
ols C 0 Y
genr alfa = $coeff(0)
genr beta = $coeff(Y)
genr gamma = 1
\end{code}

E in relazione al secondo esempio si userebbe:

\begin{code}
ols y 0 x1 x2
genr alfa = $coeff(0)
genr beta = $coeff(x1)
\end{code}

\section{Finestra di dialogo NLS}
\label{nls-gui}

Probabilmente il modo pi� comodo di formulare i comandi per una stima
NLS consiste nell'usare uno script per \app{gretl}, ma � possibile
anche procedere interattivamente, selezionando il comando
``Minimi quadrati non lineari'' dal men� ``Modello, Modelli non lineari''. In questo modo,
si aprir� una finestra di dialogo in cui � possibile scrivere la
specificazione della funzione (opzionalmente preceduta da linee
\cmd{genr} per impostare i valori iniziali dei parametri) e le
derivate, se sono disponibili. Un esempio � mostrato nella
figura~\ref{fig-nls-dialog}.  Si noti che in questo contesto non
occorre scrivere i comandi \cmd{nls} e \cmd{end nls}.

\begin{figure}[htbp]
  \begin{center}
    \includegraphics[scale=0.75]{figures/nls_window}
  \end{center}
  \caption{Finestra di dialogo NLS}
  \label{fig-nls-dialog}
\end{figure}


\section{Derivate analitiche e numeriche}
\label{nls-deriv}

Se si � in grado di calcolare le derivate dalla funzione di
regressione rispetto ai parametri, � consigliabile indicarle come
mostrato negli esempi precedenti. Se ci� non � possibile, \app{gretl}
calcoler� delle derivate approssimate numericamente. Le propriet�
dell'algoritmo NLS in questo caso potrebbero non essere ottimali (si
veda la sezione~\ref{nls-accuracy}).

Se vengono fornite delle derivate analitiche, ne viene controllata la
coerenza con la funzione non lineare data. Se esse sono chiaramente
scorrette, la stima viene annullata con un messaggio di errore. Se le
derivate sono ``sospette'', viene mostrato un messaggio di
avvertimento, ma la stima prosegue.  Questo avvertimento pu� essere
causato da derivate scorrette, ma pu� anche essere dovuto a un alto
grado di collinearit� tra le derivate.

Si noti che non � possibile mischiare derivate numeriche e analitiche:
se si indicano espressioni analitiche per una derivata, occorre farlo
per tutte.

\section{Arresto della procedura}
\label{nls-toler}

La procedura di stima NLS � iterativa: l'iterazione viene arrestata quando
si verifica una qualunque delle seguenti condizioni: viene raggiunto il
criterio di convergenza, o si supera il massimo numero di iterazioni
impostato.

Se $k$ denota il numero di parametri da stimare, il numero massimo di iterazioni
� $100 \times (k+1)$ quando si indicano le derivate analitiche, mentre � $200
\times (k+1)$ quando si usano le derivate numeriche.

Preso un numero $\epsilon$ piccolo a piacere, la convergenza si ritiene
raggiunta se � soddisfatta almeno una delle due condizioni seguenti:

\begin{itemize}
\item Entrambe le riduzioni, effettiva o prevista, della somma dei quadrati
  degli errori sono minori di $\epsilon$.
\item L'errore relativo tra due iterazioni consecutive � minore di
  $\epsilon$.
\end{itemize}

Il valore predefinito per $\epsilon$ � pari alla precisione della macchina
elevata alla potenza 3/4\footnote{Su una macchina Intel Pentium a 32-bit, il
valore � pari a circa $1.82\times 10^{-12}$.}, ma pu� essere modificato usando
il comando \cmd{set} con il parametro \verb+nls_toler+. Ad esempio
%

\begin{code}
set nls_toler .0001
\end{code}
%
imposter� il valore di $\epsilon$ a 0.0001.

\section{Dettagli sul codice}
\label{nls-code}

Il motore sottostante la stima NLS � basato sulla suite di funzioni
\app{minpack}, disponibile su
\href{http://www.netlib.org/minpack/}{netlib.org}.  Nello specifico,
sono usate le seguenti funzioni \app{minpack}:

\begin{center}
  \begin{tabular}{ll}
    \verb+lmder+ & 
    Algoritmo Levenberg--Marquandt con derivate analitiche
    \\
    \verb+chkder+ & 
    Controllo delle derivate analitiche fornite
    \\
    \verb+lmdif+ & 
    Algoritmo Levenberg--Marquandt con derivate numeriche
    \\
    \verb+fdjac2+ & 
    Calcolo del Jacobiano approssimato finale se si usano le derivate 
    numeriche
    \\
    \verb+dpmpar+ & 
    Determinazione della precisione della macchina
    \\
  \end{tabular}
\end{center}

In caso di successo nell'iterazione Levenberg--Marquandt, viene usata
una regressione Gauss--Newton per calcolare la matrice di covarianza
per le stime dei parametri. Se si usa l'opzione \verb|--robust|, viene
calcolata una variante robusta. La documentazione del comando \cmd{set}
spiega le opzioni disponibili a questo proposito.

Poich� i risultati NLS sono asintotici, si pu� discutere sulla necessit� di
applicare una correzione per i gradi di libert� nel calcolo dell'errore standard
della regressione (e di quello delle stime dei parametri). Per confrontabilit�
con OLS, e seguendo il ragionamento in Davidson e MacKinnon (1993), le stime
calcolate in \app{gretl} \emph{usano} una correzione per i gradi di libert�.

\section{Accuratezza numerica}
\label{nls-accuracy}

La tabella~\ref{tab-nls} mostra i risultati dell'uso della procedura
NLS di \app{gretl} sui 27 ``Statistical Reference Dataset'' forniti
dal National Institute of Standards and Technology (NIST)
statunitense, per il test del software di regressione non
lineare.\footnote{Per una discussione dell'accuratezza di \app{gretl}
  nella stima di modelli lineari, si veda
  l'appendice~\ref{app-accuracy}.} Per ogni dataset, il file di test
indicano due valori iniziali per i parametri, quindi il test completo
riporta 54 stime. Sono stati eseguiti due test completi, uno usando
derivate analitiche e uno usando approssimazioni numeriche; in
entrambi i casi si � usata la tolleranza predefinita.\footnote{I dati
  mostrati nella tabella derivano dalla versione 1.0.9 di \app{gretl},
  compilato con \app{gcc} 3.3, collegato a \app{glibc} 2.3.2 ed
  eseguito in Linux su un PC i686 (IBM ThinkPad A21m).}

Sulle 54 stime, \app{gretl} non riesce a produrre una soluzione in 4
casi, se vengono usate le derivate analitiche, e in 5 casi se vengono
usate le approssimazioni numeriche. Dei quattro fallimenti in modalit�
derivate analitiche, due sono dovuti alla non convergenza
dell'algoritmo Levenberg--Marquandt dopo il numero massimo di
iterazioni (su \verb+MGH09+ e \verb+Bennett5+, entrambi descritti dal
NIST come di ``alta difficolt�'') e due sono dovuti ad errori di
intervallo (valori in virgola mobile al di fuori dei limiti) occorsi
durante il calcolo del Jacobiano (su \verb+BoxBOD+ e \verb+MGH17+,
rispettivamente descritti come di ``alta difficolt�'' e di ``media
difficolt�''). In modalit� approssimazione numerica, l'ulteriore caso
di fallimento � rappresentato da \verb+MGH10+ (``alta difficolt�'',
massimo numero di iterazioni raggiunto).

La tabella mostra informazioni su vari aspetti dei test: numero di
fallimenti, numero medio di iterazioni richieste per produrre una
soluzione, e due tipi di misura dell'accuratezza dei risultati per i
parametri e per i loro errori standard.

Per ognuno dei 54 test eseguiti in ogni modalit�, se � stata prodotta
una soluzione, sono state confrontate le stime dei parametri ottenute da
\app{gretl} con i valori certificati dal NIST. � stata definita la
variabile ``numero minimo di cifre corrette'' per una data stima come il
numero di cifre significative per cui la \emph{meno accurata} delle
stime di \app{gretl} coincide con il valore certificato. La tabella
mostra i valori medio e minimo di questa variabile, calcolati sulle
stime che hanno prodotto una soluzione; la stessa informazione � fornita
per gli errori standard stimati.\footnote{Per gli errori standard, dalle
  statistiche mostrate nella tabella, � stato escluso l'outlier
  costituito da \verb+Lanczos1+, che rappresenta un caso strano,
  composto da dati generati con un adattamento quasi esatto; gli
  errori standard sono di 9 o 10 ordini di grandezza pi� piccoli dei
  coefficienti. In questo caso \app{gretl} riesce a riprodurre gli
  errori standard solo per 3 cifre (con derivate analitiche) e per 2
  cifre (con derivate numeriche).}  

La seconda misura di accuratezza mostrata � la percentuale di casi,
tenendo conto di tutti i parametri di tutte le stime giunte a buon
fine, in cui la stima di \app{gretl} concorda con il valore
certificato per almeno 6 cifre significative, che sono mostrate in
modo predefinito nei risultati delle regressioni di \app{gretl}.

\begin{table}[htbp]
  \caption{Regressione non lineare: i test NIST}
  \label{tab-nls}
  \begin{center}
    \begin{tabular}{lcc}
      � & \textit{Derivate analitiche} 
        & \textit{Derivate numeriche} \\ [4pt]
        Fallimenti in 54 test & 4 & 5\\
        Iterazioni medie & 32 & 127\\
        Media del "numero minimo di cifre corrette", & 8.120 & 6.980\\
        stima dei parametri \\
        Valore minimo del "numero minimo di cifre corrette", & 4 & 3 \\
        stima dei parametri \\
        Media del "numero minimo di cifre corrette", & 8.000 & 5.673\\
        stima degli errori standard \\
        Valore minimo del "numero minimo di cifre corrette", & 5 & 2\\
        stima degli errori standard \\
        Percentuale delle stime corrette a 6 cifre, & 96.5 & 91.9\\
        stima dei parametri \\
        Percentuale delle stime corrette a 6 cifre, & 97.7 & 77.3\\
        stima degli errori standard \\
      \end{tabular}
    \end{center}
  \end{table}

  Usando derivate analitiche, i valori dei casi peggiori sia per le
  stime dei parametri che per gli errori standard sono stati
  migliorati a 6 cifre corrette restringendo il valore di tolleranza a
  1.0e$-$14.  Usando derivate numeriche, la stessa modifica del limite
  di tolleranza ha innalzato la precisione dei valori peggiori a 5
  cifre corrette per i parametri e 3 cifre per gli errori standard, al
  costo di un fallimento in pi� nella convergenza.  

  Si noti la tendenziale superiorit� delle derivate analitiche: in
  media le soluzioni ai problemi dei test sono state ottenute con
  molte meno iterazioni e i risultati sono pi� accurati (in modo
  evidente per gli errori standard stimati). Si noti anche che i
  risultati a 6 cifre mostrati da \app{gretl} non sono affidabili al
  100 per cento per i problemi non lineari difficili (in particolare
  se si usano derivate numeriche). Tenendo presente questi limiti, la
  percentuale dei casi in cui i risultati sono accurati alla sesta
  cifra o pi� sembra sufficiente per giustificarne l'utilizzo in
  questa forma.

%%% Local Variables: 
%%% mode: latex
%%% TeX-master: "gretl-guide-it"
%%% End: 


\chapter{Stima di massima verosimiglianza}

\newcommand{\LogLik}{\ensuremath\ell}
\newcommand{\stackunder}[2]{\ensuremath\mathrel{\mathop{#2}\limits_{#1}}}
\newcommand{\pder}[2]{\frac{\ensuremath\partial #1}{\partial #2}}

\section{Stima di massima verosimiglianza con gretl}

La stima di massima verosimiglianza � una pietra angolare delle procedure
moderne di inferenza. Gretl fornisce un modo per implementare questa metodologia
per un grande numero di problemi di stima, usando il comando \texttt{mle}.
Seguono alcuni esempi.

\subsection{Introduzione}
\label{sec:background}
Per illustrare gli esempi seguenti, inizieremo con un breve ripasso degli
aspetti basilari della stima di massima verosimiglianza. Dato un campione di
ampiezza $T$, � possibile definire la funzione di densit�\footnote{Stiamo
  supponendo che i nostri dati siano una realizzazione di variabili casuali
  continue. Per variabili discrete, la trattazione rimane valida, riferendosi
  alla funzione di probabilit� invece che a quella di densit�. In entrambi i
  casi, la distribuzione pu� essere condizionale su alcune variabili esogene.}
per l'intero campione, ossia la distribuzione congiunta di tutte le
osservazioni $f(\mathbf{Y} ; \theta)$, dove $\mathbf{Y} =
\left\{ y_1, \ldots, y_T \right\}$.  La sua forma � determinata da un vettore di
parametri sconosciuti $\theta$, che assumiamo contenuti in un insieme $\Theta$,
e che possono essere usati per stimare la probabilit� di osservare un campione
con qualsiasi data caratteristica.

Dopo aver osservato i dati, i valori di $\mathbf{Y}$ sono noti, e questa
funzione pu� essere valutata per tutti i possibili valori di $\theta$. Quando
usiamo $y_t$ come argomento e $\theta$ come parametro, la funzione �
interpretabile come una densit�, mentre � preferibile chiamarla funzione di
\emph{verosimiglianza} quando $\theta$ � considerato argomento della funzione
e i valori dei dati $\mathbf{Y}$ hanno il solo compito di determinarne la forma.

Nei casi pi� comuni, questa funzione possiede un massimo unico, la cui posizione
non viene alterata dal fatto di considerare il logaritmo della verosimiglianza
(ossia la log-verosimiglianza): questa funzione si esprime come
\[
  \LogLik(\theta) = \log  f(\mathbf{Y}; \theta) .
\] 
Le funzionidi log-verosimiglianza gestite da gretl sono quelle in cui
$\LogLik(\theta)$ pu� essere scritto come
\[
  \LogLik(\theta) = \sum_{t=1}^T \ell_t(\theta) ,
\] 
che � vero nella maggior parte dei casi di interesse. Le funzioni
$\ell_t(\theta)$ vengono chiamati contributi di log-verosimiglianza.

Inoltre, la posizione del massimo � ovviamente determinata dai dati
$\mathbf{Y}$. Ci� significa che il valore
\begin{equation}
  \label{eq:maxlik}
  \hat{\theta}(\mathbf{Y}) = \stackunder{\theta \in \Theta}{\mathrm{Argmax}} \LogLik(\theta)
\end{equation}
� una qualche funzione dei dati osservati (ossia una statistica), che ha la
propriet�, sotto alcune condizioni deboli, di essere uno stimatore consistente,
asintoticamente normale e asintoticamente efficiente, di $\theta$.

In alcuni casi � possibile scrivere esplicitamente la funzione
$\hat{\theta}(\mathbf{Y})$, ma in generale ci� non � sempre vero, e il massimo
va cercato con tecniche numeriche. Queste si basano spesso sul fatto che la
log-verosimiglianza � una funzione continuamente differenziabile di $\theta$, e
quindi nel massimo le sue derivate parziali devono essere tutte pari a 0.
Il gradiente della log-verosimiglianza � chiamato il vettore degli \emph{score},
ed � una funzione con molte propriet� interessanti dal punto di vista
statistico, che verr� denotata con $\mathbf{s}(\theta)$.

I metodi basati sul gradiente possono essere illustrati brevemente:

\begin{enumerate}
\item scegliere un punto $\theta_0 \in \Theta$ ;
\item valutare $\mathbf{s}(\theta_0)$;
\item se $\mathbf{s}(\theta_0)$ � ``piccolo'', fermarsi; altrimenti calcolare
  un vettore di direzione $d(\mathbf{s}(\theta_0))$;
\item valutare $\theta_1 = \theta_0 + d(\mathbf{s}(\theta_0))$ ;
\item sostituire $\theta_0$ con $\theta_1$ ;
\item ricominciare dal punto 2.
\end{enumerate}

Esistono molti algoritmi di questo tipo; si differenziano nel modo con cui
calcolano il vettore di direzione
$d(\mathbf{s}(\theta_0))$, per assicurarsi che sia $\LogLik(\theta_1) >
\LogLik(\theta_0)$ (in modo che prima o poi si arrivi a un massimo).

Il metodo usato da \app{gretl} per massimizzare la log-verosimiglianza � un algoritmo
basato sul gradiente, noto come metodo di \textbf{BFGS} (Broyden,
Fletcher, Goldfarb e Shanno). Questa tecnica � usata in molti pacchetti
statistici ed econometrici, visto che � ritenuta valida e molto potente.
Ovviamente, per rendere operativa questa tecnica, deve essere possibile
calcolare il vettore $\mathbf{s}(\theta)$ per ogni valore di $\theta$.  In
alcuni casi, la funzione $\mathbf{s}(\theta)$ pu� essere vista esplicitamente in
termini di $\mathbf{Y}$. Talvolta questo non � possibile, o � troppo difficile,
quindi la funzione $\mathbf{s}(\theta)$ � valutata numericamente.

La scelta del valore iniziale $\theta_0$ � cruciale in alcuni contesti e
ininfluente in altri. In generale � consigliabile far partire l'algoritmo da
valori ``sensibili'', quando � possibile. Se � disponibile uno stimatore
consistente, di solito � una scelta valida ed efficiente: ci si assicura che per
grandi campioni il punto di partenza sar� probabilmente vicino a $\hat{\theta}$
e la convergenza sar� raggiunta in poche iterazioni.

\section{Stima di una Gamma}
\label{sec:gamma}

Si supponga di avere un campione di $T$ osservazioni indipendenti e
identicamente distribuite da una distribuzione Gamma. La funzione di densit� per
ogni osservazione $x_t$ �
\begin{equation}
  \label{eq:gammadens}
  f(x_t) = \frac{\alpha^p}{\Gamma(p)} x_t^{p-1} \exp\left({-\alpha
      x_t}\right) .
\end{equation}
La log-verosimiglianza per l'intero campione pu� essere scritta come il
logaritmo della densit� congiunta di tutte le osservazioni. Visto che queste
sono indipendenti e identiche, la densit� congiunta � il prodotto delle densit�
individuali, e quindi il suo logaritmo �
\begin{equation}
  \label{eq:gammaloglik}
  \LogLik(\alpha, p) = \sum_{t=1}^T \log \left[ \frac{\alpha^p}{\Gamma(p)} x_t^{p-1} \exp\left({-\alpha
      x_t}\right) \right] = 
      \sum_{t=1}^T \ell_t ,
\end{equation}
dove
\[
  \ell_t = p \cdot \log (\alpha x_t) - \gamma(p) - \log x_t - \alpha x_t
\]
e $\gamma(\cdot)$ � il logaritmo della funzione gamma.
Per stimare i parametri $\alpha$ e $p$ con la massima verosimiglianza, occorre
massimizzare (\ref{eq:gammaloglik}) rispetto ad essi. Il frammento di codice da
eseguire in \app{gretl} �

\begin{code}
    scalar alpha = 1
    scalar p = 1

    mle logl =  p*ln(alpha * x) - lngamma(p) - ln(x) - alpha * x 
    end mle 
\end{code}

I due comandi

\begin{code}
    alpha = 1
    p = 1
\end{code}

sono necessari per assicurarsi che le variabili \texttt{p} e \texttt{alpha}
esistano prima di tentare il calcolo di \texttt{logl}. Il loro valore sar�
modificato dall'esecuzione del comando \texttt{mle} e sar� sostituito dalle
stime di massima verosimiglianza se la procedura � andata a buon fine. Il valore
iniziale � 1 per entrambi; � arbitrario e non conta molto in questo esempio (ma
si veda oltre).

Il codice visto pu� essere reso pi� leggibile, e leggermente pi� efficiente,
definendo una variabile in cui memorizzare $\alpha \cdot x_t$. Questo comando
pu� essere inserito nel blocco \texttt{mle} nel modo seguente:
\begin{code}
    scalar alpha = 1
    scalar p = 1

    mle logl =  p*ln(ax) - lngamma(p) - ln(x) - ax 
    series ax = alpha*x
    params alpha p
    end mle 
\end{code}
In questo caso, � necessario includere la riga \texttt{params alpha
  p} per impostare i simboli \texttt{p} e \texttt{alpha} separatamente da
\texttt{ax}, che � una variabile generata temporaneamente, e non un parametro da
stimare.

In un semplice esempio come questo, la scelta dei valori iniziali � quasi
ininfluente, visto che l'algoritmo converger� a prescindere dai valori iniziali.
In ogni caso, stimatori consistenti basati sul metodo dei momenti
per $p$ e $\alpha$ possono essere ricavati dalla media campionaria
$m$ e dalla varianza $V$: visto che si pu� dimostrare che
\[
  E(x_t) = p/\alpha \qquad  V(x_t) = p/\alpha^2 ,
\]
segue che gli stimatori seguenti
\begin{eqnarray*}
  \bar{\alpha} & = &  m/V \\
  \bar{p} & = & m \cdot \bar{\alpha} 
\end{eqnarray*}
sono consistenti, e quindi pi� appropriati da usare come punti di partenza per
l'algoritmo.
Lo script per \app{gretl} diventa quindi
\begin{code}
    scalar m = mean(x)
    scalar alpha = var(x)/m
    scalar p = m*alpha

    mle logl =  p*ln(ax) - lngamma(p) - ln(x) - ax 
    series ax = alpha*x
    params alpha p
    end mle 
\end{code}

\section{Stochastic frontier cost function}
\label{sec:frontier}

When modelling a cost function, it is sometimes worthwhile to
incorporate explicitly into the statistical model the notion that
firms may be inefficient, so that the observed cost deviates from the
theoretical figure not only because of unobserved heterogeneity
between firms, but also because two firms could be operating at a
different efficiency level, despite being identical under all other
respects. In this case we may write
\[
  C_i = C^*_i + u_i + v_i ,
\]
where $C_i$ is some variable cost indicator, $C_i^*$ is its
``theoretical'' value, $u_i$ is a zero-mean disturbance term and
$v_i$ is the inefficiency term, which is supposed to be nonnegative
by its very nature.

A linear specification for $C_i^*$ is often chosen. For example, the
Cobb-Douglas cost function arises when $C_i^*$ is a linear function of
the logarithms of the input prices and the output quantities.

The \emph{stochastic frontier} model is a linear model of the form
$y_i = x_i \beta + \varepsilon_i$ in which the error term
$\varepsilon_i$ is the sum of $u_i$ and $v_i$.  A common postulate is
that $u_i \sim N(0,\sigma_u^2)$ and $v_i \sim
\left|N(0,\sigma_v^2)\right|$. If independence between $u_i$ and $v_i$
is also assumed, then it is possible to show that the density function
of $\varepsilon_i$ has the form:
\begin{equation}
  \label{eq:frontdens}
  f(\varepsilon_i) = 
   \sqrt{\frac{2}{\pi}} 
   \Phi\left(\frac{\lambda \varepsilon_i}{\sigma}\right)
   \frac{1}{\sigma} \phi\left(\frac{\varepsilon_i}{\sigma}\right) ,
\end{equation}
where $\Phi(\cdot)$ and $\phi(\cdot)$ are, respectively, the distribution and density
function of the standard normal, $\sigma =
\sqrt{\sigma^2_u + \sigma^2_v}$ and $\lambda = \frac{\sigma_u}{\sigma_v}$.

As a consequence, the log-verosimiglianza for one observation takes the
form (apart form an irrelevant constant)
\[
  \ell_t = 
  \log\Phi\left(\frac{\lambda \varepsilon_i}{\sigma}\right) -
  \left[ \log(\sigma) + \frac{\varepsilon_i^2}{2 \sigma^2} \right];
\]
therefore, a Cobb-Douglas cost function with stochastic frontier is the
model described by the following equations: 
\begin{eqnarray*}
  \log C_i & = & \log C^*_i + \varepsilon_i \\
  \log C^*_i & = & c + \sum_{j=1}^m \beta_j \log y_{ij} + \sum_{j=1}^n \alpha_j \log p_{ij} \\
  \varepsilon_i & = & u_i + v_i \\
  u_i & \sim & N(0,\sigma_u^2) \\
  v_i & \sim & \left|N(0,\sigma_v^2)\right|  .
\end{eqnarray*}

In most cases, one wants to ensure that the homogeneity of the cost
function with respect to the prices holds by construction. Since this
requirement is equivalent to $\sum_{j=1}^n \alpha_j = 1$, the above
equation for $C^*_i$ can be rewritten as

\begin{equation}
  \label{eq:CobbDouglasFrontier}
  \log C_i - \log p_{in}  = c + \sum_{j=1}^m \beta_j \log y_{ij} +
  \sum_{j=2}^n \alpha_j (\log p_{ij} - \log p_{in})  + \varepsilon_i.
\end{equation}

The above equation could be estimated by OLS, but it would suffer from
two drawbacks: first, the OLS estimator for the intercept $c$ is
inconsistent because the disturbance term has a non-zero expected
value; second, the OLS estimators for the other parameters are
consistent, but inefficient in view of the non-normality of
$\varepsilon_i$. Both issues can be addressed by estimating
(\ref{eq:CobbDouglasFrontier}) by maximum likelihood. Nevertheless,
OLS estimation is a quick and convenient way to provide starting
values for the MLE algorithm.

The following gretl script code shows how to implement the model
described so far. The \texttt{banks91} file contains part of the data
used in Lucchetti, Papi and Zazzaro (2001).

\begin{code}
open banks91

# Cobb-Douglas cost function

ols cost const y p1 p2 p3

# Cobb-Douglas cost function with homogeneity restrictions

genr rcost = cost - p3
genr rp1 = p1 - p3
genr rp2 = p2 - p3

ols rcost const y rp1 rp2

# Cobb-Douglas cost function with homogeneity restrictions 
# and inefficiency 

scalar b0 = coeff(const)
scalar b1 = coeff(y)
scalar b2 = coeff(rp1)
scalar b3 = coeff(rp2)

scalar su = 0.1
scalar sv = 0.1

mle logl = ln(cnorm(e*lambda/ss)) - (ln(ss) + 0.5*(e/ss)^2)
  scalar ss = sqrt(su^2 + sv^2)
  scalar lambda = su/sv
  series e = rcost - b0*const - b1*y - b2*rp1 - b3*rp2
  params b0 b1 b2 b3 su sv
end mle
\end{code}

\section{GARCH models}
\label{sec:garch}

GARCH models are handled by gretl via a native function. However, it is
instructive to see how they can be estimated through the \texttt{mle}
command.

The following equations provide the simplest example of a GARCH(1,1)
model:
\begin{eqnarray*}
  y_t & = & \mu + \varepsilon_t \\
  \varepsilon_t & = & u_t \cdot \sigma_t \\
  u_t & \sim & N(0,1) \\
  h_t & = & \omega + \alpha \varepsilon^2_{t-1} + \beta h_{t-1}.
\end{eqnarray*}
Since the variance of $y_t$ depends on past values, writing down the
log-verosimiglianza function is not simply a matter of summing the log
densities for individual observations. As is common in time series
models, $y_t$ cannot be considered independent of the other
observations in our sample, and consequently the density function for
the whole sample (the joint density for all observations) is not just
the product of the marginal densities.

Maximum likelihood estimation, in these cases, is achieved by
considering \emph{conditional} densities, so what we maximise is a
conditional likelihood function. If we define the information set at
time $t$ as
\[
  F_t = \left\{ y_t, y_{t-1}, \ldots \right\} ,
\]
then the density of $y_t$ conditional on $F_{t-1}$ is normal:
\[
  y_t | F_{t-1} \sim N\left[ \mu, h_{t} \right].
\]

By means of the properties of conditional distributions, the joint
density can be factorised as follows
\[
  f(y_t, y_{t-1}, \ldots) = \left[ \prod_{t=1}^T f(y_t |F_{t-1})
  \right] \cdot f(y_0) ;
\]
if we treat $y_0$ as fixed, then the term $f(y_0)$ does not depend on
the unknown parameters, and therefore the conditional log-verosimiglianza
can then be written as the sum of the individual contributions as
\begin{equation}
  \label{eq:garchloglik}
  \LogLik(\mu,\omega,\alpha,\beta) = \sum_{t=1}^T \ell_t ,
\end{equation}
where 
\[
  \ell_t = \log \left[ \frac{1}{\sqrt{h_t}} \phi\left( \frac{y_t - \mu}{\sqrt{h_t}}
    \right) \right] = 
    - \frac{1}{2} \left[ \log(h_t) + \frac{(y_t - \mu)^2}{h_t} \right] .
\]

The following script shows a simple application of this technique,
which uses the data file \texttt{djclose}; it is one of the example
dataset supplied with gretl and contains daily data from the Dow Jones
stock index.

\begin{code}
open djclose

series y = 100*ldiff(djclose)

scalar mu = 0.0
scalar omega = 1
scalar alpha = 0.4
scalar beta = 0.0

mle ll = -0.5*(log(h) + (e^2)/h)
  series e = y - mu
  series h = var(y)
  series h = omega + alpha*(e(-1))^2 + beta*h(-1)
  params mu omega alpha beta
end mle
\end{code}

\section{Analytical derivatives}
\label{sec:anal-der}

Computation of the score vector is essential for the working of the
BFGS method. In all the previous examples, no explicit formula for the
computation of the score was given, so the algorithm was fed
numerically evaluated gradients. Numerical computation of the score for
the $i$-th parameter is performed via a finite approximation of the
derivative, namely
\[
  \pder{\LogLik(\theta_1, \ldots, \theta_n)}{\theta_i} \simeq 
  \frac{\LogLik(\theta_1, \ldots, \theta_i + h, \ldots, \theta_n) -
    \LogLik(\theta_1, \ldots, \theta_i - h, \ldots, \theta_n)}{2h} ,
\]
where $h$ is a small number. 

In many situations, this is rather efficient and accurate. However,
one might want to avoid the approximation and specify an exact
function for the derivatives. As an example, consider the following
script:
\begin{code}
nulldata 1000

genr x1 = normal()
genr x2 = normal()
genr x3 = normal()

genr ystar = x1 + x2 + x3 + normal()
genr y = (ystar > 0)

scalar b0 = 0
scalar b1 = 0
scalar b2 = 0
scalar b3 = 0

mle logl = y*ln(P) + (1-y)*ln(1-P)
  series ndx = b0 + b1*x1 + b2*x2 + b3*x3
  series P = cnorm(ndx)
  params b0 b1 b2 b3
end mle --verbose
\end{code}

Here, 1000 data points are artificially generated for an ordinary
probit model\footnote{Again, gretl does provide a native
  \texttt{probit} command, but a probit model makes for a nice
  example here.}: $y_t$ is a binary variable, which takes the value 1 if
$y_t^* = \beta_1 x_{1t} + \beta_2 x_{2t} + \beta_3 x_{3t} +
\varepsilon_t > 0$ and 0 otherwise. Therefore, $y_t = 1$ with
probability $\Phi(\beta_1 x_{1t} + \beta_2 x_{2t} + \beta_3 x_{3t}) =
\pi_t$.  The probability function for one observation can be written
as
\[
  P(y_t) = \pi_t^{y_t} ( 1 -\pi_t )^{1-y_t} ;
\]
since the observations are independent and identically distributed,
the log-verosimiglianza is simply the sum of the individual
contributions. Hence
\[
  \LogLik = \sum_{t=1}^T y_t \log(\pi_t) + (1 - y_t) \log(1 - \pi_t) .
\]
The \texttt{--verbose} switch at the end of the \texttt{end mle}
statement produces a detailed account of the iterations done by the
BFGS algorithm.

In this case, numerical differentiation works rather well;
nevertheless, computation of the analytical score is straightforward,
since the derivative $\pder{\LogLik}{\beta_i}$ can be written as
\[
  \pder{\LogLik}{\beta_i} = \pder{\LogLik}{\pi_t} \cdot \pder{\pi_t}{\beta_i}
\]
via the chain rule, and it is easy to see that
\begin{eqnarray*}
  \pder{\LogLik}{\pi_t} & = & \frac{y_t}{\pi_t} - \frac{1 - y_t}{1 -
    \pi_t} \\
  \pder{\pi_t}{\beta_i} & = & \phi(\beta_1 x_{1t} + \beta_2 x_{2t} +
  \beta_3 x_{3t}) \cdot x_{it} .
\end{eqnarray*}

The \texttt{mle} block in the above script can therefore be modified
as follows:

\begin{code}
mle logl = y*ln(P) + (1-y)*ln(1-P)
  series ndx = b0 + b1*x1 + b2*x2 + b3*x3
  series P = cnorm(ndx)
  series tmp = dnorm(ndx)*(y/P - (1-y)/(1-P))
  deriv b0 = tmp
  deriv b1 = tmp*x1
  deriv b2 = tmp*x2
  deriv b3 = tmp*x3
end mle --verbose
\end{code}

Note that the \texttt{params} statement has been replaced by a series
of \texttt{deriv} statements; these have the double function of
identifying the parameters over which to optimise and providing an
analytical expression for their respective score elements.

%%% Local Variables: 
%%% mode: latex
%%% TeX-master: "gretl-guide"
%%% End: 



\chapter{GMM estimation}
\label{chap:gmm}

\section{Introduction and terminology}
\label{sec:gmm-intro}

The Generalized Method of Moments (GMM) is a very powerful and general
estimation method, which encompasses practically all the parametric
estimation techniques used in econometrics. It was introduced in
\cite{hansen82} and \cite{hansen-singleton82}; an excellent and
thorough treatment is given in chapter 17 of
\cite{davidson-mackinnon93}.

The basic principle on which GMM is built is rather straightforward.
Suppose we wish to estimate a scalar parameter $\theta$ based on
a sample $x_1, x_2, \ldots, x_T$.  Let $\theta_0$ indicate the ``true''
value of $\theta$. Theoretical considerations (either of statistical
or economic nature) may suggest that a relationship like the following
holds:
\begin{equation}
  \label{eq:simple}
  E \left[ x_t - g(\theta) \right] = 0 \Leftrightarrow \theta =
  \theta_0 ,
\end{equation}
with $g(\cdot)$ a continuous and invertible function. That is to say,
there exists a function of the data and the parameter, with the
property that it has expectation zero if and only if it is evaluated
at the true parameter value.  For example, economic models with
rational expectations lead to expressions like (\ref{eq:simple}) quite
naturally.

If the sampling model for the $x_t$s is such that some version of the
Law of Large Numbers holds, then
\[
  \bar{X} = \frac{1}{T} \sum_{t=1}^T x_t \convp g(\theta_0) ;
\]
hence, since $g(\cdot)$ is invertible, the statistic
\[
  \hat{\theta} = g^{-1}(\bar{X}) \convp \theta_0 ,
\]
so $\hat{\theta}$ is a consistent estimator of $\theta$. A different
way to obtain the same outcome is to choose, as an estimator of
$\theta$, the value that minimizes the objective function
\begin{equation}
  \label{eq:obj-simple}
  F(\theta) = \left[ \frac{1}{T} \sum_{t=1}^T (x_t  - g(\theta)) \right]^2 =
  \left[ \bar{X} - g(\theta) \right]^2 ;
\end{equation}
the minimum is trivially reached at $\hat{\theta} = g^{-1}(\bar{X})$,
since the expression in square brackets equals 0.

The above reasoning can be generalized as follows: suppose $\theta$ is
an $n$-vector and we have $m$ relations like
\begin{equation}
  \label{eq:GMMgeneral}
  E \left[ f_i(x_t, \theta) \right] = 0 \quad\textrm{for\ } i=1 \ldots
  m ,
\end{equation}
where $E[\cdot]$ is a conditional expectation on a set of $p$
variables $z_t$, called the \emph{instruments}. In the above simple
example, $m=1$ and $f(x_t, \theta) = x_t - g(\theta)$, and the only
instrument used is $z_t = 1$. Then, it must also be true that
\begin{equation}
  \label{eq:oc}
  E \left[ f_i(x_t, \theta) \cdot z_{j,t} \right] = E \left[ f_{i,j,t}(\theta) \right] = 
  0 \quad\textrm{for\ } i=1 \ldots
  m \quad\textrm{and \ } j=1 \ldots p;
\end{equation}
equation (\ref{eq:oc}) is known as an \emph{orthogonality condition},
or \emph{moment condition}. The GMM estimator is defined as the
minimum of the quadratic form
\begin{equation}
  \label{eq:obj-general}
  F(\theta, W) = \bar{\mathbf{f}}' W \bar{\mathbf{f}},
\end{equation}
where $\bar{\mathbf{f}}$ is a $(1 \times m\cdot p)$ vector holding the
average of the orthogonality conditions and $W$ is some symmetric,
positive definite matrix, known as the \emph{weights} matrix. A
necessary condition for the minimum to exist is the order condition $n
\le m \cdot p$. 

The statistic
\begin{equation}
  \label{eq:gmmestimator}
  \hat{\theta} = \argmin_{\theta} F(\theta, W)
\end{equation}
is a consistent estimator of $\theta$ whatever the choice of $W$.
However, to achieve maximum asymptotic efficiency $W$ must be
proportional to the inverse of the long-run covariance matrix of the
orthogonality conditions; if $W$ is not known, a consistent estimator
will suffice.

These considerations lead to the following empirical strategy:
\begin{enumerate}
\item Choose a positive definite $W$ and compute the
  \emph{one-step} GMM estimator $\hat{\theta}_1$. Customary choices
  for $W$ are $I_{m\cdot p}$ or $I_{m} \otimes (Z'Z)^{-1}$.
\item Use $\hat{\theta}_1$ to estimate $V(f_{i,j,t}(\theta))$ and use its
  inverse as the weights matrix. The resulting estimator
  $\hat{\theta}_2$ is called the \emph{two-step} estimator.
\item Re-estimate $V(f_{i,j,t}(\theta))$ by means of $\hat{\theta}_2$ and
  obtain $\hat{\theta}_3$; iterate until convergence. Asymptotically,
  these extra steps are unnecessary, since the two-step estimator is
  consistent and efficient; however, the iterated estimator often has
  better small-sample properties and should be independent of the
  choice of $W$ made at step 1. 
\end{enumerate}

In the special case when the number of parameters $n$ is equal to the
total number of orthogonality conditions $m \cdot p$, the GMM
estimator $\hat{\theta}$ is the same for any choice of the weights
matrix $W$, so the first step is sufficient; in this case, the
objective function is 0 at the minimum. 

If, on the contrary, $n < m \cdot p$, the second step (or successive
iterations) is needed to achieve efficiency, and the estimator so
obtained can be very different, in finite samples, from the one-step
estimator. Moreover, the value of the objective function at the
minimum, suitably scaled by the number of observations, yields
\emph{Hansen's J statistic}; this statistic can be interpreted as a
test statistic that has a $\chi^2$ distribution with $m \cdot p -n $
degrees of freedom under the null hypothesis of correct specification.
See Davidson and MacKinnon (\citeyear{davidson-mackinnon93}, section
17.6) for details.

In the following sections we will show how these ideas are
implemented in gretl through some examples.

\section{GMM as Method of Moments}
\label{sec:gmm-as-mom}

\emph{We thank Alecos Papadopoulos, who kindly contributed a document
  on which this section is based.}

A very simple illustration of GMM can be given by dropping the ``G'',
via an example of the time-honored statistical technique known as the
method of moments. Let's see how to estimate the parameters of a gamma
distribution, a task which we used to exemplify ML estimation in
section~\ref{sec:ml-gamma}.

Suppose that we have an i.i.d. sample of size $T$ from a gamma
distribution. The gamma density can be parameterized in terms of the
two parameters $k$ (shape) and $\theta$ (scale), both real and
positive.  In order to estimate them by the method of moments, we need
two moment conditions so that we have two equations in the two
unknowns (in GMM terms, this amounts to exact identification). The two
relations we need are
\[
  E(x_i) = k \cdot \theta \qquad V(x_i) = k \cdot \theta^2
\]

Substituting the finite-sample counterparts of the theoretical moments
we have
\begin{eqnarray}
  \label{eq:mm-ex-mean}
  \bar{X} & = & \hat{k} \cdot \hat{\theta} \\
  \label{eq:mm-ex-var}
  \hat{V} & = & \hat{k} \cdot \hat{\theta}^2
\end{eqnarray}
These two equations are easy to solve analytically, giving
$\hat{\theta} = \hat{V}/\bar{X}$ and
$\hat{k} = \bar{X}/\hat{\theta}$, ($\hat{V}$ being the sample
variance of $x_i$), but it's instructive to see how the \cmd{gmm}
command will solve this system of equations numerically.

We feed gretl the necessary ingredients for GMM estimation in a
command block that starts with \texttt{gmm} and ends with \texttt{end
  gmm}. The following elements are compulsory within a \texttt{gmm}
block:
\begin{enumerate}
\item one or more statements to calculate the left-hand side of the
  orthogonality conditions.
\item one or more \texttt{orthog} statements
\item one \texttt{weights} statement
\item one \texttt{params} statement
\end{enumerate}
These elements should be given in the stated order.

The \texttt{orthog} statements are used to specify the orthogonality
conditions.  They must follow the syntax
\begin{code}
  orthog x ; Z
\end{code}
where \texttt{x} may be a series, matrix or list of series (given either
by name or \textit{in extenso}) \texttt{Z} may also be a series, matrix or
list. Note the structure of the statement: it is assumed that the term to
the left of the semicolon represents a quantity that depends on the
estimated parameters (and so must be updated in the process of iterative
estimation), while the term on the right is a constant function of the data.

The \texttt{weights} statement is used to specify the initial
weighting matrix and its syntax is straightforward. 

The \texttt{params} statement specifies the parameters with respect to
which the GMM criterion should be minimized; it follows the same logic
and rules as in the \texttt{mle} and \texttt{nls} commands.

The minimum is found numerically using BFGS (see
chapters~\ref{chap:numerical} and~\ref{chap:mle}).  The progress of
the optimization procedure can be observed by appending the
\option{verbose} switch to the \texttt{end gmm} line.

\bigskip

Equations \ref{eq:mm-ex-mean} and \ref{eq:mm-ex-var} are not yet in
the ``moment condition'' form required by the \cmd{gmm} command. We
need to transform them and arrive at something looking like
$E(e_{j,i} z_{j,i}) = 0$ for $j=1, 2$. We therefore need two variables
$e_1$ and $e_2$ with associated instruments $z_1$ and $z_2$; we can
then tell gretl that $\hat{E}(e_j z_j) = 0$ must be satisfied (where
the $\hat{E}(\cdot)$ notation indicates sample moments).

If we define the instrument as a series of ones, and define a series
\texttt{e1} such that $e_{1,i} = x_i - k \theta$, we can rewrite the
first moment condition as
\[
\hat{E}[e_{1,i} \cdot 1] = 0
\]
In similar manner we can define a series \texttt{e2} such that
$e_{2,i} = (x_i - \bar{X})^2 - k \theta^2$ (and set $z_2 = z_1$), so
that the second moment condition is
\[
  \hat{E}[e_{2,i} \cdot 1] = 0
\]
Since the built-in \texttt{const} is just a series of 1s we could
express these moment conditions in the notation that gretl expects as
\begin{code}
orthog e1 ; const
orthog e2 ; const
\end{code}
But given that the right-hand side is the same for each the conditions
can be combined, using a list on the left:
\begin{code}
orthog e1 e2 ; const
\end{code}

The required weighting matrix can be set to any positive definite
$2 \times 2$ matrix, since under exact identification the choice
doesn't matter and its dimension is determined by the number of
orthogonality conditions. So we'll use the $I_2$ identity
matrix. Example code is shown in Listing~\ref{ex:gamma-gmm}, along with
the output it produces.

\begin{script}[htbp]
  \scriptinfo{gamma-gmm}{MM estimation of Gamma parameters}
\begin{scode}
# create an empty data set with 200 observations
nulldata 200

# fix a random seed for replicability
set seed 1707138404

# generate a Gamma random variable x with shape k = 3 and scale theta = 2
series x = randgen(G, 3, 2)  

# declare and initialize the parameter estimates
scalar k = 1				
scalar theta = 1
								
# create the weight matrix as the identity matrix
matrix W = I(2)

# declare two series for use in the orthogonality conditions
series e1 = 0				
series e2 = 0

# obtain the sample mean of x
scalar m = mean(x)

gmm
    series e1 = x - k*theta
    series e2 = (x - m)^2 - k*theta^2
    orthog e1 e2 ; const
    weights W
    params k theta
end gmm
\end{scode}
The \texttt{gmm} output is:
\begin{scodebot}
Model 1: 1-step GMM, using observations 1-200

             estimate   std. error     z     p-value 
  ---------------------------------------------------
  p          3.08539     0.412149    7.486   7.09e-14 ***
  theta      1.97898     0.286796    6.900   5.19e-12 ***

  GMM criterion: Q = 1.19302e-28 (TQ = 2.38604e-26)
\end{scodebot}
\end{script}

In order to use the \textit{unbiased} estimator of the sample
variance, we have to modify the second moment condition by
substituting
\begin{code}
series e2 = (x - m)^2 - p*theta^2
\end{code}
with
\begin{code}
scalar adj = $nobs / ($nobs - 1)
series e2 = adj * (x - m)^2 - p*theta^2
\end{code}
The output then becomes:
\begin{code}
Model 1: 1-step GMM, using observations 1-200

             estimate   std. error     z     p-value 
  ---------------------------------------------------
  p          3.06997     0.410088    7.486   7.09e-14 ***
  theta      1.98892     0.288237    6.900   5.19e-12 ***

  GMM criterion: Q = 1.66926e-28 (TQ = 3.33852e-26)
\end{code}
In this case both of the point estimates are marginally closer to the
true values. But this is a small-sample effect, not something to
be expected in large samples.

\section{OLS as GMM}
\label{sec:gmm-ols}

We now move to an example closer to econometrics proper: the linear
model $y_t = x_t \beta + u_t$.  Most of us are used to reading it
informally as the sum of a ``systematic part'' and a ``disturbance'',
but a more rigorous interpretation of this familiar expression rests
on the \emph{hypothesis} that the conditional mean $E(y_t|x_t)$ is
linear plus the \emph{definition} of $u_t$ as $y_t - E(y_t|x_t)$.

From the definition of $u_t$, it follows that $E(u_t|x_t) = 0$. 
The following orthogonality condition is therefore available:
\begin{equation}
  \label{eq:oc-ols}
  E \left[ f(\beta) \right] = 0 ,
\end{equation}
where $f(\beta) = (y_t - x_t \beta) x_t$. The definitions given in
section \ref{sec:gmm-intro} therefore specialize here to:
\begin{itemize}
\item $\theta$ is $\beta$;
\item the instrument is $x_t$;
\item $f_{i,j,t}(\theta)$ is $(y_t - x_t \beta) x_t = u_t
  x_t$; the orthogonality condition is interpretable as the
  requirement that the regressors should be uncorrelated with the
  disturbances;
\item $W$ can be any symmetric positive definite matrix, since
  the number of parameters equals the number of orthogonality
  conditions. Let's say we choose $I$.
\item The function $F(\theta, W)$ is in this case
  \[
    F(\theta, W) = \left[ \frac{1}{T} \sum_{t=1}^T (\hat{u}_t x_t) \right]^2
  \]
  and it is easy to see why OLS and GMM coincide here: the GMM
  objective function has the same minimizer as the objective function
  of OLS, the residual sum of squares. Note, however, that the two
  functions are not equal to one another: at the minimum, $F(\theta,
  W) = 0$ while the minimized sum of squared residuals is zero only in
  the special case of a perfect linear fit.
\end{itemize}

The code snippet below uses gretl's \texttt{gmm} command to make the
above operational.  The series \texttt{e} holds the ``residuals'' and
the series \texttt{x} holds the regressor.  If \texttt{x} had been a
list (or a matrix), the \texttt{orthog} statement would have generated
one orthogonality condition for each element (or column) of
\texttt{x}.
%
\begin{code}
/* initialize stuff */
series e = 0
scalar beta = 0
matrix W = I(1)

/* proceed with estimation */
gmm 
  series e = y - x*beta
  orthog e ; x
  weights W
  params beta
end gmm
\end{code}


\section{TSLS as GMM}
\label{sec:gmm-tsls}

Moving closer to the proper domain of GMM, we now consider two-stage
least squares (TSLS) as a case of GMM.  

TSLS is employed in the case where one wishes to estimate a linear
model of the form $y_t = X_t \beta + u_t$, but where one or more of
the variables in the matrix $X$ are potentially
endogenous---correlated with the error term, $u$.  We proceed by
identifying a set of instruments, $Z_t$, which are explanatory for the
endogenous variables in $X$ but which are plausibly uncorrelated with
$u$.  The classic two-stage procedure is (1) regress the endogenous
elements of $X$ on $Z$; then (2) estimate the equation of interest,
with the endogenous elements of $X$ replaced by their fitted values
from (1).

An alternative perspective is given by GMM.  We define the residual
$\hat{u}_t$ as $y_t - X_t \hat{\beta}$, as usual.  But instead of
relying on $E(u|X) = 0$ as in OLS, we base estimation on the condition
$E(u|Z) = 0$.  In this case it is natural to base the initial
weighting matrix on the covariance matrix of the instruments.
Listing~\ref{ex:tsls-gmm} presents a model from Stock and Watson's
\textit{Introduction to Econometrics}.  The demand for cigarettes is
modeled as a linear function of the logs of price and income; income
is treated as exogenous while price is taken to be endogenous and two
measures of tax are used as instruments.  Since we have two
instruments and one endogenous variable the model is over-identified.

In the GMM context, this happens when you have more orthogonality
conditions than parameters to estimate. If so, asymptotic efficiency
gains can be expected by iterating the procedure once or more. This is
accomplished by specifying, after the \cmd{end gmm} statement, two
mutually exclusive options: \option{two-step} or \option{iterate},
whose meaning should be obvious.  Note that when the problem is
over-identified, the weights matrix will influence the solution you
get from the 1- and 2-step procedures.

\tip{In cases other than one-step estimation the specified weights
  matrix will be overwritten with the \emph{final} weights on
  completion of the \texttt{gmm} command. If you wish to execute more
  than one GMM block with a common starting-point it is therefore
  necessary to reinitialize the weights matrix between runs.}


Partial output from this script is shown in~\ref{gmm-tsls-out}.  The
estimated standard errors from GMM are robust by default; if we supply
the \option{robust} option to the \texttt{tsls} command we get
identical results.\footnote{The data file used in this example is
  available in the Stock and Watson package for gretl.  See
  \url{http://gretl.sourceforge.net/gretl_data.html}.}

After the \texttt{end gmm} statement two mutually
exclusive options can be specified: \option{two-step} or
\option{iterate}, whose meaning should be obvious.

\begin{script}[htbp]
  \scriptinfo{tsls-gmm}{TSLS via GMM}
\begin{scode}
open cig_ch10.gdt

# real avg price including sales tax
ravgprs = avgprs / cpi

# real avg cig-specific tax
rtax = tax / cpi

# real average total tax
rtaxs = taxs / cpi

# real average sales tax
rtaxso = rtaxs - rtax

# logs of consumption, price, income
lpackpc = log(packpc)
lravgprs = log(ravgprs)
perinc = income / (pop*cpi)
lperinc = log(perinc)

# restrict sample to 1995 observations
smpl --restrict year==1995
# Equation (10.16) by tsls
list xlist = const lravgprs lperinc
list zlist = const rtaxso rtax lperinc
tsls lpackpc xlist ; zlist --robust

# setup for gmm
matrix Z = { zlist }
matrix W = inv(Z'Z)
series e = 0
scalar b0 = 1
scalar b1 = 1
scalar b2 = 1

gmm e = lpackpc - b0 - b1*lravgprs - b2*lperinc
  orthog e ; Z
  weights W
  params b0 b1 b2
end gmm 
\end{scode}
\end{script}

\begin{script}[htbp]
  \caption{TSLS via GMM: partial output}
  \label{gmm-tsls-out}
\begin{outbit}
Model 1: TSLS estimates using the 48 observations 1-48
Dependent variable: lpackpc
Instruments: rtaxso rtax 
Heteroskedasticity-robust standard errors, variant HC0

      VARIABLE       COEFFICIENT        STDERROR      T STAT   P-VALUE

  const                 9.89496          0.928758     10.654  <0.00001 ***
  lravgprs             -1.27742          0.241684     -5.286  <0.00001 ***
  lperinc               0.280405         0.245828      1.141   0.25401

Model 2: 1-step GMM estimates using the 48 observations 1-48
e = lpackpc - b0 - b1*lravgprs - b2*lperinc

      PARAMETER       ESTIMATE          STDERROR      T STAT   P-VALUE

  b0                    9.89496          0.928758     10.654  <0.00001 ***
  b1                   -1.27742          0.241684     -5.286  <0.00001 ***
  b2                    0.280405         0.245828      1.141   0.25401

  GMM criterion = 0.0110046
\end{outbit}
\end{script}


\section{Covariance matrix options}
\label{sec:gmm-vcv}

The covariance matrix of the estimated parameters depends on the
choice of $W$ through
\begin{equation}
  \label{eq:gmmest-vcv}
    \hat{\Sigma} = (J'WJ)^{-1} J'W\Omega W J (J'WJ)^{-1}
\end{equation}
where $J$ is a Jacobian term
\[
  J_{ij} = \pder{\bar{f}_i}{\theta_j}
\]
and $\Omega$ is the long-run covariance matrix of the orthogonality
conditions. 

Gretl computes $J$ by numeric differentiation (there is no
provision for specifying a user-supplied analytical expression for $J$
at the moment). As for $\Omega$, a consistent estimate is needed. The
simplest choice is the sample covariance matrix of the $f_t$s:
\begin{equation}
  \label{eq:gmm-hcvar}
    \hat{\Omega}_0(\theta) = \frac{1}{T} \sum_{t=1}^T f_t(\theta) f_t(\theta)'
\end{equation}

This estimator is robust with respect to heteroskedasticity, but not
with respect to autocorrelation.  A heteroskedasticity- and
autocorrelation-consistent (HAC) variant can be obtained using the
Bartlett kernel or similar.  A univariate version of this is used in
the context of the \texttt{lrvar()} function---see equation
(\ref{eq:scalar-lrvar}).  The multivariate version is set out in
equation (\ref{eq:gmm-hacvar}).

\begin{equation}
  \label{eq:gmm-hacvar}
    \hat{\Omega}_k(\theta) = \frac{1}{T} 
    \sum_{t=k}^{T-k} \left[ \sum_{i=-k}^k w_i f_t(\theta) f_{t-i}(\theta)'  \right] ,
\end{equation}

Gretl computes the HAC covariance matrix by default when a GMM
model is estimated on time series data.  You can control the kernel
and the bandwidth (that is, the value of $k$ in \ref{eq:gmm-hacvar})
using the \texttt{set} command.  See chapter~\ref{chap:robust_vcv} for
further discussion of HAC estimation.  You can also ask gretl
\emph{not} to use the HAC version by saying
%
\begin{code}
set force_hc on
\end{code}

\section{A real example: the Consumption Based Asset Pricing Model}
\label{sec:gmm-CBAPM}

To illustrate gretl's implementation of GMM, we will replicate
the example given in chapter 3 of \cite{hall05}. The model to estimate
is a classic application of GMM, and provides an example of a case
when orthogonality conditions do not stem from statistical
considerations, but rather from economic theory.

A rational individual who must allocate his income between consumption
and investment in a financial asset must in fact choose the consumption
path of his whole lifetime, since investment translates into future
consumption. It can be shown that an optimal consumption path should
satisfy the following condition:
\begin{equation}
  \label{eq:gmm-CBAPM}
  p U'(c_t) = \delta^k E\left[ r_{t+k} U'(c_{t+k}) | \mathcal{F}_t
  \right] ,
\end{equation}
where $p$ is the asset price, $U(\cdot)$ is the individual's utility
function, $\delta$ is the individual's subjective discount rate and
$r_{t+k}$ is the asset's rate of return between time $t$ and time
$t+k$. $\mathcal{F}_t$ is the \emph{information set} at time $t$;
equation (\ref{eq:gmm-CBAPM}) says that the utility ``lost'' at time
$t$ by purchasing the asset instead of consumption goods must be
matched by a corresponding increase in the (discounted) future utility
of the consumption financed by the asset's return. Since the future is
uncertain, the individual considers his expectation, conditional on
what is known at the time when the choice is made.

We have said nothing about the nature of the asset, so equation
(\ref{eq:gmm-CBAPM}) should hold whatever asset we consider; hence, it
is possible to build a system of equations like (\ref{eq:gmm-CBAPM})
for each asset whose price we observe.

If we are willing to believe that
\begin{itemize}
\item the economy as a whole can be represented as a single gigantic
  and immortal representative individual, and
\item the function $U(x) = \frac{x^{\alpha} - 1 }{\alpha}$ is a
  faithful representation of the individual's preferences,
\end{itemize}
then, setting $k=1$, equation (\ref{eq:gmm-CBAPM}) implies the
following for any asset $j$:
\begin{equation}
  \label{eq:gmm-CBAPM-est}
  E\left[ \delta \frac{r_{j,t+1}}{p_{j,t}} \left(\frac{C_{t+1}}{C_{t}}
    \right)^{\alpha - 1} \bigg| \mathcal{F}_t \right] = 1 ,
\end{equation}
where $C_t$ is aggregate consumption and $\alpha$ and $\delta$ are the
risk aversion and discount rate of the representative individual. In
this case, it is easy to see that the ``deep'' parameters $\alpha$ and
$\delta$ can be estimated via GMM by using
\[
  e_t = \delta \frac{r_{j,t+1}}{p_{j,t}} \left(\frac{C_{t+1}}{C_{t}}
    \right)^{\alpha - 1} - 1
\]
as the moment condition, while any variable known at time $t$ may serve as
an instrument.


\begin{script}[htbp]
  \scriptinfo{cbapm}{Estimation of the Consumption Based Asset Pricing Model}
\begin{scode}
open hall.gdt
set force_hc on

scalar alpha = 0.5
scalar delta = 0.5
series e = 0

list inst = const consrat(-1) consrat(-2) ewr(-1) ewr(-2)

matrix V0 = 100000*I(nelem(inst))
matrix Z = { inst }
matrix V1 = $nobs*inv(Z'Z)

gmm e = delta*ewr*consrat^(alpha-1) - 1
  orthog e ; inst
  weights V0
  params alpha delta
end gmm

gmm e = delta*ewr*consrat^(alpha-1) - 1
  orthog e ; inst
  weights V1
  params alpha delta
end gmm

gmm e = delta*ewr*consrat^(alpha-1) - 1
  orthog e ; inst
  weights V0
  params alpha delta
end gmm --iterate

gmm e = delta*ewr*consrat^(alpha-1) - 1
  orthog e ; inst
  weights V1
  params alpha delta
end gmm --iterate
\end{scode}
\end{script}

\begin{script}[htbp]
  \caption{Estimation of the Consumption Based Asset Pricing Model --
  output}
  \label{gmm-CBAPM-out}
\begin{outbit}
Model 1: 1-step GMM estimates using the 465 observations 1959:04-1997:12
e = d*ewr*consrat^(alpha-1) - 1

      PARAMETER       ESTIMATE          STDERROR      T STAT   P-VALUE

  alpha                -3.14475          6.84439      -0.459   0.64590
  d                     0.999215         0.0121044    82.549  <0.00001 ***

  GMM criterion = 2778.08

Model 2: 1-step GMM estimates using the 465 observations 1959:04-1997:12
e = d*ewr*consrat^(alpha-1) - 1

      PARAMETER       ESTIMATE          STDERROR      T STAT   P-VALUE

  alpha                 0.398194         2.26359       0.176   0.86036
  d                     0.993180         0.00439367  226.048  <0.00001 ***

  GMM criterion = 14.247

Model 3: Iterated GMM estimates using the 465 observations 1959:04-1997:12
e = d*ewr*consrat^(alpha-1) - 1

      PARAMETER       ESTIMATE          STDERROR      T STAT   P-VALUE

  alpha                -0.344325         2.21458      -0.155   0.87644
  d                     0.991566         0.00423620  234.070  <0.00001 ***

  GMM criterion = 5491.78
  J test: Chi-square(3) = 11.8103 (p-value 0.0081)

Model 4: Iterated GMM estimates using the 465 observations 1959:04-1997:12
e = d*ewr*consrat^(alpha-1) - 1

      PARAMETER       ESTIMATE          STDERROR      T STAT   P-VALUE

  alpha                -0.344315         2.21359      -0.156   0.87639
  d                     0.991566         0.00423469  234.153  <0.00001 ***

  GMM criterion = 5491.78
  J test: Chi-square(3) = 11.8103 (p-value 0.0081)
\end{outbit}
\end{script}

In the example code given in \ref{ex:cbapm}, we replicate
selected portions of table 3.7 in \cite{hall05}.  The variable
\texttt{consrat} is defined as the ratio of monthly consecutive real
per capita consumption (services and nondurables) for the US, and
\texttt{ewr} is the return--price ratio of a fictitious asset
constructed by averaging all the stocks in the NYSE.  The instrument
set contains the constant and two lags of each variable.

The command \texttt{set force\_hc on} on the second line of the script
has the sole purpose of replicating the given example: as mentioned
above, it forces gretl to compute the long-run variance of the
orthogonality conditions according to equation (\ref{eq:gmm-hcvar})
rather than (\ref{eq:gmm-hacvar}).

We run \texttt{gmm} four times: one-step estimation for each of two
initial weights matrices, then iterative estimation starting from each
set of initial weights.  Since the number of orthogonality conditions
(5) is greater than the number of estimated parameters (2), the choice
of initial weights should make a difference, and indeed we see fairly
substantial differences between the one-step estimates (Models 1 and
2).  On the other hand, iteration reduces these differences almost to
the vanishing point (Models 3 and 4). 

Part of the output is given in \ref{gmm-CBAPM-out}.  It should be
noted that the $J$ test leads to a rejection of the hypothesis of
correct specification.  This is perhaps not surprising given the
heroic assumptions required to move from the microeconomic principle
in equation (\ref{eq:gmm-CBAPM}) to the aggregate system that is
actually estimated.


\section{Caveats}
\label{sec:gmm-caveat}

A few words of warning are in order: despite its ingenuity, GMM is
possibly the most fragile estimation method in econometrics. The
number of non-obvious choices one has to make when using GMM is large,
and in finite samples each of these can have dramatic consequences for
the eventual output. Some of the factors that may affect the results
are:
\begin{enumerate}
\item Orthogonality conditions can be written in more than one way:
  for example, if $E(x_t - \mu) = 0$, then $E(x_t/\mu - 1) =
  0$ holds too. It is possible that a different specification of the
  moment conditions leads to different results.
\item As with all other numerical optimization algorithms, weird
  things may happen when the objective function is nearly flat in some
  directions or has multiple minima. BFGS is usually quite good, but
  there is no guarantee that it always delivers a sensible solution,
  if one at all.
\item The 1-step and, to a lesser extent, the 2-step estimators may be
  sensitive to apparently trivial details, like the re-scaling of the
  instruments. Different choices for the initial weights matrix can
  also have noticeable consequences.
\item With time-series data, there is no hard rule on the appropriate
  number of lags to use when computing the long-run covariance matrix
  (see section \ref{sec:gmm-vcv}). Our advice is to go by trial and
  error, since results may be greatly influenced by a poor choice.
\end{enumerate}

One of the consequences of this state of things is that replicating
well-known published studies may be extremely difficult. Any
non-trivial result is virtually impossible to reproduce unless all
details of the estimation procedure are carefully recorded.

%%% Local Variables: 
%%% mode: latex
%%% TeX-master: "gretl-guide"
%%% End: 

\chapter{Criteri di selezione dei modelli}
\label{select-criteria}

\section{Introduzione}
\label{select-intro}

In alcuni contesti, l'econometrico deve scegliere tra modelli alternativi
basandosi su test di ipotesi formali. Ad esempio, si pu� scegliere un
modello pi� generale rispetto ad uno pi� ristretto, se la restrizione in
questione pu� essere formulata sotto forma di ipotesi nulla testabile e
l'ipotesi nulla viene rifiutata da un apposito test.

In altri contesti si ha bisogno invece di un criterio di selezione dei modelli
che tenga conto da una parte dell'accuratezza dell'adattamento ai dati, o della
verosimiglianza del modello, e dall'altra parte della sua parsimonia. �
necessario mantenere questo equilibrio perch� l'aggiunta di variabili a un
modello pu� solo aumentare la sua capacit� di adattamento o la sua
verosimiglianza, ma � possibile che ci� avvenga anche se le variabili aggiuntive
non sono veramente rilevanti per il processo che ha generato i dati.

Il pi� famoso tra questi criteri di selezione, per modelli lineari stimati con i
minimi quadrati, � l'$R^2$ corretto,
%
\[
\bar{R}^2 = 1 - \frac{{\rm SSR} / (n-k)}{{\rm TSS} / (n-1)}
\]
%
dove $n$ � il numero di osservazioni nel campione, $k$ denota il numero di
parametri stimati, SSR e TSS denotano rispettivamente la somma dei quadrati dei
residui e la somma dei quadrati della variabile dipendente. Confrontata con il
classico coefficiente di determinazione $R^2$
%
\[
R^2 = 1 - \frac{{\rm SSR}}{{\rm TSS}}
\]
%
la versione ``corretta'' penalizza l'inclusione di variabili aggiuntive, a
parit� di altre condizioni.

\section{Criteri di informazione}
\label{select-aic}

Un criterio pi� generale, che segue un'impostazione simile, � il ``criterio di
informazione di Akaike'' (AIC) del 1974. La formulazione originale di questa
misura �
%
\begin{equation}
\label{aic-orig}
{\rm AIC} = -2 \ell(\hat{\theta}) + 2k
\end{equation}
%
dove $\ell(\hat{\theta})$ rappresenta la massima log-verosimiglianza come
funzione del vettore delle stime dei parametri, $\hat{\theta}$, e $k$
(come sopra) indica il numero di ``parametri indipendenti all'interno del
modello''. In questa formulazione, con AIC correlato negativamente alla
verosimiglianza e positivamente al numero dei parametri, il ricercatore mira a
minimizzare il suo valore.

L'AIC pu� generare confusione, dal momento che sono diffuse varie versioni per
il suo calcolo; ad esempio, Davidson e MacKinnon (2004) ne presentano una versione
semplificata,
%
\[
{\rm AIC} = \ell(\hat{\theta}) - k
\]
%
che vale quanto l'originale moltiplicata per $-2$: in questo caso, ovviamente,
si cercher� di massimizzare l'AIC.

Nel caso di modelli stimati con i minimi quadrati, la log-verosimiglianza pu�
essere scritta come
%
\begin{equation}
\label{ols-loglik}
\ell(\hat{\theta}) = -\frac{n}{2}(1 + \log 2\pi - \log n)
 - \frac{n}{2} \log {\rm SSR}
\end{equation}
%
Sostituendo (\ref{ols-loglik}) in (\ref{aic-orig}) otteniamo
%
\[
{\rm AIC} = n(1 + \log 2\pi - \log n) + n\log {\rm SSR} + 2k
\]
%
che pu� essere scritta anche come
%
\begin{equation}
\label{full-aic-alt}
{\rm AIC} = n\log \left( \frac{\rm SSR}{n} \right) + 2k + 
  n(1 + \log 2\pi)
\end{equation}
%

Alcuni autori semplificano la formula nel caso di modelli stimati con i minimi
quadrati. Ad esempio William Greene scrive
%
\begin{equation}
\label{aic-greene}
{\rm AIC} = \log \left( \frac{\rm SSR}{n} \right) + \frac{2k}{n}
\end{equation}
%
Questa variante pu� essere derivata da (\ref{full-aic-alt}) dividendo per
$n$ e sottraendo la costante $1 + \log 2\pi$.  Ossia, chiamando 
AIC$_G$ la versione proposta da Greene, abbiamo
%
\[
{\rm AIC}_G = \frac{1}{n} {\rm AIC} - (1 + \log 2\pi)
\]
%

Infine, Ramanathan offre un'altra variante:
%
\[
{\rm AIC}_R = \left( \frac{\rm SSR}{n} \right) e^{2k/n}
\]
%
che � l'esponenziale della versione di Greene.  

All'inizio, gretl usava la versione di Ramanathan, ma a partire dalla versione
1.3.1 del programma, viene usata la formula originale di Akaike
(\ref{aic-orig}), e pi� specificamente (\ref{full-aic-alt}) per i modelli
stimati con i minimi quadrati.

\vspace{1ex}

Anche se il criterio di Akaike � progettato per favorire la parsimonia, non lo
fa in modo eccessivo. Ad esempio, se abbiamo due modelli annidati con
rispettivamente $k-1$ e $k$ parametri, e se l'ipotesi nulla che il parametro
$k$ valga 0 � vera, per grandi campioni l'AIC tender� comunque a far preferire
il modello meno parsimonioso in circa il 16 per cento dei casi (si veda
Davidson e MacKinnon, 2004, capitolo 15).

Un criterio alternativo all'AIC che non risente di questo problema � il
``Criterio di informazione Bayesiana'' (BIC) di Schwarz (1978). Il BIC pu�
essere scritto (in modo simile alla formulazione di Akaike di AIC) come
%
\[
{\rm BIC} = -2 \ell(\hat{\theta}) + k \log n
\]
Il prodotto di $k$ per $\log n$ nel BIC significa che la penalizzazione per
l'aggiunta di parametri addizionali aumenta con l'ampiezza campionaria. Ci�
assicura che, asintoticamente, un modello troppo esteso non verr� mai scelto al
posto di un modello parsimonioso ma correttamente specificato.

Un'altra alternativa all'AIC che tende a favorire modelli pi� parsimoniosi � il
criterio di Hannan--Quinn, o HQC (Hannan e Quinn, 1979). Volendo essere coerenti
con la formulazione usata finora, pu� essere scritto nel modo seguente:
%
\[
{\rm HQC} = -2 \ell(\hat{\theta}) + 2k \log \log n
\]
%
Il calcolo di Hannan--Quinn si basa sulla regola del logaritmo iterato (si noti
che l'ultimo termine � il logaritmo del logaritmo dell'ampiezza campionaria).
Gli autori affermano che questa procedura fornisce un ``procedura di stima
consistente in senso forte per l'ordine di una autoregressione'', e che
``confrontata con altre procedure consistenti in senso forte, questa sottostimer�
l'ordine in modo minore''.

\vspace{1ex}

Gretl mostra AIC, BIC e HQC (calcolati nel modo spiegato sopra) per la maggior
parte dei modelli. Quando si interpretano questi valori occorre sempre
ricordarsi se sono calcolati in modo da essere massimizzati o minimizzati. In
gretl essi sono sempre calcolati per essere minimizzati: valori minori sono
da preferire.






\chapter{Modelli per serie storiche}
\label{chap:timeser}

\section{Modelli ARIMA}
\label{arma-estimation}

\subsection{Rappresentazione e sintassi}
\label{arma-repr}

Il comando \cmd{arma} effettua la stima di modelli autoregressivi a media mobile
(ARMA); la rappresentazione pi� generale di un modello ARMA stimabile con \app{gretl}
� la seguente:
\begin{equation}
  \label{eq:general-arma}
  A(L) B(L^s) y_t = x_t \beta + C(L) D(L^s) \epsilon_t ,
\end{equation}
dove $L$ � l'operatore di ritardo ($L^n x_t = x_{t-n}$), $s$ � il numero di
sotto-periodi per le serie stagionali (ad esempio, 12 per serie mensili),
$x_t$ � un vettore di variabili esogene, e $\epsilon_t$ � un processo a
rumore bianco.

Il modello ARMA ``basilare'' si ottiene quando vale $x_t = 1$ e non ci sono
operatori stagionali. In questo caso, $B(L^s) = D(L^s) = 1$ e il modello diventa
\begin{equation}
  \label{eq:plain-arma}
  A(L) y_t = \mu + C(L) \epsilon_t ,
\end{equation}
dove, nella solita notazione, il vettore $\beta$ si riduce all'intercetta $\mu$.
� possibile scrivere l'equazione precedente in modo pi� esplicito come
\[
  y_t = \mu + \phi_1 y_{t-1} + \ldots + \phi_p y_{t-p} + 
  \epsilon_t + \theta_1 \epsilon_{t-1} + \ldots + \theta_q
  \epsilon_{t-q}
\]
La sintassi corrispondente in \app{gretl} � semplicemente
\begin{code}
  gretl p q ; y
\end{code}
dove \verb|p| e \verb|q| sono gli ordini di ritardo desiderati; questi possono
essere sia numeri, sia scalari pre-definiti. Il parametro $\mu$ pu� essere
omesso se necessario, aggiungendo l'opzione \cmd{--nc} al comando.

Se si vuole stimare un modello con variabili esplicative, la sintassi vista
sopra pu� essere estesa in questo modo
\begin{code}
  gretl p q ; y const x1 x2
\end{code}
Questo comando stimerebbe il seguente modello:
\[
  y_t = \beta_0 + x_{1,t} \beta_1 + x_{2,t} \beta_2 + 
  \phi_1 y_{t-1} + \ldots + \phi_p y_{t-p} + 
  \epsilon_t + \theta_1 \epsilon_{t-1} + \ldots + \theta_q \epsilon_{t-q} .
\]
Quello che \app{gretl} stima in questo caso � un modello ARMAX (ARMA +
variabili esogene), che � diverso da ci� che alcuni altri programmi chiamano
``modello di regressione con errori ARMA''. La differenza � evidente
considerando il modello stimato da \app{gretl}
\begin{equation}
  \label{eq:armax}
  A(L) y_t = x_t \beta + C(L) \epsilon_t ,
\end{equation}
e un modello di regressione con errori ARMA, ossia
\begin{eqnarray}
  \label{eq:reg-arma}
  y_t & = & x_t \beta + u_t \\
  A(L) u_t & = & C(L) \epsilon_t ;
\end{eqnarray}
l'ultimo si tradurrebbe nell'espressione seguente
\[
  A(L) y_t = A(L) \left(x_t \beta \right) + C(L) \epsilon_t ;
\]
la formulazione ARMAX ha il vantaggio che il coefficiente $\beta$
pu� essere interpretato immediatamente come l'effetto marginale delle
variabili $x_t$ sulla media condizionale di $y_t$. Si noti comunque, che
i modelli di regressione che contengono errori puramente autoregressivi
possono essere stimati (sebbene non con tecniche di massima verosimiglianza) con
altri comandi di \app{gretl}, come \cmd{corc} e \cmd{pwe}.

Il comando \cmd{arma} pu� quindi essere usato anche per stimare \emph{Modelli a
funzione di trasferimento}, un tipo di generalizzazione dei modelli autoregressivi
a media mobile (ARMA) che aggiunge l'effetto di variabili esogene distribuite
nel tempo, come in questo esempio:
\begin{equation}
  \label{eq:tfunc-model}
  \phi (L) \cdot \Phi (L^s) y_t = \sum_{i=1}^kv_{i}(L)x_{it} + \theta (L)\cdot \Theta (L^s) \epsilon_t ,
\end{equation}

La struttura del comando \cmd{arma} non permette di specificare modelli con
buchi nella struttura dei ritardi. Una struttura di ritardi pi� flessibile
pu� essere necessaria quando si analizzano serie che mostrano un marcato
andamento stagionale. In questo caso � possibile usare il modello completo
(\ref{eq:general-arma}). Ad esempio, la sintassi
\begin{code}
  arma 1 1 ; 1 1 ; y
\end{code}
permette di stimare un modello con quattro parametri:
\[
  ( 1 - \phi L )  ( 1 - \Phi L^s ) y_t = \mu + ( 1 + \theta L ) ( 1 + \Theta L^s ) \epsilon_t;
\]
assumendo che $y_t$ � una serie trimestrale (e quindi $s=4$), l'equazione
precedente pu� essere scritta in modo pi� esplicito come
\[
  y_t = \mu + \phi y_{t-1} + \Phi y_{t-4} - (\phi \cdot \Phi) y_{t-5} + 
  \epsilon_t + \theta \epsilon_{t-1} + \Theta \epsilon_{t-4} +
  (\theta \cdot \Theta) \epsilon_{t-5} .
\]

Un tale modello � di solito chiamato un ``modello ARMA stagionale
moltiplicativo''.

Per modelli pi� generali questa limitazione pu� essere aggirata per quanto
riguarda la parte autoregressiva, includendo ritardi della variabile dipendente
nella lista delle variabili esogene.

Ad esempio, il comando seguente
\begin{code}
  arma 0 0 ; 0 1 ; y const y(-2)
\end{code}
su una serie trimestrale stimerebbe i parametri del modello
\[
  y_t = \mu + \phi y_{t-2} + \epsilon_t + \Theta \epsilon_{t-4}.
\]
Questo modo di procedere per� non � consigliato: anche se si ottengono stime
corrette, la funzionalit� di previsione risulter� inutilizzabile.

La discussione svolta presuppone che la serie storica $y_t$ sia gi� stata
soggetta a tutte le trasformazioni necessarie per assicurarne la stazionariet�
(si veda anche il capitolo \ref{sec:uroot}).  La trasformazione pi� usata in
questi casi � la differenziazione, quindi \app{gretl} fornisce un meccanismo per
includere questo passo nel comando \cmd{arma}: la sintassi
\begin{code}
  arma p d q ; y 
\end{code}
stimerebbe un modello ARMA(p,q) su $\Delta^d y_t$ ed � equivalente a
\begin{code}
  series tmp = y
  loop for i=1..d
    tmp = diff(tmp)
  end loop
  arma p q ; tmp 
\end{code}
Un tale modello � noto come ARIMA (autoregressivo integrato a media mobile); per
questo motivo \app{gretl} fornisce anche il comando \cmd{arima} come sinonimo di
\cmd{arma}. L'operazione di differenziazione stagionale viene gestita in modo
simile, con la sintassi
\begin{code}
  arma p d q ; P D Q ; y 
\end{code}
Cos�, il comando
\begin{code}
  arma 1 0 0 ; 1 1 1 ; y 
\end{code}
produce gli stessi risultati di
\begin{code}
  genr dsy = sdiff(y)
  arma 1 0 ; 1 1 ; dsy 
\end{code}

\subsection{Stima}
\label{arma-est}

L'algoritmo usato da \app{gretl} per stimare i parametri di un modello ARMA �
quello della massima verosimiglianza condizionale (CML), noto anche come ``somma
dei quadrati condizionale'' (si veda Hamilton 1994,
pagina 132).  Questo metodo � esemplificato nello script \ref{jack-arma}, quindi
ne verr� data solo una breve descrizione qui: dato un campione di ampiezza
$T$, il metodo CML minimizza la somma dei quadrati degli errori delle previsioni
generate dal modello per l'osservazione successiva, sull'intervallo $t_0,
\ldots, T$. Il punto di partenza $t_0$ dipende dall'ordine dei polinomi AR del
modello.

Questo metodo � quasi equivalente a quello della massima verosimiglianza in
ipotesi di normalit�; la differenza sta nel fatto che le prime $(t_0 - 1)$
osservazioni sono considerate fisse ed entrano nella funzione di verosimiglianza
solo come variabili condizionanti. Di conseguenza, i due metodi sono
asintoticamente equivalenti sotto le consuete ipotesi.

Il metodo numerico usato per massimizzare la log-verosimiglianza � quello BHHH.
La matrice di covarianza per i parametri (e quindi gli errori standard) sono
calcolati col metodo del prodotto esterno dei gradienti (OPG, Outer Product of
the Gradients).

La stima di massima verosimiglianza vera e propria in ipotesi di normalit� �
disponibile in \app{gretl} attraverso il plugin \verb|x-12|, che viene usato
automaticamente se si aggiunge l'opzione \verb|--x-12-arima| al comando
\cmd{arma}. Ad esempio, il codice seguente
\begin{code}
  open data10-1
  arma 1 1 ; r
  arma 1 1 ; r --x-12-arima
\end{code}
produce le stime mostrate nella tabella~\ref{tab:cml-ml}.

\begin{table}[htbp]
\caption{Stime CML e ML}
\label{tab:cml-ml}
\begin{center}
  \begin{tabular}{crrrr}
    \hline
    Parametro & \multicolumn{2}{c}{CML} &
    \multicolumn{2}{c}{ML (\app{X-12-ARIMA} plugin)} \\
    \hline 
    $\mu$ & 1.07322 & (0.488661) &  1.00232 & (0.133002) \\
    $\phi$ & 0.852772 & (0.0450252) & 0.855373  & (0.0496304) \\
    $\theta$ & 0.591838 & (0.0456662) & 0.587986 & (0.0799962) \\
    \hline
  \end{tabular}
\end{center}
\end{table}

Per confrontabilit� con le stime di \app{gretl}, � possibile impostare
\app{X-12-ARIMA} per produrre le stime CML aggiungendo l'opzione
\verb|--conditional| oltre a \verb|--x-12-arima|.

\subsection{Previsione}
\label{arma-fcast}

To be written

\section{Unit root tests}
\label{sec:uroot}

To be completed

\subsection{The ADF test}
\label{sec:ADFtest}

Il test ADF (Augmented Dickey-Fuller) � implementanto in \app{gretl} sotto forma
della statistica $t$ su $\varphi$ nella regressione seguente:
\begin{equation}
  \label{eq:ADFtest}
  \Delta y_t = \mu_t + \varphi y_{t-1} + \sum_{i=1}^p \gamma_i \Delta
  y_{t-i} + \epsilon_t .
\end{equation}

Questa statistica test � probabilmente il pi� famoso e utilizzato test per
radici unitarie. � un test a una coda la cui ipotesi nulla �
$\varphi = 0$, mentre quella alternativa � $\varphi < 0$. Sotto l'ipotesi nulla,
$y_t$ deve essere differenziata almeno una volta per raggiungere la
stazionariet�. Sotto l'ipotesi alternativa, $y_t$ � gi� stazionaria e non
richiede differenziazione. Quindi, grandi valori negativi della statistica test
portano a rifiutare l'ipotesi nulla.

Un aspetto peculiare di questo test � che la sua distribuzion limite non �
standard sotto l'ipotesi nulla: inoltre, la forma della distribuzione, e quindi
i valori critici per il test, dipendono dalla forma del termine
$\mu_t$. Un'eccellente analisi di tutti i casi possibili � contenuta in
Hamilton (1994), ma il soggetto � trattato anche in qualsiasi testo recente
sulle serie storiche. Per quanto riguarda \app{gretl}, esso permette all'utente
di scegliere la specificazione di $\mu_t$ tra quanttro alternative:

\begin{center}
  \begin{tabular}{cc}
    \hline
    $\mu_t$ & Opzione del comando \\
    \hline
    0 & \verb|--nc| \\
    $\mu_0$ &  \verb|--c| \\
    $\mu_0 + \mu_1 t$ &  \verb|--ct| \\
    $\mu_0 + \mu_1 t + \mu_1 t^2$ &  \verb|--ctt| \\
    \hline
  \end{tabular}
\end{center}

Queste opzioni non sono mutualmente esclusive e possono essere usate insieme; in
questo caso, la statistica verr� calcolata separatamente per ognuno dei casi.
La scelta predefinita in \app{gretl} � quella di usare la combinazione
\verb|--c --ct --ctt|. Per ognuno dei casi, vengono calcolati p-value
approssimativi usando l'algoritmo descritto in MacKinnon 1996.

Il comando di \app{gretl} da usare per eseguire il test � \cmd{adf}; ad esempio
\begin{code}
  adf 4 x1 --c --ct
\end{code}
calcola la statistica test come statistica-t per $\varphi$ nell'equazione
\ref{eq:ADFtest} con $p=4$ nei due casi $\mu_t = \mu_0$ e
$\mu_t = \mu_0 + \mu_1 t$.

Il numero di ritardi ($p$ nell'equazione \ref{eq:ADFtest}) deve essere scelto
per assicurarsi che (\ref{eq:ADFtest}) sia una parametrizzazione abbastanza
flessibile per rappresentare adeguatamente la persistenza di breve termine di
$\Delta y_t$. Scegliere un $p$ troppo basso pu� portare a distorsioni di
dimensione nel test, mentre sceglierlo troppo alto porta a una perdita di
potenza del test. Per comodit� dell'utente, il parametro $p$ pu� essere
determinato automaticamente. Impostando $p$ a un numero negativo viene attivata
una procedura sequenziale che parte da $p$ ritardi e decrementa $p$ fino a quando la statistica $t$
per il parametro $\gamma_p$ supera 1.645 in valore assoluto.

\subsection{The KPSS test}
\label{sec:KPSStest}

\begin{equation}
  \label{eq:KPSStest}
  \eta = \frac{\sum_{i=1}^T S_t^2 }{ T^2 \bar{\sigma}^2 }
\end{equation}
where $S_t = \sum_{s=1}^t e_s$ and $\bar{\sigma}^2$ is an
estimate of the long-run variance of $e_t = (y_t - \bar{y})$.
\begin{itemize}
\item $H_0$: $y_t$ is I(0);
\item one-sided test: $H_0$ is rejected if $\eta$ is ``big'';
\item extension to deterministic trend ($e_t$ are the residuals from
  an OLS regression of $y_t$ on a constant and a linear trend);
\item nonstandard distribution --- we provide 90\%, 95\%, 97.5\% and
  99\% quantiles.
\end{itemize}
Syntax:
\begin{code}
  kpss n x1
\end{code}
\begin{itemize}
\item \verb|--trend| option;
\item \verb|n| is used for estimating $\bar{\sigma}^2$; in the GUI,
  default is the integer part of $4 \left( \frac{T}{100}
  \right)^{1/4}$.
\end{itemize}

\subsection{I test di Johansen}
\label{sec:Joh-test}

In senso stretto, questi sono test per la cointegrazione, ma possono essere
usati anche come test multivariati per radici unitarie, visto che sono la
generalizzazione multivariata del test ADF.
\begin{equation}
  \label{eq:Joh-tests}
  \Delta y_t = \mu_t + \Pi y_{t-1} + \sum_{i=1}^p \Gamma_i \Delta
  y_{t-i} + \epsilon_t
\end{equation}
Se il rango di $\Pi$ � 0, i processi sono tutti I(1); se il rango di
$\Pi$ � pieno, i processi sono tutti I(0); nei casi intermedi, si ha
cointegrazione.

Il rango di $\Pi$ viene analizzato calcolando gli autovalori di una matrice ad
essa strettamente legata (chiamata $M$) che ha rango pari a quello di $\Pi$:
i test sul rango di $\Pi$ possono quindi essere condotti verificando quanti
autovalori di $M$ sono pari a 0. I due test di Johansen sono i test
``$\lambda$-max'', per le ipotesi sui singoli autovalori, e il test
``trace'', per le ipotesi congiunte.

Il comando \cmd{coint2} di \app{gretl} esegue questi due test.
Esempio:
\begin{code}
  coint2 n x1 x2 x3
\end{code}
imposta $p$ nell'equazione (\ref{eq:Joh-tests}) a $(n-1)$ e mostra una tabella
con le due batterie di test per le tre serie \verb|x1|, \verb|x2| e \verb|x3|.
 
\ldots Options for the deterministic kernel \ldots
More on this in section \ref{sec:johansen-test}

\section{ARCH e GARCH}
\label{sec:arch}

Il fenomeno dell'eteroschedasticit� rappresenta la varianza non costante del
termine di errore in un modello di regressione. L'eteroschedasticit�
condizionale autoregressiva (ARCH) � un fenomeno specifico dei modelli per serie
storiche, in cui la varianza del termine di errore presenta un comportamento
autoregressivo, ad esempio, la serie presenta periodi in cui la varianza
dell'errore � relativamente ampia e periodi in cui � relativamente piccola.

Un modello GARCH pu� essere descritto brevemente dalle equazioni seguenti:
\begin{eqnarray}
  \label{eq:garch-meaneq}
  y_t &  = & x_t \beta + \epsilon_t \\
  \label{eq:garch-epseq}
  \epsilon_t &  = & u_t \sigma_t \\
  \label{eq:garch-vareq}
  A(L) \sigma^2_t &  = & \omega + B(L) \epsilon_{t-1}^2 ,
\end{eqnarray}
dove $u_t$ � una sequenza iid con varianza unitaria. Al momento
\app{gretl} gestisce solo modelli in cui $u_t$ � ipotizzato essere un
rumore bianco gaussiano. In ogni caso, l'opzione \verb|--robust| calcola la
matrice di covarianza dei parametri usando lo stimatore
``sandwich'' di Bollerslev-Wooldridge, quindi le stime prodotte da
\app{gretl} possono essere considerate QML anche con disturbi non normali.

Esempio:
\begin{code}
  garch p q ; y const x
\end{code}
dove \verb|p| � il grado (non negativo) di $A(L)$ e \verb|q| � il grado
(strettamente positivo) di $B(L)$.

\section{Cointegrazione e modelli vettoriali a correzione d'errore}
\label{vecm-explanation}

\subsection{Il test di cointegrazione di Johansen}
\label{sec:johansen-test}

Il test di Johansen per la cointegrazione deve tenere conto di quali
ipotesi vengono fatte a proposito dei termini deterministici, per cui
si possono individuare i ben noti ``cinque casi''. Una presentazione
approfondita dei cinque casi richiede una certa quantit� di algebra
matriciale, ma � possibile dare un'intuizione del problema per mezzo
di un semplice esempio.
    
Si consideri una serie $x_t$ che si comporta nel modo seguente
%      
\[ x_t = m + x_{t-1} + \varepsilon_t \]
%
dove $m$ � un numero reale e $\varepsilon_t$ � un processo ``white
noise''.  Come si pu� facilmente mostrare, $x_t$ � un ``random walk''
che fluttua intorno a un trend deterministico con pendenza $m$. Nel
caso particolare in cui $m$ = 0, il trend deterministico scompare e
$x_t$ � un puro random walk.
    
Si consideri ora un altro processo $y_t$, definito da
%      
\[ y_t = k + x_t + u_t \]
%
dove, ancora, $k$ � un numero reale e $u_t$ � un processo white noise.
Poich� $u_t$ � stazionario per definizione, $x_t$ e $y_t$ sono
cointegrate, ossia la loro differenza
%      
\[ z_t = y_t - x_t = k + u_t \]
%	
� un processo stazionario. Per $k$ = 0, $z_t$ � un semplice white
noise a media zero, mentre per $k$ $\ne$ 0 il processo $z_t$ � white
noise con media diversa da zero.
  
Dopo alcune semplici sostituzioni, le due equazioni precedenti possono
essere rappresentate congiuntamente come un sistema VAR(1)
%      
\[ \left[ \begin{array}{c} y_t \\ x_t \end{array} \right] = \left[
  \begin{array}{c} k + m \\ m \end{array} \right] + \left[
  \begin{array}{rr} 0 & 1 \\ 0 & 1 \end{array} \right] \left[
  \begin{array}{c} y_{t-1} \\ x_{t-1} \end{array} \right] + \left[
  \begin{array}{c} u_t + \varepsilon_t \\ \varepsilon_t \end{array}
\right] \]
%	
o in forma VECM
%      
\begin{eqnarray*}
  \left[  \begin{array}{c} \Delta y_t \\ \Delta x_t \end{array} \right]  & = & 
  \left[  \begin{array}{c} k + m \\ m \end{array} \right] +
  \left[  \begin{array}{rr} -1 & 1 \\ 0 & 0 \end{array} \right] 
  \left[  \begin{array}{c} y_{t-1} \\ x_{t-1} \end{array} \right] + 
  \left[  \begin{array}{c} u_t + \varepsilon_t \\ \varepsilon_t \end{array} \right] = \\
  & = & 
  \left[  \begin{array}{c} k + m \\ m \end{array} \right] +
  \left[  \begin{array}{r} -1 \\ 0 \end{array} \right]
  \left[  \begin{array}{rr} 1 & -1 \end{array} \right] 
  \left[  \begin{array}{c} y_{t-1} \\ x_{t-1} \end{array} \right] + 
  \left[  \begin{array}{c} u_t + \varepsilon_t \\ \varepsilon_t \end{array} \right] = \\
  & = &
  \mu_0 + \alpha \beta^{\prime} \left[  \begin{array}{c} y_{t-1} \\ x_{t-1} \end{array} \right] + \eta_t = 
  \mu_0 + \alpha z_{t-1} + \eta_t ,
\end{eqnarray*}
%	
dove $\beta$ � il vettore di cointegrazione e $\alpha$ � il vettore
dei ``loading'' o ``aggiustamenti''.
     
Possiamo ora considerare tre casi possibili:
    
\begin{enumerate}
\item $m \ne 0$: In questo caso $x_t$ ha un trend, come abbiamo appena
  visto; ne consegue che anche $y_t$ segue un trend lineare perch� in
  media si mantiene a una distanza di $k$ da $x_t$.  Il vettore
  $\mu_0$ non ha restrizioni. Questo � il caso predefinito per il
  comando \cmd{vecm} di gretl.
	
\item $m = 0$ e $k \ne 0$: In questo caso, $x_t$ non ha un trend, e di
  conseguenza neanche $y_t$.  Tuttavia, la distanza media tra $y_t$ e
  $x_t$ � diversa da zero. Il vettore $\mu_0$ � dato da
%	  
  \[
  \mu_0 = \left[ \begin{array}{c} k \\ 0 \end{array} \right]
  \]
%	    
  che � non nullo, quindi il VECM mostrato sopra ha un termine
  costante.  La costante, tuttavia � soggetta alla restrizione che il
  suo secondo elemento deve essere pari a 0. Pi� in generale,
  $\mu$\ensuremath{_{0}} � un multiplo del vettore $\alpha$. Si noti
  che il VECM potrebbe essere scritto anche come
%	  
  \[
  \left[ \begin{array}{c} \Delta y_t \\ \Delta x_t \end{array} \right]
  = \left[ \begin{array}{r} -1 \\ 0 \end{array} \right] \left[
    \begin{array}{rrr} 1 & -1 & -k \end{array} \right] \left[
    \begin{array}{c} y_{t-1} \\ x_{t-1} \\ 1 \end{array} \right] +
  \left[ \begin{array}{c} u_t + \varepsilon_t \\ \varepsilon_t
    \end{array} \right]
  \]
%	   
  che incorpora l'intercetta nel vettore di cointegrazione. Questo �
  il caso di ``costante vincolata''; pu� essere specificato nel
  comando \cmd{vecm} di gretl usando l'opzione \verb+--rc+.
	
\item $m = 0$ e $k = 0$: Questo caso � il pi� vincolante: chiaramente,
  n� $x_t$ n� $y_t$ hanno un trend, e la loro distanza media � zero.
  Anche il vettore $\mu_0$ vale 0, quindi questo caso pu� essere
  chiamato ``senza costante''.  Questo caso � specificato usando
  l'opzione \verb+--nc+ con \cmd{vecm}.
	
\end{enumerate}


Nella maggior parte dei casi, la scelta tra le tre possibilit� si basa
su un misto di osservazione empirica e di ragionamento economico. Se
le variabili in esame sembrano seguire un trend lineare, � opportuno
non imporre alcun vincolo all'intercetta. Altrimenti occorre chiedersi
se ha senso specificare una relazione di cointegrazione che includa
un'intercetta diversa da zero. Un esempio appropriato potrebbe essere
la relazione tra due tassi di interesse: in generale questi non hanno
un trend, ma il VAR potrebbe comunque avere un'intercetta perch� la
differenza tra i due (lo ``spread'' sui tassi d'interesse) potrebbe
essere stazionaria attorno a una media diversa da zero (ad esempio per
un premio di liquidit� o di rischio).
    
L'esempio precedente pu� essere generalizzato in tre direzioni:
    
\begin{enumerate}
\item Se si considera un VAR di ordine maggiore di 1, l'algebra si
  complica, ma le conclusioni sono identiche.
\item Se il VAR include pi� di due variabili endogene, il rango di
  cointegrazione $r$ pu� essere maggiore di 1. In questo caso $\alpha$
  � una matrice con $r$ colonne e il caso con la costante vincolata
  comporta che $\mu_0$ sia una combinazione lineare delle colonne di
  $\alpha$.
\item Se si include un trend lineare nel modello, la parte
  deterministica del VAR diventa $\mu_0$ + $\mu_1$. Il ragionamento �
  praticamente quello visto sopra, tranne per il fatto che
  l'attenzione � ora posta su $\mu_1$ invece che su $\mu_0$. La
  controparte del caso con ``costante vincolata'' discusso sopra � un
  caso con ``trend vincolato'', cos� che le relazioni di
  cointegrazione includono un trend, ma la differenza prima delle
  variabili in questione no. Nel caso di un trend non vincolato, il
  trend appare sia nelle relazioni di cointegrazione sia nelle
  differenze prime, il che corrisponde alla presenza di un trend
  quadratico nelle variabili (espresse in livelli). Questi due casi
  sono specificati rispettivamente con le opzioni \verb+--crt+ e
  \verb+--ct+ del comando \cmd{vecm}.
\end{enumerate}


%%% Local Variables: 
%%% mode: latex
%%% TeX-master: "gretl-guide-it"
%%% End: 


\chapter{Cointegrazione e modelli vettoriali a correzione d'errore}
\label{chap:vecm}

\section{Introduzione}
\label{sec:VECM-intro}

I concetti correlati di cointegrazione e correzione d'errore sono stati al
centro della ricerca in macroeconometria negli ultimi anni. L'aspetto
interessante del Modello Vettoriale a Correzione di Errore (VECM) consiste nel
fatto che permette al ricercatore di inserire una rappresentazione di relazioni
di equilibrio economico in una specificazione abbastanza ricca basata sulle
serie storiche. Questo approccio supera l'antica dicotomia tra i modelli
strutturali, che rappresentavano fedelmente la teoria macroeconomica ma non si
adattavano ai dati, e l'analisi delle serie storiche, che era pi� precisa nel
riprodurre l'andamento dei dati, ma di difficile, se non impossibile,
interpretazione in termini di teoria economica.

L'idea basilare della cointegrazione � strettamente collegata al concetto di
radici unitarie (si veda la sezione~\ref{sec:uroot}).  Si supponga di avere un
insieme di variabili macroeconomiche di interesse, e di non poter rifiutare
l'ipotesi che alcune di queste variabili, considerate individualmente, siano
non-stazionarie. In particolare, si supponga che un sottoinsieme di queste
variabili siano individualmente integrate di ordine 1, o I(1), ossia che non
siano stazionarie, ma che la loro differenza prima sia stazionaria. Dati i
problemi di tipo statistico che sono associati all'analisi dei dati non
stazionari (ad esempio il problema della regressione spuria), l'approccio
tradizionale in questo caso consiste nel prendere la differenza prima delle
variabili prima di procedere con l'analisi statistica.

In questo modo per�, si perde informazione importante. Pu� darsi che mentre le
variabili sono I(1) prese singolarmente, esista una loro combinazione lineare
che sia invece stazionaria, ossia I(0) (potrebbe esserci anche pi� di una
combinazione lineare). In altri termini, mentre l'insieme delle variabili �
libero di muoversi nel tempo, esistono comunque delle relazioni che legano fra
di loro le variabili; � possibile interpretare queste relazioni, o
\emph{vettori di cointegrazione} come condizioni di equilibrio.

Ad esempio, ipotizziamo di scoprire che la quantit� di moneta, $M$, il livello
dei prezzi, $P$, il tasso di interesse nominale, $R$, e l'output, $Y$, siano
tutti I(1). Secondo la teoria standard della domanda di moneta, dovremmo
comunque aspettarci una relazione di equilibrio tra la quantit� di moneta reale,
il tasso d'interesse e l'output; ad esempio
\[
m - p = \gamma_0 + \gamma_1 y + \gamma_2 r \qquad \gamma_1 > 0,
\gamma_2 < 0
\]
dove le variabili in minuscolo indicano i logaritmi. In equilibrio si ha quindi
\[
m - p - \gamma_1 y - \gamma_2 r = \gamma_0
\]
Nella realt� non ci si aspetta che questa condizione sia soddisfatta in ogni
periodo, ma occorre ammettere la possibilit� di disequilibri di breve periodo.
Ma se il sistema ritorna all'equilibrio dopo un disturbo, ne consegue che
il vettore $x = (m, p, y, r)'$ � limitato da un vettore di cointegrazione
$\beta' = (\beta_1, \beta_2, \beta_3, \beta_4)$, tale che $\beta'x$ �
stazionario (con una media pari a $\gamma_0$). Inoltre, se l'equilibrio �
caratterizzato correttamente dal semplice modello visto sopra, si ha $\beta_2 =
-\beta_1$, $\beta_3 < 0$ e $\beta_4 > 0$. Queste propriet� sono testabili
attraverso l'analisi di cointegrazione.

Questa analisi consiste tipicamente in tre passi:
\begin{enumerate}
\item Test per verificare il numero di vettori di cointegrazione, ossia il 
  \emph{rango di cointegrazione} del sistema.
\item Stima di un VECM di rango appropriato, non soggetto ad altre restrizioni.
\item Test dell'interpretazione dei vettori di cointegrazione come condizioni di
  equilibrio, usando le restrizioni sugli elementi di questi vettori.
\end{enumerate}

Le sezioni seguenti approfondiscono ognuno dei passi, aggiungendo altre
considerazioni econometriche e spiegando come implementare l'analisi usando
\app{gretl}.

\section{Modelli vettoriali a correzione di errore (VECM) come rappresentazione di
un sistema cointegrato}
\label{sec:VECM-rep}

Si consideri un VAR di ordine $p$ con una parte deterministica data da $\mu_t$
(tipicamente un polinomio nel tempo). � possibile scrivere il processo
$n$-variato $y_t$ come
\begin{equation}
  \label{eq:VECM-VAR}
  y_t = \mu_t + A_1 y_{t-1} + A_2 y_{t-2} + \cdots + A_p y_{t-p} +
  \epsilon_t 
\end{equation}
Ma poich� $y_t \equiv y_{t-1} - \Delta y_t$ e $y_{t-i} \equiv
y_{t-1} - (\Delta y_{t-1} + \Delta y_{t-2} + \cdots + \Delta
y_{t-i+1})$, � possibile riscrivere l'equazione nel modo seguente:
\begin{equation}
  \label{eq:VECM}
  \Delta y_t = \mu_t + \Pi y_{t-1} + \sum_{i=1}^{p-1} \Gamma_i \Delta
  y_{t-i} + \epsilon_t ,
\end{equation}
dove $\Pi = \sum_{i=1}^p A_i$ e $\Gamma_k = -\sum_{i=k}^p A_i$.
Questa � la rappresentazione VECM della (\ref{eq:VECM-VAR}).

L'interpretazione della (\ref{eq:VECM}) dipende in modo cruciale da $r$, il
rango della matrice $\Pi$.
\begin{itemize}
\item Se $r = 0$, i processi sono tutti I(1) e non cointegrati.
\item Se $r = n$, $\Pi$ � invertibile e i processi sono tutti I(0).
\item La cointegrazione accade in tutti i casi intermedi, quando
  $0 < r < n$ e $\Pi$ pu� essere scritta come $\alpha \beta'$. In 
  questo caso, $y_t$ � I(1), ma la combinazione $z_t = \beta'y_t$ � I(0).
  Se, ad esempio, $r=1$ e il primo elemento di $\beta$ fosse $-1$, si
  potrebbe scrivere $z_t = -y_{1,t} + \beta_2 y_{2,t} + \cdots + \beta_n y_{n,t}$,
  che � equivalente a dire che
  \[
    y_{1_t} = \beta_2 y_{2,t} + \cdots + \beta_n y_{n,t} - z_t
  \]
  � una relazione di equilibrio di lungo periodo: le deviazioni
  $z_t$ possono non essere pari a zero, ma sono stazionarie. In questo caso,
  la (\ref{eq:VECM}) pu� essere scritta come
  \begin{equation}
    \label{eq:VECMab}
    \Delta y_t = \mu_t + \alpha \beta' y_{t-1} + \sum_{i=1}^{p-1} \Gamma_i 
    \Delta y_{t-i} + \epsilon_t .
  \end{equation}
  Se $\beta$ fosse noto, $z_t$ sarebbe osservabile e tutti i restanti parametri
  potrebbero essere stimati con OLS. In pratica, la procedura stima per prima
  cosa $\beta$ e tutto il resto poi.
\end{itemize}

Il rango di $\Pi$ viene analizzato calcolando gli autovalori di una matrice ad
essa strettamente legata che ha rango pari a quello di $\Pi$, ma che per
costruzione � simmetrica e semidefinita positiva. Di conseguenza, tutti i suoi
autovalori sono reali e non negativi e i test sul rango di $\Pi$ possono quindi
essere condotti verificando quanti autovalori sono pari a 0.

Se tutti gli autovalori sono significativamente diversi da 0, tutti i processi
sono stazionari. Se, al contrario, c'� almeno un autovalore pari a 0, allora
il processo $y_t$ � integrato, anche se qualche combinazione lineare $\beta'y_t$
potrebbe essere stazionaria. All'estremo opposto, se non ci sono autovalori
significativamente diversi da 0, non solo il processo $y_t$ � non-stazionario,
ma vale lo stesso per qualsiasi combinazione lineare $\beta'y_t$; in altre
parole non c'� alcuna cointegrazione.

La stima procede tipicamente in due passi: per prima cosa si esegue una serie di
test per determinare $r$, il rango di cointegrazione. Quindi, per un certo rango
vengono stimati i parametri dell'equazione (\ref{eq:VECMab}). I due comandi
offerti da \app{gretl} per compiere queste operazioni sono rispettivamente
\texttt{coint2} e \texttt{vecm}.

La sintassi di \texttt{coint2} �
\begin{code}
  coint2 p listay [ ; listax [ ; listaz ] ]
\end{code}
dove \texttt{p} � il numero di ritardi nella (\ref{eq:VECM-VAR}),
\texttt{listay} � una lista che contiene le variabili $y_t$,
\texttt{listax} � una lista opzionale di variabili esogene, e
\texttt{listaz} � un'altra lista opzionale di variabili esogene il cui
effetto � ipotizzato essere confinato alle relazioni di cointegrazione.

La sintassi di \texttt{vecm} �
\begin{code}
  vecm p r listay [ ; listax [ ; listaz ] ]
\end{code}
dove \texttt{p} � il numero di ritardi nella (\ref{eq:VECM-VAR}),
\texttt{r} � il rango di cointegrazione, e le liste \texttt{listay},
\texttt{listax} e \texttt{listaz} hanno la stessa funzione che hanno nel comando
\texttt{coint2}.

Entrambi i comandi supportano opzioni specifiche per trattare la componente
deterministica $\mu_t$; queste sono illustrate nella sezione seguente.

\section{Interpretazione delle componenti deterministiche}
\label{sec:coint-5cases}

L'inferenza statistica nell'ambito dei sistemi cointegrati dipende dalle ipotesi
fatte a proposito dei termini deterministici, per cui si possono individuare i
ben noti ``cinque casi''.

Nell'equazione (\ref{eq:VECM}), il termine $\mu_t$ di solito assume la forma
seguente:
\[
  \mu_t = \mu_0 + \mu_1 \cdot t .
\]
Affinch� il modello si adatti nel modo migliore alle caratteristiche dei dati,
occorre risolvere una questione preliminare. I dati sembrano seguire un trend
deterministico? In caso positivo, si tratta di un trend lineare o quadratico?

Una volta stabilito questo, bisogna imporre restrizioni coerenti su $\mu_0$
e $\mu_1$. Ad esempio, se i dati non esibiscono un trend evidente, ci�
significa che $\Delta y_t$ vale zero in media, quindi � ragionevole assumere
che anche il suo valore atteso sia zero. Scriviamo l'equazione
(\ref{eq:VECM}) come
\begin{equation}
  \label{eq:VECM-poly}
  \Gamma(L) \Delta y_t = \mu_0 + \mu_1 \cdot t + \alpha z_{t-1} +
  \epsilon_t ,
\end{equation}
dove si ipotizza che $z_{t} = \beta' y_{t}$ sia stazionario e quindi possieda
momenti finiti. Prendendo i valori attesi non condizionati, otteniamo
\[ 
  0 = \mu_0 + \mu_1 \cdot t + \alpha m_z .
\]
Visto che il termine al primo membro non dipende da $t$, il vincolo
$\mu_1 = 0$ pare appropriato. Per quanto riguarda $\mu_0$, ci sono solo due
modi per rendere vera l'espressione vista sopra: $\mu_0 = 0$ con $m_z = 0$,
oppure $\mu_0$ esattamente uguale a $-\alpha m_z$. Questa ultima possibilit�
� meno restrittiva nel senso che il vettore $\mu_0$ pu� non essere pari a zero,
ma � vincolato ad essere una combinazione lineare delle colonne di
$\alpha$. Ma in questo caso $\mu_0$ pu� essere scritto come
$\alpha \cdot c$, si pu� scrivere la (\ref{eq:VECM-poly}) come
\[
  \Gamma(L) \Delta y_t = \alpha \left[ \beta' \quad c \right] 
  \left[ \begin{array}{c} y_{t-1} \\ 1 \end{array} \right]  
  + \epsilon_t .
\]
La relazione di lungo periodo contiene quindi un'intercetta. Questo tipo
di restrizione � scritta di solito nel modo seguente:
\[
  \alpha'_{\perp} \mu_0 = 0 ,
\]
dove $\alpha_{\perp}$ � lo spazio nullo sinistro della matrice $\alpha$.

� possibile dare un'intuizione del problema per mezzo di un semplice esempio.
Si consideri una serie $x_t$ che si comporta nel modo seguente:
%      
\[ x_t = m + x_{t-1} + \varepsilon_t \] 
%
dove $m$ � un numero reale e $\varepsilon_t$ � un processo ``rumore bianco''
(``white noise''): $x_t$ � quindi una ``passeggiata aleatoria'' (``random
walk'') con deriva pari a $m$. Nel caso particolare in cui $m$ = 0, la deriva
scompare e $x_t$ � una passeggiata aleatoria pura.
    
Si consideri ora un altro processo $y_t$, definito da
%      
\[ y_t = k + x_t + u_t \] 
%
dove, ancora, $k$ � un numero reale e $u_t$ � un processo a rumore bianco.
Poich� $u_t$ � stazionario per definizione, $x_t$ e $y_t$ sono
cointegrate, ossia la loro differenza
%      
\[ z_t = y_t - x_t = k + u_t \]
%	
� un processo stazionario. Per $k$ = 0, $z_t$ � un semplice rumore bianco
a media zero, mentre per $k$ $\ne$ 0 il processo $z_t$ � un rumore bianco
con media diversa da zero.
  
Dopo alcune semplici sostituzioni, le due equazioni precedenti possono
essere rappresentate congiuntamente come un sistema VAR(1)
%      
\[ \left[ \begin{array}{c} y_t \\ x_t \end{array} \right] = \left[
  \begin{array}{c} k + m \\ m \end{array} \right] + \left[
  \begin{array}{rr} 0 & 1 \\ 0 & 1 \end{array} \right] \left[
  \begin{array}{c} y_{t-1} \\ x_{t-1} \end{array} \right] + \left[
  \begin{array}{c} u_t + \varepsilon_t \\ \varepsilon_t \end{array}
\right] \]
%	
o in forma VECM
%      
\begin{eqnarray*}
  \left[  \begin{array}{c} \Delta y_t \\ \Delta x_t \end{array} \right]  & = & 
  \left[  \begin{array}{c} k + m \\ m \end{array} \right] +
  \left[  \begin{array}{rr} -1 & 1 \\ 0 & 0 \end{array} \right] 
  \left[  \begin{array}{c} y_{t-1} \\ x_{t-1} \end{array} \right] + 
  \left[  \begin{array}{c} u_t + \varepsilon_t \\ \varepsilon_t \end{array} \right] = \\
  & = & 
  \left[  \begin{array}{c} k + m \\ m \end{array} \right] +
  \left[  \begin{array}{r} -1 \\ 0 \end{array} \right]
  \left[  \begin{array}{rr} 1 & -1 \end{array} \right] 
  \left[  \begin{array}{c} y_{t-1} \\ x_{t-1} \end{array} \right] + 
  \left[  \begin{array}{c} u_t + \varepsilon_t \\ \varepsilon_t \end{array} \right] = \\
  & = & 
  \mu_0 + \alpha \beta^{\prime} \left[  \begin{array}{c} y_{t-1} \\ x_{t-1} \end{array} \right] + \eta_t = 
  \mu_0 + \alpha z_{t-1} + \eta_t ,
\end{eqnarray*}
%	
dove $\beta$ � il vettore di cointegrazione e $\alpha$ � il vettore
dei ``loading'' o ``aggiustamenti''.
     
Possiamo ora considerare tre casi possibili:
    
\begin{enumerate}
\item $m$ $\ne$ 0: in questo caso $x_t$ ha un trend, come abbiamo appena
  visto; ne consegue che anche $y_t$ segue un trend lineare perch� in
  media si mantiene a una distanza fissa da $x_t$ pari a $k$.  Il vettore
  $\mu_0$ non ha restrizioni.
	
\item $m$ = 0 e $k$ $\ne$ 0: in questo caso, $x_t$ non ha un trend, e di
  conseguenza neanche $y_t$.  Tuttavia, la distanza media tra $y_t$ e
  $x_t$ � diversa da zero. Il vettore $\mu_0$ � dato da
%	  
  \[
  \mu_0 = \left[ \begin{array}{c} k \\ 0 \end{array} \right]
  \]
%	    
  che � non nullo, quindi il VECM mostrato sopra ha un termine
  costante.  La costante, tuttavia � soggetta alla restrizione che il
  suo secondo elemento deve essere pari a 0. Pi� in generale,
  $\mu_0$ � un multiplo del vettore $\alpha$. Si noti
  che il VECM potrebbe essere scritto anche come
%	  
  \[
  \left[ \begin{array}{c} \Delta y_t \\ \Delta x_t \end{array} \right]
  = \left[ \begin{array}{r} -1 \\ 0 \end{array} \right] \left[
    \begin{array}{rrr} 1 & -1 & -k \end{array} \right] \left[
    \begin{array}{c} y_{t-1} \\ x_{t-1} \\ 1 \end{array} \right] +
  \left[ \begin{array}{c} u_t + \varepsilon_t \\ \varepsilon_t
    \end{array} \right]
  \]
%	   
  che incorpora l'intercetta nel vettore di cointegrazione. Questo �
  il caso chiamato ``costante vincolata''.
	
\item $m$ = 0 e $k$ = 0: questo caso � il pi� vincolante: chiaramente,
  n� $x_t$ n� $y_t$ hanno un trend, e la loro distanza media � zero.
  Anche il vettore $\mu_0$ vale 0, quindi questo caso pu� essere
  chiamato ``senza costante''.

\end{enumerate}

Nella maggior parte dei casi, la scelta tra queste tre possibilit� si basa
su un misto di osservazione empirica e di ragionamento economico. Se
le variabili in esame sembrano seguire un trend lineare, � opportuno
non imporre alcun vincolo all'intercetta. Altrimenti occorre chiedersi
se ha senso specificare una relazione di cointegrazione che includa
un'intercetta diversa da zero. Un esempio appropriato potrebbe essere
la relazione tra due tassi di interesse: in generale questi non hanno
un trend, ma il VAR potrebbe comunque avere un'intercetta perch� la
differenza tra i due (lo ``spread'' sui tassi d'interesse) potrebbe
essere stazionaria attorno a una media diversa da zero (ad esempio per
un premio di liquidit� o di rischio).
    
L'esempio precedente pu� essere generalizzato in tre direzioni:
    
\begin{enumerate}
\item Se si considera un VAR di ordine maggiore di 1, l'algebra si
  complica, ma le conclusioni sono identiche.
\item Se il VAR include pi� di due variabili endogene, il rango di
  cointegrazione $r$ pu� essere maggiore di 1. In questo caso $\alpha$
  � una matrice con $r$ colonne e il caso con la costante vincolata
  comporta che $\mu_0$ sia una combinazione lineare delle colonne di
  $\alpha$.
\item Se si include un trend lineare nel modello, la parte
  deterministica del VAR diventa $\mu_0$ + $\mu_1$. Il ragionamento �
  praticamente quello visto sopra, tranne per il fatto che
  l'attenzione � ora posta su $\mu_1$ invece che su $\mu_0$. La
  controparte del caso con ``costante vincolata'' discusso sopra � un
  caso con ``trend vincolato'', cos� che le relazioni di
  cointegrazione includono un trend, ma la differenza prima delle
  variabili in questione no. Nel caso di un trend non vincolato, il
  trend appare sia nelle relazioni di cointegrazione sia nelle
  differenze prime, il che corrisponde alla presenza di un trend
  quadratico nelle variabili (espresse in livelli).
\end{enumerate}

Per gestire i cinque casi, \app{gretl} fornisce le seguenti opzioni per i
comandi \texttt{coint2} e \texttt{vecm}:
\begin{center}
  \begin{tabular}{ccl}
    $\mu_t$ & \textit{Opzione} & \textit{Descrizione} \\ [4pt]
    0 & \verb|--nc| & Senza costante \\
    $\mu_0, \alpha_{\perp}'\mu_0 = 0 $ &  \verb|--rc| & Costante vincolata \\
    $\mu_0$ & (predefinito) & Costante non vincolata \\
    $\mu_0 + \mu_1 t , \alpha_{\perp}'\mu_1 = 0$ &  \verb|--crt| &
     Costante + trend vincolato \\
    $\mu_0 + \mu_1 t$ &  \verb|--ct| & 
    Costante + trend non vincolato 
   \end{tabular}
\end{center}
Si noti che le opzioni viste sopra sono mutualmente esclusive. Inoltre, �
possibile usare l'opzione \verb|--seasonal| per aggiungere a $\mu_t$ delle
dummy stagionali centrate. In ogni caso, i p-value sono calcolati con le
approssimazioni indicate in Doornik (1998).

\section{I test di cointegrazione di Johansen}
\label{sec:johansen-test}

I due test di cointegrazione di Johansen vengono usati per stabilire il
rango di $\beta$, in altre parole il numero di vettori di cointegrazione
del sistema. Essi sono il test ``$\lambda$-max'', per le ipotesi sui singoli
autovalori, e il test ``traccia'', per le ipotesi congiunte.
Si supponga che gli autovalori $\lambda_i$ siano ordinati dal maggiore al
minore. L'ipotesi nulla per il test  ``$\lambda$-max'' sul
$i$-esimo autovalore � che sia $\lambda_i = 0$. Invece, il test traccia
corrispondente considera l'ipotesi che sia $\lambda_j = 0$ per ogni
$j \ge i$.

Il comando \cmd{coint2} di \app{gretl} esegue questi due test; la voce
corrispondente nel men� dell'interfaccia grafica � ``Modello, Serie Storiche,
COINT - Test di cointegrazione, Johansen''.

Come nel test ADF, la distribuzione asintotica dei test varia a seconda
della componente deterministica $\mu_t$ incluso nel VAR (si veda la sezione
\ref{sec:coint-5cases}). Il codice seguente usa il file di dati
\cmd{denmark}, fornito insieme a \app{gretl}, per replicare l'esempio proposto
da Johansen nel suo libro del 1995.
%
\begin{code}
open denmark
coint2 2 LRM LRY IBO IDE --rc --seasonal
\end{code}
%
In questo caso, il vettore $y_t$ nell'equazione (\ref{eq:VECM}) comprende le
quattro variabili \cmd{LRM}, \cmd{LRY}, \cmd{IBO}, \cmd{IDE}. Il numero dei
ritardi equivale a $p$ nella (\ref{eq:VECM}) (ossia, il numero dei ritardi del
modello scritto in forma VAR). Di seguito � riportata parte
dell'output:

\begin{center}
\begin{code}
Test di Johansen:
Numero di equazioni = 4
Ordine dei ritardi = 2
Periodo di stima: 1974:3 - 1987:3 (T = 53)

Caso 2: costante vincolata
Rango Autovalore Test traccia p-value   Test Lmax  p-value
   0    0,43317     49,144 [0,1284]     30,087 [0,0286]
   1    0,17758     19,057 [0,7833]     10,362 [0,8017]
   2    0,11279     8,6950 [0,7645]     6,3427 [0,7483]
   3   0,043411     2,3522 [0,7088]     2,3522 [0,7076]
\end{code}
\end{center}

Sia il test traccia, sia quello $\lambda$-max portano ad accettare l'ipotesi nulla
che il pi� piccolo autovalore valga 0 (ultima riga della tabella), quindi possiamo
concludere che le serie non sono stazionarie. Tuttavia, qualche loro combinazione
lineare potrebbe essere I(0), visto che il test $\lambda$-max rifiuta l'ipotesi che
il rango di $\Pi$ sia 0 (anche se il test traccia d� un'indicazione meno netta
in questo senso, con un p-value pari a $0.1284$).

\section{Identificazione dei vettori di cointegrazione}
\label{sec:johansen-ident}

Il problema centrale nella stima dell'equazione (\ref{eq:VECM}) consiste nel
trovare una stima di $\Pi$ che abbia rango $r$ per costruzione, cos� che possa
essere scritta come $\Pi = \alpha \beta'$, dove $\beta$ � la matrice che
contiene i vettori di cointegrazione e $\alpha$ contiene i coefficienti di
``aggiustamento'' o ``loading'', per cui le variabili endogene rispondono
a una deviazione dall'equilibrio nel periodo precedente.

Senza ulteriore specificazione, il problema ha molte soluzioni (in effetti
ne ha infinite). I parametri $\alpha$ e $\beta$ sono sotto-identificati: se
tutte le colonne di $\beta$ sono vettori di cointegrazione, anche qualsiasi loro
combinazione lineare arbitraria � un vettore di cointegrazione. In altre parole,
se $\Pi = \alpha_0 \beta_0'$ per specifiche matrici $\alpha_0$ e $\beta_0$, allora
$\Pi$ � anche uguale a $(\alpha_0 \Sigma)(\Sigma^{-1} \beta_0')$ per qualsiasi
matrice conformabile e non singolare $\Sigma$. Per trovare una soluzione unica,
� quindi necessario imporre alcune restrizioni su $\alpha$ e/o
$\beta$. Si pu� dimostrare che il numero minimo di restrizioni necessarie per
garantire l'identificazione � pari a $r^2$. Un primo passo banale consiste nel
normalizzare un coefficiente per ogni colonna rendendolo pari a 1 (o a $-1$, a
seconda dei gusti), il che aiuta anche a interpretare i restanti coefficienti
come parametri delle relazioni di equilibrio; tuttavia, questo basta solo nel
caso in cui $r=1$.

Il metodo usato da \app{gretl} in modo predefinito � chiamato ``normalizzazione
di Phillips'', o ``rappresentazione triangolare''\footnote{Per fare confronti
  con altri studi, potrebbe essere necessario normalizzare $\beta$ in modo
  diverso. Usando il comando \texttt{set} � possibile scrivere
  \verb|set vecm_norm diag| per scegliere la normalizzazione che scala
  le colonne della $\beta$ originaria in modo che valga $\beta_{ij} = 1$ per $i=j$
  e $i \leq r$, come � mostrato nella sezione empirica di Boswijk e
  Doornik (2004).  Una soluzione alternativa � \verb+set vecm_norm first+,
  che scala $\beta$ in modo che gli elementi della prima riga siano pari a 1.
  Per sopprimere del tutto la normalizzazione basta usare
  \verb+set vecm_norm none+, mentre per tornare all'impostazione predefinita
  \texttt{set vecm\_norm phillips}.}.
Il punto di partenza consiste nello scrivere $\beta$ in forma partizionata, come in
\[
  \beta = \left[
    \begin{array}{c} \beta_1 \\ \beta_2  \end{array}
    \right] ,
\]
dove $\beta_1$ � una matrice $r \times r$ e  $\beta_2$ � $(n-r)
\times r$. Assumendo che $\beta_1$ abbia rango pieno, $\beta$ pu� essere
post-moltiplicata da $\beta_1^{-1}$, ottenendo
\[
  \hat{\beta} = \left[
    \begin{array}{c} I \\ \beta_2 \beta_1^{-1}  \end{array}
    \right] =
    \left[
    \begin{array}{c} I \\ -B \end{array}
  \right]  ,
\]

I coefficienti prodotti da \app{gretl} sono i $\hat{\beta}$, mentre
$B$ � nota come matrice dei coefficienti non vincolati. Nei termini della
relazione di equilibrio sottostante, la normalizzazione di Phillips
esprime il sistema di $r$ relazioni di equilibrio come
  \begin{eqnarray*}
    y_{1,t} & = & b_{1,r+1} y_{r+1,t} + \ldots + b_{1,n} y_{n,t} \\
    y_{2,t} & = & b_{2,r+1} y_{r+1,t} + \ldots + b_{2,n} y_{n,t} \\
    & \vdots & \\
    y_{r,t} & = & b_{r,r+1} y_{r+1,t} + \ldots + b_{r,n} y_{r,t} 
  \end{eqnarray*}
dove le prime $r$ variabili sono espresse come funzione delle restanti
$n-r$.

Anche se la rappresentazione triangolare assicura la soluzione del problema
statistico della stima di $\beta$, le relazioni di equilibrio che risultano
possono essere difficili da interpretare. In questo caso, l'utente potrebbe
voler cercare l'identificazione specificando manualmente il sistema di
$r^2$ vincoli che \app{gretl} user� per produrre una stima di
$\beta$.

Come esempio, si consideri il sistema di domanda di moneta presentato nella
sezione 9.6 di Verbeek (2004). Le variabili usate sono \texttt{m} (il logaritmo
della quantit� di moneta reale M1), \texttt{infl} (l'inflazione), \texttt{cpr}
(i tassi di interesse sui ``commercial paper''), \texttt{y} (il logaritmo del
PIL reale) e \texttt{tbr} (i tassi di interesse sui ``Treasury bill'')\footnote{Questo
dataset � disponibile nel pacchetto \texttt{verbeek}; si veda la pagina
  \url{http://gretl.sourceforge.net/gretl_data_it.html}.}.

La stima di $\beta$ si pu� ottenere con questi comandi:
\begin{code}
open money.gdt 
smpl 1954:1 1994:4 
vecm 6 2 m infl cpr y tbr --rc
\end{code}
e la parte rilevante dei risultati � questa:
\begin{code}
Stime Massima verosimiglianza usando le osservazioni 1954:1-1994:4 (T = 164)
Rango di cointegrazione = 2
Caso 2: costante vincolata

Vettori di cointegrazione (errori standard tra parentesi)

m           1.0000       0.0000 
           (0.0000)     (0.0000) 
infl        0.0000       1.0000 
           (0.0000)     (0.0000) 
cpr        0.56108      -24.367 
          (0.10638)     (4.2113) 
y         -0.40446     -0.91166 
          (0.10277)     (4.0683) 
tbr       -0.54293       24.786 
          (0.10962)     (4.3394) 
const      -3.7483       16.751 
          (0.78082)     (30.909) 
\end{code}
L'interpretazione dei coefficienti della matrice di cointegrazione $\beta$
sarebbe pi� semplice se si potesse associare un significato ad ognuna delle sue
colonne. Questo � possibile ipotizzando l'esistenza di due relazioni di lungo
periodo: una equazione di domanda di moneta
\[
  \mbox{\tt m} = c_1 + \beta_1 \mbox{\tt infl} + \beta_2 \mbox{\tt
    y} + \beta_3 \mbox{\tt tbr}
\]
e una equazione per premio sul rischio
\[
 \mbox{\tt cpr} = c_2 + \beta_4 \mbox{\tt infl} +
   \beta_5 \mbox{\tt y} + \beta_6 \mbox{\tt tbr}
\]
che implica che la matrice di cointegrazione pu� essere normalizzata come
\[
  \beta = \left[
    \begin{array}{rr}
      -1 & 0 \\ \beta_1 & \beta_4 \\ 0 & -1 \\ \beta_2 & \beta_5
      \\ \beta_3 & \beta_6 \\ c_1 & c_2
    \end{array}
    \right]
\]

Questa rinormalizzazione pu� essere compiuta per mezzo del comando
\texttt{restrict}, da eseguire dopo il comando \texttt{vecm}, oppure, se si usa
l'interfaccia grafica, scegliendo la voce dal men� ``Test, Vincoli lineari''.
La sintassi per specificare i vincoli � abbastanza intuitiva\footnote{Si noti che
  in questo contesto, stiamo trasgredendo la convenzione usata di solito per gli
  indici delle matrici, visto che usiamo il primo indice per riferirci alla
  \textit{colonna} di $\beta$ (il particolare vettore di cointegrazione).
  Questa � la pratica usata di solito nella letteratura sulla cointegrazione,
  visto che sono le colonne di $\beta$ (le relazioni di cointegrazione o gli
  errori di equilibrio) che interessano in modo particolare.}:
\begin{code}
restrict
  b[1,1] = -1
  b[1,3] = 0
  b[2,1] = 0
  b[2,3] = -1
end restrict
\end{code}
che produce

\begin{code}
Vettori di cointegrazione (errori standard tra parentesi)

m          -1.0000       0.0000 
           (0.0000)     (0.0000) 
infl     -0.023026     0.041039 
        (0.0054666)   (0.027790) 
cpr         0.0000      -1.0000 
           (0.0000)     (0.0000) 
y          0.42545    -0.037414 
         (0.033718)    (0.17140) 
tbr      -0.027790       1.0172 
        (0.0045445)   (0.023102) 
const       3.3625      0.68744 
          (0.25318)     (1.2870) 
\end{code}

\section{Restrizioni sovra-identificanti}
\label{sec:johansen-overid}

Uno degli scopi dell'imporre restrizioni su un sistema VECM � quello di
raggiungere l'identificazione del sistema. Se queste restrizioni sono semplici
normalizzazioni, esse non sono testabili e non dovrebbero avere effetti sulla
verosimiglianza massimizzata. Tuttavia, si potrebbe anche voler formulare
vincoli su $\beta$ e/o $\alpha$ che derivano dalla teoria economica che � alla
base delle relazioni di equilibrio; restrizioni sostanziali di questo tipo sono
quindi testabili attraverso il metodo del rapporto di verosimiglianza.

\app{Gretl} pu� testare restrizioni lineari generali nella forma
\begin{equation}
\label{eq:Rb}
R_b \vec{\beta} = q
\end{equation}
e/o
\begin{equation}
\label{eq:Ra}
R_a \vec{\alpha} = 0
\end{equation}
%
Si noti che la restrizione su $\beta$ pu� essere non-omogenea ($q \neq 0$)
ma la restrizione su $\alpha$ deve essere omogenea. Non sono supportate le
restrizioni non-lineari, n� quelle incrociate tra $\beta$ e $\alpha$.
Nel caso in cui $r > 1$ queste restrizioni possono essere in comune tra tutte le
colonne di $\beta$ (o $\alpha$) o possono essere specifiche a certe colonne
di queste matrici. Questo � il caso discusso in Boswijk (1995) e in Boswijk e Doornik
(2004, capitolo 4.4).

Le restrizioni (\ref{eq:Rb}) e (\ref{eq:Ra}) possono essere scritte in forma
esplicita come
\begin{equation}
\label{eq:vecbeta}
\vec{\beta} = H\phi + h_0
\end{equation}
e
\begin{equation}
\label{eq:vecalpha}
\vec{\alpha'} = G\psi
\end{equation}
rispettivamente, dove $\phi$ e $\psi$ sono i vettori parametrici liberi
associati con $\beta$ e $\alpha$ rispettivamente. Possiamo riferirci
collettivamente a tutti i parametri liberi come $\theta$ (il vettore colonna
formato dalla concatenazione di $\phi$ e $\psi$).  \app{Gretl} usa internamente
questa rappresentazione per testare le restrizioni.

Se la lista delle restrizioni passata al comando \texttt{restrict} contiene pi�
restrizioni di quelle necessarie a garantire l'identificazione, viene eseguito
un test LR; inoltre � possibile usare l'opzione \verb|--full| con il comando
\texttt{restrict}, in modo che vengano mostrate le stime complete per il sistema
vincolato (inclusi i termini $\Gamma_i$), e che il sistema vincolato diventi
cos� il ``modello attuale'' ai fini di ulteriori test. In questo modo �
possibile eseguire test cumulativi, come descritto nel capitolo 7 di
Johansen (1995).

\subsection{Sintassi}
\label{sec:vecm-restr-syntax}

La sintassi completa per specificare le restrizioni � un'estensione di quella
mostrata nella sezione precedente. All'interno di un blocco
\texttt{restrict}\ldots\texttt{end restrict} � possibile usare dichiarazione
della forma
\begin{center}
  \texttt{\emph{combinazione lineare di parametri}} = \emph{\texttt{scalare}}
\end{center}
dove la combinazione lineare di parametri comprende la somma ponderata di
elementi individuali di $\beta$ o $\alpha$ (ma non di entrambi nella stessa
combinazione); lo scalare al secondo membro deve essere pari a 0 per
combinazioni che riguardano $\alpha$, ma pu� essere qualsiasi numero reale per
combinazioni che riguardano $\beta$. Di seguito vengono mostrati alcuni esempi
di restrizioni valide:
\begin{code}
  b[1,1] = 1.618
  b[1,4] + 2*b[2,5] = 0
  a[1,3] = 0
  a[1,1] - a[1,2] = 0
\end{code}

Una sintassi speciale � riservata al caso in cui una certa restrizione deve essere
applicata a tutte le colonne di $\beta$: in questo caso, viene usato un indice
per ogni termine \texttt{b} e non si usano le parentesi quadre. Quindi la
sintassi seguente
\begin{code}
restrict
  b1 + b2 = 0
end restrict
\end{code}
corresponde a
\[
\beta = \left[
\begin{array}{rr}
\beta_{11} & \beta_{21} \\
-\beta_{11} & -\beta_{21} \\
\beta_{13} & \beta_{23} \\
\beta_{14} & \beta_{24}
\end{array}
\right]
\]
La stessa convenzione viene usata per $\alpha$: quando si usa solo un indice per
ogni termine \texttt{a}, la restrizione viene applicata a tutte le $r$ righe di
$\alpha$, o in altre parole le variabili indicate sono debolmente esogene. Ad
esempio la formulazione
%
\begin{code}
restrict
  a3 = 0
  a4 = 0
end restrict
\end{code}
%
specifica che le variabili 3 e 4 non rispondono alla deviazione dall'equilibrio
nel periodo precedente.

Infine, � disponibile una scorciatoia per definire restrizioni complesse (al
momento solo in relazione a $\beta$): � possibile specificare $R_b$ e $q$,
come in $R_b \vec{\beta} = q$, dando i nomi di matrici definite in precedenza.
Ad esempio,
%
\begin{code}
matrix I4 = I(4)
matrix vR = I4**(I4~zeros(4,1))
matrix vq = mshape(I4,16,1)
restrict
  R = vR
  q = vq
end restrict
\end{code}
%
impone manualmente la normalizzazione di Phillips sulle stime di $\beta$ per un sistema con
rango di cointegrazione 4.
 
\subsection{Un esempio}
\label{sec:vecm-overid-ex}

Brand e Cassola (2004) propongono un sistema di domanda di moneta per l'area
Euro, in cui postulano tre relazioni di equilibrio di lungo periodo:
%
\begin{center}
\begin{tabular}{ll}
  Domanda di moneta & $m = \beta_l l + \beta_y y$ \\
  Equazione di Fisher & $\pi = \phi l$ \\
  Teoria delle aspettative sui & $l = s$ \\ [-4pt]
  tassi di interesse
\end{tabular}
\end{center}
%
dove $m$ � la domanda di moneta reale, $l$ e $s$ sono i tassi di interesse a
lungo e a breve termine, $y$ � l'output e $\pi$ � l'inflazione\footnote{Una
  formulazione tradizionale dell'equazione di Fisher invertirebbe i ruoli delle
  variabili nella seconda equazione, ma questo dettaglio non � rilevante in
  questo contesto; inoltre, la teoria delle aspettative sui tassi implica che la
  terza relazione di equilibrio dovrebbe includere una costante per il premio di
  liquidit�. In ogni caso, visto che in questo esempio il sistema � stimato con
  il termine costante non vincolato, il premio per la liquidit� viene assorbito
  nell'intercetta del sistema e scompare da $z_t$.}. I nomi di queste variabili
nel file di dati di \app{gretl} sono rispettivamente \verb|m_p|, \texttt{rl},
\texttt{rs}, \texttt{y} e \texttt{infl}.

Il rango di cointegrazione ipotizzato dagli autori � pari a 3, e ci sono 5
variabili, per un totale di 15 elementi nella  matrice $\beta$. Per
l'identificazione sono richieste $3 \times 3 = 9$ restrizioni, e un sistema
esattamente identificato avrebbe $15 - 9 = 6$ parametri liberi. Tuttavia, le
tre relazioni di lungo periodo ipotizzate contengono solo 3 parametri liberi,
quindi il rango di sovraidentificazione � 3.

\begin{script}[htbp]
  \caption{Stima di un sistema di domanda di moneta con vincoli su $\beta$}
  \label{brand-cassola-script}
Input:
\begin{scodebit}
open brand_cassola.gdt

# Alcune trasformazioni
m_p = m_p*100
y = y*100
infl = infl/4
rs = rs/4
rl = rl/4

# Replica della tabella 4 a pagina 824
vecm 2 3 m_p infl rl rs y -q
genr ll0 = $lnl

restrict --full
  b[1,1] = 1
  b[1,2] = 0
  b[1,4] = 0
  b[2,1] = 0
  b[2,2] = 1
  b[2,4] = 0
  b[2,5] = 0
  b[3,1] = 0
  b[3,2] = 0
  b[3,3] = 1
  b[3,4] = -1
  b[3,5] = 0
end restrict
genr ll1 = $rlnl
\end{scodebit}
Output parziale:
\begin{scodebit}
Log-verosimiglianza non vincolata (lu) = 116.60268
Log-verosimiglianza vincolata (lr) = 115.86451
2 * (lu - lr) = 1.47635
P(Chi-quadro(3) > 1.47635) = 0.68774

Beta (vettori di cointegrazione, errori standard tra parentesi)

m_p        1.0000       0.0000       0.0000 
          (0.0000)     (0.0000)     (0.0000) 
infl       0.0000       1.0000       0.0000 
          (0.0000)     (0.0000)     (0.0000) 
rl         1.6108     -0.67100       1.0000 
         (0.62752)   (0.049482)     (0.0000) 
rs         0.0000       0.0000      -1.0000 
          (0.0000)     (0.0000)     (0.0000) 
y         -1.3304       0.0000       0.0000 
        (0.030533)     (0.0000)     (0.0000) 
\end{scodebit}
%$
\end{script}

L'esempio \ref{brand-cassola-script} replica la tabella 4 a pagina 824
dell'articolo di Brand e Cassola\footnote{Fatta eccezione per quelli che
  sembrano alcuni errori di stampa nell'articolo.}. Si noti che viene usato
l'accessorio \verb|$lnl| dopo il comando \texttt{vecm} per salvare la
log-verosimiglianza non vincolata e l'accessorio \verb|$rlnl| dopo il comando
\texttt{restrict} per la sua controparte vincolata.

L'esempio continua nello script~\ref{brand-cassola-tab5}, dove vengono eseguiti
ulteriori test per controllare (a) se l'elasticit� al reddito nell'equazione di
domanda di moneta � pari a 1 ($\beta_y = 1$) e (b) se la relazione di Fisher �
omogenea ($\phi = 1$). Visto che � stata usata l'opzione \verb|--full| con il
comando \texttt{restrict} iniziale, � possibile applicare le restrizioni
aggiuntive senza dover ripetere quelle precedenti. Il secondo script contiene
alcuni comandi \texttt{printf} che non sono strettamente necessari, servono solo
a formattare meglio i risultati. I dati portano a rifiutare entrambe le ipotesi
aggiuntive, con p-value di $0.002$ e $0.004$.

\begin{script}[htbp]
  \caption{Test ulteriori sul sistema di domanda di moneta}
  \label{brand-cassola-tab5}
Input:
\begin{scodebit}
restrict
  b[1,5] = -1
end restrict
genr ll_uie = $rlnl

restrict
  b[2,3] = -1
end restrict
genr ll_hfh = $rlnl

# Replica della tabella 5 a pagina 824
printf "Test su zero restrizioni nello spazio di cointegrazione:\n"
printf "  test LR, rango = 3: chi^2(3) = %6.4f [%6.4f]\n", 2*(ll0-ll1), \
	pvalue(X, 3, 2*(ll0-ll1))

printf "Elasticit� al reddito unitaria: test LR, rango = 3:\n"
printf "  chi^2(4) = %g [%6.4f]\n", 2*(ll0-ll_uie), \
	pvalue(X, 4, 2*(ll0-ll_uie))

printf "Omogeneit� nell'ipotesi di Fisher:\n"
printf "  test LR, rango = 3: chi^2(4) = %6.3f [%6.4f]\n", 2*(ll0-ll_hfh), \
	pvalue(X, 4, 2*(ll0-ll_hfh))
\end{scodebit}
Output:
\begin{scodebit}
Test su zero restrizioni nello spazio di cointegrazione:
  test LR, rango = 3: chi^2(3) = 1.4763 [0.6877]
Elasticit� al reddito unitaria: test LR, rango = 3:
  chi^2(4) = 17.2071 [0.0018]
Omogeneit� nell'ipotesi di Fisher:
  test LR, rango = 3: chi^2(4) = 15.547 [0.0037]  
\end{scodebit}
\end{script}

Un altro tipo di test eseguito spesso � quello dell'``esogeneit� debole''.
In questo contesto, si dice che una variabile � debolmente esogena se tutti i
coefficienti nella riga corrispondente della matrice $\alpha$ sono pari a zero.
In questo caso, la variabile non reagisce alle deviazioni da alcuno degli
equilibri di lungo periodo e pu� essere considerata una forza autonoma dal resto
del sistema.

Il codice nell'Esempio~\ref{brand-cassola-exog} esegue questo test per ognuna
delle variabili, replicando la prima colonna della Tabella 6 a pagina 825
di Brand e Cassola (2004). I risultati mostrano che l'ipotesi di esogeneit�
debole pu� essere accettata per i tassi di interesse a lungo termine e il PIL
reale (rispettivamente con p-value $0.07$ e $0.08$).

\begin{script}[htbp]
  \caption{Test per l'esogeneit� debole}
  \label{brand-cassola-exog}
Input:
\begin{scodebit}
restrict
  a1 = 0
end restrict
ts_m = 2*(ll0 - $rlnl)

restrict
  a2 = 0
end restrict
ts_p = 2*(ll0 - $rlnl)

restrict
  a3 = 0
end restrict
ts_l = 2*(ll0 - $rlnl)

restrict
  a4 = 0
end restrict
ts_s = 2*(ll0 - $rlnl)

restrict
  a5 = 0
end restrict
ts_y = 2*(ll0 - $rlnl)

loop foreach i m p l s y --quiet
  printf "\Delta $i\t%6.3f [%6.4f]\n", ts_$i, pvalue(X, 6, ts_$i)
end loop
\end{scodebit}
Output (variabile, test LR, p-value):
\begin{scodebit}
\Delta m	18.111 [0.0060]
\Delta p	21.067 [0.0018]
\Delta l	11.819 [0.0661]
\Delta s	16.000 [0.0138]
\Delta y	11.335 [0.0786]
\end{scodebit}
%$
\end{script}

\subsection{Identificazione e testabilit�}
\label{sec:ident-test}

Un punto che pu� risultare poco chiaro a proposito delle restrizioni sui VECM
� che l'identificazione (la restrizione identifica il sistema?) e la testabilit�
(la restrizione � testabile?) sono questioni abbastanza distinte. Le restrizioni
possono essere identificanti ma non testabili; in modo meno ovvio, possono anche
essere testabili ma non identificanti.

Questo pu� essere visto facilmente in un sistema a rango 1. La restrizione
$\beta_1 = 1$ � identificante (identifica la scala di $\beta$) ma, essendo un
semplice cambiamento di scala, non � testabile. D'altra parte, la restrizione
$\beta_1 + \beta_2 = 0$ � testabile --- il sistema con questo requisito imposto
avr� quasi sicuramente una minore verosimiglianza massimizzata --- ma non �
identificante; lascia ancora aperta la scelta della scala di $\beta$.  

Abbiamo visto sopra che il numero di restrizioni deve essere pari ad almeno
$r^2$, dove $r$ � il rango di cointegrazione. Questa � una condizione necessaria
ma non sufficiente. In effetti, quando $r>1$ pu� essere abbastanza difficile
giudicare quando un certo insieme di restrizioni � identificante.
\app{Gretl} usa il metodo suggerito da Doornik (1995), dove l'identificazione �
valutata tramite il rango della matrice di informazione.

Pu� essere dimostrato che per le restrizioni del tipo della (\ref{eq:vecbeta})
e della (\ref{eq:vecalpha}) la matrice di informazione ha lo stesso rango della
matrice Jacobiana
%
\[
{\cal J}(\theta) = \left[ (I_p \otimes \beta) G : 
                   (\alpha \otimes I_{p_1}) H \right]
\]

Una condizione sufficiente per l'identificazione � che il rango di ${\cal
  J}(\theta)$ sia pari al numero dei parametri liberi. Il rango di questa
matrice � valutato esaminando i suoi valori singolari in un punto scelto a caso
dello spazio dei parametri. Agli scopi pratici trattiamo questa condizione come
se fosse sia necessaria che sufficiente, ossia tralasciamo i casi speciali in
cui l'identificazione potrebbe essere raggiunta senza che questa condizione sia
soddisfatta\footnote{Si veda Boswijk e Doornik (2004, pp.\ 447--8) per la
discussione di questo problema.}.

\section{Metodi di soluzione numerica}
\label{sec:vecm-opt}

In generale, il problema della stima ML per il VECM vincolato non ha una
soluzione in forma chiusa, quindi il massimo deve essere cercato con metodi
numerici\footnote{L'eccezione � costituita dal caso in cui le restrizioni
  sono omogenee, sono comuni a tutti i $\beta$ o a tutti gli $\alpha$ (nel caso
  $r>1$), e includono solo $\beta$ o solo $\alpha$. Queste restrizioni sono
  gestite con il metodo degli autovalori modificati proposto da Johansen (1995):
  risolviamo direttamente lo stimatore ML, senza aver bisogno di metodi
  iterativi.}. In alcuni casi, la convergenza pu� risultare difficile da
raggiungere, e \app{gretl} fornisce vari modi di risolvere il problema.

\subsection{Algoritmi switching e LBFGS}
\label{sec:vecm-algorithms}

Sono disponibili due metodi di massimizzazione in \app{gretl}: quello
predefinito � l'algoritmo di switching proposto in Boswijk e Doornik (2004);
l'alternativa � una variante a memoria limitata dell'algoritmo BFGS, che usa
derivate analitiche: LBFGS. Per usare questo metodo occorre aggiungere l'opzione
\verb+--lbfgs+ al comando \texttt{restrict}.

L'algoritmo di switching funziona massimizzando esplicitamente la
verosimiglianza ad ogni iterazione rispetto a $\hat{\phi}$, $\hat{\psi}$ e 
$\hat{\Omega}$ (la matrice di covarianza dei residui). Questo metodo condivide
una caratteristica con la classica procedura degli autovalori di Johansen,
ossia: pu� gestire un insieme di restrizioni che non identificano completamente
i parametri.

D'altra parte, LBFGS richiede che il modello sia completamente identificato,
quindi in alcuni casi occorrer� aggiungere alle restrizioni che interessano
alcune normalizzazioni con lo scopo di identificare i parametri.
Ad esempio, si pu� usare in tutto o in parte la normalizzazione di Phillips
(si veda la sezione \ref{sec:johansen-ident}).

N� l'algoritmo di switching n� quello LBFGS garantiscono di raggiungere la soluzione ML
globale\footnote{Nello sviluppo del codice per il test dei VECM in \app{gretl}
  sono stati considerati un certo numero di ``casi difficili'' individuati da
  varie fonti. Vogliamo ringraziare Luca Fanelli dell'Universit� di Bologna e 
  Sven Schreiber della Goethe University di Francoforte, per aver suggerito
  vari test per il codice VECM di \app{gretl}}: l'ottimizzatore potrebbe fermarsi
su un massimo locale (o, nel caso dell'algoritmo di switching, su un punto di
sella).

La soluzione (o la sua mancanza) pu� essere sensibile al valore iniziale scelto
per $\theta$. Per impostazione predefinita, \app{gretl} sceglie un valore
iniziale usando un metodo deterministico basato su Boswijk (1995), ma sono
disponibili due altre opzioni: � possibile aggiustare l'inizializzazione usando
il metodo della ricottura simulata (con l'opzione \verb+--jitter+), oppure �
possibile fornire esplicitamente un valore iniziale per $\theta$.

Il metodo di inizializzazione predefinito � il seguente:
%
\begin{enumerate}
\item Calcolare la ML non vincolata $\hat{\beta}$ usando la procedura di Johansen.
\item Se la restrizione su $\beta$ � non-omogenea, usare il metodo proposto da
  Boswijk (1995):
\begin{equation}
\phi_0 = -[(I_r \otimes \hat{\beta}_{\perp})'H]^+ 
  (I_r \otimes \hat{\beta}_{\perp})' h_0
\end{equation}
dove $\hat{\beta}'_{\perp} \hat{\beta} = 0$ e $A^+$ denota l'inversa di Moore--Penrose di $A$.  Altrimenti
\begin{equation}
\phi_0 = (H'H)^{-1} H' \vec{\hat{\beta}}
\end{equation}
\item $\vec{\beta_0} = H\phi_0 + h_0$.
\item Calcolare la ML non vincolata $\hat{\alpha}$ condizionale su
  $\beta_0$, come da Johansen:
\begin{equation}
\label{eq:Jalpha}
\hat{\alpha} = S_{01} \beta_0 (\beta'_0S_{11}\beta_0)^{-1}
\end{equation}
\item Se $\alpha$ � vincolata da $\vec{\alpha'} = G\psi$, allora
  $\psi_0 = (G'G)^{-1}G'\,{\rm vec}(\hat{\alpha}')$ e
  $\vec{\alpha'_0} = G\psi_0$.
\end{enumerate}

\subsection{Metodi di inizializzazione alternativi}
\label{sec:vecm-alt-init}

Come � stato detto in precedenza, \app{gretl} offre la possibilit� di
impostare l'inizializzazione usando la ricottura simulata.

L'idea di base � questa: si inizia in un certo punto dello spazio parametrico,
e per ognuna di $n$ iterazioni (al momento $n=4096$) si sceglie a caso un
punto all'interno di un certo raggio di distanza dal precedente e si determina la
verosimiglianza nel nuovo punto. Se la verosimiglianza � maggiore, si salta
nel nuovo punto, altrimenti si salta con probabilit� $P$ (e si rimane nel
punto precedente con probabilit� $1-P$). Man mano che le iterazioni procedono,
il sistema ``si raffredda'' gradualmente, ossia il raggio delle perturbazioni
casuali si riduce, cos� come la probabilit� di fare un salto quando la
verosimiglianza non aumenta.

Nel corso di questa procedura vengono valutati molti punti dello spazio dei
parametri, cominciando dal punto scelto dalla procedura deterministica, che
chiameremo $\theta_0$. Uno di questi punti sar� il ``migliore'' nel senso che
produce la pi� alta verosimiglianza: lo si chiami $\theta^*$. Questo punto pu�
avere o non avere una verosimiglianza maggiore di quella di $\theta_0$. La
procedura ha inoltre un punto finale, $\theta_n$, che pu� essere o non essere il
``migliore''.

La regola seguita da \app{gretl} per scegliere un valore iniziale per $\theta$
basandosi sulla ricottura simulata � questa: usare $\theta^*$ se $\theta^* >
\theta_0$, altrimenti usare $\theta_n$.  Ossia: se otteniamo un miglioramento
nella verosimiglianza tramite la ricottura simulata, ne approfittiamo; d'altra
parte, se non otteniamo un miglioramento usiamo comunque la ricottura per
la randomizzazione del punto iniziale. La sperimentazione ha indicato che
quest'ultimo metodo pu� aiutare.

Se l'inizializzazione deterministica fallisce, l'alternativa alla
randomizzazione consiste nello scegliere esplicitamente dei valori iniziali.
L'utente pu� farlo passando un vettore predefinito al comando \texttt{set} con
il parametro \texttt{initvals}, come nell'esempio
%
\begin{verbatim}
set initvals myvec
\end{verbatim}

I dettagli dipendono dall'algoritmo scelto. Nel caso dell'algoritmo di
switching ci sono due opzioni per sceglire i valori iniziali. Quella pi� comoda
(per la maggior parte delle persone supponiamo) consiste nello specificare una
matrice che contiene $\vec{\beta}$ seguito da $\vec{\alpha}$. Ad esempio:
\begin{code}
open denmark.gdt
vecm 2 1 LRM LRY IBO IDE --rc --seasonals

matrix BA = {1, -1, 6, -6, -6, -0.2, 0.1, 0.02, 0.03}
set initvals BA
restrict
  b[1] = 1
  b[1] + b[2] = 0
  b[3] + b[4] = 0
end restrict
\end{code}

In questo esempio tratto da Johansen (1995) il rango di cointegrazione � 1 e
ci sono 4 variabili, ma il modello contiene una costante vincolata
(opzione \verb|--rc|) cos� che $\beta$ ha 5 elementi. La matrice
$\alpha$ ha 4 elementi, uno per equazione. Quindi la matrice
\texttt{BA} pu� essere letta come
\[
\left(\beta_1, \beta_2, \beta_3, \beta_4, \beta_5,
 \alpha_1, \alpha_2, \alpha_3, \alpha_4 \right)
\]

L'altra opzione, obbligatoria quando si usa LBFGS, consiste nello specificare i
valori iniziali in termini dei parametri non vincolati, $\phi$ e $\psi$. Questo
metodo � meno ovvio da comprendere: come ricordato sopra, la restrizione in
forma implicita $R\vec{\beta} = q$ ha la forma esplicita $\vec{\beta} = H\phi +
h_0$, dove $H = R_{\perp}$, lo spazio nullo destro di $R$. Il vettore $\phi$ �
pi� corto, per il numero di restrizioni, di $\vec{\beta}$. L'utente attento
capir� cosa occorre fare. L'altro punto di cui tenere conto � che se $\alpha$
non � vincolato, la lunghezza \textit{effettiva} di $\psi$ � 0, visto che �
ottimale calcolare $\alpha$ con la formula di Johansen, condizionale su $\beta$
(equazione \ref{eq:Jalpha} sopra). Segue un esempio:
\begin{code}
open denmark.gdt
vecm 2 1 LRM LRY IBO IDE --rc --seasonals

matrix phi = {-8, -6}
set initvals phi
restrict --lbfgs
  b[1] = 1
  b[1] + b[2] = 0
  b[3] + b[4] = 0
end restrict
\end{code}

In questa formulazione pi� economica, l'inizializzatore specifica solo i due
parametri liberi di $\phi$ (5 elementi in $\beta$ meno 3 restrizioni). Non c'�
motivo di dare valori a $\psi$ visto che $\alpha$ non � vincolato.

\subsection{Rimozione della scala}
\label{sec:vecm-scale-removal}

Si consideri una versione pi� semplice della restrizione discussa nella sezione
precedente, ossia,
%
\begin{code}
restrict
  b[1] = 1
  b[1] + b[2] = 0
end restrict
\end{code}

Questa restrizione comprende un vincolo sostanziale e testabile --- che la somma di
$\beta_1$ e $\beta_2$ sia pari a zero --- e una normalizzazione, o scalatura,
$\beta_1 = 1$.  Sorge la questione se non sia pi� semplice e affidabile
massimizzare la verosimiglianza senza imporre $\beta_1 = 1$\footnote{Dal punto di vista
  numerico � pi� semplice. In linea di principio non dovrebbe fare differenza.}.
Se cos� fosse, potremmo tenere nota di questa normalizzazione, rimuoverla allo
scopo di massimizzare la verosimiglianza, e quindi re-imporla scalando il
risultato.

Purtroppo non � possibile dire in anticipo se una ``rimozione di scala'' di
questo tipo dar� risultati migliori per un certo problema di stima, ma nella
maggior parte dei casi sembra essere cos�. Gretl quindi opera una rimozione di
scala nei casi in cui � fattibile, a meno che
\begin{itemize}
\item l'utente non lo vieti, usando l'opzione \verb|--no-scaling| del comando
  restrict; oppure
\item l'utente fornisca uno specifico vettore di valori iniziali; oppure
\item l'utente scelga l'algoritmo LBFGS per la massimizzazione.
\end{itemize}

La rimozione di scala viene giudicata non fattibile se ci sono restrizioni
incrociate tra le colonne di $\beta$, o se ci sono restrizioni non-omogenee
che coinvolgono pi� di un elemento di $\beta$.  

In aggiunta, l'esperienza ha suggerito che la rimozione della scala non �
consigliabile se il sistema � esattamente identificato con le normalizzazioni
incluse, quindi in questo caso non viene eseguita. Per ``esattamente
identificato'' si intende che il sistema non sarebbe identificato se una
qualsiasi delle restrizioni fosse rimossa. Da questo punto di vista, l'esempio
visto sopra non � esattamente identificato, visto che rimuovendo la seconda
restrizione non si intaccherebbe l'identificazione; \app{gretl} in questo caso
eseguirebbe la rimozione di scala a meno che l'utente non indichi altrimenti.

%%% Local Variables: 
%%% mode: latex
%%% TeX-master: "gretl-guide"
%%% End: 

\chapter{Discrete and censored dependent variables}
\label{chap:discr-models}

\section{Logit and probit models}
\label{sec:logit-probit}

It often happens that one wants to specify and estimate a model in
which the dependent variable is not continuous, but discrete. A
typical example is a model in which the dependent variable is the
occupational status of an individual (1 = employed, 0 = unemployed). A
convenient way of formalizing this situation is to consider the
variable $y_i$ as a Bernoulli random variable and analyze its
distribution conditional on the explanatory variables $x_i$.  That is,
%
\begin{equation}
  \label{eq:qr-Bernoulli}
  y_i = \left\{ 
    \begin{array}{ll} 
      1 & P_i \\ 0 & 1 - P_i 
    \end{array}
    \right.
\end{equation}
%
where $P_i = P(y_i = 1 | x_i) $ is a given function of the explanatory
variables $x_i$.

In most cases, the function $P_i$ is a cumulative distribution
function $F$, applied to a linear combination of the $x_i$s. In the
probit model, the normal cdf is used, while the logit model employs
the logistic function $\Lambda()$. Therefore, we have
%
\begin{eqnarray}
  \label{eq:qr-link}
  \textrm{probit} & \qquad & P_i = F(z_i) = \Phi(z_i)  \\
  \textrm{logit}  & \qquad & P_i = F(z_i) = \Lambda(z_i) = \frac{1}{1 + e^{-z_i}} \\
  & &z_i = \sum_{j=1}^k x_{ij} \beta_j
\end{eqnarray}
%
where $z_i$ is commonly known as the \emph{index} function. Note that
in this case the coefficients $\beta_j$ cannot be interpreted as the
partial derivatives of $E(y_i | x_i)$ with respect to
$x_{ij}$.  However, for a given value of $x_i$ it is possible to
compute the vector of ``slopes'', that is
\[
  \mathrm{slope}_j(\bar{x}) = \left. \pder{F(z)}{x_j} \right|_{z =
    \bar{z}}
\]
\app{Gretl} automatically computes the slopes, setting each
explanatory variable at its sample mean.

Another, equivalent way of thinking about this model is in terms of
an unobserved variable $y^*_i$ which can be described thus:
%
\begin{equation}
  \label{eq:qr-latent}
  y^*_i = \sum_{j=1}^k x_{ij} \beta_j + \varepsilon_i = z_i  +
  \varepsilon_i 
\end{equation}
%
We observe $y_i = 1$ whenever $y^*_i > 0$ and $y_i = 0$ otherwise. If
$\varepsilon_i$ is assumed to be normal, then we have the probit
model. The logit model arises if we assume that the density function
of $\varepsilon_i$ is
%
\[
  \lambda(\varepsilon_i) =
  \pder{\Lambda(\varepsilon_i)}{\varepsilon_i} =
  \frac{e^{-\varepsilon_i}}{(1 + e^{-\varepsilon_i})^2}
\]

Both the probit and logit model are estimated in \app{gretl} via
maximum likelihood, where the log-likelihood can be written as
\begin{equation}
  \label{eq:qr-loglik}
  L(\beta) = \sum_{y_i=0} \ln [ 1 - F(z_i)] + \sum_{y_i=1} \ln F(z_i),
\end{equation}
which is always negative, since $0 < F(\cdot) < 1$.  Since the score
equations do not have a closed form solution, numerical optimization
is used. However, in most cases this is totally transparent to the
user, since usually only a few iterations are needed to ensure
convergence. The \option{verbose} switch can be used to track the
maximization algorithm.

\begin{script}[htbp]
  \caption{Estimation of simple logit and probit models}
  \label{simple-QR}
\begin{scode}
open greene19_1

logit GRADE const GPA TUCE PSI
probit GRADE const GPA TUCE PSI
\end{scode}
\end{script}

As an example, we reproduce the results given in Greene (2000),
chapter 21, where the effectiveness of a program for teaching
economics is evaluated by the improvements of students' grades.
Running the code in example \ref{simple-QR} gives the following output:
\begin{code}

Model 1: Logit estimates using the 32 observations 1-32
Dependent variable: GRADE

      VARIABLE       COEFFICIENT        STDERROR      T STAT       SLOPE
                                                                  (at mean)
  const               -13.0213           4.93132      -2.641
  GPA                   2.82611          1.26294       2.238      0.533859   
  TUCE                  0.0951577        0.141554      0.672      0.0179755  
  PSI                   2.37869          1.06456       2.234      0.449339   

  Mean of GRADE = 0.344
  Number of cases 'correctly predicted' = 26 (81.2%)
  f(beta'x) at mean of independent vars = 0.189
  McFadden's pseudo-R-squared = 0.374038
  Log-likelihood = -12.8896
  Likelihood ratio test: Chi-square(3) = 15.4042 (p-value 0.001502)
  Akaike information criterion (AIC) = 33.7793
  Schwarz Bayesian criterion (BIC) = 39.6422
  Hannan-Quinn criterion (HQC) = 35.7227

           Predicted
             0    1
  Actual 0  18    3
         1   3    8

Model 2: Probit estimates using the 32 observations 1-32
Dependent variable: GRADE

      VARIABLE       COEFFICIENT        STDERROR      T STAT       SLOPE
                                                                  (at mean)
  const                -7.45232          2.54247      -2.931
  GPA                   1.62581          0.693883      2.343      0.533347   
  TUCE                  0.0517288        0.0838903     0.617      0.0169697  
  PSI                   1.42633          0.595038      2.397      0.467908   

  Mean of GRADE = 0.344
  Number of cases 'correctly predicted' = 26 (81.2%)
  f(beta'x) at mean of independent vars = 0.328
  McFadden's pseudo-R-squared = 0.377478
  Log-likelihood = -12.8188
  Likelihood ratio test: Chi-square(3) = 15.5459 (p-value 0.001405)
  Akaike information criterion (AIC) = 33.6376
  Schwarz Bayesian criterion (BIC) = 39.5006
  Hannan-Quinn criterion (HQC) = 35.581

           Predicted
             0    1
  Actual 0  18    3
         1   3    8

\end{code}

In this context, the \verb+$uhat+ accessor function takes a
special meaning: it returns generalized residuals as defined in
Gourieroux \textit{et al} (1987), which can be interpreted as unbiased
estimators of the latent disturbances $\varepsilon_t$. These are
defined as
%
\begin{equation}
  \label{eq:QR-genres}
  u_i = \left\{
    \begin{array}{ll}
      y_i - \hat{P}_i & \textrm{for the logit model} \\
      y_i\cdot \frac{\phi(\hat{z}_i)}{\Phi(\hat{z}_i)} - 
      ( 1 - y_i ) \cdot \frac{\phi(\hat{z}_i)}{1 - \Phi(\hat{z}_i)}
      & \textrm{for the probit model} \\
    \end{array}
    \right.
\end{equation}

Among other uses, generalized residuals are often used for diagnostic
purposes.  For example, it is very easy to set up an omitted variables
test equivalent to the familiar LM test in the context of a linear
regression; example \ref{QR-add} shows how to perform a variable
addition test.

\begin{script}[htbp]
  \caption{Variable addition test in a probit model}
  \label{QR-add}
\begin{scode}
open greene19_1

probit GRADE const GPA PSI
series u = $uhat 
%$
ols u const GPA PSI TUCE -q
printf "Variable addition test for TUCE:\n"
printf "Rsq * T = %g (p. val. = %g)\n", $trsq, pvalue(X,1,$trsq) 
\end{scode}
\end{script}

\subsection{The perfect prediction problem}
\label{sec:perfpred}

One curious characteristic of logit and probit models is that (quite
paradoxically) estimation is not feasible if a model fits the data
perfectly; this is called the \emph{perfect prediction problem}. The
reason why this problem arises is easy to see by considering equation
(\ref{eq:qr-loglik}): if for some vector $\beta$ and scalar $k$ it's
the case that $z_i < k$ whenever $y_i=0$ and $z_i > k$ whenever
$y_i=1$, the same thing is true for any multiple of $\beta$. Hence,
$L(\beta)$ can be made arbitrarily close to 0 simply by choosing
enormous values for $\beta$. As a consequence, the log-likelihood has
no maximum, despite being bounded.

\app{Gretl} has a mechanism for preventing the algorithm from
iterating endlessly in search of a non-existent maximum. One sub-case
of interest is when the perfect prediction problem arises because of
a single binary explanatory variable. In this case, the offending
variable is dropped from the model and estimation proceeds with the
reduced specification. Nevertheless, it may happen that no single
``perfect classifier'' exists among the regressors, in which case
estimation is simply impossible and the algorithm stops with an
error. This behavior is triggered during the iteration process if
\[
  \stackunder{i: y_i = 0}{\max z_i} \, < \,
  \stackunder{i: y_i = 1}{\min z_i}  
\]
If this happens, unless your model is trivially mis-specified (like
predicting if a country is an oil exporter on the basis of oil
revenues), it is normally a small-sample problem: you probably just
don't have enough data to estimate your model. You may want to drop
some of your explanatory variables.

\subsection{Ordered models}
\label{sec:ordered}

These models are simple variations of ordinary logit/probit models,
and are usually applied in case the dependent variable is a discrete
and ordered measurement, not necessarily quantitative. For example,
this sort of model can be applied when the dependent variable is a
qualitative assessment like ``Good'', ``Average'' and ``Bad''.
Assuming we have $p$ categories, the probability that individual $i$
falls in the $j$-th category is given by
%
\begin{equation}
  \label{eq:QR-ordered}
  P(y_i = j | x_i) = \left\{
    \begin{array}{ll}
      F(z_i + \mu_0) & \textrm{for } j = 0 \\
      F(z_i + \mu_j) -  F(z_i + \mu_{j-1}) & \textrm{for } 0 < j < p \\
      1 -  F(z_i + \mu_{p-1}) & \textrm{for } j = p 
    \end{array}
    \right.
\end{equation}
%
The unknown parameters $\mu_j$ are called the ``cutoff
points'' and are estimated together with the $\beta$s. For
identification purposes, $\mu_0$ is assumed to be 0. In terms of the
unobserved variable $y^*_i$, the model can be equivalently cast as
$P(y_i = j | x_i) = P(\mu_{j-1} \le y^*_i < \mu_j)$. 

\begin{script}[htbp]
  \caption{Ordered probit model}
  \label{ex:oprobit}
\begin{scode}
open pension.gdt
series pctstck = pctstck/50
discrete pctstck
probit pctstck const choice age educ female black married finc25 finc35 \
  finc50 finc75 finc100 finc101 wealth89 prftshr
\end{scode}
\end{script}

In order to apply these models, the dependent variable must be marked
as discrete. Example \ref{ex:oprobit} reproduces the estimation given
in chapter 15 of Wooldridge (2002a). Note that \app{gretl} does not
provide a separate command for ordered models: the \texttt{logit} and
\texttt{probit} commands automatically estimate the ordered version if
the dependent variable is not binary (provided it has already been
marked as discrete).

After estimating ordered models, the \verb+$uhat+ accessor yields
generalized residuals as in binary models; additionally, the
\verb+$yhat+ accessor function returns $\hat{z}_i$, so it is
possible to compute an unbiased estimator of the latent variable
$y^*_i$ simply by adding the two together.

\subsection{Multinomial logit}
\label{sec:mlogit}

When the dependent variable is not binary and does not have a natural
ordering, \emph{multinomial} models are used. \app{Gretl} does not
provide a native implementation of these yet, but simple models can be
handled via the \texttt{mle} command (see chapter \ref{chap:mle}). We
give here an example of a multinomial logit model.  Let the dependent
variable, $y_i$, take on integer values $0,1,\dots p$.  The
probability that $y_i = k$ is given by
\[
  P(y_i = k |  x_i) = \frac{\exp(x_i \beta_k)}{\sum_{j=0}^p \exp(x_i \beta_j)}
\]
For the purpose of identification one of the outcomes must be taken as
the ``baseline''; it is usually assumed that $\beta_0 = 0$, in which case
\[
  P(y_i = k |  x_i) = \frac{\exp(x_i \beta_k)}{1 + \sum_{j=1}^p \exp(x_i \beta_j)} 
\]
and
\[
  P(y_i = 0 |  x_i) = \frac{1}{1 + \sum_{j=1}^p \exp(x_i \beta_j)} .
\]

Example~\ref{ex:mlogit} reproduces Table 15.2 in Wooldridge (2002a),
based on data on career choice from Keane and Wolpin (1997).  The
dependent variable is the occupational status of an individual (0 = in
school; 1 = not in school and not working; 2 = working), and the
explanatory variables are education and work experience (linear and
square) plus a ``black'' binary variable.  The full data set is a
panel; here the analysis is confined to a cross-section for 1987.  For
explanations of the matrix methods employed in the script, see
chapter~\ref{chap:matrices}.

\begin{script}[htbp]
  \caption{Multinomial logit}
  \label{ex:mlogit}
\begin{scode}
function mlogitlogprobs(series y, matrix X, matrix theta)

  scalar n = max(y)
  scalar k = cols(X)
  matrix b = mshape(theta,k,n)

  matrix tmp = X*b
  series ret = -ln(1 + sumr(exp(tmp)))

  loop for i=1..n --quiet
    series x = tmp[,i]
    ret += (y=$i) ? x : 0 
  end loop

  return series ret

end function

open Keane.gdt
status = status-1 # dep. var. must be 0-based
smpl (year=87 & ok(status)) --restrict

matrix X = { educ exper expersq black const }
scalar k = cols(X)
matrix theta = zeros(2*k, 1)

mle loglik = mlogitlogprobs(status,X,theta)
  params theta
end mle --verbose --hessian
\end{scode}
%$
\end{script}


\section{The Tobit model}
\label{sec:tobit}

The Tobit model is used when the dependent variable of a model is
\emph{censored}.\footnote{We assume here that censoring occurs from
  below at 0. Censoring from above, or at a point different from zero,
  can be rather easily handled by re-defining the dependent variable
  appropriately. The more general case of two-sided censoring is not
  handled by \app{gretl} via a native command yet, but it is possible
  to estimate such models using the \texttt{mle} command (see chapter
  \ref{chap:mle}).}  
Assume a latent variable $y^*_i$ can be described
as
%
\[
  y^*_i = \sum_{j=1}^k x_{ij} \beta_j + \varepsilon_i ,
\]
%
where $\varepsilon_i \sim N(0,\sigma^2)$. If $y^*_i$ were observable,
the model's parameters could be estimated via ordinary least squares.
On the contrary, suppose that we observe $y_i$, defined as
%
\begin{equation}
  \label{eq:tobit}
  y_i = \left\{ 
    \begin{array}{ll} 
      y^*_i & \mathrm{for} \quad y^*_i > 0 \\ 
      0 & \mathrm{for} \quad y^*_i \le 0 
    \end{array}
    \right. 
\end{equation}
%
In this case, regressing $y_i$ on the $x_i$s does not yield
consistent estimates of the parameters $\beta$, because the
conditional mean $E(y_i|x_i)$ is not equal to $\sum_{j=1}^k x_{ij}
\beta_j$.  It can be shown that restricting the sample to non-zero
observations would not yield consistent estimates either. The solution
is to estimate the parameters via maximum likelihood. The syntax is
simply
%
\begin{code}
tobit depvar indvars
\end{code}

As usual, progress of the maximization algorithm can be tracked via
the \option{verbose} switch, while \verb+$uhat+ returns the
generalized residuals. Note that in this case the generalized residual
is defined as $\hat{u}_i = E(\varepsilon_i | y_i = 0)$ for censored
observations, so the familiar equality $\hat{u}_i = y_i - \hat{y}_i$
only holds for uncensored observations, that is, when $y_i>0$.

An important difference between the Tobit estimator and OLS is that
the consequences of non-normality of the disturbance term are much
more severe: non-normality implies inconsistency for the Tobit
estimator. For this reason, the output for the tobit model includes
the Chesher--Irish (1987) test for normality by default.

\subsection{Sample selection model}
\label{sec:heckit}

In the sample selection model (also known as ``Tobit II'' model),
there are two latent variables:
%
\begin{eqnarray}
  \label{eq:heckit1}
  y^*_i & = & \sum_{j=1}^k x_{ij} \beta_j + \varepsilon_i \\
  \label{eq:heckit2}
  s^*_i & = & \sum_{j=1}^p z_{ij} \gamma_j + \eta_i 
\end{eqnarray}
%
and the observation rule is given by
%
\begin{equation}
  \label{eq:tobitII}
  y_i = \left\{ 
    \begin{array}{ll} 
      y^*_i & \mathrm{for} \quad s^*_i > 0 \\ 
      \diamondsuit & \mathrm{for} \quad s^*_i \le 0 
    \end{array}
    \right. 
\end{equation}

In this context, the $\diamondsuit$ symbol indicates that for some
observations we simply do not have data on $y$: $y_i$ may be 0, or
missing, or anything else. A dummy variable $d_i$ is normally used to
set censored observations apart.

One of the most popular applications of this model in econometrics is
a wage equation coupled with a labor force participation equation: we
only observe the wage for the employed. If $y^*_i$ and $s^*_i$ were
(conditionally) independent, there would be no reason not to use OLS
for estimating equation (\ref{eq:heckit1}); otherwise, OLS does not
yield consistent estimates of the parameters $\beta_j$.

Since conditional independence between $y^*_i$ and $s^*_i$ is
equivalent to conditional independence between $\varepsilon_i$ and
$\eta_i$, one may model the co-dependence between $\varepsilon_i$ and
$\eta_i$ as 
\[
  \varepsilon_i = \lambda \eta_i + v_i ;
\]
substituting the above expression in (\ref{eq:heckit1}), you obtain
the model that is actually estimated:
\[
  y_i = \sum_{j=1}^k x_{ij} \beta_j + \lambda \hat{\eta}_i + v_i ,
\]
so the hypothesis that censoring does not matter is equivalent to the
hypothesis $H_0: \lambda = 0$, which can be easily tested.

The parameters can be estimated via maximum likelihood under the
assumption of joint normality of $\varepsilon_i$ and $\eta_i$;
however, a widely used alternative method yields the so-called
\emph{Heckit} estimator, named after Heckman (1979). The procedure can
be briefly outlined as follows: first, a probit model is fit on
equation (\ref{eq:heckit2}); next, the generalized residuals are
inserted in equation (\ref{eq:heckit1}) to correct for the effect of
sample selection.

\app{Gretl} provides the \texttt{heckit} command to carry out
estimation; its syntax is
%
\begin{code}
heckit y X ; d Z
\end{code}
%
where \texttt{y} is the dependent variable, \texttt{X} is a list of
regressors, \texttt{d} is a dummy variable holding 1 for uncensored
observations and \texttt{Z} is a list of explanatory variables for the
censoring equation.

Since in most cases maximum likelihood is the method of
choice, by default \app{gretl} computes ML estimates. The 2-step
Heckit estimates can be obtained by using the \option{two-step}
option. After estimation, the \verb|$uhat| accessor contains the
generalized residuals. As in the ordinary Tobit model, the residuals
equal the difference between actual and fitted $y_i$ only for
uncensored observations (those for which $d_i = 1$).

Example \ref{ex:heckit} shows two estimates from the dataset used in
Mroz (1987): the first one replicates Table 22.7 in Greene
(2003),\footnote{Note that the estimates given by \app{gretl} do not
  coincide with those found in the printed volume.  They do, however,
  match those found on the errata web page for Greene's book:
  \url{http://pages.stern.nyu.edu/~wgreene/Text/Errata/ERRATA5.htm}.}
while the second one replicates table 17.1 in Wooldridge (2002a).

\begin{script}[htbp]
  \caption{Heckit model}
  \label{ex:heckit}
\begin{scode}
open mroz87.gdt

genr EXP2 = AX^2
genr WA2 = WA^2
genr KIDS = (KL6+K618)>0

# Greene's specification

list X = const AX EXP2 WE CIT
list Z = const WA WA2 FAMINC KIDS WE

heckit WW X ; LFP Z --two-step 
heckit WW X ; LFP Z 

# Wooldridge's specification

series NWINC = FAMINC - WW*WHRS
series lww = log(WW)
list X = const WE AX EXP2
list Z = X NWINC WA KL6 K618

heckit lww X ; LFP Z --two-step 
\end{scode}
\end{script}

% \section{Count data}
% \label{sec:poisson}

% also include example script for negative binomial (done in Verbeek
% example files).

%%% Local Variables: 
%%% mode: latex
%%% TeX-master: "gretl-guide"
%%% End: 


\part{Dettagli tecnici}

\chapter{Gretl e \TeX}
\label{gretltex}


\section{Introduzione}
\label{tex-intro}

\TeX, sviluppato inizialmente da Donald Knuth della Stanford University
e poi migliorato da centinaia di sviluppatori di tutto il mondo, �
il punto di riferimento per la composizione di testi scientifici.
\app{Gretl} fornisce vari comandi che consentono di vedere un'anteprima
o di stampare i risultati delle analisi econometriche usando 
\TeX, oltre che di salvare i risultati in modo da poter essere
successivamente modificati con \TeX.

Questo capitolo spiega in dettaglio le funzionalit� di \app{gretl}
che interessano \TeX. La prima sezione descrive i men� nell'interfaccia
grafica; la sezione~\ref{tex-tune} discute vari modi di personalizzare
i risultati prodotti da \TeX, mentre la sezione~\ref{tex-install}
fornisce alcune indicazioni per installare \TeX, se ancora non lo si ha
sul proprio computer. Per chiarezza: \TeX\ non � incluso nella distribuzione
standard di \app{gretl}, � un pacchetto separato, che comprende vari programmi
e file di supporto.

Prima di procedere, pu� essere utile chiarire brevemente il modo in cui
viene prodotto un documento usando \TeX. La maggior parte di questi dettagli
non necessitano di alcuna preoccupazione da parte dell'utente, visto che
sono gestiti automaticamente da \app{gretl}, ma una comprensione di base dei
meccanismi all'opera pu� aiutare nella scelta delle opzioni a disposizione.

Il primo passo consiste nella creazione di un file ``sorgente'' testuale,
che contiene il testo o le espressioni matematiche che compongono il documento,
assieme ad alcuni comandi mark-up che regolano la formattazione del documento.
Il secondo passo consiste nel fornire il file a un motore che esegue la
formattazione vera e propria. Tipicamente, si tratta di:
\begin{itemize}
\item un programma chiamato \app{latex}, che genera un risultato in formato
  DVI (device-independent), oppure
\item un programma chiamato \app{pdflatex}, che genera un risultato in formato
  PDF\footnote{Gli utenti pi� esperti hanno dimestichezza anche con il
  cosiddetto ``plain \TeX'', che viene processato dal programma \app{tex}.
  La maggioranza degli utenti di \TeX, comunque, usa le macro fornite dal
  programma \LaTeX, sviluppato inizialmente da Leslie Lamport. \app{Gretl}
  non supporta il plain \TeX.}.
\end{itemize}

Per visualizzare le anteprime, si pu� utilizzare un visualizzatore DVI,
(tipicamente \app{xdvi} sui sistemi GNU/Linux) o PDF (tipicamente Adobe Acrobat
Reader o \app{xpdf}, a seconda di come � stato processato il sorgente.  Se si
sceglie la via del DVI, c'� un ultimo passo per creare un documento stampabile,
che tipicamente richiede il programma \app{dvips} per generare un file in
formato PostScript. Se si sceglie la via del PDF, il documento � pronto per
essere stampato.

Sulle piattaforme MS Windows e Mac OS X, \app{gretl} richiama
\app{pdflatex} per processare il file sorgente e si aspetta che il sistema
operativo sia in grado di eseguire un programma per visualizzare il file PDF; il
formato DVI non � supportato. Su GNU/Linux, la scelta predefinita � per il formato
DVI, ma se si preferisce usare il PDF basta seguire queste istruzioni:
dal men� ``Strumenti, Preferenze, Generali'', nella sezione ``Programmi'',
impostare il ``Comando per compilare i file TeX'' a \texttt{pdflatex}; inoltre
assicurarsi che il ``Comando per visualizzare i file PDF'' sia ben impostato.

\section{I comandi \TeX nei men�}
\label{tex-menus}

La maggior parte dei comandi di \app{gretl} relativi a \TeX\ si trova nel men�
``LaTeX'' della finestra dei modelli, che contiene le voci ``Visualizza'',
``Copia'', ``Salva'' e ``Opzioni equazione''. Ognuna delle prime tre voci, a sua
volta, contiene due opzioni, intitolate ``Tabella'' ed ``Equazione''.
Selezionando ``Tabella'', il modello viene rappresentato in forma tabulare,
consentendo la rappresentazione pi� completa ed esplicita dei risultati;
l'esempio mostrato nella Tabella~\ref{tab:mod1} � stato creato proprio
scegliendo il comando ``Copia, Tabella'' di \app{gretl} (per brevit� sono state
omesse alcune righe).

\begin{table}[htbp]
\caption{Esempio dell'output \LaTeX\ tabulare}
\label{tab:mod1}
\begin{center}

Modello 1: Stime OLS usando le 51 osservazioni 1-51\\
Variabile dipendente: ENROLL\\

\vspace{1em}

\begin{tabular*}{\textwidth}{@{\extracolsep{\fill}}
l% col 1: varname
  D{.}{.}{-1}% col 2: coeff
    D{.}{.}{-1}% col 3: sderr
      D{.}{.}{-1}% col 4: t-stat
        D{.}{.}{4}}% col 5: p-value (or slope)
Variabile &
  \multicolumn{1}{c}{Coefficiente} &
    \multicolumn{1}{c}{Errore\ Std.} &
      \multicolumn{1}{c}{statistica $t$} &
        \multicolumn{1}{c}{p-value} \\[1ex]
const &
  0.241105 &
    0.0660225 &
      3.6519 &
        0.0007 \\
CATHOL &
  0.223530 &
    0.0459701 &
      4.8625 &
        0.0000 \\
PUPIL &
  -0.00338200 &
    0.00271962 &
      -1.2436 &
        0.2198 \\
WHITE &
  -0.152643 &
    0.0407064 &
      -3.7499 &
        0.0005 \\
\end{tabular*}

\vspace{1em}

\begin{tabular}{lD{.}{.}{-1}}
Media della variabile dipendente & 0.0955686 \\
 D.S. della variabile dipendente & 0.0522150 \\
Somma dei quadrati dei residui & 0.0709594 \\
Errore standard dei residui ($\hat{\sigma}$) & 0.0388558 \\
$R^2$ & 0.479466 \\
$\bar{R}^2$ corretto & 0.446241 \\
$F(3, 47)$ & 14.4306 \\
\end{tabular}
\end{center}
\end{table}

L'opzione ``Equazione'' si spiega da s�: i risultati della stima vengono scritti
sotto forma di equazione, nel modo seguente

%%% the following needs the amsmath LaTeX package

\begin{gather}
\widehat{\rm ENROLL} = 
\underset{(0.066022)}{0.241105}
+\underset{(0.04597)}{0.223530}\,\mbox{CATHOL}
-\underset{(0.0027196)}{0.00338200}\,\mbox{PUPIL}
-\underset{(0.040706)}{0.152643}\,\mbox{WHITE}
 \notag \\
T = 51 \quad \bar{R}^2 = 0.4462 \quad F(3,47) = 14.431 \quad \hat{\sigma} = 0.038856\notag \\
\centerline{(errori standard tra parentesi)} \notag
\end{gather}

La differenza tra i comandi ``Copia'' e ``Salva'' � di due tipi. Innanzitutto,
``Copia'' copia il sorgente \TeX\ negli appunti, mentre ``Salva'' richiede il
nome di un file in cui verr� salvato il sorgente. In secondo luogo, con
``Copia'' il materiale viene copiato come ``frammento'' di file \TeX, mentre con
``Salva'' viene salvato come file completo. La differenza si spiega tenendo
conto che un file sorgente \TeX\ completo comprende un preambolo che contiene
comandi come \verb|\documentclass|, che definisce il tipo di documento (ad
esempio articolo, rapporto, libro, ecc.) e come \verb|\begin{document}| e
\verb|\end{document}|, che delimitano l'inizio e la fine del documento. Questo
materiale aggiuntivo viene incluso quando si sceglie ``Salva'', ma non quando si
sceglie ``Copia'', visto che in questo caso si presume che l'utente incoller� i
dati in un file sorgente \TeX\ gi� provvisto del preambolo richiesto.

I comandi del men� ``Opzioni equazione'' sono chiari: determinano se, quando
si stampa il modello sotto forma di equazione, verranno mostrati gli errori
standard o i rapporti $t$ tra parentesi sotto le stime dei parametri.
L'impostazione predefinita � quella di mostrare gli errori standard.


\section{Personalizzazione dei documenti}
\label{tex-tune}

\section{Installazione di \TeX}
\label{tex-install}




%%% Local Variables: 
%%% mode: latex
%%% TeX-master: "gretl-guide"
%%% End: 

\chapter{Risoluzione dei problemi}
\label{trouble}



\section{Segnalazione dei bug}
\label{trouble-bugs}


Le segnalazioni dei bug sono benvenute. � difficile trovare errori di
calcolo in \app{gretl} (ma questa affermazione non costituisce alcuna
sorta di garanzia), ma � possibile imbattersi in bug o stranezze nel
comportamento dell'interfaccia grafica. Si tenga presente che
l'utilit� delle segnalazioni aumenta quanto pi� si � precisi nella
descrizione: cosa \emph{esattamente} non funziona, in che condizioni,
con quale sistema operativo? Se si ricevono messaggi di errore, cosa
dicono esattamente?

\section{Programmi ausiliari}
\label{trouble-programs}

Come detto in precedenza, \app{gretl} richiama alcuni altri programmi
per eseguire alcune operazioni (gnuplot per i grafici, {\LaTeX} per la
stampa ad alta qualit� dei risultati delle regressioni, GNU R).  Se
succede qualche problema durante questi collegamenti esterni, non �
sempre facile per \app{gretl} fornire un messaggio di errore abbastanza informativo.
Se il problema si verifica durante l'uso di \app{gretl} con
l'interfaccia grafica, pu� essere utile avviare \app{gretl} da un terminale
invece che da un men� o da un'icona del desktop. Se si usa il sistema
X window, � sufficiente avviare gretl dal prompt dei comandi di un 
\app{xterm}, mentre se si usa MS Windows occorre digitare \cmd{gretlw32.exe} da
una finestra di terminale, o ``Prompt di MS-DOS'', usando le opzioni \verb|-g|
o \verb|--debug|. Il terminale conterr� quindi messaggi di errore aggiuntivi.

Si tenga anche presente che nella maggior parte dei casi \app{gretl}
assume che i programmi in questione siano disponibili nel ``percorso
di esecuzione'' (path) dell'utente, ossia che possano essere invocati
semplicemente con il nome del programma, senza indicare il percorso
completo\footnote{L'eccezione a questa regola � costituita
  dall'invocazione di gnuplot in MS Windows, dove occorre indicare il
  percorso completo del programma.}.  Quindi se un certo programma non
si avvia, conviene provare ad eseguirlo da un prompt dei comandi, come
descritto di seguito.
      
\begin{center}
  \begin{tabular}{llll}
    & \textit{Grafica} & \textit{Stampa} & \textit{GNU R}\\
    Sistema X window & gnuplot & latex, xdvi & R\\
    MS Windows & wgnuplot.exe & pdflatex & RGui.exe\\
  \end{tabular}
\end{center}

Se il programma non si avvia dal prompt, non � un problema di
\app{gretl}: probabilmente la directory principale del programma non �
contenuta nel path dell'utente, oppure il programma non � stato
installato correttamente.  Per i dettagli su come modificare il path
dell'utente, si veda la documentazione o l'aiuto online per il proprio
sistema operativo.
    
%%% Local Variables: 
%%% mode: latex
%%% TeX-master: "gretl-guide-it"
%%% End: 


\chapter{The command line interface}
\label{cli}



\section{Gretl at the console}
\label{cli-console}


The \app{gretl} package includes the command-line program
\app{gretlcli}. On Linux it can be run from the console, or in an
xterm (or similar).  Under MS Windows it can be run in a console
window (sometimes inaccurately called a ``DOS box'').  \app{gretlcli}
has its own help file, which may be accessed by typing ``help'' at the
prompt. It can be run in batch mode, sending outout directly to a file
(see alse the \emph{Gretl Command Reference}).
    
If \app{gretlcli} is linked to the \app{readline} library (this is
automatically the case in the MS Windows version; also see [?]), the
command line is recallable and editable, and offers command
completion.  You can use the Up and Down arrow keys to cycle through
previously typed commands.  On a given command line, you can use the
arrow keys to move around, in conjunction with Emacs editing
keystokes.\footnote{Actually, the key bindings shown below are only
  the defaults; they can be customized.  See the
  \href{http://cnswww.cns.cwru.edu/~chet/readline/readline.html}{readline
    manual}.} The most common of these are:
%    
\begin{center}
  \begin{tabular}{ll}
    Keystroke & Effect\\
    \verb+Ctrl-a+ & go to start of line\\
    \verb+Ctrl-e+ & go to end of line\\
    \verb+Ctrl-d+ & delete character to right\\
  \end{tabular}
\end{center}
%
where ``\verb+Ctrl-a+'' means press the ``\verb+a+'' key while the
``\verb+Ctrl+'' key is also depressed.  Thus if you want to change
something at the beginning of a command, you \emph{don't} have to
backspace over the whole line, erasing as you go.  Just hop to the
start and add or delete characters.  If you type the first letters of
a command name then press the Tab key, readline will attempt to
complete the command name for you.  If there's a unique completion it
will be put in place automatically.  If there's more than one
completion, pressing Tab a second time brings up a list.

\section{Changes from Ramanathan's ESL}
\label{cli-syntax}

\app{gretlcli} inherits its basic command syntax from Ramu
Ramanathan's \app{ESL}, and command scripts developed for \app{ESL}
should be usable with few or no changes: the only things to watch for
are multi-line commands and the \cmd{freq} command.
    
\begin{itemize}
\item In \app{ESL}, a semicolon is used as a terminator for many
  commands.  I decided to remove this in \app{gretlcli}. Semicolons
  are simply ignored, apart from a few special cases where they have a
  definite meaning: as a separator for two lists in the \cmd{ar} and
  \cmd{tsls} commands, and as a marker for an unchanged starting or
  ending observation in the \cmd{smpl} command. In \app{ESL} semicolon
  termination gives the possibility of breaking long commands over
  more than one line; in \app{gretlcli} this is done by putting a
  trailing backslash \verb+\+ at the end of a line that is to be
  continued.
\item With \cmd{freq}, you can't at present specify user-defined
  ranges as in \app{ESL}.  A chi-square test for normality has been
  added to the output of this command.
\end{itemize}

Note also that the command-line syntax for running a batch job is
simplified. For \app{ESL} you typed, e.g.
      
\begin{code} 
        esl -b datafile < inputfile > outputfile
\end{code}
%
while for \app{gretlcli} you type:
%      
\begin{code}
	gretlcli -b inputfile > outputfile
\end{code}

The inputfile is treated as a program argument; it should specify a
datafile to use internally, using the syntax \cmd{open datafile} or
the special comment \verb+(* !+ datafile \verb+*)+
    
%%% Local Variables: 
%%% mode: latex
%%% TeX-master: "gretl-guide"
%%% End: 



\part{Appendici}

\begin{appendices}
\chapter{Data file details}
\label{app-datafile}

\section{Basic native format}
\label{native}

In \app{gretl}'s native data format, a data set is stored in XML
(extensible mark-up language). Data files correspond to the simple DTD
(document type definition) given in \verb+gretldata.dtd+, which is
supplied with the \app{gretl} distribution and is installed in the
system data directory (e.g.\ \url{/usr/share/gretl/data} on Linux.)
Data files may be plain text or gzipped.  They contain the actual data
values plus additional information such as the names and descriptions
of variables, the frequency of the data, and so on.

Most users will probably not have need to read or write such files
other than via \app{gretl} itself, but if you want to manipulate them
using other software tools you should examine the DTD and also take a
look at a few of the supplied practice data files: \verb+data4-1.gdt+
gives a simple example; \verb+data4-10.gdt+ is an example where
observation labels are included.

\section{Traditional ESL format}
\label{traddata}

For backward compatibility, \app{gretl} can also handle data files in
the ``traditional'' format inherited from Ramanathan's \app{ESL}
program.  In this format (which was the default in \app{gretl} prior
to version 0.98) a data set is represented by two files.  One contains
the actual data and the other information on how the data should be
read.  To be more specific:

\begin{enumerate}
\item \emph{Actual data}: A rectangular matrix of white-space
  separated numbers.  Each column represents a variable, each row an
  observation on each of the variables (spreadsheet style). Data
  columns can be separated by spaces or tabs. The filename should have
  the suffix \verb+.gdt+.  By default the data file is ASCII (plain
  text).  Optionally it can be gzip-compressed to save disk space. You
  can insert comments into a data file: if a line begins with the hash
  mark (\verb+#+) the entire line is ignored. This is consistent with
  gnuplot and octave data files.
\item \emph{Header}: The data file must be accompanied by a header
  file which has the same basename as the data file plus the suffix
  \verb+.hdr+.  This file contains, in order:
  \begin{itemize}
  \item (Optional) \emph{comments} on the data, set off by the opening
    string \verb+(*+ and the closing string \verb+*)+, each of these
    strings to occur on lines by themselves.
  \item (Required) list of white-space separated \emph{names of the
      variables} in the data file. Names are limited to 8 characters,
    must start with a letter, and are limited to alphanumeric
    characters plus the underscore.  The list may continue over more
    than one line; it is terminated with a semicolon, \verb+;+.
  \item (Required) \emph{observations} line of the form \verb+1 1 85+.
    The first element gives the data frequency (1 for undated or
    annual data, 4 for quarterly, 12 for monthly).  The second and
    third elements give the starting and ending observations.
    Generally these will be 1 and the number of observations
    respectively, for undated data.  For time-series data one can use
    dates of the form \cmd{1959.1} (quarterly, one digit after the
    point) or \cmd{1967.03} (monthly, two digits after the point).
    See Chapter~\ref{chap-panel} for special use of this line in the
    case of panel data.
  \item The keyword \verb+BYOBS+.
  \end{itemize}
\end{enumerate}

Here is an example of a well-formed data header file.

\begin{code} 
(* 
  DATA9-6: 
  Data on log(money), log(income) and interest rate from US. 
  Source: Stock and Watson (1993) Econometrica 
  (unsmoothed data) Period is 1900-1989 (annual data). 
  Data compiled by Graham Elliott. 
*) 
lmoney lincome intrate ; 
1 1900 1989 BYOBS
\end{code}

The corresponding data file contains three columns of data, each
having 90 entries.  Three further features of the ``traditional'' data
format may be noted.
    
\begin{enumerate}
\item If the \verb+BYOBS+ keyword is replaced by \verb+BYVAR+, and
  followed by the keyword \verb+BINARY+, this indicates that the
  corresponding data file is in binary format.  Such data files can be
  written from \app{gretlcli} using the \cmd{store} command with the
  \cmd{-s} flag (single precision) or the \cmd{-o} flag (double
  precision).
\item If \verb+BYOBS+ is followed by the keyword \verb+MARKERS+,
  \app{gretl} expects a data file in which the \emph{first column}
  contains strings (8 characters maximum) used to identify the
  observations.  This may be handy in the case of cross-sectional data
  where the units of observation are identifiable: countries, states,
  cities or whatever.  It can also be useful for irregular time series
  data, such as daily stock price data where some days are not trading
  days --- in this case the observations can be marked with a date
  string such as \cmd{10/01/98}.  (Remember the 8-character maximum.)
  Note that \cmd{BINARY} and \cmd{MARKERS} are mutually exclusive
  flags.  Also note that the ``markers'' are not considered to be a
  variable: this column does not have a corresponding entry in the
  list of variable names in the header file.
\item If a file with the same base name as the data file and header
  files, but with the suffix \verb+.lbl+, is found, it is read to fill
  out the descriptive labels for the data series. The format of the
  label file is simple: each line contains the name of one variable
  (as found in the header file), followed by one or more spaces,
  followed by the descriptive label. Here is an example:
  \verb+price New car price index, 1982 base year+
\end{enumerate}

If you want to save data in traditional format, use the \cmd{-t} flag
with the \cmd{store} command, either in the command-line program or in
the console window of the GUI program.


\section{Binary database details}
\label{dbdetails}

A \app{gretl} database consists of two parts: an ASCII index file
(with filename suffix \verb+.idx+) containing information on the
series, and a binary file (suffix \verb+.bin+) containing the actual
data.  Two examples of the format for an entry in the \verb+idx+ file
are shown below:

\begin{code}
G0M910  Composite index of 11 leading indicators (1987=100) 
M 1948.01 - 1995.11  n = 575
currbal Balance of Payments: Balance on Current Account; SA 
Q 1960.1 - 1999.4 n = 160
\end{code}

The first field is the series name.  The second is a description of
the series (maximum 128 characters).  On the second line the first
field is a frequency code: \verb+M+ for monthly, \verb+Q+ for
quarterly, \verb+A+ for annual, \verb+B+ for business-daily (daily
with five days per week) and \verb+D+ for daily (seven days per week).
No other frequencies are accepted at present.  Then comes the starting
date (N.B. with two digits following the point for monthly data, one
for quarterly data, none for annual), a space, a hyphen, another
space, the ending date, the string ``\verb+n = +'' and the integer
number of observations. In the case of daily data the starting and
ending dates should be given in the form \verb+YYYY/MM/DD+. This
format must be respected exactly.

Optionally, the first line of the index file may contain a short
comment (up to 64 characters) on the source and nature of the data,
following a hash mark.  For example:

\begin{code}
# Federal Reserve Board (interest rates)
\end{code}

The corresponding binary database file holds the data values,
represented as ``floats'', that is, single-precision floating-point
numbers, typically taking four bytes apiece.  The numbers are packed
``by variable'', so that the first \emph{n} numbers are the
observations of variable 1, the next \emph{m} the observations on
variable 2, and so on.

\chapter{Technical notes}
\label{app-technote}

\app{Gretl} is written in the C programming language, abiding as far
as possible by the ISO/ANSI C Standard (C90) although the graphical
user interface and some other components necessarily make use of
platform-specific extensions.
  
The program was developed under Linux. The shared library and
command-line client should compile and run on any platform that (a)
supports ISO/ANSI C, and (b) has the following libraries installed:
zlib (data compression), libxml2 (XML manipulation), and LAPACK
(linear algebra support). The homepage for zlib can be found at
\href{http://www.info-zip.org/pub/infozip/zlib/}{info-zip.org};
libxml2 is at \href{http://xmlsoft.org/}{xmlsoft.org}; LAPACK is at
\href{http://www.netlib.org/lapack/}{netlib.org}. If the GNU readline
library is found on the host system this will be used for
\app{gretcli}, providing a much enhanced editable command line.  See
the
\href{http://cnswww.cns.cwru.edu/~chet/readline/rltop.html}{readline
  homepage}.

The graphical client program should compile and run on any system
that, in addition to the above requirements, offers GTK version 2.4.0
or higher (see \href{http://www.gtk.org/}{gtk.org}).  As of this
writing there are two main variants of the GTK libraries: the 1.2
series and the 2.0 series which was launched in summer 2002.  These
variants are mutually incompatible.  Up to version 1.5.1, \app{gretl}
could be built using either variant of GTK, but at version 1.6.0 we
dropped support for GTK 1.2.
  
\app{Gretl} calls gnuplot for graphing. You can find gnuplot at
\href{http://www.gnuplot.info/}{gnuplot.info}.  As of this writing the
most recent official release is 4.0 (of April, 2004).  The MS Windows
version of \app{gretl} comes with a Windows version gnuplot 4.0; the
gretl website also offers an rpm of gnuplot 3.8j0 for x86 Linux
systems.
  
Some features of \app{gretl} make use of portions of Adrian Feguin's
\app{gtkextra} library.  The relevant parts of this package are
included (in slightly modified form) with the \app{gretl} source
distribution.
  
A binary version of the program is available for the Microsoft Windows
platform (Windows 98 or higher). This version was cross-compiled under
Linux using mingw (the GNU C compiler, \app{gcc}, ported for use with
win32) and linked against the Microsoft C library, \verb+msvcrt.dll+.
It uses Tor Lillqvist's port of GTK 2.0 to win32.  The (free,
open-source) Windows installer program is courtesy of Jordan Russell
(\href{http://www.jrsoftware.org/}{jrsoftware.org}).

We're hopeful that some users with coding skills may consider
\app{gretl} sufficiently interesting to be worth improving and
extending.  The documentation of the libgretl API is by no means
complete, but you can find some details by following the link
``Libgretl API docs'' on the \app{gretl} homepage. People interested
in the \app{gretl} development are welcome to subscribe to the
\href{http://gretl.sourceforge.net/lists.html}{gretl-devel} mailing
list.

\chapter{Numerical accuracy}
\label{app-accuracy}

\app{Gretl} uses double-precision arithmetic throughout --- except for
the multiple-precision plugin invoked by the menu item ``Model, Other
linear models, High precision OLS'' which represents floating-point values using a number
of bits given by the environment variable \verb+GRETL_MP_BITS+
(default value 256).  The normal equations of Least Squares are by
default solved via Cholesky decomposition, which is accurate enough
for most purposes (with the option of using QR decomposition instead).
The program has been tested rather thoroughly on the statistical
reference datasets provided by NIST (the U.S. National Institute of
Standards and Technology) and a full account of the results may be
found on the gretl website (follow the link ``Numerical accuracy'').

Giovanni Baiocchi and Walter Distaso published a review of \app{gretl}
in the \emph{Journal of Applied Econometrics} (2003).  We are grateful
to Baiocchi and Distaso for their careful examination of the program,
which prompted the following modifications.

\begin{enumerate}
\item The reviewers pointed out that there was a bug in \app{gretl}'s
  ``p-value finder'', whereby the program printed the complement of
  the correct probability for negative values of \emph{z}.  This was
  fixed in version 0.998 of the program (released July 9, 2002).
\item They also noted that the p-value finder produced inaccurate
  results for extreme values of \emph{x} (e.g.\ values of around 8 to
  10 in the \emph{t} distribution with 100 degrees of freedom).  This
  too was fixed in \app{gretl} version 0.998, with a switch to more
  accurate probability distribution code.
\item The reviewers noted a flaw in the presentation of regression
  coefficients in \app{gretl}, whereby some coefficients could be
  printed to an unacceptably small number of significant figures.
  This was fixed in version 0.999 (released August 25, 2002): now all
  the statistics associated with a regression are printed to 6
  significant figures.
\item It transpired from the reviewer's tests that the numerical
  accuracy of \app{gretl} on MS Windows was less than on Linux.  For
  example, on the Longley data --- a well-known ``ill-conditioned''
  dataset often used for testing econometrics programs --- the Windows
  version of gretl was getting some coefficients wrong at the 7th
  digit while the same coefficients were correct on Linux.  This
  anomaly was fixed in \app{gretl} version 1.0pre3 (released October
  10, 2002).
\end{enumerate}

The current version of \app{gretl} includes a ``plugin'' that runs the
NIST linear regression test suite.  You can find this under the
``Tools'' menu in the main window.  When you run this test, the
introductory text explains the expected result.  If you run this test
and see anything other than the expected result, please send a bug
report to \verb+cottrell@wfu.edu+.  

As mentioned above, all regression statistics are printed to 6
significant figures in the current version of \app{gretl} (except when
the multiple-precision plugin is used, then results are given to 12
figures).  If you want to examine a particular value more closely,
first save it (for example, using the \cmd{genr} command) then print
it using \cmd{print --long} (see the \emph{Gretl Command Reference}).
This will show the value to 10 digits (or more, if you set the
internal variable \texttt{longdigits} to a higher value via the
\cmd{set} command).

\chapter{Related free software}
\label{app-advanced}

\app{Gretl}'s capabilities are substantial, and are expanding.
Nonetheless you may find there are some things you can't do in
\app{gretl}, or you may wish to compare results with other programs.
If you are looking for complementary functionality in the realm of
free, open-source software we recommend the following programs.  The
self-description of each program is taken from its website.

\begin{itemize}

\item \textbf{GNU R} \href{http://www.r-project.org/}{r-project.org}:
  ``R is a system for statistical computation and graphics. It
  consists of a language plus a run-time environment with graphics, a
  debugger, access to certain system functions, and the ability to run
  programs stored in script files\dots\ It compiles and runs on a wide
  variety of UNIX platforms, Windows and MacOS.''  Comment: There are
  numerous add-on packages for R covering most areas of statistical
  work.

\item \textbf{GNU Octave}
  \href{http://www.octave.org/}{www.octave.org}:
  ``GNU Octave is a high-level language, primarily intended for
  numerical computations. It provides a convenient command line
  interface for solving linear and nonlinear problems numerically, and
  for performing other numerical experiments using a language that is
  mostly compatible with Matlab. It may also be used as a
  batch-oriented language.''

\item \textbf{JMulTi} \href{http://www.jmulti.de/}{www.jmulti.de}:
  ``JMulTi was originally designed as a tool for certain econometric
  procedures in time series analysis that are especially difficult to
  use and that are not available in other packages, like Impulse
  Response Analysis with bootstrapped confidence intervals for VAR/VEC
  modelling. Now many other features have been integrated as well to
  make it possible to convey a comprehensive analysis.''  Comment:
  JMulTi is a java GUI program: you need a java run-time environment to
  make use of it.

\end{itemize}

As mentioned above, \app{gretl} offers the facility of exporting
data in the formats of both Octave and R.  In the case of Octave, the
\app{gretl} data set is saved as a single matrix, \verb+X+. You can
pull the \verb+X+ matrix apart if you wish, once the data are loaded
in Octave; see the Octave manual for details.  As for R, the exported
data file preserves any time series structure that is apparent to
\app{gretl}.  The series are saved as individual structures. The data
should be brought into R using the \cmd{source()} command.
  
In addition, \app{gretl} has a convenience function for moving data
quickly into R.  Under \app{gretl}'s ``Tools'' menu, you will find the
entry ``Start GNU R''.  This writes out an R version of the current
\app{gretl} data set (in the user's gretl directory), and sources it
into a new R session.  The particular way R is invoked depends on the
internal \app{gretl} variable \verb+Rcommand+, whose value may be set
under the ``Tools, Preferences'' menu.  The default command is
\cmd{RGui.exe} under MS Windows. Under X it is \cmd{xterm -e R}.
Please note that at most three space-separated elements in this
command string will be processed; any extra elements are ignored.

\chapter{Listing of URLs}
\label{app-urls}

Below is a listing of the full URLs of websites mentioned in the text.

\begin{description}

\item[Estima (RATS)] \url{http://www.estima.com/}
\item[FFTW3] \url{http://www.fftw.org/}
\item[Gnome desktop homepage] \url{http://www.gnome.org/}
\item[GNU Multiple Precision (GMP) library]
  \url{http://swox.com/gmp/}
\item[GNU Octave homepage] \url{http://www.octave.org/}
\item[GNU R homepage] \url{http://www.r-project.org/}
\item[GNU R manual]
  \url{http://cran.r-project.org/doc/manuals/R-intro.pdf}
\item[Gnuplot homepage] \url{http://www.gnuplot.info/}
\item[Gnuplot manual] \url{http://ricardo.ecn.wfu.edu/gnuplot.html}
\item[Gretl data page]
  \url{http://gretl.sourceforge.net/gretl_data.html}
\item[Gretl homepage] \url{http://gretl.sourceforge.net/}
\item[GTK+ homepage] \url{http://www.gtk.org/}
\item[GTK+ port for win32]
  \url{http://www.gimp.org/~tml/gimp/win32/}
\item[Gtkextra homepage] \url{http://gtkextra.sourceforge.net/}
\item[InfoZip homepage]
  \url{http://www.info-zip.org/pub/infozip/zlib/}
\item[JMulTi homepage] \url{http://www.jmulti.de/}
\item[JRSoftware] \url{http://www.jrsoftware.org/}
\item[Mingw (gcc for win32) homepage] \url{http://www.mingw.org/}
\item[Minpack] \url{http://www.netlib.org/minpack/}
\item[Penn World Table] \url{http://pwt.econ.upenn.edu/}
\item[Readline homepage]
  \url{http://cnswww.cns.cwru.edu/~chet/readline/rltop.html}
\item[Readline manual]
  \url{http://cnswww.cns.cwru.edu/~chet/readline/readline.html}
\item[Xmlsoft homepage] \url{http://xmlsoft.org/}

\end{description}


%%% Local Variables: 
%%% mode: latex
%%% TeX-master: "gretl-guide"
%%% End: 


\end{appendices}

\clearpage

\bibliographystyle{gretl}
\bibliography{gretl}

%%% Local Variables: 
%%% mode: latex
%%% TeX-master: "gretl-guide"
%%% End: 



\end{document}
