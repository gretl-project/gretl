\chapter{Comentarios en los guiones}
\label{chap:comments}

Cuando un guion hace algo que no es obvio, es una buena idea añadir
comentarios que expliquen lo que está sucediendo. En concreto esto es
útil si planeas compartir el guion con los demás, pero también lo es
como recordatorio para uno mismo --- cuando vuelves a retomar un guion
algunos meses más tarde y te preguntas que se suponía que hacía.

El mecanismo de hacer comentarios también puede ser de ayuda cuando estás
desarrollando un guion. Pues puede llegar un punto en el que quieres
ejecutar un guion, pero saltándote la ejecución de una parte del mismo.
Obviamente, puedes eliminar la parte que quieres saltarte, pero en lugar
de perder esa sección puedes ``ponerle un comentario'' de forma que
\app{GRETL} no la tenga en cuenta.

Se admiten dos tipos de comentarios en los guiones de \app{GRETL}.
El más simple es este:

\begin{itemize}
\item Cuando se encuentra una marca de almohadilla, \texttt{\#}, en un
  guion de \app{GRETL}, se trata todo lo que está desde ese lugar hasta
  el final de la línea considerada como un comentario, y se deja de lado.
\end{itemize}

Si quieres ``comentar'' varias líneas utilizando esta modalidad, tendrás
que colocar una marca de almohadilla al principio de cada una de esas líneas.

El segundo tipo de comentarios sigue el patrón del lenguaje C de programación:

\begin{itemize}
\item Cuando se encuentra en un guion la secuencia \texttt{/*}, todo
  lo que la sigue se trata como un comentario hasta que se encuentre la
  secuencia \texttt{*/}.
\end{itemize}

Los comentarios de este tipo pueden extenderse por varias líneas. Utilizando
esta modalidad es fácil añadir un texto explicativo extenso, o conseguir
que \app{GRETL} deje de lado a bloques considerables de instrucciones. Como
en C, los comentarios de este tipo no pueden anidarse.

¿Como interactúan estas dos modalidades de hacer comentarios? Puedes imaginar
que \app{GRETL} empieza en el principio de un guion, y que intenta decidir en
cada punto si debe o no debe estar en la ``modalidad de omisión''. Cuando
se hace esto, se aplican las siguientes reglas:

\begin{itemize}
\item Si no estamos en la modalidad de omisión, entonces \texttt{\#} nos coloca
  en la modalidad de omisión hasta el final de la línea vigente.
\item Si no estamos en la modalidad de omisión, entonces \texttt{/*} nos coloca
  en la modalidad de omisión hasta que se encuentre \texttt{*/}.
\end{itemize}

Esto significa que cada tipo de comentario puede estar enmascarado por el otro.

\begin{itemize}
\item Si \texttt{/*} sigue a \texttt{\#} en una determinada línea que no
  empieza en la modalidad de omisión, entonces no hay nada de especial en
  cuanto a \texttt{/*}; es solo una parte de un comentario de estilo \texttt{\#}.
\item Si aparece \texttt{\#} cuando ya estamos en la modalidad de omisión,
  ello solo es una parte más de un comentario.
\end{itemize}

A continuación tenemos algunos ejemplos.
%
\begin{code}
/* Comentario de varias líneas
   # Hola
   # Hola */
\end{code}
%
En el ejemplo de arriba, las marcas de almohadilla no tienen nada de especial;
en concreto, la marca de almohadilla de la tercera línea no evita que ese
comentario de varias líneas acabe en \texttt{*/}.
%
\begin{code}
# Comentario de una única línea /* Hola
\end{code}
%
Suponiendo que no estamos en la modalidad de omisión antes de la línea que
se muestra arriba, esta indica un comentario de una única línea: el
\texttt{/*} está enmascarado, y no abre un comentario de varias líneas.

Puedes añadir un comentario a una instrucción:
%
\begin{code}
ols 1 0 2 3 # Estimar el modelo de partida
\end{code}
%
Ejemplo de ``comentario'':
%
\begin{code}
/*
# Vamos a saltarnos esto por el momento
ols 1 0 2 3 4
omit 3 4
*/
\end{code}
%
